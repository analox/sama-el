\subsection{Self Adaptative Methods}

%---------------------------------------------------------------------------%
\index{mutateLogNormal!( double overallStdDev, double indivStdDev )}
\setNormalInstance
\setCorrectWidthThree{8pt}
\setParamOne{overallStdDev}{double}{Standard deviation for the mutation
of the stepsize for all components of {\em this}.}
\setParamTwo{indivStdDev}{double}{Standard deviation for the mutation
of the stepsize for one component of {\em this}.}
\printMethodWithParamsSaved
{void}
{}
{mutateLogNormal}
{Mutates {\em this} by adding a normally distributed random number between
 ``0'' and $\sigma_i^2$ to each allele. The stepsize $\sigma_i$ will also
 be mutated for each component of {\em this} separately (parameter 
 {\em indivStdDev}) and for all components (parameter {\em overallStdDev}).
 For more information cf. \cite{EALibRef}, page 20.}
{}
\setCorrectWidthThree{4pt}
%---------------------------------------------------------------------------%

\clearpage

%---------------------------------------------------------------------------%
\index{mutateRotate!( ChromosomeT$<$ double $>$\& sigma )}
\setNormalInstance
\printMethodWithOneParam
{void}
{mutateRotate}
{ChromosomeT$<$ double $>$ \&}
{sigma}
{Chromosome with standard deviations for the normal distribution.}
{Mutates {\em this} and the strategy variables by using a
 rotation matrix. For more information cf.
 \cite{EALibRef}, p. 20 ff.}
{None.}
{None.}
%---------------------------------------------------------------------------%

\vspace*{4ex}

%---------------------------------------------------------------------------%
\index{mutateRotate!( ChromosomeT$<$ double $>$\& sigma, double tau1, double tau2, double beta, int sigmaCheck, double epsi )}
\setNormalInstance
\setCorrectWidthThree{8pt}
\setParamOne{sigma}{ChromosomeT$<$ double $>$\&}
{Chromosome with standard deviations for the normal distribution.}
\setParamTwo{tau1}{double}{Stepsize adaptation for all individuals.}
\setParamThree{tau2}{double}{Stepsize adaptation for one individual.}
\setParamFour{beta}{double}{Parameter $\beta$ for damping the
stepsize variation between successive generations.}
\setParamFive{sigmaCheck}{int}{If ``true''
check {\em sigma} for too small values.} 
\setParamSix{epsi}{double}{Lower boundary for the {\em sigma} values.}
\printMethodWithParamsSaved 
{void}
{}
{mutateRotate}
{Same as above, but with more parameters.}
{}
\setCorrectWidthThree{4pt}
%---------------------------------------------------------------------------%

\clearpage

%---------------------------------------------------------------------------%
\index{mutateNormalRotAngles!( const Chromosome\& sigma, const Chromosome\& alpha )}
\setNormalInstance
\setCorrectWidthThree{8pt}
\setParamOne{sigma}{const Chromosome\&}
{Chromosome with standard deviations for the normal distribution.}
\setParamTwo{alpha}{const Chromosome\&}
{Chromosome with anchor points for the rotations.}
\printMethodWithParamsSaved
{void}
{}
{mutateNormalRotAngles}
{Mutates {\em this} and the strategy variables by using a
 rotation matrix. For the rotations the rotation-anchor-points in
 {\em alpha} will be used. For more information cf.
 \cite{GSA}.}
{}
\setCorrectWidthThree{4pt}
%---------------------------------------------------------------------------%

\vspace*{4ex}

%---------------------------------------------------------------------------%
\index{mutateDerandom!( vector$<$ double $>$\& v, const DerandomConst\& K )}
\setNormalInstance
\setCorrectWidthThree{8pt}
\setParamOne{v}{vector$<$ double $>$\&}{Vector with parameter values for
the mutation method.}
\setParamTwo{K}{const DerandomConst\&}{Instance of the class 
{\em DerandomConst} with some constants for the mutation method.}
\printMethodWithParamsSaved
{void}
{}
{mutateDerandom}
{Mutates {\em this} by using the {\em Derandom} method. For more
 information cf. \cite{GSA}.}
{}
\setCorrectWidthThree{4pt}
%---------------------------------------------------------------------------%

\vspace*{4ex}

%---------------------------------------------------------------------------%
\index{mutateDerandom!( Chromosome\& chrom, const DerandomConst\& K )}
\setNormalInstance
\setCorrectWidthThree{8pt}
\setParamOne{chrom}{Chromosome\&}{Chromosome with parameter values for
the mutation method.}
\setParamTwo{K}{const DerandomConst\&}{See above.}
\printMethodWithParamsSaved
{void}
{}
{mutateDerandom}
{Same as above, but a chromosome is used to store the parameter
 values.}
{}
\setCorrectWidthThree{4pt}
%---------------------------------------------------------------------------%

\clearpage

%---------------------------------------------------------------------------%
\index{mutateCMA!( ChromosomeT$<$ double $>$\& sigma )}
\setNormalInstance
\printMethodWithOneParam
{void}
{mutateCMA}
{ChromosomeT$<$ double $>$ \&}
{sigma}
{Chromosome with standard deviations for the normal distribution.}
{Mutates {\em this} and the strategy parameters by using the mutation
 method with covariance matrix ({\em Covariance Matrix Adaptation}).
 For more information cf. \cite{CMA}.}
{None.}
{None.}
%---------------------------------------------------------------------------%

\vspace*{4ex}

%---------------------------------------------------------------------------%
\index{mutateCMA!( ChromosomeT$<$ double $>$\& sigma, double c, double cu, double ccov, double beta )}
\setNormalInstance
\setCorrectWidthThree{8pt}
\setParamOne{sigma}{ChromosomeT$<$ double $>$ \&}{Chromosome with standard 
deviations for the normal distribution.}
\setParamTwo{c}{double}{Value for determination of the cumulation
time for the sum vector {\em s}.}
\setParamThree{cu}{double}{Normalizes the variance of the sum vector {\em s}.}
\setParamFour{ccov}{double}{Time for the evaluation of the average
of the distribution $ss^t$ over the generation sequence.}
\setParamFive{beta}{double}{Parameter $\beta$ for damping the stepsize
varation between successive generations.}
\printMethodWithParamsSaved
{void}
{}
{mutateCMA}
{Same as above, but with more parameters.}
{}
\setCorrectWidthThree{4pt}
%---------------------------------------------------------------------------%

\vspace*{4ex}

%---------------------------------------------------------------------------%
\index{mutateMSR!( double xi\_prob )}
\setNormalInstance
\printMethodWithOneParam
{void}
{mutateMSR}
{double}
{xi\_prob}
{Probability to assign the value $\alpha = 1.5$ to the general
 stepsize variation factor $\xi$. The alternative value for
 $\xi$ is $\frac{1}{\alpha}$.}
{Mutates {\em this} and the strategy parameters by using the method
 with {\em mutative stepsize regulation}.
 For more information cf. \cite{MSR}.}
{None.}
{None.}
%---------------------------------------------------------------------------%

\clearpage

%---------------------------------------------------------------------------%
\index{mutateGSA!( ChromosomeT$<$ double $>$\& sigma )}
\setNormalInstance
\printMethodWithOneParam
{void}
{mutateGSA}
{ChromosomeT$<$ double $>$ \&}
{sigma}
{Chromosome with standard deviation for the normal distribution.}
{Mutates {\em this} and the strategy parameters by using the method
 with the {\em Generating Set Adaptation}.
 For more information cf. \cite{GSA}.}
{None.}
{None.}
%---------------------------------------------------------------------------%

\vspace*{4ex}

%---------------------------------------------------------------------------%
\index{mutateGSA!( ChromosomeT$<$ double $>$\& sigma, double beta, double xi\_const, double cu, double cm )}
\setNormalInstance
\setCorrectWidthThree{8pt}
\setParamOne{sigma}{ChromosomeT$<$ double $>$ \&}
{Chromosome with standard deviation for the normal distribution.}
\setParamTwo{beta}{double}{Parameter $\beta$ for damping the stepsize
variation between successive generations.}
\setParamThree{xi\_const}{double}{Factor for the stepsize adaptation.
Normally the values $1.5$ and $\frac{1}{1.5}$ will be assigned to 
{\em xi\_const} with the same probability. Here the factor can be
stated explicitly.}
\setParamFour{cu}{double}{Normalizing factor for the variances.}
\setParamFive{cm}{double}{Length adaptation factor.}
\printMethodWithParamsSaved
{void}
{}
{mutateGSA}
{Same as above, but with more parameters.} 
{}
\setCorrectWidthThree{4pt}
%---------------------------------------------------------------------------%

\vspace*{4ex}

%---------------------------------------------------------------------------%
\index{mutateIDA!( ChromosomeT$<$ double $>$\& sigma )}
\setNormalInstance
\printMethodWithOneParam
{void}
{mutateIDA}
{ChromosomeT$<$ double $>$ \&}
{sigma}
{Chromosome with standard deviations for the normal distribution.}
{Mutates {\em this} and the strategy parameters by using the {\em Individual 
 step sizes and one Direction Adaptation}.\\
 For more informations cf. \cite{EALibRef}, p. 22 ff.}
{None.}
{None.}
%---------------------------------------------------------------------------%

\clearpage

%---------------------------------------------------------------------------%
\index{mutateIDA!( ChromosomeT$<$ double $>$\& sigma, double c, double c\_r, double beta, double beta\_ind, double beta\_r, double cu, double xi )}
\setNormalInstance
\setCorrectWidthThree{8pt}
\setParamOne{sigma}{ChromosomeT$<$ double $>$ \&}
{Chromosome with standard deviations for the normal distribution.}
\setParamTwo{c}{double}{Defines the cumulation time.}
\setParamThree{c\_r}{double}{Also defines the cumulation time.}
\setParamFour{beta}{double}{Parameter $\beta$ for damping the stepsize
variation between successive generations.}
\setParamFive{beta\_ind}{double}{Parameter for damping the stepsize
variation between successive generations for one individual.}
\setParamSix{beta\_r}{double}{Parameter for damping the stepsize
variation between successive generations for the direction.}
\setParamSeven{cu}{double}{Normalizing factor for variances.}
\setParamEight{xi}{double}{Factor for the stepsize adaptation.
Normally the values $1.5$ and $\frac{1}{1.5}$ will be assigned to 
{\em xi} with the same probability. Here the factor can be stated
explicitly.}
\printMethodWithParamsSaved
{void}
{}
{mutateIDA}
{Same as above, but with more parameters.}
{}
\setCorrectWidthThree{4pt}
%---------------------------------------------------------------------------%

\vspace*{4ex}

%---------------------------------------------------------------------------%
\index{mutateIDAiso!( ChromosomeT$<$ double $>$\& sigma )}
\setNormalInstance
\printMethodWithOneParam
{void}
{mutateIDAiso}
{ChromosomeT$<$ double $>$ \&}
{sigma}
{Chromosome with standard deviations for the normal distribution.}
{This method does the same as {\em mutateIDA}, but here 
 no adaptation of the preference direction takes place.}
{Keiner.}
{Keine.}
%---------------------------------------------------------------------------%

\clearpage

%---------------------------------------------------------------------------%
\index{mutateIDAiso!( ChromosomeT$<$ double $>$\& sigma, double c, double beta, double beta\_ind, double cu )}
\setNormalInstance
\setCorrectWidthThree{8pt}
\setParamOne{sigma}{ChromosomeT$<$ double $>$ \&}
{Chromosome with standard deviations for the normal distribution.}
\setParamTwo{c}{double}{Defines the cumulation time.}
\setParamThree{beta}{double}{Parameter $\beta$ for damping the stepsize
variation between successive generations.}
\setParamFour{beta\_ind}{double}{Parameter for damping the stepsize
variation between successive generations for one individual.}
\setParamFive{cu}{double}{Normalizing factor for the variances.}
\printMethodWithParamsSaved
{void}
{}
{mutateIDAiso}
{Same as above, but with more parameters.}
{}
\setCorrectWidthThree{4pt}
%---------------------------------------------------------------------------%



