\subsection{Methods for the Extraction of Individuals}

%---------------------------------------------------------------------------%
\index{random!( )}
\setNormalInstance
\printEmptyMethodReturnSpecial
{Individual\&}
{random}
{Returns a randomly chosen individual of {\em this}.}
{An individual of {\em this}.}
{None.}
%---------------------------------------------------------------------------%

\vspace*{4ex}

%---------------------------------------------------------------------------%
\index{matingPool!( unsigned chrom, unsigned from = 0, unsigned to = 0 )}
\setConstInstance
\setCorrectWidthThree{8pt}
\setParamOne{chrom}{unsigned}{Index of the chromosome to be
extracted from the set of individuals.}
\setParamTwo{from}{unsigned}{Index of the first individual of {\em this}
from which the chromosome {\em chrom} shall be extracted. The default 
index is 0. It is not checked, whether {\em from} is greater
than the number of individuals in {\em this}.}
\setParamThree{to}{unsigned}{Index of the last individual of {\em this}
from which the chromosome {\em chrom} shall be extracted. The default
index is 0. If {\em to} is less than {\em from}, an empty vector
will be returned. It is not checked, whether {\em to} is greater
than the number of individuals in {\em this}.}
\printMethodWithParamsSaved
{vector$<$ const Chromosome $\ast$ $>$}
{Vector with the extracted chromosomes.}
{matingPool}
{Extracts all chromosomes with index {\em chrom} out of the set
 of chromosomes from index {\em from} to index {\em to}, stores
 them in a vector and then returns this vector. If {\em from}
 and {\em to} are set to their default value 0, then the
 chromosomes with index {\em chrom} will be extracted from all
 individuals of {\em this}.\\
 Example: {\em chrom} = 3, {\em from} = 1, {\em to} = 6. From
 the individuals with indices 1 to 6, the chromosome with index
 3 will be extracted and stored in the chromosome pool.}
{}
\setCorrectWidthThree{4pt}
%---------------------------------------------------------------------------%
