% 24th, Jan, 2001 Ver.1     Tatsuya Okabe
%                 Ver.2
%                 Ver.3
%                 Ver.4
%                 Ver.5
%
%---------------------------------------------------------------------------%
% Made by Tatsuya Okabe ( HONDA R&D Europe ( Deutschland ) GmbH )           %
% Checked by Bernhard Sendhoff ( HONDA R&D Europe ( Deutschland ) GmbH )    %
%---------------------------------------------------------------------------%
% arraybase

\noindent
In the class {\em arraybase}, functions are defined for treating the
index and changing the structure of an array. Using this class, we can
get information about e.g. the number of dimensions and the number of
elements. Additionally, the structure of the array can be changed.

\vspace*{10mm}

\section{Internal Variables}

\begin{itemize}
\item unsigned* d - The pointer to the dimension vector.
\item unsigned nd - The number of dimensions.
\item unsigned ne - The number of elements
\item bool stat - The flag which signals whether object is a static ref.
\end{itemize}

%********************
\index{*d (Variable)}
\index{nd (Variable)}
\index{ne (Variable)}
\index{stat (Variable)}
%********************

\clearpage

\section{Public Methods}

\subsection{Destructor}

%---------------------------------------------------------------------------%
% 014
\index{$\sim$arraybase!( )}
\setNormalInstance
\printEmptyMethodReturn
{}
{$\sim$arraybase}
{The destructor. Destructs {\em this} object if there is anything to
destruct and it is not a static reference (signalled by the flag
``{\em stat}'').}
{None.}
\setCorrectWidthThree{4pt}
%---------------------------------------------------------------------------%

\vspace*{10mm}

\subsection{Information Retrieval Methos}

%---------------------------------------------------------------------------%
% 001
\index{ndim!( )} 
\setConstInstance
\printEmptyMethodReturnSpecial
{unsigned}
{ndim}
{Returns the number of dimensions, i.e. the length of the index vector.}
{The number of dimensions.}
{None.}
%---------------------------------------------------------------------------%

%---------------------------------------------------------------------------%
% 002
\index{nelem!( )} 
\setConstInstance
\printEmptyMethodReturnSpecial
{unsigned}
{nelem}
{Returns the total number of elements, i.e. the product over all dimensions.}
{The total number of elements.}
{None.}
%---------------------------------------------------------------------------%

\clearpage

\subsection{Structure Changing Methods}

%---------------------------------------------------------------------------%
% 003
\index{samedim!( const arraybase\& v )} 
\setConstInstance
\printMethodWithOneParam
{bool}
{samedim}
{const arraybase\&}
{v}
{The reference array.}
{Checks whether two arrays have the same dimensions independent from
their respective types, returns true if both dimensions are the same.}
{The result of this check.}
{None.}
%---------------------------------------------------------------------------%

%---------------------------------------------------------------------------%
% 004
\index{resize!( unsigned i, bool copy )}
\setNormalInstance
\setCorrectWidthThree{8pt}
\setParamOne{i}{unsigned}{The size of an one-dimensional array.} 
\setParamTwo{copy}{bool}{The flag of copy.}
\printMethodWithParamsSaved
{void}
{None.}
{resize}
{Resizes an array to an one-dimensional array of the size ``{\em i}'', the
flag ``{\em copy}'' signals whether existing elements are copied ( if
possible ).}
{None.}
\setCorrectWidthThree{4pt}
%---------------------------------------------------------------------------%

%---------------------------------------------------------------------------%
% 005
\index{resize!( unsigned i, unsigned j, bool copy )}
\setNormalInstance
\setCorrectWidthThree{8pt}
\setParamOne{i}{unsigned}{The size of an one dimensional array.}
\setParamTwo{j}{unsigned}{The size of the other dimensional array.} 
\setParamThree{copy}{bool}{The flag of copy.}
\printMethodWithParamsSaved
{void}
{None.}
{resize}
{Resizes an array to an two-dimensional array of the size ``{\em i $\times$
j}'', the
flag ``{\em copy}'' signals whether existing elements are copied ( if
possible ).}
{None.}
\setCorrectWidthThree{4pt}
%---------------------------------------------------------------------------%

\clearpage

%---------------------------------------------------------------------------%
% 006
\index{resize!( unsigned i, unsigned j, unsigned k, bool copy )}
\setNormalInstance
\setCorrectWidthThree{8pt}
\setParamOne{i}{unsigned}{The size of an one-dimensional array.}
\setParamTwo{j}{unsigned}{The size of a two-dimensional array.} 
\setParamThree{k}{unsigned}{The size of a three-dimensional array.}
\setParamThree{copy}{bool}{The flag of copy.}
\printMethodWithParamsSaved
{void}
{None.}
{resize}
{Resizes an array to an three-dimensional array of the size ``{\em i $\times$
j $\times$ k}'', the flag ``{\em copy}'' signals whether existing elements
are copied ( if possible ).}
{None.}
\setCorrectWidthThree{4pt}
%---------------------------------------------------------------------------%

%---------------------------------------------------------------------------%
% 007
\index{resize!( const std::vector$<$ unsigned $>$\& i, bool copy )}
\setNormalInstance
\setCorrectWidthThree{8pt}
\setParamOne{i}{const std::vector$<$ unsigned $>$\&}{The vector
setting dimensions.} 
\setParamTwo{copy}{bool}{The flag of copy.}
\printMethodWithParamsSaved
{void}
{None.}
{resize}
{Resizes an array to an array with dimensions defined in vector ``{\em
i}'',
the flag ``{\em copy}'' signals whether existing elements are copied.}
{None.}
\setCorrectWidthThree{4pt}
%---------------------------------------------------------------------------%

%---------------------------------------------------------------------------%
% 009
\index{dim!( unsigned i )} 
\setNormalInstance
\printMethodWithOneParam
{unsigned}
{dim}
{unsigned}
{i}
{The dimension whose value you want to know.}
{Returns the ``{\em i}''th dimension, range check may be performed if not
disabled by the preprocessor directive ``-DNDEBUG''.}
{The ``{\em i}''th dimension.}
{None.}
%---------------------------------------------------------------------------%

\clearpage

%---------------------------------------------------------------------------%
% 010
\index{dimvec!( )} 
\setNormalInstance
\printEmptyMethodReturnSpecial
{unsigned*}
{dimvec}
{Returns the pointer to the dimension vector.}
{The pointer to the dimension vector.}
{Handle with care.}
%---------------------------------------------------------------------------%

%---------------------------------------------------------------------------%
% 011
\index{dimvec!( )} 
\setConstInstance
\printEmptyMethodReturnSpecial
{unsigned*}
{dimvec}
{Returns the pointer to the dimension vector for const objects.}
{The pointer to the dimension vector for const objects.}
{Handle with care.}
%---------------------------------------------------------------------------%

\vspace*{10mm}

\subsection{Generating Methods}

%---------------------------------------------------------------------------%
% 012
\index{clone!( )} 
\setConstInstance
\printEmptyMethodReturnSpecial
{arraybase*}
{clone}
{The pure virtual function which is defined in template class array$<$
T $>$. Returns an identical copy of this object.}
{The pointer.}
{None.}
%---------------------------------------------------------------------------%

%---------------------------------------------------------------------------%
% 013
\index{empty!( )} 
\setConstInstance
\printEmptyMethodReturnSpecial
{arraybase*}
{empty}
{The pure virtual function which is defined in template class array$<$
T $>$. Returns an empty array with the same type as this object.}
{The pointer.}
{None.}
%---------------------------------------------------------------------------%

\clearpage

\section{Private Methods}

\subsection{Constructors}

%---------------------------------------------------------------------------%
% 015
\index{arraybase!( )}
\setNormalInstance
\printEmptyMethodReturnSpecial
{}
{arraybase}
{The default constructor. Initializes the number of dimensions, the
total number of elements and the flag ``{\em stat}'' to false.}
{None.}
{This function should never be used directly.}
\setCorrectWidthThree{4pt}
%---------------------------------------------------------------------------%

%---------------------------------------------------------------------------%
% 016
\index{arraybase!( const arraybase\& v )} 
\setNormalInstance
\printMethodWithOneParam
{}
{arraybase}
{const arraybase\&}
{v}
{The dimension vector.}
{The copy constructor. Only the dimension vector is copied, the data
vector must be copied in the copy constructor of template class
array$<$ T $>$}
{None.}
{This function should never be used directly.}
%---------------------------------------------------------------------------%

\vspace*{10mm}

\subsection{Structure Changing Methods}

%---------------------------------------------------------------------------%
% 017
\index{resize\_i!( unsigned*, unsigned, bool )}
\setNormalInstance
\setCorrectWidthThree{8pt}
\setParamOne{}{unsigned}{} 
\setParamTwo{}{unsigned}{}
\setParamThree{}{bool}{}
\printMethodWithParamsSaved
{void}
{None.}
{resize\_i}
{The pure virtual function. Handles memory allocation of the
respective template type in case of resizing.}
{None.}
\setCorrectWidthThree{4pt}
%---------------------------------------------------------------------------%

\vspace*{10mm}

\subsection{Input and Output Methods}

%---------------------------------------------------------------------------%
% 018
\index{readFrom!( std::istream\& is )} 
\setNormalInstance
\printMethodWithOneParam
{void}
{readFrom}
{std::istream\&}
{is}
{Input stream.}
{Replaces the data of {\em this} with the data read from the input stream
{\em is}.}
{None.}
{None.}
%---------------------------------------------------------------------------%

%---------------------------------------------------------------------------%
% 019
\index{writeTo!( std::ostream\& os )} 
\setConstInstance
\printMethodWithOneParam
{void}
{writeTo}
{std::ostream\&}
{os}
{Output stream.}
{Writes all data of the array {\em this} to the output stream {\em os}.}
{None.}
{None.}
%---------------------------------------------------------------------------%




