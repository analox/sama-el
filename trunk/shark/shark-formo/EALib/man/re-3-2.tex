\section{Public Methods}

These methods can be used by all \cpp\ -programs that have included
the header file {\em ChromosomeT.h} and the library {\em EA}.

\subsection{Initializing Methods}

%---------------------------------------------------------------------------%
\index{initialize!( T min, T max )}
\setNormalInstance
\setCorrectWidthThree{8pt}
\setParamOne{min}{T}{Minimum random value.}
\setParamTwo{max}{T}{Maximum random value.}
\printMethodWithParamsSaved
{void}
{}
{initialize}
{Initializes all alleles of {\em this} with random values, ranging from
 {\em min} to {\em max} (including {\em mix} and {\em max}).}
{}
\setCorrectWidthThree{4pt}
%---------------------------------------------------------------------------%

\clearpage

%---------------------------------------------------------------------------%
\index{initialize!( const Chromosome\& minChrom, const Chromosome\& maxChrom )}
\setNormalInstance
\setCorrectWidthThree{8pt}
\setParamOne{minChrom}{const Chromosome\&}{Chromosome, that contains
the minimum possible random number values for the alleles of {\em this}.
{\em minChrom} must include at least one allele, otherwise no initialization
will take place.}
\setParamTwo{maxChrom}{const Chromosome\&}{Chromosome, that contains
the maximum possible random number values for the alleles of {\em this}.
{\em maxChrom} must have as many alleles as {\em minChrom}, otherwise
the method will be aborted with an error. Additionally {\em maxChrom}
must contain at least one allele or no initialization will take place.}
\printMethodWithParamsSaved
{void}
{}
{initialize}
{Same as above, but here the minimum and maximum possible random
 numbers are given for each allele of {\em this} separately, using the
 chromosomes {\em minChrom} and {\em maxChrom}. The size of
{\em this} will be adapted to the size of one of the range values
chromosomes.}
{}
\setCorrectWidthThree{4pt}
%---------------------------------------------------------------------------%

\subsection{Adapting Methods}

%---------------------------------------------------------------------------%
\index{cutOff!( T min, T max )}
\setNormalInstance
\setCorrectWidthThree{8pt}
\setParamOne{min}{T}{Minimum possible value for an allele.}
\setParamTwo{max}{T}{Maximum possible value for an allele.}
\printMethodWithParamsSaved
{void}
{}
{cutOff}
{Checks for each allele of {\em this}, whether its value is between
 {\em min} and {\em max}. If an allele is out of range, it will
 be set to the value of the nearest boundary.}
{}
\setCorrectWidthThree{4pt}
%---------------------------------------------------------------------------%

\clearpage

%---------------------------------------------------------------------------%
\index{cutOff!( const Chromosome\& minChrom, const Chromosome\& maxChrom )}
\setNormalInstance
\setCorrectWidthThree{8pt}
\setParamOne{minChrom}{const Chromosome\&}{Chromosome, which contains
the minimum possible values for each single allele of {\em this}.
{\em minChrom} must contain at least one chromosome, otherwise
no adaptation will take place.}
\setParamTwo{maxChrom}{const Chromosome\&}{Chromosome, which contains
the maximum possible values for each single allele of {\em this}.
{\em maxChrom} must have as many alleles as {\em minChrom},
otherwise the method will be aborted with an error. Additionally
{\em maxChrom} must include at least one allele or no adaptation
will take place.}
\printMethodWithParamsSaved
{void}
{}
{cutOff}
{Same as above, but here range values are given by the chromosomes
 {\em minChrom} and {\em maxChrom} for each single
 allele of {\em this}. Besides the size of {\em this} will be
 adapted to the size of one of the range values chromosomes.}
{}
\setCorrectWidthThree{4pt}
%---------------------------------------------------------------------------%

\subsection{Mutation Methods}

%---------------------------------------------------------------------------%
\index{mutateUniform!( T min, T max, double p )}
\setNormalInstance
\setCorrectWidthThree{8pt}
\setParamOne{min}{T}{Minimum possible random value.}
\setParamTwo{max}{T}{Maximum possible random value.}
\setParamThree{p}{double}{Probability to replace an allele value.}
\printMethodWithParamsSaved
{void}
{}
{mutateUniform}
{Replaces each allele value of {\em this} with probability {\em p} 
 with a random number ranging from {\em min} to {\em max}.}
{}
\setCorrectWidthThree{4pt}
%---------------------------------------------------------------------------%

\clearpage

%---------------------------------------------------------------------------%
\index{mutateUniform!( T min, T max, const vector$<$ double $>$\& p, bool cycle = false )}
\setNormalInstance
\setCorrectWidthThree{8pt}
\setParamOne{min}{T}{Minimum possible random value.}
\setParamTwo{max}{T}{Maximum possible random value.}
\setParamThree{p}{const vector$<$ double $>$\&}{Vector, that contains
probability values for each allele of {\em this} to be replaced.
{\em p} shall contain at most as many values as {\em this} has
alleles. Otherwise the method will be aborted with an error message.}
\setParamFour{cycle}{bool}{Specifies, whether {\em p} can be
used circular (``true'') or not (``false'', this is also the
default).}
\printMethodWithParamsSaved
{void}
{}
{mutateUniform}
{Same as above, but here a probability value is given for each
 single allele of {\em this} separately by using the vector {\em p}.
 Because {\em p} can contain less values than {\em this} has alleles
 the flag {\em cycle} specifies whether {\em p} can be used
 circular.}
{}
\setCorrectWidthThree{4pt}
%---------------------------------------------------------------------------%

\vspace*{4ex}

%---------------------------------------------------------------------------%
\index{mutateUniform!( T min, T max, const Chromosome\& p, bool cycle = false )}
\setNormalInstance
\setCorrectWidthThree{8pt}
\setParamOne{min}{T}{See above.}
\setParamTwo{max}{T}{See above.}
\setParamThree{p}{const Chromosome\&}{Chromosome that contains the
probability values for each allele of {\em this}. {\em p} must include
at most as many alleles as {\em this}, otherwise the method will
be aborted with an error message.}
\setParamFour{cycle}{bool}{See above.}
\printMethodWithParamsSaved
{void}
{}
{mutateUniform}
{Same as above, but here a chromosome is used 
 to store the probability values.}
{}
\setCorrectWidthThree{4pt}
%---------------------------------------------------------------------------%

\clearpage

%---------------------------------------------------------------------------%
\index{mutateUniform!( const Chromosome\& minChrom, const Chromosome\& maxChrom, double p )}
\setNormalInstance
\setCorrectWidthThree{8pt}
\setParamOne{minChrom}{const  Chromosome\&}{Chromosome, which contains
the minimum possible random numbers for each allele of {\em this}.
{\em minChrom} must include as many alleles as {\em this}, otherwise
the method will be aborted with an error message. Additionally,
{\em minChrom} must have at least one allele or no mutation will
take place.}
\setParamTwo{maxChrom}{const  Chromosome\&}{Chromosome, which contains
the maximum possible random numbers for each allele of {\em this}.
{\em maxChrom} must include as many alleles as {\em this} and
{\em minChrom}, otherwise the method will be aborted with an error
message. Additionally {\em maxChrom} must have at least one allele
or no mutation will take place.}
\setParamThree{p}{double}{Probability to replace one allele of {\em this}.}
\printMethodWithParamsSaved
{void}
{}
{mutateUniform}
{Replaces each allele of {\em this} with probability {\em p} with a random
 value. The range for the random number is given by {\em minChrom} and
{\em maxChrom} for each allele of {\em this} separately. {\em this} must
contain at least one allele or no mutation will take place.}
{}
\setCorrectWidthThree{4pt}
%---------------------------------------------------------------------------%

\vspace*{4ex}

%---------------------------------------------------------------------------%
\index{mutateUniform!( const Chromosome\& minChrom, const Chromosome\& maxChrom, const vector$<$ double $>$\& p, bool cycle = false )}
\setNormalInstance
\setCorrectWidthThree{8pt}
\setParamOne{minChrom}{const Chromosome\&}{See above.}
\setParamTwo{maxChrom}{const Chromosome\&}{See above.}
\setParamThree{p}{const vector$<$ double $>$\&}{Vector, that gives
probabilities for each single allele of {\em this}. {\em p} shall
contain at most as many values as there are alleles in {\em this}. 
Otherwise
the method will be aborted with an error message.}
\setParamFour{cycle}{bool}{Specifies whether {\em p} can be used
circular (``true'') or not (``false'', this is also the default).}
\printMethodWithParamsSaved
{void}
{}
{mutateUniform}
{Same as above, but here a vector {\em p} is used to specify probability
 values for each single allele of {\em this} separately. Because
 {\em p} can contain less values than {\em this} has alleles the flag
{\em cycle} specifies whether {\em p} can be used circular
or not. {\em this} must include at least one allele or no mutation will
take place.} 
{}
\setCorrectWidthThree{4pt}
%---------------------------------------------------------------------------%

\vspace*{4ex}

%---------------------------------------------------------------------------%
\index{mutateUniform!( const Chromosome\& min, const Chromosome\& max, const Chromosome\& p, bool cycle = false )}
\setNormalInstance
\setCorrectWidthThree{8pt}
\setParamOne{min}{const Chromosome\&}{See above.}
\setParamTwo{max}{const Chromosome\&}{See above.}
\setParamThree{p}{const Chromosome\&}{Chromosome that contains the
probability values for each allele of {\em this}. {\em p} shall include
at most as many alleles as {\em this}, otherwise the method will
be aborted with an error message.}
\setParamFour{cycle}{bool}{See above.}
\printMethodWithParamsSaved
{void}
{}
{mutateUniform}
{Same as above, but here a chromosome is used 
to store the probability values.}
{}
\setCorrectWidthThree{4pt}
%---------------------------------------------------------------------------%










