\section{Public Methods}

These methods can be used by all \cpp\ -programs that have included
the header file {\em ChromosomeT.h} and the library {\em EA}.

\subsection{Initializing Method}

%---------------------------------------------------------------------------%
\index{initialize!( )}
\setNormalInstance
\printEmptyMethod
{initialize}
{Randomly initializes the alleles of {\em this} with the
 value ``true'' or ``false''. Each value has the same chance
 to be chosen.}
%---------------------------------------------------------------------------%

\clearpage

\subsection{Encoding Methods}

%---------------------------------------------------------------------------%
\index{encode!( double val, const Interval\& range, unsigned nbits, bool useGray = false )}
\setNormalInstance
\setCorrectWidthThree{8pt}
\setParamOne{val}{double}{Value to be encoded.}
\setParamTwo{range}{const Interval\&}{{\em val} $\in [${\em range}$]$}
\setParamThree{nbits}{unsigned}{Number of bits that shall be used for encoding.}
\setParamFour{useGray}{bool}{Specifies the encoding method:\\
 {\em true}\hspace{2pt} - The Gray encoding method is used,\\
 {\em false} - The standard encoding method is used, this is also the default.}
\printMethodWithParamsSaved
{void}
{}
{encode}
{Encodes the double value {\em val} out of the interval {\em range} into
 a bitstring of length {\em nbits} and stores it in {\em this}.
 The flag {\em useGray} specifies, whether the Gray method is used
 for encoding or not.}
{}
\setCorrectWidthThree{4pt}
%---------------------------------------------------------------------------%

\vspace*{4ex}

%---------------------------------------------------------------------------%
\index{encodeBinary!( onst vector$<$ double $>$\& chrom, const Interval\& range, unsigned nbits, bool useGray = false )} 
\setNormalInstance
\setCorrectWidthThree{8pt}
\setParamOne{chrom}{const vector$<$ double $>$\&}{Vector that contains the
values, to be encoded.}
\setParamTwo{range}{const Interval\&}{See above.}
\setParamThree{nbits}{unsigned}{Number of bits, that are used for
each value to encode it.}
\setParamFour{useGray}{bool}{See above.}
\printMethodWithParamsSaved
{void}
{}
{encodeBinary}
{Same as above, but here not only one, but several double values,
 contained in the vector {\em chrom}, are encoded.}
{}
\setCorrectWidthThree{4pt}
%---------------------------------------------------------------------------%

\clearpage

%---------------------------------------------------------------------------%
\index{encodeBinary!( const Chromosome\& chrom, const Interval\& range, unsigned nbits, bool useGray = false )}
\setNormalInstance
\setCorrectWidthThree{8pt}
\setParamOne{chrom}{const Chromosome\&}{Chromosome that contains
the values, to be encoded.}
\setParamTwo{range}{const Interval\&}{See above.}
\setParamThree{nbits}{unsigned}{See above.}
\setParamFour{useGray}{bool}{See above.}
\printMethodWithParamsSaved
{void}
{}
{encodeBinary}
{Same as above, but here the values to be encoded are stored
 in a chromosome.}
{}
\setCorrectWidthThree{4pt}
%---------------------------------------------------------------------------%

\subsection{Decoding Method}

%---------------------------------------------------------------------------%
\index{decode!( const Interval\& range, bool useGray = false )}
\setConstInstance
\setCorrectWidthThree{8pt}
\setParamOne{range}{const Interval\&}{{\em val} $\in [${\em range}$]$}
\setParamTwo{useGray}{bool}{Specifies the encoding method:\\
 {\em true}\hspace{2pt} - the Gray encoding method was used,\\
 {\em false} - the standard encoding method was used, this is also the default.}
\printMethodWithParamsSaved
{double}
{The decoded value.}
{decode}
{Decodes a value out of the interval {\em range}, that is encoded as
 bitstring and stored in {\em this}. The flag {\em useGray} specifies,
 whether the Gray method was used for the encoding process or not.}
{}
\setCorrectWidthThree{4pt}
%---------------------------------------------------------------------------%

\clearpage
 
\subsection{Mutation methods}

%---------------------------------------------------------------------------%
\index{flip!( double p )}
\setNormalInstance
\printMethodWithOneParam
{void}
{flip}
{double}
{p}
{Probability for flipping an allele value.}
{Each allele of {\em this} is flipped with probability {\em p},
 that means a value of ``true'' is flipped to ``false'' and
 vice versa.}
{None.}
{None.}
%---------------------------------------------------------------------------%

\vspace*{4ex}

%---------------------------------------------------------------------------%
\index{flip!( const vector$<$ double $>$\& p, bool cycle = false )}
\setNormalInstance
\setCorrectWidthThree{8pt}
\setParamOne{p}{const vector$<$ double $>$\&}{Vector with probabilities
for each allele of {\em this} to be flipped. {\em p} should contain
at most as many values as there are alleles in {\em this}. Otherwise the
method will be aborted with an error message.}
\setParamTwo{cycle}{bool}{Specifies, whether {\em p} is used
circular ({\em cycle} = ``true'') or not ({\em cycle} = ``false'', this
is also the default).}

\printMethodWithParamsSaved
{void}
{}
{flip}
{Same as above but {\em p} contains separate probabilities for each
 allele of {\em this}. As {\em p} can contain less values than {\em this}
 has alleles, the flag {\em cycle} is used to specify, whether {\em p}
 can is used circular or not.}
{}
\setCorrectWidthThree{4pt}
%---------------------------------------------------------------------------%

\clearpage

%---------------------------------------------------------------------------%
\index{flip!( const Chromosome\& p, bool cycle = false )}
\setNormalInstance
\setCorrectWidthThree{8pt}
\setParamOne{p}{const Chromosome\&}{See above.}
\setParamTwo{cycle}{bool}{See above.}

\printMethodWithParamsSaved
{void}
{}
{flip}
{Same as above, but here a chromosome is used to store the
 probability values.}
{}
\setCorrectWidthThree{4pt}
%---------------------------------------------------------------------------%

