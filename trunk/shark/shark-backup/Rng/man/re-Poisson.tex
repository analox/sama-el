% 24th, Jan, 2001 Ver.1     Tatsuya Okabe
%                 Ver.2
%                 Ver.3
%                 Ver.4
%                 Ver.5
%
%---------------------------------------------------------------------------%
% Made by Tatsuya Okabe ( HONDA R&D Europe ( Deutschland ) GmbH )           %
% Checked by Bernhard Sendhoff ( HONDA R&D Europe ( Deutschland ) GmbH )    %
%---------------------------------------------------------------------------%
% Class Poisson

\section{Abstract}

\noindent
With the class {\em Poisson}, we can simulate the ``Poisson''
distribution. Now let's guess the situation where we try somethings
$n$ times and we saw the factor $x$ times with the probability
$p$. This distribution follows ``Binomial'' distribution as the next
equation. 

\begin{equation}
f(x) = _n \hspace{-2mm} C_x p^x (1-p)^{n-x} = \frac{n!}{x! (n-x)!} p^x (1-p)^{n-x}
\end{equation} 

\noindent
If the number of experiences is sufficiently large, this equation can
be approximated by ``Poisson'' distribution, which will be shown in
the next equation.

\begin{equation}
f(x) = \frac{e^\lambda \lambda^x}{x!}
\end{equation}

\noindent
In this equation, $\lambda$ and x mean the constant and the factor respectively.

\vspace*{10mm}

\section{Internal Variables}

\begin{itemize}
\item pMean - Corresponding to $\lambda$ in upper equation.
\end{itemize}

%********************
\index{pMean (Variable)}
%********************

\clearpage

\section{Public Methods}

\noindent
These methods can be used by all \cpp - programs, that have included the
header file Poisson.h and the library EA.

\subsection{Constructors}

%---------------------------------------------------------------------------%
% 001
\index{Poisson!( double mean )}
\setNormalInstance
\printMethodWithOneParam
{}
{Poisson}
{double}
{mean}
{The constant $\lambda$.}
{The default constructor. Generates the random generator of Poisson distribution.}
{None.}
{None.}
%---------------------------------------------------------------------------%

%---------------------------------------------------------------------------%
% 002
\index{Poisson!( double mean, RNG\& r )}
\setNormalInstance
\setCorrectWidthThree{8pt}
\setParamOne{mean}{double}{The constant $\lambda$.} 
\setParamTwo{r}{RNG\&}{RNG class.}
\printMethodWithParamsSaved
{}
{None.}
{Poisson}
{The constructor. Generates the random generator of Poisson distribution.}
{None.}
\setCorrectWidthThree{4pt}
%---------------------------------------------------------------------------%

\vspace*{10mm}

\subsection{Operators}

%---------------------------------------------------------------------------%
% 003
\index{operetor( )!( double mean )}
\setNormalInstance
\printMethodWithOneParam
{long}
{operator( )}
{double}
{mean}
{The constant $\lambda$.}
{Gets the result of Poisson distribution.}
{The result of Poisson distribution.}
{None.}
%---------------------------------------------------------------------------%

\clearpage

%---------------------------------------------------------------------------%
% 004
\index{operator( )!( )} 
\setNormalInstance
\printEmptyMethodReturnSpecial
{long}
{operator( )}
{Gets the result of Poisson distribution {\em ( $\lambda$ = pMean )}.}
{The result of Poisson distribution {\em ( $\lambda$ = pMean )}.}
{None.}
%---------------------------------------------------------------------------%

\vspace*{10mm}

\subsection{Information Retrieval Methods}

%---------------------------------------------------------------------------%
% 005
\index{mean!( )} 
\setConstInstance
\printEmptyMethodReturnSpecial
{double}
{mean}
{Returns the variable {\em $\lambda$} {\em pMean}.}
{The variable {\em $\lambda$} {\em pMean}.}
{None.}
%---------------------------------------------------------------------------%

%---------------------------------------------------------------------------%
% 006
\index{mean!( double newMean )} 
\setNormalInstance
\printMethodWithOneParam
{void}
{mean}
{double}
{newMean}
{New $\lambda$.}
{Sets the mean {\em pMean} using new mean {\em newMean}.}
{None.}
{None.}
%---------------------------------------------------------------------------%

\vspace*{10mm}

\subsection{The probability}

%---------------------------------------------------------------------------%
% 007
\index{p!( const long\& x )} 
\setConstInstance
\printMethodWithOneParam
{double}
{p}
{const long\&}
{x}
{The factor which you want to calculate the probability.}
{Returns the probability of {\em x}.}
{The probability.}
{None.}
%---------------------------------------------------------------------------%



