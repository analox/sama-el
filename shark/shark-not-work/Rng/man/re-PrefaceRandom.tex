\noindent
Genetic Algorithms (GA), Evolution Strategies (ES) and Evolutionary
Programming (EP) were developed almost 40 years ago. Since then many
different variants and combinations of the first appreaches have been
proposed, which are now commonly referred to as Evolutionary
Algorithms (EA) or methods from Evolutionary Computation
(EC). Evolutionary algorithms use genetic operators, e.g. crossover,
mutation, recombination. In order to simulate these operators, random
number generators are very important.

\vspace*{5mm}

\noindent
Furthermore, ``Monte Calro simulations'', which are very popular in
the field of material science, heavily rely on the ``true'' randomness
of random number generators.

\vspace*{5mm}

\noindent
In both cases, the results depend on the quality of the random number
generators, which are prepared as a C++ template in the EALib. However,
this random number only produces uniform distributions. Thus, from the
uniform random numbers, we have to calculate other distributions.

\vspace*{5mm}

\noindent
In the EALib, the following distributions are available.

\vspace*{10mm}

\begin{enumerate}
\item Bernoulli trial
\item Binomial distribution
\item Cauchy distribution
\item Differentiative Geometric distribution
\item Discrete uniform distribution
\item Geometric distribution
\item Log normal distribution
\item Negative exponential distribution
\item Normal distribution
\item Poisson distribution
\item Uniform distribution
\item Weibull distribution
\end{enumerate}

\vspace*{10mm}

\noindent
Of course, these classes cannot only be used for Evolutionary
Computation and Monte Carlo simulation, but also for other simulations
where random number generators are necessary. Thus, starting with the
next section, I will introduce the different functions. Additionary, I
will also introduce two classes $Rng$ and $RNG$. These two classes
control the random number generator.
