%
%%
%% file: mutation.tex
%%
%

% =======================================================================
        \subsection{Mutation\index{mutation}}
% =======================================================================

After selection and recombination the reproduced individuals are
subject to mutation.  The main differences between the mutation
operators used in GAs and ESs are due to the different representation
of the genome. In GAs the genome is encoded into symbol strings and
therefore discrete mutations apply, whereas in ESs floating point
numbers are used combined with continuous mutations\index{mutation!continuous} (see
\secref{mutation:subsubs:continuesMutations}).


% -----------------------------------------------------------------------
        \subsubsection{Discrete Mutations\index{mutation!discrete}}
% -----------------------------------------------------------------------

The standard mutation operator for GAs was first proposed by Holland
\cite{Holland:75}. Each position of the genome is given a chance $p_m$
(the mutation rate) of undergoing mutation.  In the case of mutation a
random value is chosen from the set of allowed symbols for the
selected allele.  Using binary strings a simpler mechanism can be
employed.  For each position to be mutated the corresponding bit is
simply flipped.


% .......................................................................
        \paragraph{Adaptive/Variable Mutation Rates\index{mutation!adaptive}}

Adopting the idea of self-adaptation of the strategy parameters from
ES, variable mutation rates can be defined.  This has been proposed by
B\"ack, results can be found in \cite{Baeck:91}.


% -----------------------------------------------------------------------
        \subsubsection{Continuous Mutations\index{mutation!continuous}}
        \label{mutation:subsubs:continuesMutations}
% -----------------------------------------------------------------------

ESs rely heavily on the mutation operator, which adds a normally distributed
random number to each allele.
The step size of the mutation is controlled by the standard deviation of
the \myindex{normal distribution}.


% .......................................................................
        \paragraph{Correlated Mutations\index{mutation!correlated}}

In order to allow for more flexible distributions in the mutation
operator correlations may be considered.  Schwefel \cite{Schwefel:77}
has proposed a correlated mutation scheme where the correlations were
implicitly defined by a \myindex{rotation matrix} (see also \cite{Rudolph:92}).


% .......................................................................
%       \paragraph{Self--Adaptation of Mutations\index{mutation!self-adaptati% on}}
%       
% One essential part of mutation is the possibility of
% self-adaptation of the step sizes.  Various adaptation schemes
% have been proposed. In most cases the mutative adaptation scheme
% proposed by Schwefel
% \cite{Schwefel:77} is employed, which works well in the case of large
% populations. A better \emph{derandomised} adaptation scheme\index{mutation!% derandomised adaptation}
% has been proposed by Ostermeier
% \cite{Ostermeier:92,Ostermeier:94} which can cope with rather small
% populations and allows for a controlled adaptation of the step sizes
% as well as of the direction of mutations.



% -----------------------------------------------------------------------
        \subsubsection{Inversion}
% -----------------------------------------------------------------------

Holland \cite{Holland:75} describes an inversion operator which works
on a single chromosome.  It inverts the order of the elements between
two randomly chosen points on the chromosome.  While inversion was
inspired by a biological mechanism it has not in general been found to
be useful in genetic algorithms.  In combination with a transcription
operator which works on chromosomes that are several times longer,
inversion proves to be useful in some applications \cite{Wienholt:93}.
