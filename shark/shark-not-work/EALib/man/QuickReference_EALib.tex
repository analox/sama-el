\documentclass[12pt]{report}
\usepackage{a4}
\usepackage{pstricks}
% =======================================================================
% Dateiname:        Vorlage.tex
% Autor:            Ruediger Alberts
% EMail:            Ruediger.Alberts@neuroinformatik.ruhr-uni-bochum.de
% Erstellt am:      1999-09-07
% Letzte Aenderung: 1999-09-09
%
% Datei wird mit Befehl ``% =======================================================================
% Dateiname:        Vorlage.tex
% Autor:            Ruediger Alberts
% EMail:            Ruediger.Alberts@neuroinformatik.ruhr-uni-bochum.de
% Erstellt am:      1999-09-07
% Letzte Aenderung: 1999-09-09
%
% Datei wird mit Befehl ``% =======================================================================
% Dateiname:        Vorlage.tex
% Autor:            Ruediger Alberts
% EMail:            Ruediger.Alberts@neuroinformatik.ruhr-uni-bochum.de
% Erstellt am:      1999-09-07
% Letzte Aenderung: 1999-09-09
%
% Datei wird mit Befehl ``\input{Vorlage}'' in LaTex-Dokumente 
% (Latex2Epsilon) eingebunden und stellt Befehle zur einfachen Doku- 
% mentation von C-Funktionen/C++-Methoden bereit.
% =======================================================================

\usepackage{ifthen}

    % Definition von Konstanten:
    \newcommand{\ParamDelim}            % Trennungstext zwischen
        { \ - }                         % Parametername und -beschreibung.
    \newcommand{\NormMethodEnd}         % Endtext von Methode bei 
        {\ \ );}                        % veraenderbarer Aufrufinstanz.
    \newcommand{\ConstMethodEnd}        % Endtext von Methode bei
        {\ \ ) const;}                  % nicht veraenderbarer Aufrufinstanz.
    \newcommand{\MaxLabelTxt}           % Komponentenbeschreibungstext
        {{\bf Besonderheiten: }}        % mit maximaler Breite.
    \newcommand{\MethodBegin}           % Beginntext einer Methode.
        {(\ }
    \newcommand{\InBetween}             % Text zwischen Rueckgabewert und
        {\ \ }                          % Methodenname.

    % Definition eigener Variablen:


    \newboolean{ConstInstance}          % Gibt an, ob die Instanz, die
                                        % die Methode aufruft veraendert
                                        % werden darf oder nicht. Im
                                        % 2. Fall endet die Methode mit
                                        % ``) const;'', im 1. Fall 
                                        % mit ``);''.
    \newcounter{SavedParams}            % Anzahl der gespeicherten
                                        % Parameter.
    \setcounter{SavedParams}{0}
    \newcommand{\ParamOne}{}            % Definition der 10 
    \newcommand{\ParamTypeOne}{}        % abspeicherbaren Parameter,
    \newcommand{\ParamDescrOne}{}       % ihrer Typen und Beschreibungen.
    \newcommand{\ParamTwo}{}            
    \newcommand{\ParamTypeTwo}{}
    \newcommand{\ParamDescrTwo}{}
    \newcommand{\ParamThree}{}
    \newcommand{\ParamTypeThree}{}
    \newcommand{\ParamDescrThree}{}
    \newcommand{\ParamFour}{}
    \newcommand{\ParamTypeFour}{}
    \newcommand{\ParamDescrFour}{}
    \newcommand{\ParamFive}{}
    \newcommand{\ParamTypeFive}{}
    \newcommand{\ParamDescrFive}{}
    \newcommand{\ParamSix}{}
    \newcommand{\ParamTypeSix}{}
    \newcommand{\ParamDescrSix}{}
    \newcommand{\ParamSeven}{}
    \newcommand{\ParamTypeSeven}{}
    \newcommand{\ParamDescrSeven}{}
    \newcommand{\ParamEight}{}
    \newcommand{\ParamTypeEight}{}
    \newcommand{\ParamDescrEight}{}
    \newcommand{\ParamNine}{}
    \newcommand{\ParamTypeNine}{}
    \newcommand{\ParamDescrNine}{}
    \newcommand{\ParamTen}{}
    \newcommand{\ParamTypeTen}{}
    \newcommand{\ParamDescrTen}{}
    \newlength{\MaxParamWidth}          % Breite des breitesten 
                                        % Methodenparameters.
    \newlength{\SpecialWidth}           % Breite des Textes 
    \settowidth{\SpecialWidth}          % {\bf Besonderheiten}.
        {\MaxLabelTxt}
    \newlength{\ParamDelimWidth}        % Breite des Textes `` \ - ``.
    \settowidth{\ParamDelimWidth}
        {\ParamDelim}
    \newlength{\ParamDescrWidth}        % Max. Breite des Parametertextes.
    \newlength{\SpecialRetTxtWidth}     % Max. Breite der Texte fuer
                                        % Besonderheiten und Rueckgabewert.
    \newlength{\MethodBeginWidth}       % Breite des Methodenanfangs, d.h.
                                        % ``Rueckgabewert Methodenname ( ``.
    \newlength{\BetweenParsWidth}       % Max. zur Verfuegung stehende
                                        % Breite zwischen den beiden Klammern
                                        % der Methode.
    
    \newlength{\MethodEndWidth}         % Breite des Textes ``  );'' bzw.
    \settowidth{\MethodEndWidth}        % `` ) const;''
        {\NormMethodEnd}
    \newlength{\ParamMaxTypeWidth}      % Breite des breitesten
                                        % Parameterwert-Typs. 
    \newlength{\NoneWidthOne}           % Breite des Textes ``Keine.''.
    \settowidth{\NoneWidthOne}{Keine.}
    \newlength{\NoneWidthTwo}           % Breite des Textes ``Keiner.''.
    \settowidth{\NoneWidthTwo}{Keiner.}
    \newlength{\Temp}                   % Temporaere Hilfsvariable. 
    \newlength{\VertSpace}              % Vertikaler Abstand zwischen 
                                        % zwei Parboxen.
   
    % Trotz der automatischen Berechnung der Textbreiten treten
    % kleine Unstimmigkeiten von wenigen Punkten bei den Texten
    % zur Beschreibung der Parameter, des Rueckgabewertes und
    % der Besonderheiten auf, die mit den folgenden Variablen
    % behoben werden koennen:

    \newlength{\CorrectWidthOne}        % Korrekturbreite fuer die
                                        % Parameterbeschreibung.
    \newlength{\CorrectWidthTwo}        % Korrekturbreite fuer die
                                        % Beschreibungen von Rueckgabewert
                                        % und Besonderheiten.
    \newlength{\CorrectWidthThree}      % Korrekturbreite, da erster
                                        % Parameter einer Methode weiter
                                        % vorrueckt als die anderen.
    

% Uebersicht ueber die Parameter und ihre Bedeutung.


%
% <-- \MethodBeginWidth    -->
%------------------------------------------------------------------------
% Rueckgabewert Methodenname(   Typ_Parameter_1 Parameter_1,
%                            <->Typ_Parameter_2 Parameter_2,
%              \CorrectWideThree...
%                               Typ_Parameter_n Parameter_n   );
%                             <-------------> <--------->
%                          \ParamMaxTypeWidth \MaxParamWidth
%                             <-------------------------->
%                             \BetweenParsWidth
%                                                         <-->
%                                                         \MethodEndWidth
%------------------------------------------------------------------------
% Beschreibung der Methode...
%                             \ParamDelimWidth
%                             <->                        \CorrectWidthOne
% \VertSpace
% Parameter:       Parameter_1  - Beschreibung von Parameter 1.   <----->
%                  Parameter_2  - Beschreibung von Parameter 2.
%                  ...
%                  Parameter_n  - Beschreibung von Parameter n.
% \VertSpace                      \ParamDescrWidth
%                                 <------------------------------>
% Rueckgabewert:   Beschreibung des Rueckgabewertes bzw. der Text <-----> 
%                  Keiner.                   
%                  <----->
%                  \NoneWidthTwo               \SpecialRetTxtWidth
% \VertSpace       <--------------------------------------------->
% Besonderheiten:  Beschreibung von Besonderheiten bzw. der Text  <----->
%                  Keine.     
% <--------------> <---->
%  \SpecialWidth   \NoneWidthOne                         \CorrectWidthTwo


    % Definition eigener Befehle:


    %-----------------------------------------------------------------------
    % Gibt den Begriff ``C++'' in ansprechender Form aus.
    % Makro wurde freundlicherweise zur Verfuegung gestellt von
    % Axel W. Dietrich.
    %
    \newcommand{\cpp}{
        \mbox{\emph{\textrm{C\hspace{-1.5pt}\raisebox{1.75pt}{\scriptsize +}
        \hspace{-6pt}\raisebox{.75pt}{\scriptsize +}}}}%
    }


    %-----------------------------------------------------------------------
    % Gibt die Werte aller Parameter auf dem Bildschirm aus.
    %
    \newcommand{\showAllParams}{
        \noindent
        {\em Werte aller Parameter:}
        \begin{tabbing}
            ParamMaxTypeWidth:\ \ \=\kill
            MaxParamWidth:       \>\the\MaxParamWidth\\
            SpecialWidth:        \>\the\SpecialWidth\\
            ParamDelimWidth:     \>\the\ParamDelimWidth\\
            ParamDescrWidth:     \>\the\ParamDescrWidth\\
            SpecialRetTxtWidth:  \>\the\SpecialRetTxtWidth\\
            MethodBeginWidth:    \>\the\MethodBeginWidth\\
            BetweenParsWidth:    \>\the\BetweenParsWidth\\
            MethodEndWidth:      \>\the\MethodEndWidth\\
            ParamMaxTypeWidth:   \>\the\ParamMaxTypeWidth\\
            NoneWidthOne:        \>\the\NoneWidthOne\\
            NoneWidthTwo:        \>\the\NoneWidthTwo\\
            VertSpace:           \>\the\VertSpace\\
            CorrectWidthOne:     \>\the\CorrectWidthOne\\
            CorrectWidthTwo:     \>\the\CorrectWidthTwo\\
            CorrectWidthThree:   \>\the\CorrectWidthThree\\
            ConstInstance:       
            \ifthenelse{\boolean{ConstInstance}}
                {\>ja\\}
                {\>nein\\}
        \end{tabbing}
    }

    %-----------------------------------------------------------------------
    % Setzt einen Schalter, der angibt, dass die Instanz, die die kommende
    % Methode aufrufen kann, nicht veraendert werden darf. Dies hat fuer
    % die Dokumentation zur Folge, dass die Methode mit ``) const;'' endet.
    %
    \newcommand{\setConstInstance}{    
        \setboolean{ConstInstance}{true}
    }

    %-----------------------------------------------------------------------
    % Setzt einen Schalter, der angibt, dass die Instanz, die die kommende
    % Methode aufrufen kann, veraendert werden darf. Die Methode endet also
    % in der Dokumentation normal mit ``);''.
    %
    \newcommand{\setNormalInstance}{    
        \setboolean{ConstInstance}{false}
    }

    %-----------------------------------------------------------------------
    % Initialisiert die Breite des breitesten Methodenparametertyptextes
    % mit Null.
    %
    \newcommand{\initParamMaxTypeWidth}{
        \setlength{\ParamMaxTypeWidth}{0pt}
    }

    %-----------------------------------------------------------------------
    % Initialisiert die Breite des breitesten Parameternamens mit Null.
    %
    \newcommand{\initMaxParamWidth}{
        \setlength{\MaxParamWidth}{0pt}
    }

    %-----------------------------------------------------------------------
    % Ueberprueft, ob die Breite des uebergebenen Textes des Parametertyps
    % groesser ist als die des bisher als breitester Typtext festgelegten
    % Textes und passt den Wert evtl. an.
    % Parameter #1:  Neuer Typtext, dessen Breite zum Vergleich dient.
    %   
    \newcommand{\setNewParamMaxTypeWidth}[1]{
        \settowidth{\Temp}{#1}
        \ifthenelse{\lengthtest{\ParamMaxTypeWidth < \Temp}}
            {\setlength{\ParamMaxTypeWidth}{\Temp}}{}   
    }

    %-----------------------------------------------------------------------
    % Ueberprueft, ob die Breite des uebergebenen Parameternamens groesser
    % ist als die des bisher als am breitesten geltenden Parameternames
    % und passt den Wert evtl. an. Der Parametername steht dabei in
    % Emphasized.
    % Parameter #1:  Neuer Parametername, dessen Breite zum Vergleich dient.
    %
    \newcommand{\setNewMaxParamWidth}[1]{
        \settowidth{\Temp}{{\em #1}}
        \ifthenelse{\lengthtest{\MaxParamWidth < \Temp}}
            {\setlength{\MaxParamWidth}{\Temp}}{}
    }

    %-----------------------------------------------------------------------
    % Setzt die Breite der Beschreibungstexte fuer Rueckgabewert
    % und Besonderheiten.
    %
    \newcommand{\setSpecialRetTxtWidth}{
        \setlength{\SpecialRetTxtWidth}{\textwidth}
        \addtolength{\SpecialRetTxtWidth}{-1\SpecialWidth} 
    }

    %-----------------------------------------------------------------------
    % Setzt die Breite fuer den Beginn einer Methode, d.h. den Text
    % ``Rueckgabewert Methodenname( ``. Rueckgabewert steht dabei
    % in Emphasized, Methodenname in Bold Font.
    % Parameter #1:  Rueckgabewert der Methode.
    % Parameter #2:  Name der Methode.
    %
    \newcommand{\setMethodBeginWidth}[2]{
        \settowidth{\MethodBeginWidth}
            {{\em #1}\InBetween{\bf #2}\MethodBegin}
    }

    %-----------------------------------------------------------------------
    % Setzt die maximale Breite die an Platz zwischen der oeffnenden
    % und der schliessenden Klammer der Methode zur Verfuegung steht.
    %
    \newcommand{\setBetweenParsWidth}{
        \setlength{\BetweenParsWidth}{\textwidth}
        \addtolength{\BetweenParsWidth}{-1\MethodBeginWidth}
        \addtolength{\BetweenParsWidth}{-1\MethodEndWidth}
    }

    %-----------------------------------------------------------------------
    % Setzt die Breite fuer die Beschreibungstexte
    % der Methodenparameter.
    %
    \newcommand{\setParamDescrWidth}{
        \setlength{\ParamDescrWidth}{\textwidth}
        \addtolength{\ParamDescrWidth}{-1\SpecialWidth} 
        \addtolength{\ParamDescrWidth}{-1\MaxParamWidth}
        \addtolength{\ParamDescrWidth}{-1\ParamDelimWidth}
    }    

    %-----------------------------------------------------------------------
    % Setzt die Korrekturbreite fuer den Beschreibungstext von
    % Methodenparametern auf den uebergebenen Wert.
    % Parameter #1:  Neuer Wert fuer die Korrekturbreite.
    %
    \newcommand{\setCorrectWidthOne}[1]{
        \setlength{\CorrectWidthOne}{#1}
    }    

    %-----------------------------------------------------------------------
    % Setzt die Korrekturbreite fuer den Beschreibungstext von
    % Rueckgabewert und Besonderheiten auf den uebergebenen Wert.
    % Parameter #1:  Neuer Wert fuer die Korrekturbreite.
    %
    \newcommand{\setCorrectWidthTwo}[1]{
        \setlength{\CorrectWidthTwo}{#1}
    }    

    %-----------------------------------------------------------------------
    % Setzt die Korrekturbreite fuer den Abstand des zweiten bis n-ten
    % Parameters vom Anfang auf den uebergebenen Wert.
    % Parameter #1:  Neuer Wert fuer die Korrekturbreite.
    %
    \newcommand{\setCorrectWidthThree}[1]{
        \setlength{\CorrectWidthThree}{#1}
    }    

    %-----------------------------------------------------------------------
    % Setzt den vertikalen Abstand zwischen zwei Parboxen auf den ueber-
    % gebenen Wert.
    % Parameter #1:  Neuer Wert fuer den vertikalen Abstand.
    %
    \newcommand{\setVertSpace}[1]{
        \setlength{\VertSpace}{#1}
    }    

    %-----------------------------------------------------------------------
    % Speichert die uebergebenen Informationen ueber Methodenparameter 1
    % in internen Variablen. Achtung: Die Parameter muessen in der richtigen
    % Reihenfolge gesetzt werden!
    % Parameter #1: Der Typ des Parameters.  
    % Parameter #2: Der Name des Parameters. 
    % Parameter #3: Die Beschreibung des Parameters. 
    % 
    \newcommand{\setParamOne}[3]{
        \renewcommand{\ParamOne}{#1}
        \renewcommand{\ParamTypeOne}{#2}
        \renewcommand{\ParamDescrOne}{#3} 
        \setcounter{SavedParams}{0}
        \stepcounter{SavedParams}
    }

    %-----------------------------------------------------------------------
    % Speichert die uebergebenen Informationen ueber Methodenparameter 2
    % in internen Variablen. Achtung: Die Parameter muessen in der richtigen
    % Reihenfolge gesetzt werden!
    % Parameter #1: Der Typ des Parameters.  
    % Parameter #2: Der Name des Parameters. 
    % Parameter #3: Die Beschreibung des Parameters. 
    % 
    \newcommand{\setParamTwo}[3]{
        \renewcommand{\ParamTwo}{#1}
        \renewcommand{\ParamTypeTwo}{#2}
        \renewcommand{\ParamDescrTwo}{#3} 
        \stepcounter{SavedParams}
    }

    %-----------------------------------------------------------------------
    % Speichert die uebergebenen Informationen ueber Methodenparameter 3
    % in internen Variablen. Achtung: Die Parameter muessen in der richtigen
    % Reihenfolge gesetzt werden!
    % Parameter #1: Der Typ des Parameters.  
    % Parameter #2: Der Name des Parameters. 
    % Parameter #3: Die Beschreibung des Parameters. 
    % 
    \newcommand{\setParamThree}[3]{
        \renewcommand{\ParamThree}{#1}
        \renewcommand{\ParamTypeThree}{#2}
        \renewcommand{\ParamDescrThree}{#3} 
        \stepcounter{SavedParams}
    }

    %-----------------------------------------------------------------------
    % Speichert die uebergebenen Informationen ueber Methodenparameter 4
    % in internen Variablen. Achtung: Die Parameter muessen in der richtigen
    % Reihenfolge gesetzt werden!
    % Parameter #1: Der Typ des Parameters.  
    % Parameter #2: Der Name des Parameters. 
    % Parameter #3: Die Beschreibung des Parameters. 
    % 
    \newcommand{\setParamFour}[3]{
        \renewcommand{\ParamFour}{#1}
        \renewcommand{\ParamTypeFour}{#2}
        \renewcommand{\ParamDescrFour}{#3} 
        \stepcounter{SavedParams}
    }

    %-----------------------------------------------------------------------
    % Speichert die uebergebenen Informationen ueber Methodenparameter 5
    % in internen Variablen. Achtung: Die Parameter muessen in der richtigen
    % Reihenfolge gesetzt werden!
    % Parameter #1: Der Typ des Parameters.  
    % Parameter #2: Der Name des Parameters. 
    % Parameter #3: Die Beschreibung des Parameters. 
    % 
    \newcommand{\setParamFive}[3]{
        \renewcommand{\ParamFive}{#1}
        \renewcommand{\ParamTypeFive}{#2}
        \renewcommand{\ParamDescrFive}{#3} 
        \stepcounter{SavedParams}
    }

    %-----------------------------------------------------------------------
    % Speichert die uebergebenen Informationen ueber Methodenparameter 6
    % in internen Variablen. Achtung: Die Parameter muessen in der richtigen
    % Reihenfolge gesetzt werden!
    % Parameter #1: Der Typ des Parameters.  
    % Parameter #2: Der Name des Parameters. 
    % Parameter #3: Die Beschreibung des Parameters. 
    %     
    \newcommand{\setParamSix}[3]{
        \renewcommand{\ParamSix}{#1}
        \renewcommand{\ParamTypeSix}{#2}
        \renewcommand{\ParamDescrSix}{#3} 
        \stepcounter{SavedParams} 
    }

    %-----------------------------------------------------------------------
    % Speichert die uebergebenen Informationen ueber Methodenparameter 7
    % in internen Variablen. Achtung: Die Parameter muessen in der richtigen
    % Reihenfolge gesetzt werden!
    % Parameter #1: Der Typ des Parameters.  
    % Parameter #2: Der Name des Parameters. 
    % Parameter #3: Die Beschreibung des Parameters. 
    %    
    \newcommand{\setParamSeven}[3]{
        \renewcommand{\ParamSeven}{#1}
        \renewcommand{\ParamTypeSeven}{#2}
        \renewcommand{\ParamDescrSeven}{#3} 
        \stepcounter{SavedParams} 
    }

    %-----------------------------------------------------------------------
    % Speichert die uebergebenen Informationen ueber Methodenparameter 8
    % in internen Variablen. Achtung: Die Parameter muessen in der richtigen
    % Reihenfolge gesetzt werden!
    % Parameter #1: Der Typ des Parameters.  
    % Parameter #2: Der Name des Parameters. 
    % Parameter #3: Die Beschreibung des Parameters. 
    % 
    \newcommand{\setParamEight}[3]{
        \renewcommand{\ParamEight}{#1}
        \renewcommand{\ParamTypeEight}{#2}
        \renewcommand{\ParamDescrEight}{#3} 
        \stepcounter{SavedParams}
    }

    %-----------------------------------------------------------------------
    % Speichert die uebergebenen Informationen ueber Methodenparameter 9
    % in internen Variablen. Achtung: Die Parameter muessen in der richtigen
    % Reihenfolge gesetzt werden!
    % Parameter #1: Der Typ des Parameters.  
    % Parameter #2: Der Name des Parameters. 
    % Parameter #3: Die Beschreibung des Parameters. 
    % 
    \newcommand{\setParamNine}[3]{
        \renewcommand{\ParamNine}{#1}
        \renewcommand{\ParamTypeNine}{#2}
        \renewcommand{\ParamDescrNine}{#3}  
        \stepcounter{SavedParams}
    }

    %-----------------------------------------------------------------------
    % Speichert die uebergebenen Informationen ueber Methodenparameter 10
    % in internen Variablen. Achtung: Die Parameter muessen in der richtigen
    % Reihenfolge gesetzt werden!
    % Parameter #1: Der Typ des Parameters.  
    % Parameter #2: Der Name des Parameters. 
    % Parameter #3: Die Beschreibung des Parameters. 
    % 
    \newcommand{\setParamTen}[3]{
        \renewcommand{\ParamTen}{#1}
        \renewcommand{\ParamTypeTen}{#2}
        \renewcommand{\ParamDescrTen}{#3} 
        \stepcounter{SavedParams}
    }

    %-----------------------------------------------------------------------
    % Gibt den Text ``Rueckgabewert Methodenname(`` aus.
    % Parameter #1:  Rueckgabewert der Methode.
    % Parameter #2:  Name der Methode.
    %
    \newcommand{\printMethodBegin}[2]{
         \ifthenelse{\boolean{ConstInstance}}
             {\settowidth{\MethodEndWidth}{\ConstMethodEnd}}
             {\settowidth{\MethodEndWidth}{\NormMethodEnd}}
         \setMethodBeginWidth{#1}{#2}
         \setBetweenParsWidth
         \noindent
         \hrulefill\\
         \noindent
         \parbox[t]{\MethodBeginWidth}
             {#1\InBetween{\bf #2}\MethodBegin}
     }

    %-----------------------------------------------------------------------
    % Gibt das Ende einer Methode aus, d.h. ``\ );'' oder ``\ ) const;''.
    %
    \newcommand{\printMethodEnd}{
        \ifthenelse{\boolean{ConstInstance}}
            {\ConstMethodEnd\\}
            {\NormMethodEnd\\}
            \mbox{}\hrulefill\\
    }

    %-----------------------------------------------------------------------
    % Gibt den Typen und Namen des einzigen Parameters einer Methode
    % gefolgt von dem Text ``  );'' bzw. `` ) const;'' aus.
    % Parameter #1:  Typ des Parameters.
    % Parameter #2:  Name des Parameters.
    % 
    \newcommand{\printMethodOneParam}[2]{
        \parbox[t]{\ParamMaxTypeWidth}{#1}
        \parbox[t]{\MaxParamWidth}{{\em #2}}
        \printMethodEnd
    }

    %-----------------------------------------------------------------------
    % Gibt den Typen und den Namen des ersten Methodenparameters aus.
    % Parameter #1:  Typ des Parameters.
    % Parameter #2:  Name des Parameters.
    %
    \newcommand{\printMethodFirstParam}[2]{
        \noindent
        \parbox[t]{\ParamMaxTypeWidth}{#1}
        \parbox[t]{\MaxParamWidth}{{\em #2},}\\[\VertSpace]
    }

    %-----------------------------------------------------------------------
    % Gibt den Typen und Namen des letzten Parameters einer Methode aus.
    % Parameter #1: Typ des Parameters.
    % Parameter #2: Name des Parameters.
    %
    \newcommand{\printMethodLastParam}[2]{
        \parbox[t]{\MethodBeginWidth}{\hfill}
        \hspace{\CorrectWidthThree}
        \parbox[t]{\ParamMaxTypeWidth}{#1}
        \parbox[t]{\MaxParamWidth}{{\em #2}}
        \printMethodEnd
    }

    %-----------------------------------------------------------------------
    % Gibt den Typen und Namen eines Methodenparameters aus.
    % Parameter #1:  Typ des Parameters.
    % Parameter #2: Name des Parameters.
    % Achtung: Nicht fuer den ersten oder letzten Parameter der Methode
    % geeignet!
    %
    \newcommand{\printMethodParam}[2]{
        \parbox[t]{\MethodBeginWidth}{\hfill}
        \hspace{\CorrectWidthThree}
        \parbox[t]{\ParamMaxTypeWidth}{#1}
        \parbox[t]{\MaxParamWidth}{{\em #2},}\\[\VertSpace]
    }

    %-----------------------------------------------------------------------
    % Gibt die Beschreibung einer Methode aus.
    % Parameter #1: Beschreibung der Methode.
    %
    \newcommand{\printMethodDescr}[1]{
        \newline
        \noindent
        #1\\[\VertSpace]
    }

    %-----------------------------------------------------------------------
    % Gibt den Text ``Parameter:    `` und danach den Namen des
    % ersten Parameters der Methode, gefolgt von seiner Beschreibung
    % aus.
    % Parameter #1:  Name des Parameters. 
    % Parameter #2:  Beschreibung des Parameters.
    %
    \newcommand{\printFirstParam}[2]{
        \setlength{\Temp}{\ParamDescrWidth}
        \addtolength{\Temp}{1\CorrectWidthOne}
        \parbox[t]{\SpecialWidth}{{\bf Parameter:}}
        \parbox[t]{\MaxParamWidth}{\em #1}
        \parbox[t]{\ParamDelimWidth}{\ParamDelim}
        \parbox[t]{\Temp}{#2}\\[\VertSpace]
    }

    %-----------------------------------------------------------------------
    % Gibt einen Parameter der Methode und die dazugehoerige Beschreibung
    % aus.
    % Parameter #1:  Name des Parameters.
    % Parameter #2:  Beschreibung des Parameters.
    %
    \newcommand{\printParam}[2]{
        \setlength{\Temp}{\ParamDescrWidth}
        \addtolength{\Temp}{1\CorrectWidthOne}
        \parbox[t]{\SpecialWidth}{\hfill}
        \parbox[t]{\MaxParamWidth}{\em #1}
        \parbox[t]{\ParamDelimWidth}{\ParamDelim}
        \parbox[t]{\Temp}{#2}\\[\VertSpace]
    }

    %-----------------------------------------------------------------------
    %
    % Gibt den Text ``Rueckgabewert:  `` gefolgt von einer Beschreibung
    % des Rueckgabewertes aus.
    % Parameter #1:  Beschreibung des Rueckgabewertes.
    %
    \newcommand{\printReturn}[1]{
        \setlength{\Temp}{\SpecialRetTxtWidth}
        \addtolength{\Temp}{1\CorrectWidthTwo}
        \parbox[t]{\SpecialWidth}{{\bf R\"uckgabewert:}}
        \parbox[t]{\Temp}{#1}\\[\VertSpace]
    }

    %-----------------------------------------------------------------------
    % Gibt den Text ``Besonderheiten:  `` gefolgt von einer Beschreibung
    % der Besonderheiten aus.
    % Parameter #1:  Beschreibung der Besonderheiten.
    %
    \newcommand{\printSpecial}[1]{
        \setlength{\Temp}{\SpecialRetTxtWidth}
        \addtolength{\Temp}{1\CorrectWidthTwo}
        \parbox[t]{\SpecialWidth}{\MaxLabelTxt}
        \parbox[t]{\Temp}{#1}\\[\VertSpace]
    }

    %-----------------------------------------------------------------------
    % Gibt den Text ``Parameter:        Keine.'' aus.
    %
    \newcommand{\printNoParams}{
        \parbox[t]{\SpecialWidth}{{\bf Parameter:}}
        \parbox[t]{\NoneWidthOne}{Keine.}\\[\VertSpace]
    }

    %-----------------------------------------------------------------------
    % Gibt den Text ``Rueckgabewert:  Keiner.'' aus.
    %
    \newcommand{\printNoReturn}{
        \parbox[t]{\SpecialWidth}{{\bf R\"uckgabewert:}}
        \parbox[t]{\NoneWidthTwo}{Keiner.}\\[\VertSpace]
    }

    %-----------------------------------------------------------------------
    % Gibt den Text ``Besonderheiten:  Keine.'' aus.
    %
    \newcommand{\printNoSpecial}{
        \parbox[t]{\SpecialWidth}{\MaxLabelTxt}
        \parbox[t]{\NoneWidthOne}{Keine.}\\[\VertSpace]
    }

    %-----------------------------------------------------------------------
    % Gibt den Text:
    % ``Parameter:       Keine.
    %   Rueckgabewert:   Keiner.
    %   Besonderheiten:  Keine.''
    % aus.
    %
    \newcommand{\printNoAll}{
        \printNoParams
        \printNoReturn
        \printNoSpecial
    }

    %-----------------------------------------------------------------------
    % Gibt eine Methode ohne Parameter, Rueckgabewert und Besonderheiten
    % aus.
    % Parameter #1:  Name der Methode.
    % Parameter #2:  Beschreibung der Methode.
    %
    \newcommand{\printEmptyMethod}[2]{
        \setSpecialRetTxtWidth
        \printMethodBegin{void}{#1}
        \printMethodEnd
        \printMethodDescr{#2}
        \printNoAll
    }

    %-----------------------------------------------------------------------
    % Gibt eine Methode ohne Parameter und Besonderheiten, aber mit
    % Rueckgabewert aus.
    % Parameter #1:  Rueckgabewert der Methode.
    % Parameter #2:  Name der Methode.
    % Parameter #3:  Beschreibung der Methode.
    % Parameter #4:  Beschreibung des Rueckgabewertes der Methode.
    %
    \newcommand{\printEmptyMethodReturn}[4]{
        \setSpecialRetTxtWidth
        \printMethodBegin{#1}{#2}
        \printMethodEnd
        \printMethodDescr{#3}
        \printNoParams
        \printReturn{#4}
        \printNoSpecial
    }

    %-----------------------------------------------------------------------
    % Gibt eine Methode ohne Parameter und Rueckgabewert, aber mit
    % Besonderheiten aus.
    % Parameter #1:  Name der Methode.
    % Parameter #2:  Beschreibung der Methode.
    % Parameter #3:  Beschreibung der Besonderheiten der Methode.
    %
    \newcommand{\printEmptyMethodSpecial}[3]{
        \setSpecialRetTxtWidth
        \printMethodBegin{void}{#1}
        \printMethodEnd
        \printMethodDescr{#2}
        \printNoParams
        \printNoReturn
        \printSpecial{#3}
    }

    %-----------------------------------------------------------------------
    % Gibt eine Methode ohne Parameter, aber mit Rueckgabewert
    % und Besonderheiten aus.
    % Parameter #1:  Rueckgabewert der Methode.
    % Parameter #2:  Name der Methode.
    % Parameter #3:  Beschreibung der Methode.
    % Parameter #4:  Beschreibung des Rueckgabewertes der Methode.
    % Parameter #5:  Beschreibung der Besonderheiten der Methode.
    %
    \newcommand{\printEmptyMethodReturnSpecial}[5]{
        \setSpecialRetTxtWidth
        \printMethodBegin{#1}{#2}
        \printMethodEnd
        \printMethodDescr{#3}
        \printNoParams
        \printReturn{#4}
        \printSpecial{#5}
    }

    %-----------------------------------------------------------------------
    % Gibt eine Methode mit einem Parameter komplett mit allen
    % Beschreibungen aus.
    % Parameter #1:  Rueckgabewert der Methode.
    % Parameter #2:  Name der Methode.
    % Parameter #3:  Typ des einzigen Parameters der Methode.
    % Parameter #4:  Name des einzigen Parameters der Methode.
    % Parameter #5:  Beschreibung des einzigen Parameters der Methode.
    % Parameter #6:  Beschreibung der Methode.
    % Parameter #7:  Beschreibung des Rueckgabewertes der Methode.
    % Parameter #8:  Beschreibung der Besonderheiten der Methode.
    %
    \newcommand{\printMethodWithOneParam}[8]{
        \initParamMaxTypeWidth
        \initMaxParamWidth
        \setNewParamMaxTypeWidth{#3}
        \setNewMaxParamWidth{#4}
        \setSpecialRetTxtWidth
        \setParamDescrWidth
        \printMethodBegin{#1}{#2}
        \printMethodOneParam{#3}{#4}
        \printMethodDescr{#6}
        \printFirstParam{#4}{#5}
        \ifthenelse{\equal{#7}{Keiner.}}
            {\printNoReturn}
            {\printReturn{#7}}
        \ifthenelse{\equal{#8}{Keine.}}
            {\printNoSpecial}
            {\printSpecial{#8}}
    }

    %-----------------------------------------------------------------------
    % Gibt eine Methode mit einem Parameter komplett mit allen
    % Beschreibungen aus.
    % Parameter #1:  Rueckgabewert der Methode.
    % Parameter #2:  Beschreibung des Rueckgabewertes der Methode oder 
    %                Aufruf mit leerer Klammer. 
    % Parameter #3:  Name der Methode.
    % Parameter #4:  Beschreibung der Methode.
    % Parameter #5:  Beschreibung der Besonderheiten der Methode oder 
    %                Aufruf mit leerer Klammer.
    %
    \newcommand{\printMethodWithParamsSaved}[5]{
        \ifthenelse{\value{SavedParams} > 0}
            {\initParamMaxTypeWidth
             \initMaxParamWidth
             \setNewParamMaxTypeWidth{\ParamTypeOne}
             \setNewMaxParamWidth{\ParamOne}
             \ifthenelse{\value{SavedParams} = 2 \or \value{SavedParams} > 2}
                 {\setNewParamMaxTypeWidth{\ParamTypeTwo}
                  \setNewMaxParamWidth{\ParamTwo}}{}   
             \ifthenelse{\value{SavedParams} = 3 \or \value{SavedParams} > 3}
                 {\setNewParamMaxTypeWidth{\ParamTypeThree}
                  \setNewMaxParamWidth{\ParamThree}}{}    
             \ifthenelse{\value{SavedParams} = 4 \or \value{SavedParams} > 4}
                 {\setNewParamMaxTypeWidth{\ParamTypeFour}
                  \setNewMaxParamWidth{\ParamFour}}{}
             \ifthenelse{\value{SavedParams} = 5 \or \value{SavedParams} > 5}
                 {\setNewParamMaxTypeWidth{\ParamTypeFive}
                  \setNewMaxParamWidth{\ParamFive}}{}      
             \ifthenelse{\value{SavedParams} = 6 \or \value{SavedParams} > 6}
                 {\setNewParamMaxTypeWidth{\ParamTypeSix}
                  \setNewMaxParamWidth{\ParamSix}}{}
             \ifthenelse{\value{SavedParams} = 7 \or \value{SavedParams} > 7}
                 {\setNewParamMaxTypeWidth{\ParamTypeSeven}
                  \setNewMaxParamWidth{\ParamSeven}}{}       
             \ifthenelse{\value{SavedParams} = 8 \or \value{SavedParams} > 8}
                 {\setNewParamMaxTypeWidth{\ParamTypeEight}
                  \setNewMaxParamWidth{\ParamEight}}{}
             \ifthenelse{\value{SavedParams} = 9 \or \value{SavedParams} > 9}
                 {\setNewParamMaxTypeWidth{\ParamTypeNine}
                  \setNewMaxParamWidth{\ParamNine}}{}
             \ifthenelse{\value{SavedParams} = 10 \or 
                         \value{SavedParams} > 10}
                 {\setNewParamMaxTypeWidth{\ParamTypeTen}
                  \setNewMaxParamWidth{\ParamTen}}{}         
             \setSpecialRetTxtWidth
             \setParamDescrWidth
             \printMethodBegin{#1}{#3}
             \ifthenelse{\value{SavedParams} = 1} 
                 {\printMethodOneParam{\ParamTypeOne}{\ParamOne}}{}
                 {\printMethodFirstParam{\ParamTypeOne}{\ParamOne}}{}
             \ifthenelse{\value{SavedParams} = 2} 
                 {\printMethodLastParam{\ParamTypeTwo}{\ParamTwo}}{}
             \ifthenelse{\value{SavedParams} > 2}  
                 {\printMethodParam{\ParamTypeTwo}{\ParamTwo}}{} 
             \ifthenelse{\value{SavedParams} = 3} 
                 {\printMethodLastParam{\ParamTypeThree}{\ParamThree}}{}
             \ifthenelse{\value{SavedParams} > 3}  
                 {\printMethodParam{\ParamTypeThree}{\ParamThree}}{} 
             \ifthenelse{\value{SavedParams} = 4} 
                 {\printMethodLastParam{\ParamTypeFour}{\ParamFour}}{}
             \ifthenelse{\value{SavedParams} > 4}  
                 {\printMethodParam{\ParamTypeFour}{\ParamFour}}{} 
             \ifthenelse{\value{SavedParams} = 5} 
                 {\printMethodLastParam{\ParamTypeFive}{\ParamFive}}{}
             \ifthenelse{\value{SavedParams} > 5}  
                 {\printMethodParam{\ParamTypeFive}{\ParamFive}}{} 
             \ifthenelse{\value{SavedParams} = 6} 
                 {\printMethodLastParam{\ParamTypeSix}{\ParamSix}}{}
             \ifthenelse{\value{SavedParams} > 6}  
                 {\printMethodParam{\ParamTypeSix}{\ParamSix}}{} 
             \ifthenelse{\value{SavedParams} = 7} 
                 {\printMethodLastParam{\ParamTypeSeven}{\ParamSeven}}{}
             \ifthenelse{\value{SavedParams} > 7}  
                 {\printMethodParam{\ParamTypeSeven}{\ParamSeven}}{} 
             \ifthenelse{\value{SavedParams} = 8} 
                 {\printMethodLastParam{\ParamTypeEight}{\ParamEight}}{}
             \ifthenelse{\value{SavedParams} > 8}  
                 {\printMethodParam{\ParamTypeEight}{\ParamEight}}{}  
             \ifthenelse{\value{SavedParams} = 9} 
                 {\printMethodLastParam{\ParamTypeNine}{\ParamNine}}{}
             \ifthenelse{\value{SavedParams} > 9}  
                 {\printMethodParam{\ParamTypeNine}{\ParamNine}}{} 
             \ifthenelse{\value{SavedParams} = 10} 
                 {\printMethodLastParam{\ParamTypeTen}{\ParamTen}}{}
             \printMethodDescr{#4}
             \printFirstParam{\ParamOne}{\ParamDescrOne}
             \ifthenelse{\value{SavedParams} = 2 \or \value{SavedParams} > 2}
                  {\printParam{\ParamTwo}{\ParamDescrTwo}}{}
             \ifthenelse{\value{SavedParams} = 3 \or \value{SavedParams} > 3}
                  {\printParam{\ParamThree}{\ParamDescrThree}}{}
             \ifthenelse{\value{SavedParams} = 4 \or \value{SavedParams} > 4}
                  {\printParam{\ParamFour}{\ParamDescrFour}}{}
             \ifthenelse{\value{SavedParams} = 5 \or \value{SavedParams} > 5}
                  {\printParam{\ParamFive}{\ParamDescrFive}}{}
             \ifthenelse{\value{SavedParams} = 6 \or \value{SavedParams} > 6}
                  {\printParam{\ParamSix}{\ParamDescrSix}}{}
             \ifthenelse{\value{SavedParams} = 7 \or \value{SavedParams} > 7}
                  {\printParam{\ParamSeven}{\ParamDescrSeven}}{}
             \ifthenelse{\value{SavedParams} = 8 \or \value{SavedParams} > 8}
                  {\printParam{\ParamEight}{\ParamDescrEight}}{} 
             \ifthenelse{\value{SavedParams} = 9 \or \value{SavedParams} > 9}
                  {\printParam{\ParamNine}{\ParamDescrNine}}{}
             \ifthenelse{\value{SavedParams} = 10}
                  {\printParam{\ParamTen}{\ParamDescrTen}}{}
             \ifthenelse{\equal{#2}{}}
                 {\printNoReturn}
                 {\printReturn{#2}}
             \ifthenelse{\equal{#5}{}}
                 {\printNoSpecial}
                 {\printSpecial{#5}}}
        {\noindent
         Fehler Methode {\em printMethodWithParamsSaved}: Es sind noch keine 
         Parameter gesetzt worden!}
    }

%-----------------------------------------------------------------------
% Ende der Befehlsdefinitionen.
%-----------------------------------------------------------------------





%%% Local Variables: 
%%% mode: latex
%%% TeX-master: t
%%% End: 
'' in LaTex-Dokumente 
% (Latex2Epsilon) eingebunden und stellt Befehle zur einfachen Doku- 
% mentation von C-Funktionen/C++-Methoden bereit.
% =======================================================================

\usepackage{ifthen}

    % Definition von Konstanten:
    \newcommand{\ParamDelim}            % Trennungstext zwischen
        { \ - }                         % Parametername und -beschreibung.
    \newcommand{\NormMethodEnd}         % Endtext von Methode bei 
        {\ \ );}                        % veraenderbarer Aufrufinstanz.
    \newcommand{\ConstMethodEnd}        % Endtext von Methode bei
        {\ \ ) const;}                  % nicht veraenderbarer Aufrufinstanz.
    \newcommand{\MaxLabelTxt}           % Komponentenbeschreibungstext
        {{\bf Besonderheiten: }}        % mit maximaler Breite.
    \newcommand{\MethodBegin}           % Beginntext einer Methode.
        {(\ }
    \newcommand{\InBetween}             % Text zwischen Rueckgabewert und
        {\ \ }                          % Methodenname.

    % Definition eigener Variablen:


    \newboolean{ConstInstance}          % Gibt an, ob die Instanz, die
                                        % die Methode aufruft veraendert
                                        % werden darf oder nicht. Im
                                        % 2. Fall endet die Methode mit
                                        % ``) const;'', im 1. Fall 
                                        % mit ``);''.
    \newcounter{SavedParams}            % Anzahl der gespeicherten
                                        % Parameter.
    \setcounter{SavedParams}{0}
    \newcommand{\ParamOne}{}            % Definition der 10 
    \newcommand{\ParamTypeOne}{}        % abspeicherbaren Parameter,
    \newcommand{\ParamDescrOne}{}       % ihrer Typen und Beschreibungen.
    \newcommand{\ParamTwo}{}            
    \newcommand{\ParamTypeTwo}{}
    \newcommand{\ParamDescrTwo}{}
    \newcommand{\ParamThree}{}
    \newcommand{\ParamTypeThree}{}
    \newcommand{\ParamDescrThree}{}
    \newcommand{\ParamFour}{}
    \newcommand{\ParamTypeFour}{}
    \newcommand{\ParamDescrFour}{}
    \newcommand{\ParamFive}{}
    \newcommand{\ParamTypeFive}{}
    \newcommand{\ParamDescrFive}{}
    \newcommand{\ParamSix}{}
    \newcommand{\ParamTypeSix}{}
    \newcommand{\ParamDescrSix}{}
    \newcommand{\ParamSeven}{}
    \newcommand{\ParamTypeSeven}{}
    \newcommand{\ParamDescrSeven}{}
    \newcommand{\ParamEight}{}
    \newcommand{\ParamTypeEight}{}
    \newcommand{\ParamDescrEight}{}
    \newcommand{\ParamNine}{}
    \newcommand{\ParamTypeNine}{}
    \newcommand{\ParamDescrNine}{}
    \newcommand{\ParamTen}{}
    \newcommand{\ParamTypeTen}{}
    \newcommand{\ParamDescrTen}{}
    \newlength{\MaxParamWidth}          % Breite des breitesten 
                                        % Methodenparameters.
    \newlength{\SpecialWidth}           % Breite des Textes 
    \settowidth{\SpecialWidth}          % {\bf Besonderheiten}.
        {\MaxLabelTxt}
    \newlength{\ParamDelimWidth}        % Breite des Textes `` \ - ``.
    \settowidth{\ParamDelimWidth}
        {\ParamDelim}
    \newlength{\ParamDescrWidth}        % Max. Breite des Parametertextes.
    \newlength{\SpecialRetTxtWidth}     % Max. Breite der Texte fuer
                                        % Besonderheiten und Rueckgabewert.
    \newlength{\MethodBeginWidth}       % Breite des Methodenanfangs, d.h.
                                        % ``Rueckgabewert Methodenname ( ``.
    \newlength{\BetweenParsWidth}       % Max. zur Verfuegung stehende
                                        % Breite zwischen den beiden Klammern
                                        % der Methode.
    
    \newlength{\MethodEndWidth}         % Breite des Textes ``  );'' bzw.
    \settowidth{\MethodEndWidth}        % `` ) const;''
        {\NormMethodEnd}
    \newlength{\ParamMaxTypeWidth}      % Breite des breitesten
                                        % Parameterwert-Typs. 
    \newlength{\NoneWidthOne}           % Breite des Textes ``Keine.''.
    \settowidth{\NoneWidthOne}{Keine.}
    \newlength{\NoneWidthTwo}           % Breite des Textes ``Keiner.''.
    \settowidth{\NoneWidthTwo}{Keiner.}
    \newlength{\Temp}                   % Temporaere Hilfsvariable. 
    \newlength{\VertSpace}              % Vertikaler Abstand zwischen 
                                        % zwei Parboxen.
   
    % Trotz der automatischen Berechnung der Textbreiten treten
    % kleine Unstimmigkeiten von wenigen Punkten bei den Texten
    % zur Beschreibung der Parameter, des Rueckgabewertes und
    % der Besonderheiten auf, die mit den folgenden Variablen
    % behoben werden koennen:

    \newlength{\CorrectWidthOne}        % Korrekturbreite fuer die
                                        % Parameterbeschreibung.
    \newlength{\CorrectWidthTwo}        % Korrekturbreite fuer die
                                        % Beschreibungen von Rueckgabewert
                                        % und Besonderheiten.
    \newlength{\CorrectWidthThree}      % Korrekturbreite, da erster
                                        % Parameter einer Methode weiter
                                        % vorrueckt als die anderen.
    

% Uebersicht ueber die Parameter und ihre Bedeutung.


%
% <-- \MethodBeginWidth    -->
%------------------------------------------------------------------------
% Rueckgabewert Methodenname(   Typ_Parameter_1 Parameter_1,
%                            <->Typ_Parameter_2 Parameter_2,
%              \CorrectWideThree...
%                               Typ_Parameter_n Parameter_n   );
%                             <-------------> <--------->
%                          \ParamMaxTypeWidth \MaxParamWidth
%                             <-------------------------->
%                             \BetweenParsWidth
%                                                         <-->
%                                                         \MethodEndWidth
%------------------------------------------------------------------------
% Beschreibung der Methode...
%                             \ParamDelimWidth
%                             <->                        \CorrectWidthOne
% \VertSpace
% Parameter:       Parameter_1  - Beschreibung von Parameter 1.   <----->
%                  Parameter_2  - Beschreibung von Parameter 2.
%                  ...
%                  Parameter_n  - Beschreibung von Parameter n.
% \VertSpace                      \ParamDescrWidth
%                                 <------------------------------>
% Rueckgabewert:   Beschreibung des Rueckgabewertes bzw. der Text <-----> 
%                  Keiner.                   
%                  <----->
%                  \NoneWidthTwo               \SpecialRetTxtWidth
% \VertSpace       <--------------------------------------------->
% Besonderheiten:  Beschreibung von Besonderheiten bzw. der Text  <----->
%                  Keine.     
% <--------------> <---->
%  \SpecialWidth   \NoneWidthOne                         \CorrectWidthTwo


    % Definition eigener Befehle:


    %-----------------------------------------------------------------------
    % Gibt den Begriff ``C++'' in ansprechender Form aus.
    % Makro wurde freundlicherweise zur Verfuegung gestellt von
    % Axel W. Dietrich.
    %
    \newcommand{\cpp}{
        \mbox{\emph{\textrm{C\hspace{-1.5pt}\raisebox{1.75pt}{\scriptsize +}
        \hspace{-6pt}\raisebox{.75pt}{\scriptsize +}}}}%
    }


    %-----------------------------------------------------------------------
    % Gibt die Werte aller Parameter auf dem Bildschirm aus.
    %
    \newcommand{\showAllParams}{
        \noindent
        {\em Werte aller Parameter:}
        \begin{tabbing}
            ParamMaxTypeWidth:\ \ \=\kill
            MaxParamWidth:       \>\the\MaxParamWidth\\
            SpecialWidth:        \>\the\SpecialWidth\\
            ParamDelimWidth:     \>\the\ParamDelimWidth\\
            ParamDescrWidth:     \>\the\ParamDescrWidth\\
            SpecialRetTxtWidth:  \>\the\SpecialRetTxtWidth\\
            MethodBeginWidth:    \>\the\MethodBeginWidth\\
            BetweenParsWidth:    \>\the\BetweenParsWidth\\
            MethodEndWidth:      \>\the\MethodEndWidth\\
            ParamMaxTypeWidth:   \>\the\ParamMaxTypeWidth\\
            NoneWidthOne:        \>\the\NoneWidthOne\\
            NoneWidthTwo:        \>\the\NoneWidthTwo\\
            VertSpace:           \>\the\VertSpace\\
            CorrectWidthOne:     \>\the\CorrectWidthOne\\
            CorrectWidthTwo:     \>\the\CorrectWidthTwo\\
            CorrectWidthThree:   \>\the\CorrectWidthThree\\
            ConstInstance:       
            \ifthenelse{\boolean{ConstInstance}}
                {\>ja\\}
                {\>nein\\}
        \end{tabbing}
    }

    %-----------------------------------------------------------------------
    % Setzt einen Schalter, der angibt, dass die Instanz, die die kommende
    % Methode aufrufen kann, nicht veraendert werden darf. Dies hat fuer
    % die Dokumentation zur Folge, dass die Methode mit ``) const;'' endet.
    %
    \newcommand{\setConstInstance}{    
        \setboolean{ConstInstance}{true}
    }

    %-----------------------------------------------------------------------
    % Setzt einen Schalter, der angibt, dass die Instanz, die die kommende
    % Methode aufrufen kann, veraendert werden darf. Die Methode endet also
    % in der Dokumentation normal mit ``);''.
    %
    \newcommand{\setNormalInstance}{    
        \setboolean{ConstInstance}{false}
    }

    %-----------------------------------------------------------------------
    % Initialisiert die Breite des breitesten Methodenparametertyptextes
    % mit Null.
    %
    \newcommand{\initParamMaxTypeWidth}{
        \setlength{\ParamMaxTypeWidth}{0pt}
    }

    %-----------------------------------------------------------------------
    % Initialisiert die Breite des breitesten Parameternamens mit Null.
    %
    \newcommand{\initMaxParamWidth}{
        \setlength{\MaxParamWidth}{0pt}
    }

    %-----------------------------------------------------------------------
    % Ueberprueft, ob die Breite des uebergebenen Textes des Parametertyps
    % groesser ist als die des bisher als breitester Typtext festgelegten
    % Textes und passt den Wert evtl. an.
    % Parameter #1:  Neuer Typtext, dessen Breite zum Vergleich dient.
    %   
    \newcommand{\setNewParamMaxTypeWidth}[1]{
        \settowidth{\Temp}{#1}
        \ifthenelse{\lengthtest{\ParamMaxTypeWidth < \Temp}}
            {\setlength{\ParamMaxTypeWidth}{\Temp}}{}   
    }

    %-----------------------------------------------------------------------
    % Ueberprueft, ob die Breite des uebergebenen Parameternamens groesser
    % ist als die des bisher als am breitesten geltenden Parameternames
    % und passt den Wert evtl. an. Der Parametername steht dabei in
    % Emphasized.
    % Parameter #1:  Neuer Parametername, dessen Breite zum Vergleich dient.
    %
    \newcommand{\setNewMaxParamWidth}[1]{
        \settowidth{\Temp}{{\em #1}}
        \ifthenelse{\lengthtest{\MaxParamWidth < \Temp}}
            {\setlength{\MaxParamWidth}{\Temp}}{}
    }

    %-----------------------------------------------------------------------
    % Setzt die Breite der Beschreibungstexte fuer Rueckgabewert
    % und Besonderheiten.
    %
    \newcommand{\setSpecialRetTxtWidth}{
        \setlength{\SpecialRetTxtWidth}{\textwidth}
        \addtolength{\SpecialRetTxtWidth}{-1\SpecialWidth} 
    }

    %-----------------------------------------------------------------------
    % Setzt die Breite fuer den Beginn einer Methode, d.h. den Text
    % ``Rueckgabewert Methodenname( ``. Rueckgabewert steht dabei
    % in Emphasized, Methodenname in Bold Font.
    % Parameter #1:  Rueckgabewert der Methode.
    % Parameter #2:  Name der Methode.
    %
    \newcommand{\setMethodBeginWidth}[2]{
        \settowidth{\MethodBeginWidth}
            {{\em #1}\InBetween{\bf #2}\MethodBegin}
    }

    %-----------------------------------------------------------------------
    % Setzt die maximale Breite die an Platz zwischen der oeffnenden
    % und der schliessenden Klammer der Methode zur Verfuegung steht.
    %
    \newcommand{\setBetweenParsWidth}{
        \setlength{\BetweenParsWidth}{\textwidth}
        \addtolength{\BetweenParsWidth}{-1\MethodBeginWidth}
        \addtolength{\BetweenParsWidth}{-1\MethodEndWidth}
    }

    %-----------------------------------------------------------------------
    % Setzt die Breite fuer die Beschreibungstexte
    % der Methodenparameter.
    %
    \newcommand{\setParamDescrWidth}{
        \setlength{\ParamDescrWidth}{\textwidth}
        \addtolength{\ParamDescrWidth}{-1\SpecialWidth} 
        \addtolength{\ParamDescrWidth}{-1\MaxParamWidth}
        \addtolength{\ParamDescrWidth}{-1\ParamDelimWidth}
    }    

    %-----------------------------------------------------------------------
    % Setzt die Korrekturbreite fuer den Beschreibungstext von
    % Methodenparametern auf den uebergebenen Wert.
    % Parameter #1:  Neuer Wert fuer die Korrekturbreite.
    %
    \newcommand{\setCorrectWidthOne}[1]{
        \setlength{\CorrectWidthOne}{#1}
    }    

    %-----------------------------------------------------------------------
    % Setzt die Korrekturbreite fuer den Beschreibungstext von
    % Rueckgabewert und Besonderheiten auf den uebergebenen Wert.
    % Parameter #1:  Neuer Wert fuer die Korrekturbreite.
    %
    \newcommand{\setCorrectWidthTwo}[1]{
        \setlength{\CorrectWidthTwo}{#1}
    }    

    %-----------------------------------------------------------------------
    % Setzt die Korrekturbreite fuer den Abstand des zweiten bis n-ten
    % Parameters vom Anfang auf den uebergebenen Wert.
    % Parameter #1:  Neuer Wert fuer die Korrekturbreite.
    %
    \newcommand{\setCorrectWidthThree}[1]{
        \setlength{\CorrectWidthThree}{#1}
    }    

    %-----------------------------------------------------------------------
    % Setzt den vertikalen Abstand zwischen zwei Parboxen auf den ueber-
    % gebenen Wert.
    % Parameter #1:  Neuer Wert fuer den vertikalen Abstand.
    %
    \newcommand{\setVertSpace}[1]{
        \setlength{\VertSpace}{#1}
    }    

    %-----------------------------------------------------------------------
    % Speichert die uebergebenen Informationen ueber Methodenparameter 1
    % in internen Variablen. Achtung: Die Parameter muessen in der richtigen
    % Reihenfolge gesetzt werden!
    % Parameter #1: Der Typ des Parameters.  
    % Parameter #2: Der Name des Parameters. 
    % Parameter #3: Die Beschreibung des Parameters. 
    % 
    \newcommand{\setParamOne}[3]{
        \renewcommand{\ParamOne}{#1}
        \renewcommand{\ParamTypeOne}{#2}
        \renewcommand{\ParamDescrOne}{#3} 
        \setcounter{SavedParams}{0}
        \stepcounter{SavedParams}
    }

    %-----------------------------------------------------------------------
    % Speichert die uebergebenen Informationen ueber Methodenparameter 2
    % in internen Variablen. Achtung: Die Parameter muessen in der richtigen
    % Reihenfolge gesetzt werden!
    % Parameter #1: Der Typ des Parameters.  
    % Parameter #2: Der Name des Parameters. 
    % Parameter #3: Die Beschreibung des Parameters. 
    % 
    \newcommand{\setParamTwo}[3]{
        \renewcommand{\ParamTwo}{#1}
        \renewcommand{\ParamTypeTwo}{#2}
        \renewcommand{\ParamDescrTwo}{#3} 
        \stepcounter{SavedParams}
    }

    %-----------------------------------------------------------------------
    % Speichert die uebergebenen Informationen ueber Methodenparameter 3
    % in internen Variablen. Achtung: Die Parameter muessen in der richtigen
    % Reihenfolge gesetzt werden!
    % Parameter #1: Der Typ des Parameters.  
    % Parameter #2: Der Name des Parameters. 
    % Parameter #3: Die Beschreibung des Parameters. 
    % 
    \newcommand{\setParamThree}[3]{
        \renewcommand{\ParamThree}{#1}
        \renewcommand{\ParamTypeThree}{#2}
        \renewcommand{\ParamDescrThree}{#3} 
        \stepcounter{SavedParams}
    }

    %-----------------------------------------------------------------------
    % Speichert die uebergebenen Informationen ueber Methodenparameter 4
    % in internen Variablen. Achtung: Die Parameter muessen in der richtigen
    % Reihenfolge gesetzt werden!
    % Parameter #1: Der Typ des Parameters.  
    % Parameter #2: Der Name des Parameters. 
    % Parameter #3: Die Beschreibung des Parameters. 
    % 
    \newcommand{\setParamFour}[3]{
        \renewcommand{\ParamFour}{#1}
        \renewcommand{\ParamTypeFour}{#2}
        \renewcommand{\ParamDescrFour}{#3} 
        \stepcounter{SavedParams}
    }

    %-----------------------------------------------------------------------
    % Speichert die uebergebenen Informationen ueber Methodenparameter 5
    % in internen Variablen. Achtung: Die Parameter muessen in der richtigen
    % Reihenfolge gesetzt werden!
    % Parameter #1: Der Typ des Parameters.  
    % Parameter #2: Der Name des Parameters. 
    % Parameter #3: Die Beschreibung des Parameters. 
    % 
    \newcommand{\setParamFive}[3]{
        \renewcommand{\ParamFive}{#1}
        \renewcommand{\ParamTypeFive}{#2}
        \renewcommand{\ParamDescrFive}{#3} 
        \stepcounter{SavedParams}
    }

    %-----------------------------------------------------------------------
    % Speichert die uebergebenen Informationen ueber Methodenparameter 6
    % in internen Variablen. Achtung: Die Parameter muessen in der richtigen
    % Reihenfolge gesetzt werden!
    % Parameter #1: Der Typ des Parameters.  
    % Parameter #2: Der Name des Parameters. 
    % Parameter #3: Die Beschreibung des Parameters. 
    %     
    \newcommand{\setParamSix}[3]{
        \renewcommand{\ParamSix}{#1}
        \renewcommand{\ParamTypeSix}{#2}
        \renewcommand{\ParamDescrSix}{#3} 
        \stepcounter{SavedParams} 
    }

    %-----------------------------------------------------------------------
    % Speichert die uebergebenen Informationen ueber Methodenparameter 7
    % in internen Variablen. Achtung: Die Parameter muessen in der richtigen
    % Reihenfolge gesetzt werden!
    % Parameter #1: Der Typ des Parameters.  
    % Parameter #2: Der Name des Parameters. 
    % Parameter #3: Die Beschreibung des Parameters. 
    %    
    \newcommand{\setParamSeven}[3]{
        \renewcommand{\ParamSeven}{#1}
        \renewcommand{\ParamTypeSeven}{#2}
        \renewcommand{\ParamDescrSeven}{#3} 
        \stepcounter{SavedParams} 
    }

    %-----------------------------------------------------------------------
    % Speichert die uebergebenen Informationen ueber Methodenparameter 8
    % in internen Variablen. Achtung: Die Parameter muessen in der richtigen
    % Reihenfolge gesetzt werden!
    % Parameter #1: Der Typ des Parameters.  
    % Parameter #2: Der Name des Parameters. 
    % Parameter #3: Die Beschreibung des Parameters. 
    % 
    \newcommand{\setParamEight}[3]{
        \renewcommand{\ParamEight}{#1}
        \renewcommand{\ParamTypeEight}{#2}
        \renewcommand{\ParamDescrEight}{#3} 
        \stepcounter{SavedParams}
    }

    %-----------------------------------------------------------------------
    % Speichert die uebergebenen Informationen ueber Methodenparameter 9
    % in internen Variablen. Achtung: Die Parameter muessen in der richtigen
    % Reihenfolge gesetzt werden!
    % Parameter #1: Der Typ des Parameters.  
    % Parameter #2: Der Name des Parameters. 
    % Parameter #3: Die Beschreibung des Parameters. 
    % 
    \newcommand{\setParamNine}[3]{
        \renewcommand{\ParamNine}{#1}
        \renewcommand{\ParamTypeNine}{#2}
        \renewcommand{\ParamDescrNine}{#3}  
        \stepcounter{SavedParams}
    }

    %-----------------------------------------------------------------------
    % Speichert die uebergebenen Informationen ueber Methodenparameter 10
    % in internen Variablen. Achtung: Die Parameter muessen in der richtigen
    % Reihenfolge gesetzt werden!
    % Parameter #1: Der Typ des Parameters.  
    % Parameter #2: Der Name des Parameters. 
    % Parameter #3: Die Beschreibung des Parameters. 
    % 
    \newcommand{\setParamTen}[3]{
        \renewcommand{\ParamTen}{#1}
        \renewcommand{\ParamTypeTen}{#2}
        \renewcommand{\ParamDescrTen}{#3} 
        \stepcounter{SavedParams}
    }

    %-----------------------------------------------------------------------
    % Gibt den Text ``Rueckgabewert Methodenname(`` aus.
    % Parameter #1:  Rueckgabewert der Methode.
    % Parameter #2:  Name der Methode.
    %
    \newcommand{\printMethodBegin}[2]{
         \ifthenelse{\boolean{ConstInstance}}
             {\settowidth{\MethodEndWidth}{\ConstMethodEnd}}
             {\settowidth{\MethodEndWidth}{\NormMethodEnd}}
         \setMethodBeginWidth{#1}{#2}
         \setBetweenParsWidth
         \noindent
         \hrulefill\\
         \noindent
         \parbox[t]{\MethodBeginWidth}
             {#1\InBetween{\bf #2}\MethodBegin}
     }

    %-----------------------------------------------------------------------
    % Gibt das Ende einer Methode aus, d.h. ``\ );'' oder ``\ ) const;''.
    %
    \newcommand{\printMethodEnd}{
        \ifthenelse{\boolean{ConstInstance}}
            {\ConstMethodEnd\\}
            {\NormMethodEnd\\}
            \mbox{}\hrulefill\\
    }

    %-----------------------------------------------------------------------
    % Gibt den Typen und Namen des einzigen Parameters einer Methode
    % gefolgt von dem Text ``  );'' bzw. `` ) const;'' aus.
    % Parameter #1:  Typ des Parameters.
    % Parameter #2:  Name des Parameters.
    % 
    \newcommand{\printMethodOneParam}[2]{
        \parbox[t]{\ParamMaxTypeWidth}{#1}
        \parbox[t]{\MaxParamWidth}{{\em #2}}
        \printMethodEnd
    }

    %-----------------------------------------------------------------------
    % Gibt den Typen und den Namen des ersten Methodenparameters aus.
    % Parameter #1:  Typ des Parameters.
    % Parameter #2:  Name des Parameters.
    %
    \newcommand{\printMethodFirstParam}[2]{
        \noindent
        \parbox[t]{\ParamMaxTypeWidth}{#1}
        \parbox[t]{\MaxParamWidth}{{\em #2},}\\[\VertSpace]
    }

    %-----------------------------------------------------------------------
    % Gibt den Typen und Namen des letzten Parameters einer Methode aus.
    % Parameter #1: Typ des Parameters.
    % Parameter #2: Name des Parameters.
    %
    \newcommand{\printMethodLastParam}[2]{
        \parbox[t]{\MethodBeginWidth}{\hfill}
        \hspace{\CorrectWidthThree}
        \parbox[t]{\ParamMaxTypeWidth}{#1}
        \parbox[t]{\MaxParamWidth}{{\em #2}}
        \printMethodEnd
    }

    %-----------------------------------------------------------------------
    % Gibt den Typen und Namen eines Methodenparameters aus.
    % Parameter #1:  Typ des Parameters.
    % Parameter #2: Name des Parameters.
    % Achtung: Nicht fuer den ersten oder letzten Parameter der Methode
    % geeignet!
    %
    \newcommand{\printMethodParam}[2]{
        \parbox[t]{\MethodBeginWidth}{\hfill}
        \hspace{\CorrectWidthThree}
        \parbox[t]{\ParamMaxTypeWidth}{#1}
        \parbox[t]{\MaxParamWidth}{{\em #2},}\\[\VertSpace]
    }

    %-----------------------------------------------------------------------
    % Gibt die Beschreibung einer Methode aus.
    % Parameter #1: Beschreibung der Methode.
    %
    \newcommand{\printMethodDescr}[1]{
        \newline
        \noindent
        #1\\[\VertSpace]
    }

    %-----------------------------------------------------------------------
    % Gibt den Text ``Parameter:    `` und danach den Namen des
    % ersten Parameters der Methode, gefolgt von seiner Beschreibung
    % aus.
    % Parameter #1:  Name des Parameters. 
    % Parameter #2:  Beschreibung des Parameters.
    %
    \newcommand{\printFirstParam}[2]{
        \setlength{\Temp}{\ParamDescrWidth}
        \addtolength{\Temp}{1\CorrectWidthOne}
        \parbox[t]{\SpecialWidth}{{\bf Parameter:}}
        \parbox[t]{\MaxParamWidth}{\em #1}
        \parbox[t]{\ParamDelimWidth}{\ParamDelim}
        \parbox[t]{\Temp}{#2}\\[\VertSpace]
    }

    %-----------------------------------------------------------------------
    % Gibt einen Parameter der Methode und die dazugehoerige Beschreibung
    % aus.
    % Parameter #1:  Name des Parameters.
    % Parameter #2:  Beschreibung des Parameters.
    %
    \newcommand{\printParam}[2]{
        \setlength{\Temp}{\ParamDescrWidth}
        \addtolength{\Temp}{1\CorrectWidthOne}
        \parbox[t]{\SpecialWidth}{\hfill}
        \parbox[t]{\MaxParamWidth}{\em #1}
        \parbox[t]{\ParamDelimWidth}{\ParamDelim}
        \parbox[t]{\Temp}{#2}\\[\VertSpace]
    }

    %-----------------------------------------------------------------------
    %
    % Gibt den Text ``Rueckgabewert:  `` gefolgt von einer Beschreibung
    % des Rueckgabewertes aus.
    % Parameter #1:  Beschreibung des Rueckgabewertes.
    %
    \newcommand{\printReturn}[1]{
        \setlength{\Temp}{\SpecialRetTxtWidth}
        \addtolength{\Temp}{1\CorrectWidthTwo}
        \parbox[t]{\SpecialWidth}{{\bf R\"uckgabewert:}}
        \parbox[t]{\Temp}{#1}\\[\VertSpace]
    }

    %-----------------------------------------------------------------------
    % Gibt den Text ``Besonderheiten:  `` gefolgt von einer Beschreibung
    % der Besonderheiten aus.
    % Parameter #1:  Beschreibung der Besonderheiten.
    %
    \newcommand{\printSpecial}[1]{
        \setlength{\Temp}{\SpecialRetTxtWidth}
        \addtolength{\Temp}{1\CorrectWidthTwo}
        \parbox[t]{\SpecialWidth}{\MaxLabelTxt}
        \parbox[t]{\Temp}{#1}\\[\VertSpace]
    }

    %-----------------------------------------------------------------------
    % Gibt den Text ``Parameter:        Keine.'' aus.
    %
    \newcommand{\printNoParams}{
        \parbox[t]{\SpecialWidth}{{\bf Parameter:}}
        \parbox[t]{\NoneWidthOne}{Keine.}\\[\VertSpace]
    }

    %-----------------------------------------------------------------------
    % Gibt den Text ``Rueckgabewert:  Keiner.'' aus.
    %
    \newcommand{\printNoReturn}{
        \parbox[t]{\SpecialWidth}{{\bf R\"uckgabewert:}}
        \parbox[t]{\NoneWidthTwo}{Keiner.}\\[\VertSpace]
    }

    %-----------------------------------------------------------------------
    % Gibt den Text ``Besonderheiten:  Keine.'' aus.
    %
    \newcommand{\printNoSpecial}{
        \parbox[t]{\SpecialWidth}{\MaxLabelTxt}
        \parbox[t]{\NoneWidthOne}{Keine.}\\[\VertSpace]
    }

    %-----------------------------------------------------------------------
    % Gibt den Text:
    % ``Parameter:       Keine.
    %   Rueckgabewert:   Keiner.
    %   Besonderheiten:  Keine.''
    % aus.
    %
    \newcommand{\printNoAll}{
        \printNoParams
        \printNoReturn
        \printNoSpecial
    }

    %-----------------------------------------------------------------------
    % Gibt eine Methode ohne Parameter, Rueckgabewert und Besonderheiten
    % aus.
    % Parameter #1:  Name der Methode.
    % Parameter #2:  Beschreibung der Methode.
    %
    \newcommand{\printEmptyMethod}[2]{
        \setSpecialRetTxtWidth
        \printMethodBegin{void}{#1}
        \printMethodEnd
        \printMethodDescr{#2}
        \printNoAll
    }

    %-----------------------------------------------------------------------
    % Gibt eine Methode ohne Parameter und Besonderheiten, aber mit
    % Rueckgabewert aus.
    % Parameter #1:  Rueckgabewert der Methode.
    % Parameter #2:  Name der Methode.
    % Parameter #3:  Beschreibung der Methode.
    % Parameter #4:  Beschreibung des Rueckgabewertes der Methode.
    %
    \newcommand{\printEmptyMethodReturn}[4]{
        \setSpecialRetTxtWidth
        \printMethodBegin{#1}{#2}
        \printMethodEnd
        \printMethodDescr{#3}
        \printNoParams
        \printReturn{#4}
        \printNoSpecial
    }

    %-----------------------------------------------------------------------
    % Gibt eine Methode ohne Parameter und Rueckgabewert, aber mit
    % Besonderheiten aus.
    % Parameter #1:  Name der Methode.
    % Parameter #2:  Beschreibung der Methode.
    % Parameter #3:  Beschreibung der Besonderheiten der Methode.
    %
    \newcommand{\printEmptyMethodSpecial}[3]{
        \setSpecialRetTxtWidth
        \printMethodBegin{void}{#1}
        \printMethodEnd
        \printMethodDescr{#2}
        \printNoParams
        \printNoReturn
        \printSpecial{#3}
    }

    %-----------------------------------------------------------------------
    % Gibt eine Methode ohne Parameter, aber mit Rueckgabewert
    % und Besonderheiten aus.
    % Parameter #1:  Rueckgabewert der Methode.
    % Parameter #2:  Name der Methode.
    % Parameter #3:  Beschreibung der Methode.
    % Parameter #4:  Beschreibung des Rueckgabewertes der Methode.
    % Parameter #5:  Beschreibung der Besonderheiten der Methode.
    %
    \newcommand{\printEmptyMethodReturnSpecial}[5]{
        \setSpecialRetTxtWidth
        \printMethodBegin{#1}{#2}
        \printMethodEnd
        \printMethodDescr{#3}
        \printNoParams
        \printReturn{#4}
        \printSpecial{#5}
    }

    %-----------------------------------------------------------------------
    % Gibt eine Methode mit einem Parameter komplett mit allen
    % Beschreibungen aus.
    % Parameter #1:  Rueckgabewert der Methode.
    % Parameter #2:  Name der Methode.
    % Parameter #3:  Typ des einzigen Parameters der Methode.
    % Parameter #4:  Name des einzigen Parameters der Methode.
    % Parameter #5:  Beschreibung des einzigen Parameters der Methode.
    % Parameter #6:  Beschreibung der Methode.
    % Parameter #7:  Beschreibung des Rueckgabewertes der Methode.
    % Parameter #8:  Beschreibung der Besonderheiten der Methode.
    %
    \newcommand{\printMethodWithOneParam}[8]{
        \initParamMaxTypeWidth
        \initMaxParamWidth
        \setNewParamMaxTypeWidth{#3}
        \setNewMaxParamWidth{#4}
        \setSpecialRetTxtWidth
        \setParamDescrWidth
        \printMethodBegin{#1}{#2}
        \printMethodOneParam{#3}{#4}
        \printMethodDescr{#6}
        \printFirstParam{#4}{#5}
        \ifthenelse{\equal{#7}{Keiner.}}
            {\printNoReturn}
            {\printReturn{#7}}
        \ifthenelse{\equal{#8}{Keine.}}
            {\printNoSpecial}
            {\printSpecial{#8}}
    }

    %-----------------------------------------------------------------------
    % Gibt eine Methode mit einem Parameter komplett mit allen
    % Beschreibungen aus.
    % Parameter #1:  Rueckgabewert der Methode.
    % Parameter #2:  Beschreibung des Rueckgabewertes der Methode oder 
    %                Aufruf mit leerer Klammer. 
    % Parameter #3:  Name der Methode.
    % Parameter #4:  Beschreibung der Methode.
    % Parameter #5:  Beschreibung der Besonderheiten der Methode oder 
    %                Aufruf mit leerer Klammer.
    %
    \newcommand{\printMethodWithParamsSaved}[5]{
        \ifthenelse{\value{SavedParams} > 0}
            {\initParamMaxTypeWidth
             \initMaxParamWidth
             \setNewParamMaxTypeWidth{\ParamTypeOne}
             \setNewMaxParamWidth{\ParamOne}
             \ifthenelse{\value{SavedParams} = 2 \or \value{SavedParams} > 2}
                 {\setNewParamMaxTypeWidth{\ParamTypeTwo}
                  \setNewMaxParamWidth{\ParamTwo}}{}   
             \ifthenelse{\value{SavedParams} = 3 \or \value{SavedParams} > 3}
                 {\setNewParamMaxTypeWidth{\ParamTypeThree}
                  \setNewMaxParamWidth{\ParamThree}}{}    
             \ifthenelse{\value{SavedParams} = 4 \or \value{SavedParams} > 4}
                 {\setNewParamMaxTypeWidth{\ParamTypeFour}
                  \setNewMaxParamWidth{\ParamFour}}{}
             \ifthenelse{\value{SavedParams} = 5 \or \value{SavedParams} > 5}
                 {\setNewParamMaxTypeWidth{\ParamTypeFive}
                  \setNewMaxParamWidth{\ParamFive}}{}      
             \ifthenelse{\value{SavedParams} = 6 \or \value{SavedParams} > 6}
                 {\setNewParamMaxTypeWidth{\ParamTypeSix}
                  \setNewMaxParamWidth{\ParamSix}}{}
             \ifthenelse{\value{SavedParams} = 7 \or \value{SavedParams} > 7}
                 {\setNewParamMaxTypeWidth{\ParamTypeSeven}
                  \setNewMaxParamWidth{\ParamSeven}}{}       
             \ifthenelse{\value{SavedParams} = 8 \or \value{SavedParams} > 8}
                 {\setNewParamMaxTypeWidth{\ParamTypeEight}
                  \setNewMaxParamWidth{\ParamEight}}{}
             \ifthenelse{\value{SavedParams} = 9 \or \value{SavedParams} > 9}
                 {\setNewParamMaxTypeWidth{\ParamTypeNine}
                  \setNewMaxParamWidth{\ParamNine}}{}
             \ifthenelse{\value{SavedParams} = 10 \or 
                         \value{SavedParams} > 10}
                 {\setNewParamMaxTypeWidth{\ParamTypeTen}
                  \setNewMaxParamWidth{\ParamTen}}{}         
             \setSpecialRetTxtWidth
             \setParamDescrWidth
             \printMethodBegin{#1}{#3}
             \ifthenelse{\value{SavedParams} = 1} 
                 {\printMethodOneParam{\ParamTypeOne}{\ParamOne}}{}
                 {\printMethodFirstParam{\ParamTypeOne}{\ParamOne}}{}
             \ifthenelse{\value{SavedParams} = 2} 
                 {\printMethodLastParam{\ParamTypeTwo}{\ParamTwo}}{}
             \ifthenelse{\value{SavedParams} > 2}  
                 {\printMethodParam{\ParamTypeTwo}{\ParamTwo}}{} 
             \ifthenelse{\value{SavedParams} = 3} 
                 {\printMethodLastParam{\ParamTypeThree}{\ParamThree}}{}
             \ifthenelse{\value{SavedParams} > 3}  
                 {\printMethodParam{\ParamTypeThree}{\ParamThree}}{} 
             \ifthenelse{\value{SavedParams} = 4} 
                 {\printMethodLastParam{\ParamTypeFour}{\ParamFour}}{}
             \ifthenelse{\value{SavedParams} > 4}  
                 {\printMethodParam{\ParamTypeFour}{\ParamFour}}{} 
             \ifthenelse{\value{SavedParams} = 5} 
                 {\printMethodLastParam{\ParamTypeFive}{\ParamFive}}{}
             \ifthenelse{\value{SavedParams} > 5}  
                 {\printMethodParam{\ParamTypeFive}{\ParamFive}}{} 
             \ifthenelse{\value{SavedParams} = 6} 
                 {\printMethodLastParam{\ParamTypeSix}{\ParamSix}}{}
             \ifthenelse{\value{SavedParams} > 6}  
                 {\printMethodParam{\ParamTypeSix}{\ParamSix}}{} 
             \ifthenelse{\value{SavedParams} = 7} 
                 {\printMethodLastParam{\ParamTypeSeven}{\ParamSeven}}{}
             \ifthenelse{\value{SavedParams} > 7}  
                 {\printMethodParam{\ParamTypeSeven}{\ParamSeven}}{} 
             \ifthenelse{\value{SavedParams} = 8} 
                 {\printMethodLastParam{\ParamTypeEight}{\ParamEight}}{}
             \ifthenelse{\value{SavedParams} > 8}  
                 {\printMethodParam{\ParamTypeEight}{\ParamEight}}{}  
             \ifthenelse{\value{SavedParams} = 9} 
                 {\printMethodLastParam{\ParamTypeNine}{\ParamNine}}{}
             \ifthenelse{\value{SavedParams} > 9}  
                 {\printMethodParam{\ParamTypeNine}{\ParamNine}}{} 
             \ifthenelse{\value{SavedParams} = 10} 
                 {\printMethodLastParam{\ParamTypeTen}{\ParamTen}}{}
             \printMethodDescr{#4}
             \printFirstParam{\ParamOne}{\ParamDescrOne}
             \ifthenelse{\value{SavedParams} = 2 \or \value{SavedParams} > 2}
                  {\printParam{\ParamTwo}{\ParamDescrTwo}}{}
             \ifthenelse{\value{SavedParams} = 3 \or \value{SavedParams} > 3}
                  {\printParam{\ParamThree}{\ParamDescrThree}}{}
             \ifthenelse{\value{SavedParams} = 4 \or \value{SavedParams} > 4}
                  {\printParam{\ParamFour}{\ParamDescrFour}}{}
             \ifthenelse{\value{SavedParams} = 5 \or \value{SavedParams} > 5}
                  {\printParam{\ParamFive}{\ParamDescrFive}}{}
             \ifthenelse{\value{SavedParams} = 6 \or \value{SavedParams} > 6}
                  {\printParam{\ParamSix}{\ParamDescrSix}}{}
             \ifthenelse{\value{SavedParams} = 7 \or \value{SavedParams} > 7}
                  {\printParam{\ParamSeven}{\ParamDescrSeven}}{}
             \ifthenelse{\value{SavedParams} = 8 \or \value{SavedParams} > 8}
                  {\printParam{\ParamEight}{\ParamDescrEight}}{} 
             \ifthenelse{\value{SavedParams} = 9 \or \value{SavedParams} > 9}
                  {\printParam{\ParamNine}{\ParamDescrNine}}{}
             \ifthenelse{\value{SavedParams} = 10}
                  {\printParam{\ParamTen}{\ParamDescrTen}}{}
             \ifthenelse{\equal{#2}{}}
                 {\printNoReturn}
                 {\printReturn{#2}}
             \ifthenelse{\equal{#5}{}}
                 {\printNoSpecial}
                 {\printSpecial{#5}}}
        {\noindent
         Fehler Methode {\em printMethodWithParamsSaved}: Es sind noch keine 
         Parameter gesetzt worden!}
    }

%-----------------------------------------------------------------------
% Ende der Befehlsdefinitionen.
%-----------------------------------------------------------------------





%%% Local Variables: 
%%% mode: latex
%%% TeX-master: t
%%% End: 
'' in LaTex-Dokumente 
% (Latex2Epsilon) eingebunden und stellt Befehle zur einfachen Doku- 
% mentation von C-Funktionen/C++-Methoden bereit.
% =======================================================================

\usepackage{ifthen}

    % Definition von Konstanten:
    \newcommand{\ParamDelim}            % Trennungstext zwischen
        { \ - }                         % Parametername und -beschreibung.
    \newcommand{\NormMethodEnd}         % Endtext von Methode bei 
        {\ \ );}                        % veraenderbarer Aufrufinstanz.
    \newcommand{\ConstMethodEnd}        % Endtext von Methode bei
        {\ \ ) const;}                  % nicht veraenderbarer Aufrufinstanz.
    \newcommand{\MaxLabelTxt}           % Komponentenbeschreibungstext
        {{\bf Return Value: }}          % mit maximaler Breite.
    \newcommand{\MethodBegin}           % Beginntext einer Methode.
        {(\ }
    \newcommand{\InBetween}             % Text zwischen Rueckgabewert und
        {\ \ }                          % Methodenname.

    % Definition eigener Variablen:


    \newboolean{ConstInstance}          % Gibt an, ob die Instanz, die
                                        % die Methode aufruft veraendert
                                        % werden darf oder nicht. Im
                                        % 2. Fall endet die Methode mit
                                        % ``) const;'', im 1. Fall 
                                        % mit ``);''.
    \newcounter{SavedParams}            % Anzahl der gespeicherten
                                        % Parameter.
    \setcounter{SavedParams}{0}
    \newcommand{\ParamOne}{}            % Definition der 10 
    \newcommand{\ParamTypeOne}{}        % abspeicherbaren Parameter,
    \newcommand{\ParamDescrOne}{}       % ihrer Typen und Beschreibungen.
    \newcommand{\ParamTwo}{}            
    \newcommand{\ParamTypeTwo}{}
    \newcommand{\ParamDescrTwo}{}
    \newcommand{\ParamThree}{}
    \newcommand{\ParamTypeThree}{}
    \newcommand{\ParamDescrThree}{}
    \newcommand{\ParamFour}{}
    \newcommand{\ParamTypeFour}{}
    \newcommand{\ParamDescrFour}{}
    \newcommand{\ParamFive}{}
    \newcommand{\ParamTypeFive}{}
    \newcommand{\ParamDescrFive}{}
    \newcommand{\ParamSix}{}
    \newcommand{\ParamTypeSix}{}
    \newcommand{\ParamDescrSix}{}
    \newcommand{\ParamSeven}{}
    \newcommand{\ParamTypeSeven}{}
    \newcommand{\ParamDescrSeven}{}
    \newcommand{\ParamEight}{}
    \newcommand{\ParamTypeEight}{}
    \newcommand{\ParamDescrEight}{}
    \newcommand{\ParamNine}{}
    \newcommand{\ParamTypeNine}{}
    \newcommand{\ParamDescrNine}{}
    \newcommand{\ParamTen}{}
    \newcommand{\ParamTypeTen}{}
    \newcommand{\ParamDescrTen}{}
    \newlength{\MaxParamWidth}          % Breite des breitesten 
                                        % Methodenparameters.
    \newlength{\SpecialWidth}           % Breite des Textes 
    \settowidth{\SpecialWidth}          % {\bf Besonderheiten}.
        {\MaxLabelTxt}
    \newlength{\ParamDelimWidth}        % Breite des Textes `` \ - ``.
    \settowidth{\ParamDelimWidth}
        {\ParamDelim}
    \newlength{\ParamDescrWidth}        % Max. Breite des Parametertextes.
    \newlength{\SpecialRetTxtWidth}     % Max. Breite der Texte fuer
                                        % Besonderheiten und Rueckgabewert.
    \newlength{\MethodBeginWidth}       % Breite des Methodenanfangs, d.h.
                                        % ``Rueckgabewert Methodenname ( ``.
    \newlength{\BetweenParsWidth}       % Max. zur Verfuegung stehende
                                        % Breite zwischen den beiden Klammern
                                        % der Methode.
    
    \newlength{\MethodEndWidth}         % Breite des Textes ``  );'' bzw.
    \settowidth{\MethodEndWidth}        % `` ) const;''
        {\NormMethodEnd}
    \newlength{\ParamMaxTypeWidth}      % Breite des breitesten
                                        % Parameterwert-Typs. 
    \newlength{\NoneWidthOne}           % Breite des Textes ``Keine.''.
    \settowidth{\NoneWidthOne}{None.}
    \newlength{\NoneWidthTwo}           % Breite des Textes ``Keiner.''.
    \settowidth{\NoneWidthTwo}{None.}
    \newlength{\Temp}                   % Temporaere Hilfsvariable. 
    \newlength{\VertSpace}              % Vertikaler Abstand zwischen 
                                        % zwei Parboxen.
   
    % Trotz der automatischen Berechnung der Textbreiten treten
    % kleine Unstimmigkeiten von wenigen Punkten bei den Texten
    % zur Beschreibung der Parameter, des Rueckgabewertes und
    % der Besonderheiten auf, die mit den folgenden Variablen
    % behoben werden koennen:

    \newlength{\CorrectWidthOne}        % Korrekturbreite fuer die
                                        % Parameterbeschreibung.
    \newlength{\CorrectWidthTwo}        % Korrekturbreite fuer die
                                        % Beschreibungen von Rueckgabewert
                                        % und Besonderheiten.
    \newlength{\CorrectWidthThree}      % Korrekturbreite, da erster
                                        % Parameter einer Methode weiter
                                        % vorrueckt als die anderen.
    

% Uebersicht ueber die Parameter und ihre Bedeutung.


%
% <-- \MethodBeginWidth    -->
%------------------------------------------------------------------------
% Rueckgabewert Methodenname(   Typ_Parameter_1 Parameter_1,
%                            <->Typ_Parameter_2 Parameter_2,
%              \CorrectWideThree...
%                               Typ_Parameter_n Parameter_n   );
%                             <-------------> <--------->
%                          \ParamMaxTypeWidth \MaxParamWidth
%                             <-------------------------->
%                             \BetweenParsWidth
%                                                         <-->
%                                                         \MethodEndWidth
%------------------------------------------------------------------------
% Beschreibung der Methode...
%                             \ParamDelimWidth
%                             <->                        \CorrectWidthOne
% \VertSpace
% Parameter:       Parameter_1  - Beschreibung von Parameter 1.   <----->
%                  Parameter_2  - Beschreibung von Parameter 2.
%                  ...
%                  Parameter_n  - Beschreibung von Parameter n.
% \VertSpace                      \ParamDescrWidth
%                                 <------------------------------>
% Rueckgabewert:   Beschreibung des Rueckgabewertes bzw. der Text <-----> 
%                  Keiner.                   
%                  <----->
%                  \NoneWidthTwo               \SpecialRetTxtWidth
% \VertSpace       <--------------------------------------------->
% Besonderheiten:  Beschreibung von Besonderheiten bzw. der Text  <----->
%                  Keine.     
% <--------------> <---->
%  \SpecialWidth   \NoneWidthOne                         \CorrectWidthTwo


    % Definition eigener Befehle:


    %-----------------------------------------------------------------------
    % Gibt den Begriff ``C++'' in ansprechender Form aus.
    % Makro wurde freundlicherweise zur Verfuegung gestellt von
    % Axel W. Dietrich.
    %
    \newcommand{\cpp}{
        \mbox{\emph{\textrm{C\hspace{-1.5pt}\raisebox{1.75pt}{\scriptsize +}
        \hspace{-6pt}\raisebox{.75pt}{\scriptsize +}}}}%
    }


    %-----------------------------------------------------------------------
    % Gibt die Werte aller Parameter auf dem Bildschirm aus.
    %
    \newcommand{\showAllParams}{
        \noindent
        {\em Werte aller Parameter:}
        \begin{tabbing}
            ParamMaxTypeWidth:\ \ \=\kill
            MaxParamWidth:       \>\the\MaxParamWidth\\
            SpecialWidth:        \>\the\SpecialWidth\\
            ParamDelimWidth:     \>\the\ParamDelimWidth\\
            ParamDescrWidth:     \>\the\ParamDescrWidth\\
            SpecialRetTxtWidth:  \>\the\SpecialRetTxtWidth\\
            MethodBeginWidth:    \>\the\MethodBeginWidth\\
            BetweenParsWidth:    \>\the\BetweenParsWidth\\
            MethodEndWidth:      \>\the\MethodEndWidth\\
            ParamMaxTypeWidth:   \>\the\ParamMaxTypeWidth\\
            NoneWidthOne:        \>\the\NoneWidthOne\\
            NoneWidthTwo:        \>\the\NoneWidthTwo\\
            VertSpace:           \>\the\VertSpace\\
            CorrectWidthOne:     \>\the\CorrectWidthOne\\
            CorrectWidthTwo:     \>\the\CorrectWidthTwo\\
            CorrectWidthThree:   \>\the\CorrectWidthThree\\
            ConstInstance:       
            \ifthenelse{\boolean{ConstInstance}}
                {\>ja\\}
                {\>nein\\}
        \end{tabbing}
    }

    %-----------------------------------------------------------------------
    % Setzt einen Schalter, der angibt, dass die Instanz, die die kommende
    % Methode aufrufen kann, nicht veraendert werden darf. Dies hat fuer
    % die Dokumentation zur Folge, dass die Methode mit ``) const;'' endet.
    %
    \newcommand{\setConstInstance}{    
        \setboolean{ConstInstance}{true}
    }

    %-----------------------------------------------------------------------
    % Setzt einen Schalter, der angibt, dass die Instanz, die die kommende
    % Methode aufrufen kann, veraendert werden darf. Die Methode endet also
    % in der Dokumentation normal mit ``);''.
    %
    \newcommand{\setNormalInstance}{    
        \setboolean{ConstInstance}{false}
    }

    %-----------------------------------------------------------------------
    % Initialisiert die Breite des breitesten Methodenparametertyptextes
    % mit Null.
    %
    \newcommand{\initParamMaxTypeWidth}{
        \setlength{\ParamMaxTypeWidth}{0pt}
    }

    %-----------------------------------------------------------------------
    % Initialisiert die Breite des breitesten Parameternamens mit Null.
    %
    \newcommand{\initMaxParamWidth}{
        \setlength{\MaxParamWidth}{0pt}
    }

    %-----------------------------------------------------------------------
    % Ueberprueft, ob die Breite des uebergebenen Textes des Parametertyps
    % groesser ist als die des bisher als breitester Typtext festgelegten
    % Textes und passt den Wert evtl. an.
    % Parameter #1:  Neuer Typtext, dessen Breite zum Vergleich dient.
    %   
    \newcommand{\setNewParamMaxTypeWidth}[1]{
        \settowidth{\Temp}{#1}
        \ifthenelse{\lengthtest{\ParamMaxTypeWidth < \Temp}}
            {\setlength{\ParamMaxTypeWidth}{\Temp}}{}   
    }

    %-----------------------------------------------------------------------
    % Ueberprueft, ob die Breite des uebergebenen Parameternamens groesser
    % ist als die des bisher als am breitesten geltenden Parameternames
    % und passt den Wert evtl. an. Der Parametername steht dabei in
    % Emphasized.
    % Parameter #1:  Neuer Parametername, dessen Breite zum Vergleich dient.
    %
    \newcommand{\setNewMaxParamWidth}[1]{
        \settowidth{\Temp}{{\em #1}}
        \ifthenelse{\lengthtest{\MaxParamWidth < \Temp}}
            {\setlength{\MaxParamWidth}{\Temp}}{}
    }

    %-----------------------------------------------------------------------
    % Setzt die Breite der Beschreibungstexte fuer Rueckgabewert
    % und Besonderheiten.
    %
    \newcommand{\setSpecialRetTxtWidth}{
        \setlength{\SpecialRetTxtWidth}{\textwidth}
        \addtolength{\SpecialRetTxtWidth}{-1\SpecialWidth} 
    }

    %-----------------------------------------------------------------------
    % Setzt die Breite fuer den Beginn einer Methode, d.h. den Text
    % ``Rueckgabewert Methodenname( ``. Rueckgabewert steht dabei
    % in Emphasized, Methodenname in Bold Font.
    % Parameter #1:  Rueckgabewert der Methode.
    % Parameter #2:  Name der Methode.
    %
    \newcommand{\setMethodBeginWidth}[2]{
        \settowidth{\MethodBeginWidth}
            {{\em #1}\InBetween{\bf #2}\MethodBegin}
    }

    %-----------------------------------------------------------------------
    % Setzt die maximale Breite die an Platz zwischen der oeffnenden
    % und der schliessenden Klammer der Methode zur Verfuegung steht.
    %
    \newcommand{\setBetweenParsWidth}{
        \setlength{\BetweenParsWidth}{\textwidth}
        \addtolength{\BetweenParsWidth}{-1\MethodBeginWidth}
        \addtolength{\BetweenParsWidth}{-1\MethodEndWidth}
    }

    %-----------------------------------------------------------------------
    % Setzt die Breite fuer die Beschreibungstexte
    % der Methodenparameter.
    %
    \newcommand{\setParamDescrWidth}{
        \setlength{\ParamDescrWidth}{\textwidth}
        \addtolength{\ParamDescrWidth}{-1\SpecialWidth} 
        \addtolength{\ParamDescrWidth}{-1\MaxParamWidth}
        \addtolength{\ParamDescrWidth}{-1\ParamDelimWidth}
    }    

    %-----------------------------------------------------------------------
    % Setzt die Korrekturbreite fuer den Beschreibungstext von
    % Methodenparametern auf den uebergebenen Wert.
    % Parameter #1:  Neuer Wert fuer die Korrekturbreite.
    %
    \newcommand{\setCorrectWidthOne}[1]{
        \setlength{\CorrectWidthOne}{#1}
    }    

    %-----------------------------------------------------------------------
    % Setzt die Korrekturbreite fuer den Beschreibungstext von
    % Rueckgabewert und Besonderheiten auf den uebergebenen Wert.
    % Parameter #1:  Neuer Wert fuer die Korrekturbreite.
    %
    \newcommand{\setCorrectWidthTwo}[1]{
        \setlength{\CorrectWidthTwo}{#1}
    }    

    %-----------------------------------------------------------------------
    % Setzt die Korrekturbreite fuer den Abstand des zweiten bis n-ten
    % Parameters vom Anfang auf den uebergebenen Wert.
    % Parameter #1:  Neuer Wert fuer die Korrekturbreite.
    %
    \newcommand{\setCorrectWidthThree}[1]{
        \setlength{\CorrectWidthThree}{#1}
    }    

    %-----------------------------------------------------------------------
    % Setzt den vertikalen Abstand zwischen zwei Parboxen auf den ueber-
    % gebenen Wert.
    % Parameter #1:  Neuer Wert fuer den vertikalen Abstand.
    %
    \newcommand{\setVertSpace}[1]{
        \setlength{\VertSpace}{#1}
    }    

    %-----------------------------------------------------------------------
    % Speichert die uebergebenen Informationen ueber Methodenparameter 1
    % in internen Variablen. Achtung: Die Parameter muessen in der richtigen
    % Reihenfolge gesetzt werden!
    % Parameter #1: Der Typ des Parameters.  
    % Parameter #2: Der Name des Parameters. 
    % Parameter #3: Die Beschreibung des Parameters. 
    % 
    \newcommand{\setParamOne}[3]{
        \renewcommand{\ParamOne}{#1}
        \renewcommand{\ParamTypeOne}{#2}
        \renewcommand{\ParamDescrOne}{#3} 
        \setcounter{SavedParams}{0}
        \stepcounter{SavedParams}
    }

    %-----------------------------------------------------------------------
    % Speichert die uebergebenen Informationen ueber Methodenparameter 2
    % in internen Variablen. Achtung: Die Parameter muessen in der richtigen
    % Reihenfolge gesetzt werden!
    % Parameter #1: Der Typ des Parameters.  
    % Parameter #2: Der Name des Parameters. 
    % Parameter #3: Die Beschreibung des Parameters. 
    % 
    \newcommand{\setParamTwo}[3]{
        \renewcommand{\ParamTwo}{#1}
        \renewcommand{\ParamTypeTwo}{#2}
        \renewcommand{\ParamDescrTwo}{#3} 
        \stepcounter{SavedParams}
    }

    %-----------------------------------------------------------------------
    % Speichert die uebergebenen Informationen ueber Methodenparameter 3
    % in internen Variablen. Achtung: Die Parameter muessen in der richtigen
    % Reihenfolge gesetzt werden!
    % Parameter #1: Der Typ des Parameters.  
    % Parameter #2: Der Name des Parameters. 
    % Parameter #3: Die Beschreibung des Parameters. 
    % 
    \newcommand{\setParamThree}[3]{
        \renewcommand{\ParamThree}{#1}
        \renewcommand{\ParamTypeThree}{#2}
        \renewcommand{\ParamDescrThree}{#3} 
        \stepcounter{SavedParams}
    }

    %-----------------------------------------------------------------------
    % Speichert die uebergebenen Informationen ueber Methodenparameter 4
    % in internen Variablen. Achtung: Die Parameter muessen in der richtigen
    % Reihenfolge gesetzt werden!
    % Parameter #1: Der Typ des Parameters.  
    % Parameter #2: Der Name des Parameters. 
    % Parameter #3: Die Beschreibung des Parameters. 
    % 
    \newcommand{\setParamFour}[3]{
        \renewcommand{\ParamFour}{#1}
        \renewcommand{\ParamTypeFour}{#2}
        \renewcommand{\ParamDescrFour}{#3} 
        \stepcounter{SavedParams}
    }

    %-----------------------------------------------------------------------
    % Speichert die uebergebenen Informationen ueber Methodenparameter 5
    % in internen Variablen. Achtung: Die Parameter muessen in der richtigen
    % Reihenfolge gesetzt werden!
    % Parameter #1: Der Typ des Parameters.  
    % Parameter #2: Der Name des Parameters. 
    % Parameter #3: Die Beschreibung des Parameters. 
    % 
    \newcommand{\setParamFive}[3]{
        \renewcommand{\ParamFive}{#1}
        \renewcommand{\ParamTypeFive}{#2}
        \renewcommand{\ParamDescrFive}{#3} 
        \stepcounter{SavedParams}
    }

    %-----------------------------------------------------------------------
    % Speichert die uebergebenen Informationen ueber Methodenparameter 6
    % in internen Variablen. Achtung: Die Parameter muessen in der richtigen
    % Reihenfolge gesetzt werden!
    % Parameter #1: Der Typ des Parameters.  
    % Parameter #2: Der Name des Parameters. 
    % Parameter #3: Die Beschreibung des Parameters. 
    %     
    \newcommand{\setParamSix}[3]{
        \renewcommand{\ParamSix}{#1}
        \renewcommand{\ParamTypeSix}{#2}
        \renewcommand{\ParamDescrSix}{#3} 
        \stepcounter{SavedParams} 
    }

    %-----------------------------------------------------------------------
    % Speichert die uebergebenen Informationen ueber Methodenparameter 7
    % in internen Variablen. Achtung: Die Parameter muessen in der richtigen
    % Reihenfolge gesetzt werden!
    % Parameter #1: Der Typ des Parameters.  
    % Parameter #2: Der Name des Parameters. 
    % Parameter #3: Die Beschreibung des Parameters. 
    %    
    \newcommand{\setParamSeven}[3]{
        \renewcommand{\ParamSeven}{#1}
        \renewcommand{\ParamTypeSeven}{#2}
        \renewcommand{\ParamDescrSeven}{#3} 
        \stepcounter{SavedParams} 
    }

    %-----------------------------------------------------------------------
    % Speichert die uebergebenen Informationen ueber Methodenparameter 8
    % in internen Variablen. Achtung: Die Parameter muessen in der richtigen
    % Reihenfolge gesetzt werden!
    % Parameter #1: Der Typ des Parameters.  
    % Parameter #2: Der Name des Parameters. 
    % Parameter #3: Die Beschreibung des Parameters. 
    % 
    \newcommand{\setParamEight}[3]{
        \renewcommand{\ParamEight}{#1}
        \renewcommand{\ParamTypeEight}{#2}
        \renewcommand{\ParamDescrEight}{#3} 
        \stepcounter{SavedParams}
    }

    %-----------------------------------------------------------------------
    % Speichert die uebergebenen Informationen ueber Methodenparameter 9
    % in internen Variablen. Achtung: Die Parameter muessen in der richtigen
    % Reihenfolge gesetzt werden!
    % Parameter #1: Der Typ des Parameters.  
    % Parameter #2: Der Name des Parameters. 
    % Parameter #3: Die Beschreibung des Parameters. 
    % 
    \newcommand{\setParamNine}[3]{
        \renewcommand{\ParamNine}{#1}
        \renewcommand{\ParamTypeNine}{#2}
        \renewcommand{\ParamDescrNine}{#3}  
        \stepcounter{SavedParams}
    }

    %-----------------------------------------------------------------------
    % Speichert die uebergebenen Informationen ueber Methodenparameter 10
    % in internen Variablen. Achtung: Die Parameter muessen in der richtigen
    % Reihenfolge gesetzt werden!
    % Parameter #1: Der Typ des Parameters.  
    % Parameter #2: Der Name des Parameters. 
    % Parameter #3: Die Beschreibung des Parameters. 
    % 
    \newcommand{\setParamTen}[3]{
        \renewcommand{\ParamTen}{#1}
        \renewcommand{\ParamTypeTen}{#2}
        \renewcommand{\ParamDescrTen}{#3} 
        \stepcounter{SavedParams}
    }

    %-----------------------------------------------------------------------
    % Gibt den Text ``Rueckgabewert Methodenname(`` aus.
    % Parameter #1:  Rueckgabewert der Methode.
    % Parameter #2:  Name der Methode.
    %
    \newcommand{\printMethodBegin}[2]{
         \ifthenelse{\boolean{ConstInstance}}
             {\settowidth{\MethodEndWidth}{\ConstMethodEnd}}
             {\settowidth{\MethodEndWidth}{\NormMethodEnd}}
         \setMethodBeginWidth{#1}{#2}
         \setBetweenParsWidth
         \noindent
         \hrulefill\\
         \noindent
         \parbox[t]{\MethodBeginWidth}
             {#1\InBetween{\bf #2}\MethodBegin}
     }

    %-----------------------------------------------------------------------
    % Gibt das Ende einer Methode aus, d.h. ``\ );'' oder ``\ ) const;''.
    %
    \newcommand{\printMethodEnd}{
        \ifthenelse{\boolean{ConstInstance}}
            {\ConstMethodEnd\\}
            {\NormMethodEnd\\}
            \mbox{}\hrulefill\\
    }

    %-----------------------------------------------------------------------
    % Gibt den Typen und Namen des einzigen Parameters einer Methode
    % gefolgt von dem Text ``  );'' bzw. `` ) const;'' aus.
    % Parameter #1:  Typ des Parameters.
    % Parameter #2:  Name des Parameters.
    % 
    \newcommand{\printMethodOneParam}[2]{
        \parbox[t]{\ParamMaxTypeWidth}{#1}
        \parbox[t]{\MaxParamWidth}{{\em #2}}
        \printMethodEnd
    }

    %-----------------------------------------------------------------------
    % Gibt den Typen und den Namen des ersten Methodenparameters aus.
    % Parameter #1:  Typ des Parameters.
    % Parameter #2:  Name des Parameters.
    %
    \newcommand{\printMethodFirstParam}[2]{
        \noindent
        \parbox[t]{\ParamMaxTypeWidth}{#1}
        \parbox[t]{\MaxParamWidth}{{\em #2},}\\[\VertSpace]
    }

    %-----------------------------------------------------------------------
    % Gibt den Typen und Namen des letzten Parameters einer Methode aus.
    % Parameter #1: Typ des Parameters.
    % Parameter #2: Name des Parameters.
    %
    \newcommand{\printMethodLastParam}[2]{
        \parbox[t]{\MethodBeginWidth}{\hfill}
        \hspace{\CorrectWidthThree}
        \parbox[t]{\ParamMaxTypeWidth}{#1}
        \parbox[t]{\MaxParamWidth}{{\em #2}}
        \printMethodEnd
    }

    %-----------------------------------------------------------------------
    % Gibt den Typen und Namen eines Methodenparameters aus.
    % Parameter #1:  Typ des Parameters.
    % Parameter #2: Name des Parameters.
    % Achtung: Nicht fuer den ersten oder letzten Parameter der Methode
    % geeignet!
    %
    \newcommand{\printMethodParam}[2]{
        \parbox[t]{\MethodBeginWidth}{\hfill}
        \hspace{\CorrectWidthThree}
        \parbox[t]{\ParamMaxTypeWidth}{#1}
        \parbox[t]{\MaxParamWidth}{{\em #2},}\\[\VertSpace]
    }

    %-----------------------------------------------------------------------
    % Gibt die Beschreibung einer Methode aus.
    % Parameter #1: Beschreibung der Methode.
    %
    \newcommand{\printMethodDescr}[1]{
        \newline
        \noindent
        #1\\[\VertSpace]
    }

    %-----------------------------------------------------------------------
    % Gibt den Text ``Parameter:    `` und danach den Namen des
    % ersten Parameters der Methode, gefolgt von seiner Beschreibung
    % aus.
    % Parameter #1:  Name des Parameters. 
    % Parameter #2:  Beschreibung des Parameters.
    %
    \newcommand{\printFirstParam}[2]{
        \setlength{\Temp}{\ParamDescrWidth}
        \addtolength{\Temp}{1\CorrectWidthOne}
        \parbox[t]{\SpecialWidth}{{\bf Parameters:}}
        \parbox[t]{\MaxParamWidth}{\em #1}
        \parbox[t]{\ParamDelimWidth}{\ParamDelim}
        \parbox[t]{\Temp}{#2}\\[\VertSpace]
    }

    %-----------------------------------------------------------------------
    % Gibt einen Parameter der Methode und die dazugehoerige Beschreibung
    % aus.
    % Parameter #1:  Name des Parameters.
    % Parameter #2:  Beschreibung des Parameters.
    %
    \newcommand{\printParam}[2]{
        \setlength{\Temp}{\ParamDescrWidth}
        \addtolength{\Temp}{1\CorrectWidthOne}
        \parbox[t]{\SpecialWidth}{\hfill}
        \parbox[t]{\MaxParamWidth}{\em #1}
        \parbox[t]{\ParamDelimWidth}{\ParamDelim}
        \parbox[t]{\Temp}{#2}\\[\VertSpace]
    }

    %-----------------------------------------------------------------------
    %
    % Gibt den Text ``Rueckgabewert:  `` gefolgt von einer Beschreibung
    % des Rueckgabewertes aus.
    % Parameter #1:  Beschreibung des Rueckgabewertes.
    %
    \newcommand{\printReturn}[1]{
        \setlength{\Temp}{\SpecialRetTxtWidth}
        \addtolength{\Temp}{1\CorrectWidthTwo}
        \parbox[t]{\SpecialWidth}{{\bf Return Value:}}
        \parbox[t]{\Temp}{#1}\\[\VertSpace]
    }

    %-----------------------------------------------------------------------
    % Gibt den Text ``Besonderheiten:  `` gefolgt von einer Beschreibung
    % der Besonderheiten aus.
    % Parameter #1:  Beschreibung der Besonderheiten.
    %
    \newcommand{\printSpecial}[1]{
        \setlength{\Temp}{\SpecialRetTxtWidth}
        \addtolength{\Temp}{1\CorrectWidthTwo}
        \parbox[t]{\SpecialWidth}{\bf Caveats:}
        \parbox[t]{\Temp}{#1}\\[\VertSpace]
    }

    %-----------------------------------------------------------------------
    % Gibt den Text ``Parameter:        Keine.'' aus.
    %
    \newcommand{\printNoParams}{
        \parbox[t]{\SpecialWidth}{{\bf Parameters:}}
        \parbox[t]{\NoneWidthOne}{None.}\\[\VertSpace]
    }

    %-----------------------------------------------------------------------
    % Gibt den Text ``Rueckgabewert:  Keiner.'' aus.
    %
    \newcommand{\printNoReturn}{
        \parbox[t]{\SpecialWidth}{{\bf Return Value:}}
        \parbox[t]{\NoneWidthTwo}{None.}\\[\VertSpace]
    }

    %-----------------------------------------------------------------------
    % Gibt den Text ``Besonderheiten:  Keine.'' aus.
    %
    \newcommand{\printNoSpecial}{
        \parbox[t]{\SpecialWidth}{\bf Caveats:}
        \parbox[t]{\NoneWidthOne}{None.}\\[\VertSpace]
    }

    %-----------------------------------------------------------------------
    % Gibt den Text:
    % ``Parameter:       Keine.
    %   Rueckgabewert:   Keiner.
    %   Besonderheiten:  Keine.''
    % aus.
    %
    \newcommand{\printNoAll}{
        \printNoParams
        \printNoReturn
        \printNoSpecial
    }

    %-----------------------------------------------------------------------
    % Gibt eine Methode ohne Parameter, Rueckgabewert und Besonderheiten
    % aus.
    % Parameter #1:  Name der Methode.
    % Parameter #2:  Beschreibung der Methode.
    %
    \newcommand{\printEmptyMethod}[2]{
        \setSpecialRetTxtWidth
        \printMethodBegin{void}{#1}
        \printMethodEnd
        \printMethodDescr{#2}
        \printNoAll
    }

    %-----------------------------------------------------------------------
    % Gibt eine Methode ohne Parameter und Besonderheiten, aber mit
    % Rueckgabewert aus.
    % Parameter #1:  Rueckgabewert der Methode.
    % Parameter #2:  Name der Methode.
    % Parameter #3:  Beschreibung der Methode.
    % Parameter #4:  Beschreibung des Rueckgabewertes der Methode.
    %
    \newcommand{\printEmptyMethodReturn}[4]{
        \setSpecialRetTxtWidth
        \printMethodBegin{#1}{#2}
        \printMethodEnd
        \printMethodDescr{#3}
        \printNoParams
        \printReturn{#4}
        \printNoSpecial
    }

    %-----------------------------------------------------------------------
    % Gibt eine Methode ohne Parameter und Rueckgabewert, aber mit
    % Besonderheiten aus.
    % Parameter #1:  Name der Methode.
    % Parameter #2:  Beschreibung der Methode.
    % Parameter #3:  Beschreibung der Besonderheiten der Methode.
    %
    \newcommand{\printEmptyMethodSpecial}[3]{
        \setSpecialRetTxtWidth
        \printMethodBegin{void}{#1}
        \printMethodEnd
        \printMethodDescr{#2}
        \printNoParams
        \printNoReturn
        \printSpecial{#3}
    }

    %-----------------------------------------------------------------------
    % Gibt eine Methode ohne Parameter, aber mit Rueckgabewert
    % und Besonderheiten aus.
    % Parameter #1:  Rueckgabewert der Methode.
    % Parameter #2:  Name der Methode.
    % Parameter #3:  Beschreibung der Methode.
    % Parameter #4:  Beschreibung des Rueckgabewertes der Methode.
    % Parameter #5:  Beschreibung der Besonderheiten der Methode.
    %
    \newcommand{\printEmptyMethodReturnSpecial}[5]{
        \setSpecialRetTxtWidth
        \printMethodBegin{#1}{#2}
        \printMethodEnd
        \printMethodDescr{#3}
        \printNoParams
        \printReturn{#4}
        \printSpecial{#5}
    }

    %-----------------------------------------------------------------------
    % Gibt eine Methode mit einem Parameter komplett mit allen
    % Beschreibungen aus.
    % Parameter #1:  Rueckgabewert der Methode.
    % Parameter #2:  Name der Methode.
    % Parameter #3:  Typ des einzigen Parameters der Methode.
    % Parameter #4:  Name des einzigen Parameters der Methode.
    % Parameter #5:  Beschreibung des einzigen Parameters der Methode.
    % Parameter #6:  Beschreibung der Methode.
    % Parameter #7:  Beschreibung des Rueckgabewertes der Methode.
    % Parameter #8:  Beschreibung der Besonderheiten der Methode.
    %
    \newcommand{\printMethodWithOneParam}[8]{
        \initParamMaxTypeWidth
        \initMaxParamWidth
        \setNewParamMaxTypeWidth{#3}
        \setNewMaxParamWidth{#4}
        \setSpecialRetTxtWidth
        \setParamDescrWidth
        \printMethodBegin{#1}{#2}
        \printMethodOneParam{#3}{#4}
        \printMethodDescr{#6}
        \printFirstParam{#4}{#5}
        \ifthenelse{\equal{#7}{Keiner.}}
            {\printNoReturn}
            {\printReturn{#7}}
        \ifthenelse{\equal{#8}{Keine.}}
            {\printNoSpecial}
            {\printSpecial{#8}}
    }

    %-----------------------------------------------------------------------
    % Gibt eine Methode mit einem Parameter komplett mit allen
    % Beschreibungen aus.
    % Parameter #1:  Rueckgabewert der Methode.
    % Parameter #2:  Beschreibung des Rueckgabewertes der Methode oder 
    %                Aufruf mit leerer Klammer. 
    % Parameter #3:  Name der Methode.
    % Parameter #4:  Beschreibung der Methode.
    % Parameter #5:  Beschreibung der Besonderheiten der Methode oder 
    %                Aufruf mit leerer Klammer.
    %
    \newcommand{\printMethodWithParamsSaved}[5]{
        \ifthenelse{\value{SavedParams} > 0}
            {\initParamMaxTypeWidth
             \initMaxParamWidth
             \setNewParamMaxTypeWidth{\ParamTypeOne}
             \setNewMaxParamWidth{\ParamOne}
             \ifthenelse{\value{SavedParams} = 2 \or \value{SavedParams} > 2}
                 {\setNewParamMaxTypeWidth{\ParamTypeTwo}
                  \setNewMaxParamWidth{\ParamTwo}}{}   
             \ifthenelse{\value{SavedParams} = 3 \or \value{SavedParams} > 3}
                 {\setNewParamMaxTypeWidth{\ParamTypeThree}
                  \setNewMaxParamWidth{\ParamThree}}{}    
             \ifthenelse{\value{SavedParams} = 4 \or \value{SavedParams} > 4}
                 {\setNewParamMaxTypeWidth{\ParamTypeFour}
                  \setNewMaxParamWidth{\ParamFour}}{}
             \ifthenelse{\value{SavedParams} = 5 \or \value{SavedParams} > 5}
                 {\setNewParamMaxTypeWidth{\ParamTypeFive}
                  \setNewMaxParamWidth{\ParamFive}}{}      
             \ifthenelse{\value{SavedParams} = 6 \or \value{SavedParams} > 6}
                 {\setNewParamMaxTypeWidth{\ParamTypeSix}
                  \setNewMaxParamWidth{\ParamSix}}{}
             \ifthenelse{\value{SavedParams} = 7 \or \value{SavedParams} > 7}
                 {\setNewParamMaxTypeWidth{\ParamTypeSeven}
                  \setNewMaxParamWidth{\ParamSeven}}{}       
             \ifthenelse{\value{SavedParams} = 8 \or \value{SavedParams} > 8}
                 {\setNewParamMaxTypeWidth{\ParamTypeEight}
                  \setNewMaxParamWidth{\ParamEight}}{}
             \ifthenelse{\value{SavedParams} = 9 \or \value{SavedParams} > 9}
                 {\setNewParamMaxTypeWidth{\ParamTypeNine}
                  \setNewMaxParamWidth{\ParamNine}}{}
             \ifthenelse{\value{SavedParams} = 10 \or 
                         \value{SavedParams} > 10}
                 {\setNewParamMaxTypeWidth{\ParamTypeTen}
                  \setNewMaxParamWidth{\ParamTen}}{}         
             \setSpecialRetTxtWidth
             \setParamDescrWidth
             \printMethodBegin{#1}{#3}
             \ifthenelse{\value{SavedParams} = 1} 
                 {\printMethodOneParam{\ParamTypeOne}{\ParamOne}}{}
                 {\printMethodFirstParam{\ParamTypeOne}{\ParamOne}}{}
             \ifthenelse{\value{SavedParams} = 2} 
                 {\printMethodLastParam{\ParamTypeTwo}{\ParamTwo}}{}
             \ifthenelse{\value{SavedParams} > 2}  
                 {\printMethodParam{\ParamTypeTwo}{\ParamTwo}}{} 
             \ifthenelse{\value{SavedParams} = 3} 
                 {\printMethodLastParam{\ParamTypeThree}{\ParamThree}}{}
             \ifthenelse{\value{SavedParams} > 3}  
                 {\printMethodParam{\ParamTypeThree}{\ParamThree}}{} 
             \ifthenelse{\value{SavedParams} = 4} 
                 {\printMethodLastParam{\ParamTypeFour}{\ParamFour}}{}
             \ifthenelse{\value{SavedParams} > 4}  
                 {\printMethodParam{\ParamTypeFour}{\ParamFour}}{} 
             \ifthenelse{\value{SavedParams} = 5} 
                 {\printMethodLastParam{\ParamTypeFive}{\ParamFive}}{}
             \ifthenelse{\value{SavedParams} > 5}  
                 {\printMethodParam{\ParamTypeFive}{\ParamFive}}{} 
             \ifthenelse{\value{SavedParams} = 6} 
                 {\printMethodLastParam{\ParamTypeSix}{\ParamSix}}{}
             \ifthenelse{\value{SavedParams} > 6}  
                 {\printMethodParam{\ParamTypeSix}{\ParamSix}}{} 
             \ifthenelse{\value{SavedParams} = 7} 
                 {\printMethodLastParam{\ParamTypeSeven}{\ParamSeven}}{}
             \ifthenelse{\value{SavedParams} > 7}  
                 {\printMethodParam{\ParamTypeSeven}{\ParamSeven}}{} 
             \ifthenelse{\value{SavedParams} = 8} 
                 {\printMethodLastParam{\ParamTypeEight}{\ParamEight}}{}
             \ifthenelse{\value{SavedParams} > 8}  
                 {\printMethodParam{\ParamTypeEight}{\ParamEight}}{}  
             \ifthenelse{\value{SavedParams} = 9} 
                 {\printMethodLastParam{\ParamTypeNine}{\ParamNine}}{}
             \ifthenelse{\value{SavedParams} > 9}  
                 {\printMethodParam{\ParamTypeNine}{\ParamNine}}{} 
             \ifthenelse{\value{SavedParams} = 10} 
                 {\printMethodLastParam{\ParamTypeTen}{\ParamTen}}{}
             \printMethodDescr{#4}
             \printFirstParam{\ParamOne}{\ParamDescrOne}
             \ifthenelse{\value{SavedParams} = 2 \or \value{SavedParams} > 2}
                  {\printParam{\ParamTwo}{\ParamDescrTwo}}{}
             \ifthenelse{\value{SavedParams} = 3 \or \value{SavedParams} > 3}
                  {\printParam{\ParamThree}{\ParamDescrThree}}{}
             \ifthenelse{\value{SavedParams} = 4 \or \value{SavedParams} > 4}
                  {\printParam{\ParamFour}{\ParamDescrFour}}{}
             \ifthenelse{\value{SavedParams} = 5 \or \value{SavedParams} > 5}
                  {\printParam{\ParamFive}{\ParamDescrFive}}{}
             \ifthenelse{\value{SavedParams} = 6 \or \value{SavedParams} > 6}
                  {\printParam{\ParamSix}{\ParamDescrSix}}{}
             \ifthenelse{\value{SavedParams} = 7 \or \value{SavedParams} > 7}
                  {\printParam{\ParamSeven}{\ParamDescrSeven}}{}
             \ifthenelse{\value{SavedParams} = 8 \or \value{SavedParams} > 8}
                  {\printParam{\ParamEight}{\ParamDescrEight}}{} 
             \ifthenelse{\value{SavedParams} = 9 \or \value{SavedParams} > 9}
                  {\printParam{\ParamNine}{\ParamDescrNine}}{}
             \ifthenelse{\value{SavedParams} = 10}
                  {\printParam{\ParamTen}{\ParamDescrTen}}{}
             \ifthenelse{\equal{#2}{}}
                 {\printNoReturn}
                 {\printReturn{#2}}
             \ifthenelse{\equal{#5}{}}
                 {\printNoSpecial}
                 {\printSpecial{#5}}}
        {\noindent
         Fehler Methode {\em printMethodWithParamsSaved}: Es sind noch keine 
         Parameter gesetzt worden!}
    }

%-----------------------------------------------------------------------
% Ende der Befehlsdefinitionen.
%-----------------------------------------------------------------------





%%% Local Variables: 
%%% mode: latex
%%% TeX-master: t
%%% End: 

\begin{document}

\textheight 23cm
\topmargin -0.54cm
\textwidth 16.5cm
\oddsidemargin -0.04cm
\evensidemargin -0.54cm


\newgray{lightgray}{.97}
\newif\ifmethodgray
\methodgrayfalse
\def\method#1#2{%
  \ifmethodgray
    \psframebox*[framesep=1pt,fillcolor=lightgray,fillstyle=solid]{%
      %\begin{minipage}{.996\textwidth}
      \begin{minipage}{.873\textwidth}
        \begin{minipage}{17em} #1 \end{minipage}
        %\begin{minipage}{22.5em} #2 \end{minipage}
        \begin{minipage}{17.7em} #2 \end{minipage}
      \end{minipage}}
    \methodgrayfalse
  \else
    \psframebox*[framesep=1pt]{%
      %\begin{minipage}{.996\textwidth}
      \begin{minipage}{.873\textwidth}
        \begin{minipage}{17em} #1 \end{minipage}
        %\begin{minipage}{22.5em} #2 \end{minipage}
        \begin{minipage}{17.7em} #2 \end{minipage}
      \end{minipage}}
    \methodgraytrue
  \fi
}


\begin{center}
{\Large {\bf Quick Reference for the {\em EA}-Library}}\\
{\bf R\"udiger Alberts}\\
{\bf 2000-05-29}
\end{center}

\newpage
\tableofcontents
\newpage

\chapter{Preface}
This is a quick reference for all methods and variables
of the {\em EA}-library by Martin Kreutz. This library is
used for the programming of evolutionary algorithms.
More precisely, the data of the classes
{\tt Chromosome}, {\tt ChromosomeT\_num}, {\tt ChromosomeT\_int},
{\tt ChromosomeT\_double}, {\tt ChromosomeT\_char}, 
{\tt Chro\-mo\-someT\_bool}, {\tt Individual} and {\tt Population} 
are described.



\chapter{Class {\tt Chromosome}} 

% ######################################################################
\section{Public Methods}
% ######################################################################

% -----------------------------------------------------------------------
\subsection{Constructors and Destructor}
% -----------------------------------------------------------------------

\vspace*{10pt}

\methodgrayfalse

\parindent0mm
\hrule
\method{Method}{Description}\hrule
\method{
  \textsf{\textbf{\small ChromosomeT\_base}}( ) }{ 
  Generates an empty chromosome of arbitrary type and registers it.  
}\\
\method{
  \textsf{\textbf{\small ChromosomeT\_base}}( \emph{unsigned l} ) }{ 
  Generates and registers an empty chromosome of arbitrary type
  and reserves space for {\em l} alleles.
}\\
\method{
  \textsf{\textbf{\small ChromosomeT\_base}}( \emph{unsigned l,\\ 
  \hspace*{1.5em}const T\emph{\&} v} ) }{ 
  Generates and registers a chromosome of type {\sl T} 
  with {\em l} alleles with the value {\em v}.  
}\\
\method{
  \textsf{\textbf{\small ChromosomeT\_base}}(\\ 
  \hspace*{1.5em}\emph{const vector$<$ T $>$\emph{\&} v} ) }{
  Generates and registers a chromosome of type {\sl T} with
  the allele values {\em v}.
}\\
\method{
  \textsf{\textbf{\small $\sim$Chromosome}}( ) }{
  Destroys the current chromosome.
}\hrule

\newpage


% -----------------------------------------------------------------------
\subsection{Operators}
% -----------------------------------------------------------------------

%\vspace*{10pt}

\methodgrayfalse
\parindent0mm
\hrule
\method{Method}{Description}\hrule
\method{
  \emph{Chromosome\emph{\&}} \textsf{\textbf{\small operator =\ }}(\\ 
  \hspace*{1.5em}\emph{const Chromosome\emph{\&} c} ) }{
  Assigns the values of chromosome {\em c} to the current
  chromosome {\em this}.
}\\
\method{
  \emph{Chromosome\emph{\&}} \textsf{\textbf{\small operator =\ }}(\\ 
  \hspace*{1.5em}\emph{const vector$<$ T $>$\emph{\&} c} ) }{
  Assigns the values of vector {\em c} to the alleles of the
  current chromosome {\em this}.
}\\
\method{
  \emph{Chromosome\emph{\&}} \textsf{\textbf{\small operator =\ }}(\\ 
  \hspace*{1.5em}\emph{const T\emph{\&} c} ) }{
  Assigns the value {\em c} of type {\sl T} to all alleles of the
  current chromosome {\em this}.
}\\
\method{
  \emph{bool} \textsf{\textbf{\small operator ==\ }}(\\ 
  \hspace*{1.5em}\emph{const Chromosome\emph{\&} c} ) }{
  Checks, whether {\em this} is equal to {\em c}.
}\\
\method{
  \emph{bool} \textsf{\textbf{\small operator $<$\ }}(\\ 
  \hspace*{1.5em}\emph{const Chromosome\emph{\&} c} ) }{
  Checks whether {\em this} is less than {\em c}.
}\\
\method{
  \emph{bool} \textsf{\textbf{\small operator $<=$\ }}(\\ 
  \hspace*{1.5em}\emph{const Chromosome\emph{\&} c} ) }{
  Checks whether {\em this} is less than or equal to {\em c}.
}\\
\method{
  \emph{bool} \textsf{\textbf{\small operator $>$\ }}(\\ 
  \hspace*{1.5em}\emph{const Chromosome\emph{\&} c} ) }{
  Checks whether {\em this} is greater than {\em c}.
}\\
\method{
  \emph{bool} \textsf{\textbf{\small operator $>=$\ }}(\\
  \hspace*{1.5em}\emph{const Chromosome\emph{\&} c} ) }{
  Checks whether {\em this} is greater than or equal to {\em c}.
}\\
\method{
  \emph{bool} \textsf{\textbf{\small operator !$=$\ }}(\\ 
  \hspace*{1.5em}\emph{const Chromosome\emph{\&} c} ) }{
  Checks whether {\em this} and {\em c} are different.
}\\
\method{
  \emph{T} \textsf{\textbf{\small operator [\ ]\ }}(\\
  \hspace*{1.5em}\emph{unsigned allele} ) }{
  Returns allele number {\em allele} of type {\sl T}.
}\hrule


\vspace{10pt}

% -----------------------------------------------------------------------
\subsection{Information Retrieval Methods}
% -----------------------------------------------------------------------

\vspace*{10pt}

\methodgrayfalse
\parindent0mm
\hrule
\vspace*{1pt}
\method{Method}{Description}\hrule
\method{
  \emph{T} \textsf{\textbf{\small min}}( 
  \emph{T a, T b} ) }{
  Returns the smaller value of {\em a} and {\em b}.
}\\
\method{
  \emph{T} \textsf{\textbf{\small max}}( 
  \emph{T a, T b} ) }{
  Returns the greater value of {\em a} and {\em b}.
}\\
\method{
  \emph{bool} \textsf{\textbf{\small sameType}}(\\ 
  \hspace*{1.5em}\emph{const Chromosome\emph{\&} c} ) }{
  Checks whether {\em this} and {\em c} are of the same type.
}\\
\method{
  \emph{const char$\ast$} \textsf{\textbf{\small typeOfAlleles}}( ) }{
  Returns the type of the current chromosome.
}\\
\method{
  \emph{unsigned} \textsf{\textbf{\small sizeOfAlleles}}( ) }{
  Returns the size of one chromosome allele in bytes.
}\\
\method{
  \emph{unsigned} \textsf{\textbf{\small size}}( ) }{
  Returns the number of alleles contained in the chromosome.
}\hrule

\newpage


% -----------------------------------------------------------------------
\subsection{Structure Changing Methods}
% -----------------------------------------------------------------------

\vspace*{10pt}

\methodgrayfalse
\parindent0mm
\hrule
\method{Method}{Description}\hrule
\method{
  \emph{void} \textsf{\textbf{\small resize}}( 
  \emph{unsigned n} ) }{
  The number of alleles contained in {\em this} is set to
  the new value {\em n}. If $n < |this|$ then 
  $|this| - n$ alleles are removed from the end of 
  {\em this}. If $n > |this|$ then $n - |this|$
  new empty alleles are appended at the end of
  {\em this}.
}\\
\method{
  \emph{void} \textsf{\textbf{\small duplicate}}( 
  \emph{unsigned start,\\ 
  \hspace*{1.5em}unsigned stop, unsigned dest} ) }{
  Copies the alleles with indices in $[start,stop]$ to
  index {\em dest}.
}\\
\method{
  \emph{void} \textsf{\textbf{\small invert}}( 
  \emph{unsigned start,\\
  \hspace*{1.5em}unsigned stop,\\ 
  \hspace*{1.5em}unsigned granularity} ) }{
  Inverts the order of allele blocks of size {\em granularity} from
  index {\em start} to index {\em stop}.
}\\
\method{
  \emph{void} \textsf{\textbf{\small invert}}( 
  \emph{unsigned granularity} ) }{
  Same as above, but all alleles will be inverted.
}\\
\method{
  \emph{void} \textsf{\textbf{\small transcribe}}( 
  \emph{unsigned start,\\ 
  \hspace*{1.5em}unsigned stop,\\
  \hspace*{1.5em}const Chromosome\emph{\&} chrom} ) }{
  Copies the allele sequence from {\em start} to {\em stop}
  from chromosome {\em chrom} to {\em this}.
}\\
\method{
  \emph{void} \textsf{\textbf{\small swap}}( 
  \emph{unsigned i, unsigned j} ) }{
  Exchanges the alleles {\em i} and {\em j}.
}\\
\method{
  \emph{void} \textsf{\textbf{\small shuffle}}( ) }{
  Randomly changes the order of the alleles in {\em this}.
}\\
\method{
  \emph{void} \textsf{\textbf{\small replace}}( 
  \emph{unsigned i, const T\emph{\&} v} ) }{
  Assigns the new value {\em v} to allele number {\em i}.
}\\
\method{
  \emph{void} \textsf{\textbf{\small replace}}( 
  \emph{unsigned i,\\ 
  \hspace*{1.5em}const Chromosome\emph{\&} chrom} ) }{
  Replaces the alleles of {\em this} beginning at position {\em i} with
  the allele values of {\em chrom}.
}\\
\method{
  \emph{void} \textsf{\textbf{\small insert}}( 
  \emph{unsigned i,\\ 
  \hspace*{1.5em}const T\emph{\&} allele} ) }{
  Inserts value {\em allele} at position {\em i}.
}\\
\method{
  \emph{void} \textsf{\textbf{\small insert}}( 
  \emph{unsigned i,\\ 
  \hspace*{1.5em}const Chromosome\emph{\&} chrom} ) }{
  Inserts the values of {\em chrom} at position {\em i} in {\em this}.
}\\
\method{
  \emph{void} \textsf{\textbf{\small append}}( 
  \emph{const T\emph{\&} allele} ) }{
  Appends value {\em allele} to the end of {\em this}.
}\\
\method{
  \emph{void} \textsf{\textbf{\small append}}(\\ 
  \hspace*{1.5em}\emph{const Chromosome\emph{\&} chrom} ) }{
  Appends the values of {\em chrom} to the end of {\em this}.
}\\
\method{
  \emph{void} \textsf{\textbf{\small remove}}( 
  \emph{unsigned i} ) }{
  Removes allele number {\em i}.
}\\
\method{
  \emph{void} \textsf{\textbf{\small remove}}( 
  \emph{unsigned i, unsigned k} ) }{
  Removes all alleles from index {\em i} to index {\em k}.
}\\
\method{
  \emph{void} \textsf{\textbf{\small rotateRight}}( 
  \emph{unsigned n} ) }{
  Rotates all alleles by {\em n} position to the right.
}\\
\method{
  \emph{void} \textsf{\textbf{\small rotateLeft}}( 
  \emph{unsigned n} ) }{
    Rotates all alleles by {\em n} position to the left.
}\hrule


\newpage


% -----------------------------------------------------------------------
\subsection{Crossover Methods}
% -----------------------------------------------------------------------

\vspace*{10pt}

\methodgrayfalse
\parindent0mm
\hrule
\method{Method}{Description}\hrule
\method{
  \emph{void} \textsf{\textbf{\small crossover}}(\\
  \hspace*{1.5em}\emph{const Chromosome\emph{\&} dadChrom,\\ 
  \hspace*{1.5em}const Chromosome\emph{\&} momChrom,\\
  \hspace*{1.5em}const vector$<$unsigned$>$\emph{\&} points} ) }{
  Generates a new {\em this} by recombining {\em dadChrom}
  and {\em momChrom} at the positions marked by the vector {\em points}.
}\\
\method{
  \emph{void} \textsf{\textbf{\small crossover}}( 
  \emph{Chromosome\emph{\&} mate,\\ 
  \hspace*{1.5em}const vector$<$unsigned$>$\emph{\&} points} ) }{
  Recombines {\em this} and {\em mate} at the positions 
  marked by the vector {\em points}.
}\\
\method{%
  \emph{void} \textsf{\textbf{\small crossover}}(\\ 
  \hspace*{1.5em}\emph{const Chromosome\emph{\&} dadChrom,\\
  \hspace*{1.5em}const Chromosome\emph{\&} momChrom,\\ 
  \hspace*{1.5em}const vector$<$ bool $>$\emph{\&} pos} ) }{
  Generates a new {\em this} by recombining {\em dadChrom} and
  {\em momChrom} at the positions marked as ``true''
  in vector {\em pos}.
}\\
\method{
  \emph{void} \textsf{\textbf{\small crossover}}(\\
  \hspace*{1.5em}\emph{Chromosome\emph{\&} mate,\\
  \hspace*{1.7em}const vector$<$ bool $>$\emph{\&} pos} ) }{
  Recombines {\em this} and {\em mate} at the positions
  marked as ``true'' in {\em pos}.
}\\
\method{
  \emph{void} \textsf{\textbf{\small crossover}}(\\
  \hspace*{1.5em}\emph{const Chromosome\emph{\&} dadChrom,\\
  \hspace*{1.5em}const Chromosome\emph{\&} momChrom,\\
  \hspace*{1.5em}unsigned npoints, unsigned align} ) }{
  Randomly recombines {\em dadChrom} and {\em momChrom} in {\em this}.
  At most {\em npoints} crossover-positions are possible.
  Between these positions there will be the minimum distance {\em align}.
}\\
\method{
  \emph{void} \textsf{\textbf{\small crossover}}(\\ 
  \hspace*{1.5em}\emph{Chromosome\emph{\&} mate,\\
  \hspace*{1.5em}unsigned npoints, unsigned align} ) }{
  Randomly recombines {\em this} and {\em mate}.
  At most {\em npoints} crossover-positions are possible.
  Between these positions there will be the minimum distance {\em align}.
}\\
\method{%
  \emph{void} \textsf{\textbf{\small crossover}}(\\
  \hspace*{1.5em}\emph{const Chromosome\emph{\&} dadChrom,\\ 
  \hspace*{1.5em}const Chromosome\emph{\&} momChrom,\\
  \hspace*{1.5em}const Chromosome\emph{\&} posChrom} ) }{
  Recombines {\em dadChrom} and {\em momChrom} in {\em this} at
  the points marked as ``1'' in {\em posChrom}.
}\\
\method{
  \emph{void} \textsf{\textbf{\small crossover}}(\\
  \hspace*{1.5em}\emph{Chromosome\emph{\&} mate,\\ 
  \hspace*{1.5em}const Chromosome\emph{\&} posChrom} ) }{
  Recombines {\em this} and {\em mate} at the points
  marked as ``1'' in {\em posChrom}.
}\\
\method{
  \emph{void} \textsf{\textbf{\small crossoverUniform}}(\\
  \hspace*{1.5em}\emph{const Chromosome\emph{\&} dadChrom,\\ 
  \hspace*{1.5em}const Chromosome\emph{\&} momChrom,\\ 
  \hspace*{1.5em}const vector$<$ bool $>$\emph{\&} pos} ) }{
  Generates a new {\em this} by recombining {\em dadChrom} and
  {\em momChrom}. Position $pos_i=$\ ``true'' $\Rightarrow this_i= 
  momChrom_i$, $pos_i=$\ ``false'' $\Rightarrow this_i= 
  dadChrom_i$ (with $i\ =\ 0..n-1$, $n\ =\ |this|$).
}\\
\method{
  \emph{void} \textsf{\textbf{\small crossoverUniform}}(\\
  \hspace*{1.5em}\emph{Chromosome\emph{\&} mateChrom,\\
  \hspace*{1.5em}const vector$<$ bool $>$\emph{\&} pos} ) }{
  Crossover of {\em this} and {\em mate}. 
  Position $pos_i=$\ ``true'' $\Rightarrow this_i= 
  mateChrom_i$, $pos_i=$\ ``false'' $\Rightarrow this_i= 
  this_i$ (with $i\ =\ 0..n-1$, $n\ =\ |this|$).
}\\
\method{
  \emph{void} \textsf{\textbf{\small crossoverUniform}}(\\
  \hspace*{1.5em}\emph{const Chromosome\emph{\&} dadChrom,\\ 
  \hspace*{1.5em}const Chromosome\emph{\&} momChrom} ) }{
  Randomly recombines {\em dadChrom} and {\em momChrom} in
  {\em this} with a probability of $0.5$ per parent
  and allele.  
}\hrule

\newpage

\methodgrayfalse
\parindent0mm
\hrule
\method{Method}{Description}\hrule
\method{
  \emph{void} \textsf{\textbf{\small crossoverUniform}}(\\
  \hspace*{1.5em}\emph{Chromosome\emph{\&} mateChrom} ) }{
  Randomly recombines {\em this} and {\em mate} with a
  probability of $0.5$ per parent and allele.
}\\
\method{
  \emph{void} \textsf{\textbf{\small crossoverUniform}}(\\
  \hspace*{1.5em}\emph{const Chromosome\emph{\&} dadChrom,\\ 
  \hspace*{1.5em}const Chromosome\emph{\&} momChrom,\\
  \hspace*{1.5em}const Chromosome\emph{\&} posChrom} ) }{
  Crossover of {\em dadChrom} and {\em momChrom} in {\em this}.
  Position $posChrom_i==$ 0? $\Rightarrow this_i= 
  dadChrom_i$, $posChrom_i<>$ 0? $\Rightarrow this_i= 
  momChrom_i$ (with $i\ =\ 0..n-1$, $n\ =\ |this|$).
}\\
\method{
  \emph{void} \textsf{\textbf{\small crossoverUniform}}(\\
  \hspace*{1.5em}\emph{Chromosome\emph{\&} mateChrom,\\ 
  \hspace*{1.5em}const Chromosome\emph{\&} posChrom} ) }{
  Crossover of {\em this} and {\em mateChrom}. 
  Position $posChrom_i==$ 0? $\Rightarrow this_i= 
  mateChrom_i$, $posChrom_i<>$ 0? $\Rightarrow this_i= 
  this_i$ (with $i\ =\ 0..n-1$, $n\ =\ |this|$).
}\\
\method{
  \emph{void} \textsf{\textbf{\small recombineDiscrete}}(\\
  \hspace*{1.5em}\emph{const Chromosome\emph{\&} dadChrom,\\ 
  \hspace*{1.5em}const Chromosome\emph{\&} momChrom} ) }{
  Randomly recombines {\em dadChrom} and {\em momChrom}
  in {\em this} with a probability of $0.5$ per
  parent and allele.
}\\
\method{
  \emph{void} \textsf{\textbf{\small recombineDiscrete}}(\\
  \hspace*{1.5em}\emph{Chromosome\emph{\&} mate} ) }{
  Randomly recombines {\em mate} and {\em this} with
  a probability of $0.5$ per parent and allele.
}\hrule

\vspace{10pt}


% -----------------------------------------------------------------------
\subsection{Input Method}
% -----------------------------------------------------------------------

\vspace*{10pt}

\methodgrayfalse
\parindent0mm
\hrule
\method{Method}{Description}\hrule{\hspace{\linewidth}}\\
\method{%
  \emph{static Chromosome$\ast$} \textsf{\textbf{\small readNewFrom}}(\\
  \hspace*{1.5em}\emph{istream\emph{\&} is} ) }{
  Reads new data into {\em this} from {\em is} and registers the new {\em this}.
}\\\hrule


\newpage

% ######################################################################
\section{Private Methods}
% ######################################################################

% -----------------------------------------------------------------------
\subsection{Generating Methods}
% -----------------------------------------------------------------------



\methodgrayfalse
\parindent0mm
\hrule
%\vspace*{1pt}
\method{Method}{Description}\hrule{\hspace{\linewidth}}\\
\method{
  \emph{Chromosome$\ast$} \textsf{\textbf{\small clone}}( ) }{
  Returns a copy of the current chromosome.
}\\
\method{
  \emph{Chromosome$\ast$} \textsf{\textbf{\small empty}}( ) }{
  Returns a new empty chromosome.
}\hrule

\vspace{10pt}

% -----------------------------------------------------------------------
\subsection{Input and Output Methods}
% -----------------------------------------------------------------------

\vspace*{10pt}

\methodgrayfalse
\parindent0mm
\hrule
%\vspace*{1pt}
\method{Method}{Description}\hrule
\method{
  \emph{void} \textsf{\textbf{\small writeDataTo}}( 
  \emph{ostream\emph{\&} os} ) }{
  Writes chromosome {\em this} to output stream {\em os}.
}\\
\method{
  \emph{void} \textsf{\textbf{\small writeTo}}( 
  \emph{ostream\emph{\&} os} ) }{
  Writes chromosome {\em this} including type and size to output
  stream {\em os}.
}\\
\method{
  \emph{void} \textsf{\textbf{\small readDataFrom}}( 
  \emph{istream\emph{\&} is} ) }{
  Reads new allele values into {\em this} from {\em is}.
}\\
\method{
  \emph{void} \textsf{\textbf{\small readFrom}}( 
  \emph{istream\emph{\&} is} ) }{
  Reads new allele values, size and type into {\em this} from
  {\em is}.
}\hrule


% ######################################################################
% ######################################################################
% ######################################################################
% ######################################################################

\newpage

\chapter{Class {\tt ChromosomeT\_num}}

% ######################################################################
\section{Public Methods}
% ######################################################################

% -----------------------------------------------------------------------
\subsection{Initialization and Adaptation Methods}
% -----------------------------------------------------------------------

\vspace*{10pt}

\methodgrayfalse

\parindent0mm
\hrule
\method{Method}{Description}\hrule
\method{
  \emph{void} \textsf{\textbf{\small initialize}}( \emph{T min, T max} ) }{ 
  Initializes all alleles with random numbers ranging from {\em min}
  to {\em max}.
}\\
\method{
  \emph{void} \textsf{\textbf{\small initialize}}(\\
  \hspace*{1.5em}\emph{const Chromosome\emph{\&} minChrom,\\
  \hspace*{1.5em}const Chromosome\emph{\&} maxChrom} ) }{ 
  Initializes $this_i$ with \nolinebreak $i\ =\ 0\ ..\ n-1$ 
  \linebreak with a random value ranging from
  $minChrom_i$ to $maxChrom_i$ ($n\ =\ $ number of alleles
  in {\em this}).
}\\
\method{
  \emph{void} \textsf{\textbf{\small cutOff}}(
  \emph{T min, T max} ) }{
  Adapts all allele values in the range {\em min} to {\em max} to the
  nearest range limit.
}\\
\method{
  \emph{void} \textsf{\textbf{\small cutOff}}(\\
  \hspace*{1.5em}\emph{const Chromosome\emph{\&} minChrom,\\
  \hspace*{1.5em}const Chromosome\emph{\&} maxChrom} ) }{
  Adapts $this_i$ with $i\ =\ 0\ ..\ n-1, this_i$ less than $minChrom_i$ or
  $this_i$ greater than $maxChrom_i$ to the nearest range limit 
  ($n\ =\ $ number of alleles in {\em this}).
}\hrule

\vspace{10pt}


% -----------------------------------------------------------------------
\subsection{Mutation Methods}
% -----------------------------------------------------------------------

\vspace*{10pt}

\methodgrayfalse
\parindent0mm
\hrule
\method{Method}{Description}\hrule
\method{
  \emph{void} \textsf{\textbf{\small mutateUniform}}( 
  \emph{T min,\\
  \hspace*{1.5em}T max,\\
  \hspace*{1.5em}const vector$<$ double $>$\emph{\&} p,\\
  \hspace*{1.5em}bool cycle = ``false''} ) }{
  Replaces $this_i$ ($i\ =\ 0\ ..\ n-1$, $n\ =\ $ number of alleles
  in {\em this}) with probability $p_i$
  with a random value ranging from $min_i$ to $max_i$. 
  Is {\em cycle} set to ``true'', {\em p} can be used
  circular.
}\\
\method{
  \emph{void} \textsf{\textbf{\small mutateUniform}}( 
  \emph{T min,\\
  \hspace*{1.5em}T max,\\
  \hspace*{1.5em}const Chromosome\emph{\&} p,\\
  \hspace*{1.5em}bool cycle = ``false''} ) }{
  Replaces $this_i$ ($i\ =\ 0\ ..\ n-1$, $n\ =\ $ number of
  alleles in {\em this}) with probability $p_i$
  with a random value ranging from $min_i$ to $max_i$. 
  Is {\em cycle} set to ``true'', {\em p} can be used
  circular.
}\hrule

\newpage

\methodgrayfalse
\parindent0mm
\hrule
\method{Method}{Description}\hrule
\method{
  \emph{void} \textsf{\textbf{\small mutateUniform}}(\\
  \hspace*{1.5em}\emph{const Chromosome\emph{\&} minChrom,\\
  \hspace*{1.5em}const Chromosome\emph{\&} maxChrom,\\
  \hspace*{1.5em}double p} ) }{
  Replaces $this_i$ with \nolinebreak $i\ =\ 0\ ..\ n-1$ 
  \linebreak with probability {\em p} with a random value ranging from 
  $minChrom_i$ to $maxChrom_i$ ($n\ =\ $ number of alleles in
  {\em this}).
}\\
\method{
  \emph{void} \textsf{\textbf{\small mutateUniform}}(\\
  \hspace*{1.5em}\emph{const Chromosome\emph{\&} minChrom,\\
  \hspace*{1.5em}const Chromosome\emph{\&} maxChrom,\\
  \hspace*{1.5em}const vector$<$ double $>$\emph{\&} p,\\
  \hspace*{1.5em}bool cycle = ``false''} ) }{
  Replaces $this_i$ ($i\ =\ 0\ ..\ n-1$, $n\ =\ $ number of
  alleles in {\em this}) with probability $p_i$
  with a random value ranging from $minChrom_i$ to $maxChrom_i$. 
  Is {\em cycle} set to ``true'', {\em p} can be used
  circular.
}\\
\method{
  \emph{void} \textsf{\textbf{\small mutateUniform}}(\\
  \hspace*{1.5em}\emph{const Chromosome\emph{\&} min,\\
  \hspace*{1.5em}const Chromosome\emph{\&} max,\\
  \hspace*{1.5em}const Chromosome\emph{\&} p,\\
  \hspace*{1.5em}bool cycle = ``false''} ) }{
  Replaces $this_i$ ($i\ =\ 0\ ..\ n-1$, $n\ =\ $ number of
  alleles in {\em this}) with probability $p_i$
  with a random value ranging from $min_i$ to $max_i$. 
  Is {\em cycle} set to ``true'', {\em p} can be used
  circular.
}\hrule


% ######################################################################
% ######################################################################
% ######################################################################
% ######################################################################

\newpage

\chapter{Class {\tt ChromosomeT$<$ bool $>$}}


% ######################################################################
\section{Public Methods}
% ######################################################################

% -----------------------------------------------------------------------
\subsection{Initialization Methods}
% -----------------------------------------------------------------------

\vspace*{10pt}

\methodgrayfalse

\parindent0mm
\hrule
\method{Method}{Description}\hrule
%\vspace*{1pt}
\method{
  \emph{void} \textsf{\textbf{\small initialize}}( ) }{ 
  Randomly initializes all alleles with ``true'' or ``false''.
}\hrule

\vspace*{10pt}


% -----------------------------------------------------------------------
\subsection{Encoding and Decoding Methods}
% -----------------------------------------------------------------------

\vspace*{10pt}

\methodgrayfalse

\parindent0mm
\hrule
\method{Method}{Description}\hrule
\method{
  \emph{void} \textsf{\textbf{\small encode}}( \emph{double val,\\
  \hspace*{1.5em}const Interval\emph{\&} range,\\
  \hspace*{1.5em}unsigned nbits,\\
  \hspace*{1.5em}bool useGray = ``false''} ) }{ 
  Encodes {\em val} out of the interval {\em range} into a
  bitstring of length {\em nbits} ({\em useGray} = ``true''
  $\Rightarrow$ using Gray Encoding, {\em useGray} = ``false''
  $\Rightarrow$ using Standard Encoding) and stores it in {\em this}.
}\\
\method{
  \emph{void} \textsf{\textbf{\small encodeBinary}}(\\
  \hspace*{1.5em}\emph{const vector$<$ double $>$\emph{\&} chrom,\\
  \hspace*{1.5em}const Interval\emph{\&} range,\\
  \hspace*{1.5em}unsigned nbits,\\
  \hspace*{1.5em}bool useGray = ``false''} ) }{ 
  Encodes all numbers from {\em chrom} out of the interval
  {\em range} ({\em useGray} = ``true''
  $\Rightarrow$ using Gray Encoding, {\em useGray} = ``false''
  $\Rightarrow$ using Standard Encoding) into a bitstring
  and stores it in {\em this}. Each single number will be encoded
  in {\em nbits} Bits.
}\\
\method{
  \emph{void} \textsf{\textbf{\small encodeBinary}}(\\
  \hspace*{1.5em}\emph{const Chromosome\emph{\&} chrom,\\
  \hspace*{1.5em}const Interval\emph{\&} range,\\
  \hspace*{1.5em}unsigned nbits,\\
  \hspace*{1.5em}bool useGray = ``false''} ) }{ 
  Encodes all numbers from {\em chrom} out of the interval
  {\em range} ({\em useGray} = ``true''
  $\Rightarrow$ using Gray Encoding, {\em useGray} = ``false''
  $\Rightarrow$ using Standard Encoding) into a bitstring
  and stores it in {\em this}. Each single number will be encoded
  in {\em nbits} Bits.
}\hrule

\newpage

\parindent0mm
\hrule
\method{Method}{Description}\hrule
\method{
  \emph{double} \textsf{\textbf{\small decode}}(\\
  \hspace*{1.5em}\emph{const Interval\emph{\&} range,\\
  \hspace*{1.5em}bool useGray = ``false''} ) }{ 
  Decodes the bitstring in {\em this} and returns the decoded number.
  Parameter description see above.
}\hrule

\vspace*{10pt}


% -----------------------------------------------------------------------
\subsection{Mutation Methods}
% -----------------------------------------------------------------------

\vspace*{10pt}

\methodgrayfalse

\parindent0mm
\hrule
\method{Method}{Description}\hrule
%\vspace*{1pt}
\method{
  \emph{void} \textsf{\textbf{\small flip}}(
  \emph{double p} ) }{ 
  Each allele value will be inverted with probability {\em p}.
}\\
\method{
  \emph{void} \textsf{\textbf{\small flip}}(
  \emph{const vector$<$ double $>$\emph{\&} p\\
  \hspace*{1.5em}bool cycle = ``false''} ) }{ 
  $this_i$ ($i\ =\ 0..n-1$, $n\ =\ $ number of alleles in {\em this}) 
  will be inverted 
  with probability $p_i$. If {\em cycle} = ``true'' then {\em p} can be 
  used circular.
}\\
\method{
  \emph{void} \textsf{\textbf{\small flip}}(
  \emph{const Chromosome\emph{\&} p\\
  \hspace*{1.5em}bool cycle = ``false''} ) }{ 
  $this_i$ ($i\ =\ 0..n-1$, $n\ =\ $ number of alleles in {\em this}) 
  will be inverted 
  with probability $p_i$. If {\em cycle} = ``true'' then {\em p} can be 
  used circular.
}\hrule


% ######################################################################
% ######################################################################
% ######################################################################
% ######################################################################

\newpage

\chapter{Class {\tt ChromosomeT$<$ int $>$}}


% ######################################################################
\section{Public Methods}
% ######################################################################

% -----------------------------------------------------------------------
\subsection{Mutation Methods}
% -----------------------------------------------------------------------

\vspace*{10pt}

\methodgrayfalse

\parindent0mm
\hrule
%\vspace{1pt}
\method{Method}{Description}\hrule
\method{
  \emph{void} \textsf{\textbf{\small mutateDiffGeom}}( 
  \emph{double s}) }{ 
  Adds a random number to the value of each allele, using 
  stepsize {\em s} for a special random number distribution
  (refer to detailed reference).
}\\
\method{
  \emph{void} \textsf{\textbf{\small mutateDiffGeom}}(\\ 
  \hspace*{1.5em}\emph{const vector$<$ double $>$\emph{\&} s,\\
  \hspace*{1.5em}bool cycle = ``false''} ) }{ 
  Adds a random number to the value of each allele
  $this_i$ $(i\ =\ 0..n-1$, $n\ =\ $ number of alleles in {\em this}$)$ 
  using a special random number
  distribution (refer to detailed reference) with stepsize $s_i$. 
  Is {\em cycle} set to ``true'', {\em s} can be used
  circular.
}\\
\method{
  \emph{void} \textsf{\textbf{\small mutateDiffGeom}}(\\ 
  \hspace*{1.5em}\emph{const ChromosomeT$<$ double $>$\emph{\&} s,\\
  \hspace*{1.5em}bool cycle = ``false''} ) }{ 
  Adds a random number to the value of each allele
  $this_i$ $(i\ =\ 0..n-1$, $n\ =\ $ number of alleles in {\em this}) 
  using a special random number
  distribution (refer to detailed reference) with stepsize $s_i$. 
  Is {\em cycle} set to ``true'', {\em s} can be used
  circular.
}\\
\method{
  \emph{void} \textsf{\textbf{\small mutateDiffGeom}}(\\ 
  \hspace*{1.5em}\emph{const Chromosome\emph{\&} s,\\
  \hspace*{1.5em}bool cycle = ``false''} ) }{ 
  Adds a random number to the value of each allele
  $this_i$ $(i\ =\ 0..n-1$, $n\ =\ $ number of alleles in {\em this})
  using a special random number
  distribution (refer to detailed reference) with stepsize $s_i$. 
  Is {\em cycle} set to ``true'', {\em s} can be used
  circular.
}\hrule


% ######################################################################
% ######################################################################
% ######################################################################
% ######################################################################

\newpage

\chapter{Class {\tt ChromosomeT$<$ double $>$}}


% ######################################################################
\section{Internal Class {\tt DerandomConst}}
% ######################################################################

% -----------------------------------------------------------------------
\subsection{Variables of internal class {\tt DerandomConst}}
% -----------------------------------------------------------------------

\vspace*{10pt}

\methodgrayfalse

\parindent0mm
\hrule
%\vspace*{1pt}
\method{Variable}{Description}\hrule
\method{
  \textsf{\textbf{\small dim}}}{ 
  Dimension of {\em this}.
}\\
\method{
  \textsf{\textbf{\small C}}}{ 
  Period of cumulation, initialized with $\frac{1.0}{\sqrt{n}}$.
}\\
\method{
  \textsf{\textbf{\small Cu}}}{ 
  Normalization factor for the sum vector {\em s}, 
  initialized with $\sqrt{\frac{2.0 - C}{C}}$.
}\\
\method{
  \textsf{\textbf{\small Crr}}}{ 
  Period of cumulation for the preference direction, initialized
  with $\frac{3.0}{n}$. 
}\\
\method{
  \textsf{\textbf{\small Beta}}}{ 
  Damping of the global stepsize adaptation, initialized with
  $\frac{1.0}{\sqrt{n}}$.
}\\
\method{
  \textsf{\textbf{\small BetaI}}}{ 
  Damping of the individual stepsize adaptation, initialized with
  $\frac{1}{n}$.
}\\
\method{
  \textsf{\textbf{\small BetaR}}}{ 
  Damping of the direction, initialized with $\frac{1.0}{\sqrt{4n}}$. 
}\\
\method{
  \textsf{\textbf{\small ChiN}}}{ 
  Expectation value of the $\chi_n$-distribution, initialized with
  $\sqrt{n}(1 - \frac{1}{4n} + \frac{1}{21n^2})$.
}\\
\method{
  \textsf{\textbf{\small Chi1}}}{ 
  Expectation value of the $\chi_1$-distribution, initialized with
  $\sqrt{\frac{2}{\pi}}$.
}\hrule

\newpage

% -----------------------------------------------------------------------
\subsection{Methods of internal class {\tt DerandomConst}}
% -----------------------------------------------------------------------

\vspace*{10pt}

\methodgrayfalse
\parindent0mm
\hrule
\method{Method}{Description}\hrule
\method{
  \textsf{\textbf{\small DerandomConst}}( 
  \emph{unsigned n} ) }{
  Initializes all internal variables. {\em dim} will be set
  to {\em n}.
}\\
\method{
  \emph{unsigned} \textsf{\textbf{\small nobj}}( ) }{
  Returns the dimension {\em dim}.
}\\
\method{
  \emph{unsigned} \textsf{\textbf{\small npar}}( ) }{
  Returns the number of necessary parameters for the {\em Derandom}
  mutation method.
}\\\hrule


\newpage

% ######################################################################
\section{Public Methods}
% ######################################################################


% -----------------------------------------------------------------------
\subsection{Initialization Methods}
% -----------------------------------------------------------------------

\vspace*{10pt}

\methodgrayfalse
\parindent0mm
\hrule
%\vspace*{1pt}
\method{Method}{Description}\hrule
\method{
  \emph{void} \textsf{\textbf{\small initializeRotate}}(\\ 
  \hspace*{1.5em}\emph{double SigmaMin,\\
  \hspace*{1.5em}double SigmaMax} ) }{
  Initializes the alleles of {\em this} for the method
  {\em mutateRotate} (see below) with random numbers ranging from
  {\em SigmaMin} to {\em SigmaMax}.   
}\\
\method{
  \emph{void} \textsf{\textbf{\small initializeRotate}}(\\ 
  \hspace*{1.5em}\emph{const vector$<$ double $>$ \emph{\&} sigma} ) }{
  Initializes the alleles of {\em this} for the method 
  {\em mutateRotate} (see below) with the values stored in {\em sigma}.
}\\
\method{
  \emph{void} \textsf{\textbf{\small initializeRotate}}(\\ 
  \hspace*{1.5em}\emph{const ChromosomeT$<$ double $>$\emph{\&}\\
  \hspace*{2.5em}sigma} ) }{
  Initializes the alleles of {\em this} for the method 
  {\em mutateRotate} (see below) with the values stored in {\em sigma}.
}\\
\method{
  \emph{void} \textsf{\textbf{\small initializeDerandom}}(\\
  \hspace*{1.5em}\emph{double minVal,\\ 
  \hspace*{1.5em}double maxVal} ) }{
  Initializes the alleles of {\em this} for the method
  {\em mutateDerandom} (see below) with random numbers ranging from
  {\em minVal} to {\em maxVal}. 
}\\
\method{
  \emph{void} \textsf{\textbf{\small initializeCMA}}(\\
  \hspace*{1.5em}\emph{double SigmaMin,\\
  \hspace*{1.5em}double SigmaMax} ) }{
  Initializes the alleles of {\em this} for the method {\em mutateCMA} 
  (see below) with random numbers. A random number ranging from
  {\em SigmaMin} to {\em SigmaMax} will be assigned to the first
  allele of {\em this}.
}\\
\method{
  \emph{void} \textsf{\textbf{\small initializeCMA}}(\\ 
  \hspace*{1.5em}\emph{const vector$<$ double $>$ \emph{\&} sigma} ) }{
  Initializes the alleles of {\em this} for the method {\em mutateCMA}
  (see below) using the values stored in {\em sigma}.
}\\
\method{
  \emph{void} \textsf{\textbf{\small initializeCMA}}(\\ 
  \hspace*{1.5em}\emph{const ChromosomeT$<$ double $>$\emph{\&}\\
  \hspace*{2.5em}sigma} ) }{
  Initializes the alleles of {\em this} for the method {\em mutateCMA}
  (see below) using the values stored in {\em sigma}.
}\\
\method{
  \emph{void} \textsf{\textbf{\small initializeGSA}}(\\
  \hspace*{1.5em}\emph{double SigmaMin,\\
  \hspace*{1.5em}double SigmaMax} ) }{
  Initializes the alleles of {\em this} for the method {\em mutateGSA} 
  (see below) with random numbers. A random number ranging from
  {\em SigmaMin} to {\em SigmaMax} will be assigned to the first
  allele of {\em this}.
}\\
\method{
  \emph{void} \textsf{\textbf{\small initializeGSA}}(\\ 
  \hspace*{1.5em}\emph{const vector$<$ double $>$ \emph{\&} sigma,\\
  \hspace*{1.5em}int baseSize} ) }{
  Initializes the alleles of {\em this} for the method {\em mutateGSA}
  (see below) with the values stored in {\em sigma}. {\em baseSize}
  denotes the number of vectors of the generating set.
}\\
\method{
  \emph{void} \textsf{\textbf{\small initializeGSA}}(\\ 
  \hspace*{1.5em}\emph{const ChromosomeT$<$ double $>$\emph{\&}\\
  \hspace*{2.5em}sigma,\\
  \hspace*{1.5em}int baseSize} ) }{
  Initializes the alleles of {\em this} for the method {\em mutateGSA}
  (see below) with the values stored in {\em sigma}.
  {\em baseSize} denotes the number of vectors of the generating set.
}\hrule

\newpage

\methodgrayfalse
\parindent0mm
\hrule
%\vspace*{1pt}
\method{Method}{Description}\hrule
\method{
  \emph{void} \textsf{\textbf{\small initializeIDA}}(\\
  \hspace*{1.5em}\emph{double SigmaMin,\\
  \hspace*{1.5em}double SigmaMax} ) }{
  Initializes the alleles of {\em this} for the method {\em mutateIDA}
  (see below) with random numbers. Several alleles will be initialized
  with random numbers ranging from {\em SigmaMin} to {\em SigmaMax}.
}\\
\method{
  \emph{void} \textsf{\textbf{\small initializeIDA}}(\\
  \hspace*{1.5em}\emph{const vector$<$ double $>$ \emph{\&} sigma} ) }{
  Initializes the alleles of {\em this} for the method {\em mutateIDA}
  (see below) with the values stored in {\em sigma}.
}\\
\method{
  \emph{void} \textsf{\textbf{\small initializeIDA}}(\\
  \hspace*{1.5em}\emph{const ChromosomeT$<$ double $>$\emph{\&}\\
  \hspace*{2.5em}sigma} ) }{
  Initializes the alleles of {\em this} for the method {\em mutateIDA}
  (see below) with the values stored in {\em sigma}.
}\\
\method{
  \emph{void} \textsf{\textbf{\small initializeIDAiso}}(\\
  \hspace*{1.5em}\emph{double SigmaMin,\\
  \hspace*{1.5em}double SigmaMax} ) }{
  Initializes the alleles of {\em this} for the method {\em mutateIDAiso}
  (see below) with random numbers. Several alleles will be initialized
  with random numbers ranging from {\em SigmaMin} to {\em SigmaMax}.
}\\
\method{
  \emph{void} \textsf{\textbf{\small initializeIDAiso}}(\\
  \hspace*{1.5em}\emph{const vector$<$ double $>$ \emph{\&} sigma} ) }{
  Initializes the alleles of {\em this} for the method {\em mutateIDAiso}
  (see below) with the values stored in {\em sigma}.
}\\
\method{
  \emph{void} \textsf{\textbf{\small initializeIDAiso}}(\\
  \hspace*{1.5em}\emph{const ChromosomeT$<$ double $>$\emph{\&}\\
  \hspace*{2.5em}sigma} ) }{
  Initializes the alleles of {\em this} for the method {\em mutateIDAiso}
  (see below) with the values stored in {\em sigma}.
}\hrule

\vspace*{10pt}

% -----------------------------------------------------------------------
\subsection{Decoding Method}
% -----------------------------------------------------------------------

\vspace*{10pt}

\methodgrayfalse
\parindent0mm
\hrule
\method{Method}{Description}\hrule
\method{
  \emph{void} \textsf{\textbf{\small decodeBinary}}(\\
  \hspace*{1.5em}\emph{const Chromosome\emph{\&} chrom,\\
  \hspace*{1.5em}const Interval\emph{\&} range,\\
  \hspace*{1.5em}bool useGray = ``false''} ) }{
  Decodes the bitstring {\em chrom} that contains a value out of
  the interval {\em range} encoded in {\em nbits} Bits
  ({\em use\-Gray} = ``true'' $\Rightarrow$ used the Gray Encoding,
   {\em useGray} = ``false'' $\Rightarrow$ used the Standard Encoding).
  The decoded value is stored in {\em this}.
}\hrule

\newpage

% -----------------------------------------------------------------------
\subsection{Cumulation Methods}
% -----------------------------------------------------------------------

\vspace*{10pt}

\methodgrayfalse
\parindent0mm
\hrule
\method{Method}{Description}\hrule
\method{
  \emph{void} \textsf{\textbf{\small accumulate}}(\\
  \hspace*{1.5em}\emph{const vector$<$ double $>$ \emph{\&} acc,\\
  \hspace*{1.5em}double c} ) }{
  Adds $acc_i$ (with $i\ =\ 0\ ..\ n-1$) with the weight $c_i$ 
  to $this_i$ ($n\ =\ $ number of alleles in {\em this}).
}\\
\method{
  \emph{void} \textsf{\textbf{\small accumulate}}(\\
  \hspace*{1.5em}\emph{const Chromosome \emph{\&} acc,\\
  \hspace*{1.5em}double c} ) }{
  Adds $acc_i$ (with $i\ =\ 0\ ..\ n-1$) with the weight $c_i$ to 
  $this_i$ ($n\ =\ $ number of alleles in {\em this}).
}\hrule

\vspace*{10pt}


% -----------------------------------------------------------------------
\subsection{Mutation Methods}
% -----------------------------------------------------------------------

\vspace*{10pt}

\methodgrayfalse
\parindent0mm
\hrule
%\vspace*{1pt}
\method{Method}{Description}\hrule
\method{
  \emph{void} \textsf{\textbf{\small mutateNormal}}(
  \emph{double stddev} ) }{
  Adds a normally distributed random number ranging from 0 to
  {\em stddev} to all alleles.
}\\
\method{
  \emph{void} \textsf{\textbf{\small mutateNormal}}(\\
  \hspace*{1.5em}\emph{const vector$<$ double $>$ \emph{\&} stddev,\\
  \hspace*{1.5em}bool cycle} ) }{
  Adds a normally distributed random number, ranging from 0 to
  $stddev_i$ (with $i\ =\ 0..n-1$) to $this_i$ 
  ($n\ =\ $ number of alleles in {\em this}). 
  If {\em cycle} is set to ``true'', {\em stddev} can be used
  circular.
}\\
\method{
  \emph{void} \textsf{\textbf{\small mutateNormal}}(\\
  \hspace*{1.5em}\emph{const Chromosome \emph{\&} stddev,\\
  \hspace*{1.5em}bool cycle} ) }{
  Adds a normally distributed random number ranging from 0 to
  $stddev_i$ (with $i\ =\ 0..n-1$) to $this_i$
  ($n\ =\ $ number of alleles in {\em this}). 
  If {\em cycle} is
  set to ``true'', {\em stddev} can be used circular.
}\\
\method{
  \emph{void} \textsf{\textbf{\small mutateNormal}}(\\
  \hspace*{1.5em}\emph{const ChromosomeT$<$ double $>$\emph{\&}\\
  \hspace*{2.5em} stddev,\\
  \hspace*{1.5em}bool cycle} ) }{
  Adds a normally distributed random number ranging from 0 to
  $stddev_i$ (with $i\ =\ 0..n-1$) to $this_i$. 
  ($n\ =\ $ number of alleles in {\em this}). 
  If {\em cycle} is
  set to ``true'', {\em stddev} can be used circular.
}\\
\method{
  \emph{void} \textsf{\textbf{\small mutateCauchy}}(
  \emph{double scale} ) }{
  Adds Cauchy random numbers scaled with {\em scale} to all
  alleles.
}\\
\method{
  \emph{void} \textsf{\textbf{\small mutateCauchy}}(\\
  \hspace*{1.5em}\emph{const vector$<$ double $>$\emph{\&} scale,\\
  \hspace*{1.5em}bool cycle = ``false''} ) }{
  Adds Cauchy random numbers scaled with $scale_i$ (with i\ =\ 0..n-1)
  to $this_i$
  ($n\ =\ $ number of alleles in {\em this}). 
  If {\em cycle} is set to ``true'', {\em scale} can be
  used circular.
}\\
\method{
  \emph{void} \textsf{\textbf{\small mutateCauchy}}(\\
  \hspace*{1.5em}\emph{const Chromosome\emph{\&} scale,\\
  \hspace*{1.5em}bool cycle = ``false''} ) }{
  Adds Cauchy random numbers scaled with $scale_i$ (with i\ =\ 0..n-1)
  to $this_i$
  ($n\ =\ $ number of alleles in {\em this}). 
  If {\em cycle} is set to ``true'', {\em scale} can be
  used circular.
}\hrule

\newpage

\methodgrayfalse
\parindent0mm
\hrule
\method{Method}{Description}\hrule
\method{
  \emph{void} \textsf{\textbf{\small mutateCauchy}}(\\
  \hspace*{1.5em}\emph{const ChromosomeT$<$ double $>$\emph{\&}\\
  \hspace*{2.5em}scale,
  \hspace*{1.5em}bool cycle = ``false''} ) }{
  Adds Cauchy random numbers scaled with $scale_i$ (with i\ =\ 0..n-1)
  to $this_i$
  ($n\ =\ $ number of alleles in {\em this}). 
  If {\em cycle} is set to ``true'', {\em scale} can be
  used circular.
}\hrule


\vspace*{10pt}

% -----------------------------------------------------------------------
\subsection{Recombination Methods}
% -----------------------------------------------------------------------

\vspace*{10pt}

\methodgrayfalse
\parindent0mm
\hrule
%\vspace*{1pt}
\method{Method}{Description}\hrule
\method{
  \emph{void} \textsf{\textbf{\small recombineIntermediate}}(\\
  \hspace*{1.5em}\emph{const Chromosome\emph{\&} dadChrom,\\
  \hspace*{1.5em}const Chromosome\emph{\&} momChrom} ) }{
  Assigns the average of $dadChrom_i$ and $momChrom_i$ to 
  $this_i$ (with $i\ =\ 0\ ..\ n-1$, $n\ =\ $ number of alleles 
  in {\em this}). 
}\\
\method{
  \emph{void} \textsf{\textbf{\small recombineGenIntermediate}}(\\
  \hspace*{1.5em}\emph{const Chromosome\emph{\&} dadChrom,\\
  \hspace*{1.5em}const Chromosome\emph{\&} momChrom} ) }{
  Assigns an equally distributed random number ranging from 
  $dadChrom_i$ to $momChrom_i$ to $this_i$ 
  (with $i\ =\ 0\ ..\ n-1$, $n\ =\ $ number of alleles in {\em this}). 
}\\
\method{
  \emph{void} \textsf{\textbf{\small recombineGeomIntermediate}}(\\
  \hspace*{1.5em}\emph{const Chromosome\emph{\&} dadChrom,\\
  \hspace*{1.5em}const Chromosome\emph{\&} momChrom} ) }{
  Assigns the value $\sqrt{dadChrom_i \cdot momChrom_i}$
  (with $i\ =\ 0\ ..\ n-1$) to $this_i$
  ($n\ =\ $ number of alleles in {\em this}). 
}\\
\method{
  \emph{void} \textsf{\textbf{\small recombineIntermediate}}(\\
  \hspace*{1.5em}\emph{Chromosome\emph{\&} mate} ) }{
  Assigns the average of $mate_i$ und $this_{i_{old}}$
  (with $i\ =\ 0\ ..\ n-1$) to $this_i$
  ($n\ =\ $ number of alleles in {\em this}). 
}\\
\method{
  \emph{void} \textsf{\textbf{\small recombineGenIntermediate}}(\\
  \hspace*{1.5em}\emph{Chromosome\emph{\&} mate} ) }{
  Assigns equally distributed random numbers ranging from
  $this_{i_{old}}$ to $mate_{i_{old}}$ (with $i\ =\ 0\ ..\ n-1$)
  to $this_i$ and $mate_i$ ($n\ =\ $ number of alleles in {\em this}). 
}\\
\method{
  \emph{void} \textsf{\textbf{\small recombineGeomIntermediate}}(\\
  \hspace*{1.5em}\emph{Chromosome\emph{\&} mate} ) }{
  Assigns the values $\sqrt{this_{i_{old}} \cdot mate_{i_{old}}}$
  (with $i\ =\ 0\ ..\ n-1$) to $this_i$ and $mate_i$
  ($n\ =\ $ number of alleles in {\em this}). 
}\hrule

\newpage

% -----------------------------------------------------------------------
\subsection{Self Adaptation Methods}
% -----------------------------------------------------------------------

\vspace*{10pt}

\methodgrayfalse
\parindent0mm
\hrule
\method{Method}{Description}\hrule
\method{
  \emph{void} \textsf{\textbf{\small mutateLogNormal}}(\\
  \hspace*{1.5em}\emph{double overallStdDev,\\
  \hspace*{1.5em}double indivStdDev} ) }{
  Assigns a normally distributed random number ranging from
  0 to $sigma_i^2$ (with $i\ =\ 0\ ..\ n-1$,
  $n\ =\ $ number of alleles in {\em this}), to $this_i$.
  $sigma_i$ will be mutated for each allele separately (parameter
  {\em indivStdDev}) and for all alleles together (parameter
  {\em overallStdDev}). 
}\\
\method{
  \emph{void} \textsf{\textbf{\small mutateRotate}}(\\
  \hspace*{1.5em}\emph{ChromosomeT$<$ double $>$\emph{\&}\\
  \hspace*{2.5em}sigma} ) }{
  Mutates {\em this} and the strategy variables using a rotation
  matrix with the standard deviations stored in {\em sigma}.
}\\
\method{
  \emph{void} \textsf{\textbf{\small mutateRotate}}(\\
  \hspace*{1.5em}\emph{ChromosomeT$<$ double $>$\emph{\&} sigma,\\
  \hspace*{1.5em}double tau1, double tau2,\\
  \hspace*{1.5em}double beta, int sigmaCheck\\
  \hspace*{1.5em}double epsi} ) }{
  Mutates {\em this} and the strategy variables using a rotation
  matrix with the standard deviations for the normal distribution
  stored in {\em sigma}, the stepsize adaptation for all individuals
  ({\em tau1}) and for single individuals ({\em tau2}), the damping 
  {\em beta} of the stepsize variation between successive generations and
  the lower boundary {\em epsi} for the $\sigma$-values. {\em sigmaCheck}
  denotes, whether {\em sigma} is checked for too small values (``true'')
  or not (``false''). 
}\\
\method{
  \emph{void} \textsf{\textbf{\small mutateNormalRotAngles}}(\\
  \hspace*{1.5em}\emph{const Chromosome\emph{\&} sigma,\\
  \hspace*{1.5em}const Chromosome\emph{\&} alpha} ) }{
  Mutates {\em this} and the strategy variables using a rotation matrix
  with the standard deviations for the normal distribution stored in
  {\em sigma}, and the angle points {\em alpha} for the rotations.
}\\
\method{
  \emph{void} \textsf{\textbf{\small mutateDerandom}}(\\
  \hspace*{1.5em}\emph{vector$<$ double $>$\emph{\&} v,\\
  \hspace*{1.5em}const DerandomConst\emph{\&} K} ) }{
  Mutates {\em this} using the {\em Derandom} method with the parameter
  values {\em v} and the constants {\em K}.
}\\
\method{
  \emph{void} \textsf{\textbf{\small mutateDerandom}}(\\
  \hspace*{1.5em}\emph{Chromosome\emph{\&} chrom,\\
  \hspace*{1.5em}const DerandomConst\emph{\&} K} ) }{
  Mutates {\em this} using the {\em Derandom} method with the
  parameter values {\em chrom} and the constants {\em K}.
}\hrule

\newpage

\methodgrayfalse
\parindent0mm
\hrule
\method{Method}{Description}\hrule
\method{
  \emph{void} \textsf{\textbf{\small mutateCMA}}(\\
  \hspace*{1.5em}\emph{ChromosomeT$<$ double $>$\emph{\&}\\
  \hspace*{2.5em}sigma} ) }{
  Mutates {\em this} and the strategy parameters using the 
  {\em Covariance Matrix Adaptation} with the standard deviations
  {\em sigma} for the normal distribution.
}\\
\method{
  \emph{void} \textsf{\textbf{\small mutateCMA}}(\\
  \hspace*{1.5em}\emph{ChromosomeT$<$ double $>$ \emph{\&}\\
  \hspace*{2.5em}sigma,\\
  \hspace*{1.5em}double c, double cu,\\
  \hspace*{1.5em}double ccov, double beta} ) }{
  Mutates {\em this} and the strategy parameters using the
  {\em Covariance Matrix Adaptation} with the standard deviations
  {\em sigma} for the normal distribution, the value {\em c} for the
  determination of the cumulation time for the sum vector {\em s}, the
  normalization {\em cu} of the variance of the sum vector, the time
  {\em ccov} for the determination of the average of the distribution
  $ss^t$ over the generation sequence and the parameter {\em beta} for
  the damping of the stepsize variation between successive
  generations.
}\\
\method{
  \emph{void} \textsf{\textbf{\small mutateMSR}}(
  \emph{double xi\_prob} ) }{
  Mutates {\em this} and the strategy parameters using the method
  of {\em mutative stepsize regulation}. 
  The value $\alpha = 1.5$ will be assigned to the general stepsize 
  changing factor $\xi$ with probability
  {\em xi\_prob}.
}\\
\method{
  \emph{void} \textsf{\textbf{\small mutateGSA}}(\\
  \hspace*{1.5em}\emph{ChromosomeT$<$ double $>$\emph{\&}\\
  \hspace*{2.5em}sigma} ) }{
  Mutates {\em this} and the strategy parameters using the
  {\em Generating Set Adaptation} with the standard deviations
  {\em sigma} for the normal distribution.
}\hrule

\newpage

\methodgrayfalse
\parindent0mm
\hrule
\method{Method}{Description}\hrule
\method{
  \emph{void} \textsf{\textbf{\small mutateGSA}}(\\
  \hspace*{1.5em}\emph{ChromosomeT$<$ double $>$\emph{\&}\\
  \hspace*{2.5em}sigma\\
  \hspace*{1.5em}double beta, double xi\_const,\\
  \hspace*{1.5em}double cu, double cm} ) }{
  Mutates {\em this} and the strategy parameters using the
  {\em Generating Set Adaptation} with the standard deviations
  {\em sigma} for the normal distribution, the parameter {\em beta}
  for the damping of the stepsize variation between successive
  generations, the factor {\em xi\_const} for the
  stepsize adaptation, the normalization factor {\em cu} for the
  variances and the length adaptation factor {\em cm}.
}\\
\method{
  \emph{void} \textsf{\textbf{\small mutateIDA}}(\\
  \hspace*{1.5em}\emph{ChromosomeT$<$ double $>$\emph{\&}\\
  \hspace*{2.5em}sigma} ) }{
  Mutates {\em this} and the strategy parameters using the 
  {\em Individual step sizes and one Direction Adaptation}
  with the standard deviations {\em sigma} for the normal
  distribution.
}\\
\method{
  \emph{void} \textsf{\textbf{\small mutateIDA}}(\\
  \hspace*{1.5em}\emph{ChromosomeT$<$ double $>$\emph{\&}\\
  \hspace*{2.5em}sigma\\
  \hspace*{1.5em}double c, double c\_r\\
  \hspace*{1.5em}double beta, double beta\_ind,\\
  \hspace*{1.5em}double beta\_r, double cu,\\
  \hspace*{1.5em}double xi} ) }{
  Mutates {\em this} and the strategy parameters using the
  {\em Individual step sizes and one Direction Adaptation}
  with the standard deviations {\em sigma} for the normal
  distribution, the cumulation time {\em c\_r}, the parameter
  for the damping of the stepsize variation between successive
  generations for all individuals ({\em beta}), for a single
  individual ({\em beta\_ind}), and for the preference direction
  ({\em beta\_r}). Additionally, the normalization
  factor {\em cu} for variances and the factor {\em xi} for the 
  stepsize adaptation is used.
}\\
\method{
  \emph{void} \textsf{\textbf{\small mutateIDAiso}}(\\
  \hspace*{1.5em}\emph{ChromosomeT$<$ double $>$\emph{\&}\\
  \hspace*{2.5em}sigma} ) }{
  Mutates {\em this} and the strategy parameters using the 
  {\em Individual step sizes and one Direction Adaptation (isotropic)}
  method with the standard deviations {\em sigma} for the normal
  distribution.
}\hrule

\newpage

\methodgrayfalse
\parindent0mm
\hrule
\method{Method}{Description}\hrule
\method{
  \emph{void} \textsf{\textbf{\small mutateIDAiso}}(\\
  \hspace*{1.5em}\emph{ChromosomeT$<$ double $>$\emph{\&}\\
  \hspace*{2.5em}sigma\\
  \hspace*{1.5em}double c, double beta,\\
  \hspace*{1.5em}double beta\_ind, double cu} ) }{
  Mutates {\em this} and the strategy parameters using the
  {\em Individual step sizes and one Direction Adaptation (isotropic)}
  method with the standard deviations {\em sigma} for the normal
  distribution, the parameter for the damping of the stepsize variation
  between successive generations for all individuals ({\em beta}), for
  a single individual ({\em beta\_ind}) and the normalization
  factor {\em cu} for variances.
}\hrule

\vspace*{10pt}

% -----------------------------------------------------------------------
\subsection{Methods for Displaying Parameters}
% -----------------------------------------------------------------------

\vspace*{10pt}

\methodgrayfalse
\parindent0mm
\hrule
\method{Method}{Description}\hrule
\method{
  \emph{void} \textsf{\textbf{\small showRotate}}( ) }{
  Displays the parameters {\em sigma} and {\em alpha} of the self
  adaptation method {\em mutateRotate} (see above). 
}\\
\method{
  \emph{void} \textsf{\textbf{\small showCMA}}( ) }{
  Displays the parameters of the self adaptation method {\em mutateCMA} 
  (see above).
}\\
\method{
  \emph{void} \textsf{\textbf{\small showIDA}}( ) }{
  Displays the parameters of the self adaptation method {\em mutateIDA} 
  (see above).
}\\
\method{
  \emph{void} \textsf{\textbf{\small showGSA}}( 
  \emph{int baseSize} ) }{
  Displays the parameters of the self adaptation method {\em mutateGSA} 
  (see above) with {\em baseSize} vectors of the generating set.
}\\
\method{
  \emph{void} \textsf{\textbf{\small dumpDerandom}}( 
  \emph{ostream\emph{\&} os} ) }{
  Writes the parameters of the {\em Derandom}
  self adaptation method (method see above) to the output stream {\em os}.
}\hrule


% ######################################################################
% ######################################################################
% ######################################################################
% ######################################################################

\newpage

\chapter{Class {\tt Individual}}

\section{Internal Variables and Flags}

\vspace*{10pt}

\methodgrayfalse

\parindent0mm
\hrule
%\vspace*{1pt}
\method{Variable/Flag}{Description}\hrule
\method{
  \textsf{\textbf{\small fitness}}}{ 
  Result of the evaluation of the individual, initialized with 0.
}\\
\method{
  \textsf{\textbf{\small scaledFitness}}}{ 
  Scaled fitness -- a positive value, that will result in
  the sum 1.0, when added together with all other
  fitness values. This variable will be initialized with 1.0.
}\\
\method{
  \textsf{\textbf{\small age}}}{ 
  Age of the individual, initialized with 0.
}\\
\method{
  \textsf{\textbf{\small selProb}}}{ 
  Selection probability of the individual, will be initialized
  with 0.
}\\
\method{
  \textsf{\textbf{\small numCopies}}}{ 
  Number of reproductions of the individual during the
  last selection, will be initialized with 0.
}\\
\method{
  \textsf{\textbf{\small evalFlg}}}{ 
  If this flag is set to ``true'', the fitness of the individual
  must be evaluated. The flag will be initialized with ``false''.
}\\
\method{
  \textsf{\textbf{\small feasible}}}{ 
  If the flag is set to ``true'', the individual is a possible
  solution for the optimization problem. The flag will be initialized
  with ``false''.
}\\
\method{
  \textsf{\textbf{\small elitist}}}{ 
  If the flag is set to ``true'', the individual was selected as an
  elitist during the last selection. The flag will be initialized
  with ``false''.
}\hrule

\newpage

% ######################################################################
\section{Public Methods}
% ######################################################################

\subsection{Constructors and Destructor}

\vspace*{10pt}

\methodgrayfalse

\parindent0mm
\hrule
\method{Method}{Description}\hrule
%\vspace*{1pt}
\method{
  \textsf{\textbf{\small Individual}}( )}{ 
  Generates an empty individual.
}\\
\method{
  \emph{explicit} \textsf{\textbf{\small Individual}}( \emph{unsigned n} )}{ 
  Generates an individual with {\em n} chromosomes of type {\tt char}.
}\\
\method{
  \textsf{\textbf{\small Individual}}( \emph{unsigned n,\\
  \hspace*{1.5em}const Chromosome\emph{\&} chrom} )}{ 
  Generates an individual with {\em n} clones of {\em chrom}.
}\\
\method{
  \textsf{\textbf{\small Individual}}(\\
  \hspace*{1.5em}\emph{const Chromosome\emph{\&} chrom0} )}{ 
  Generates an individual with the chromosome {\em chrom0}.
}\\
\method{
  \textsf{\textbf{\small Individual}}(\\
  \hspace*{1.5em}\emph{const Chromosome\emph{\&}\\
  \hspace*{2.5em}chrom0, ..., chrom\emph{[1 .. 7]}} )}{ 
  Generates an individual consisting of clones of 
  {\em chrom0} to $chrom_i$ (with $i\ =\ 1\ ..\ 7$).
}\\
\method{
  \textsf{\textbf{\small Individual}}(\\
  \hspace*{1.5em}\emph{const vector$<$Chromosome$\ast>$\emph{\&}\\
  \hspace*{2.5em}chrom} 
  )}{ 
  Generates an individual consisting of clones of the
  chromosomes in {\em chrom}.
}\\
\method{
  \textsf{\textbf{\small Individual}}(\\
  \hspace*{1.5em}\emph{const Individual\emph{\&} indiv} )}{ 
  Generates an individual as an exact copy of {\em indiv}.
}\\
\method{
  \emph{$\sim$} \textsf{\textbf{\small Individual}}( )}{ 
  Removes all individuals of {\em this} and then destroys
  {\em this} itself.
}\\\hrule

\vspace*{10pt}

\subsection{Operators}

%\vspace*{10pt}

\methodgrayfalse

\parindent0mm
\hrule
%\vspace*{1pt}
\method{Method}{Description}\hrule
\method{
  \emph{Individual\emph{\&}} \textsf{\textbf{\small operator\ =\ }}(\\
  \hspace*{1.5em}\emph{const Individual\emph{\&} indiv} )}{ 
  Assigns the values of {\em indiv} to {\em this}.
}\\
\method{
  \emph{bool} \textsf{\textbf{\small operator\ ==\ }}(\\
  \hspace*{1.5em}\emph{const Individual\emph{\&} ind} )}{ 
  Checks whether {\em this} and {\em ind} are equal.
}\\
\method{
  \emph{bool} \textsf{\textbf{\small operator\ $<$\ }}(\\
  \hspace*{1.5em}\emph{const Individual\emph{\&} ind} )}{ 
  Checks whether {\em this} is less than {\em ind}.
}\\
\method{
  \emph{bool} \textsf{\textbf{\small operator\ $>$\ }}(\\
  \hspace*{1.5em}\emph{const Individual\emph{\&} ind} )}{ 
  Checks whether {\em this} is greater than {\em ind}.
}\\
\method{
  \emph{bool} \textsf{\textbf{\small operator\ $<=$\ }}(\\
  \hspace*{1.5em}\emph{const Individual\emph{\&} ind} )}{ 
  Checks whether {\em this} is less than or equal to
  {\em ind}.
}\\
\method{
  \emph{bool} \textsf{\textbf{\small operator\ $>=$\ }}(\\
  \hspace*{1.5em}\emph{const Individual\emph{\&} ind} )}{ 
  Checks whether {\em this} is greater than or equal to
  {\em ind}.
}\\
\method{
  \emph{bool} \textsf{\textbf{\small operator\ !=\ }}(\\
  \hspace*{1.5em}\emph{const Individual\emph{\&} ind} )}{ 
  Checks whether {\em this} and {\em ind} are different.
}\\
\method{
  \emph{Chromosome\emph{\&}} \textsf{\textbf{\small operator\ [\ ]\ }}(\\
  \hspace*{1.5em}\emph{unsigned i} )}{ 
  Returns the chromosome with index {\em i}.
}\hrule

\newpage

\subsection{Information Retrieval Methods}

\vspace*{10pt}

\methodgrayfalse

\parindent0mm
\hrule
%\vspace*{1pt}
\method{Method}{Description}\hrule
\method{
  \emph{unsigned} \textsf{\textbf{\small size}}( ) }{
  Returns the number of chromosomes in {\em this}.
}\\
\method{
  \emph{unsigned} \textsf{\textbf{\small totalSize}}( ) }{
  Returns the total number of all chromosome alleles in {\em this}.
}\\
\method{
  \emph{double} \textsf{\textbf{\small fitnessValue}}( ) }{
  Returns the fitness value of {\em this}.
}\\
\method{
  \emph{unsigned} \textsf{\textbf{\small getAge}}( ) }{
  Returns the age of {\em this}.
}\\
\method{
  \emph{bool} \textsf{\textbf{\small needEvaluation}}( ) }{
  Returns ``true'' when {\em this} needs evaluation.
}\\
\method{
  \emph{double} \textsf{\textbf{\small selectionProbability}}( ) }{
  Returns the selection probability of {\em this}.
}\\
\method{
  \emph{bool} \textsf{\textbf{\small isFeasible}}( ) }{
  Returns ``true'' when {\em this} is a possible solution
  for the optimization problem.
}\\
\method{
  \emph{unsigned} \textsf{\textbf{\small numberOfCopies}}( ) }{
  Denotes, how often {\em this} was reproduced during the last
  selection.
}\\
\method{
  \emph{bool} \textsf{\textbf{\small isElitist}}( ) }{
  Returns ``true'' when {\em this} was selected as an elitist 
  during the last selection.
}\hrule

\vspace*{10pt}

\subsection{Methods for Changing Internal Variables and Flags}

\vspace*{10pt}

\methodgrayfalse


\parindent0mm
\hrule
%\vspace*{1pt}
\method{Method}{Description}\hrule
\method{
  \emph{void} \textsf{\textbf{\small setFitness}}( 
  \emph{double fit} ) }{
  Sets the normal and scaled fitness of {\em this} to {\em fit}.
}\\
\method{
  \emph{void} \textsf{\textbf{\small setAge}}(
  \emph{unsigned a} ) }{
  Sets the age of {\em this} to {\em a}.
}\\
\method{
  \emph{void} \textsf{\textbf{\small incAge}}( ) }{
  Increments the age of {\em this} by 1.
}\\
\method{
  \emph{void} \textsf{\textbf{\small setEvaluationFlag}}( ) }{
  Sets {\em evalFlg} to ``true''.
}\\
\method{
  \emph{void} \textsf{\textbf{\small clearEvaluationFlag}}( ) }{
  Sets {\em evalFlg} to ``false''.
}\\
\method{
  \emph{void} \textsf{\textbf{\small setFeasible}}(
  \emph{bool f} ) }{
  Sets the {\em feasible}-flag to {\em f}.
}\\
\method{
  \emph{void} \textsf{\textbf{\small setSelectionProbability}}(\\
  \hspace*{1.5em}\emph{double ps} ) }{
  Sets the selection probability of the individual to {\em ps}.
}\\\hrule

\newpage

\subsection{Structure Modifying Methods}

\vspace*{10pt}

\methodgrayfalse


\parindent0mm
\hrule
%\vspace*{1pt}
\method{Method}{Description}\hrule
\method{
  \emph{void} \textsf{\textbf{\small replace}}( 
  \emph{unsigned i,\\ 
  \hspace*{1.5em}const Chromosome\emph{\&} chrom} ) }{
  Replaces chromosome number {\em i} of {\em this} with {\em chrom}.
}\\
\method{
  \emph{void} \textsf{\textbf{\small insert}}( 
  \emph{unsigned i,\\ 
  \hspace*{1.5em}const Chromosome\emph{\&} chrom} ) }{
  Inserts {\em chrom} at position {\em i} into {\em this}.
}\\
\method{
  \emph{void} \textsf{\textbf{\small append}}(\\
  \hspace*{1.5em}\emph{const Chromosome\emph{\&} chrom} ) }{
  Appends {\em chrom} at the end of {\em this}.
}\\
\method{
  \emph{void} \textsf{\textbf{\small remove}}( 
  \emph{unsigned i} ) }{
  Removes chromosome number {\em i} from {\em this}.
}\\
\method{
  \emph{void} \textsf{\textbf{\small remove}}( 
  \emph{unsigned from,\\ 
  \hspace*{1.5em}unsigned to} ) }{
  Removes all chromosomes with indices in the range $[from, to]$
  from {\em this}.
}\hrule

\vspace*{10pt}

\subsection{Input- and Output Methods}

\vspace*{10pt}

\methodgrayfalse


\parindent0mm
\hrule
%\vspace*{1pt}
\method{Method}{Description}\hrule
\method{
  \emph{void} \textsf{\textbf{\small readFrom}}(
  \emph{istream\emph{\&} is} ) }{ 
  The data of {\em this} will be replaced by the data read from
  input stream {\em is}. 
}\\
\method{
  \emph{void} \textsf{\textbf{\small writeTo}}(
  \emph{ostream\emph{\&} os} ) }{ 
  Writes the data of {\em this} to the output stream {\em os}.
}\hrule


% ######################################################################
% ######################################################################
% ######################################################################
% ######################################################################

\newpage

\chapter{Class {\tt Population}}

\section{Internal Variables and Flags}

\vspace*{10pt}

\methodgrayfalse

\parindent0mm
\hrule
%\vspace*{1pt}
\method{Variable/Flag}{Description}\hrule
\method{
  \textsf{\textbf{\small index}}}{ 
  Index of the individual with the best fitness value.
}\\
\method{
  \textsf{\textbf{\small subPop}}}{ 
  Is set to ``true'' if the population is a subpopulation.
  This flag defaults to ``false''.
}\\
\method{
  \textsf{\textbf{\small ascending}}}{ 
  If ``true'' all individuals are sorted
  by ascending fitness values. If the flag is set to its
  default value ``false'', the individuals are sorted by
  descending fitness values.
}\\
\method{
  \textsf{\textbf{\small spinOnce}}}{ 
  Flag for the selection me\-thod {\em se\-lectRouletteWheel}.
  If the  flag is set to its default value ``true'', the wheel will
  only spin once, ``false'' denotes multiple spins.
}\hrule

\newpage


% ######################################################################
\section{Public Methods}
% ######################################################################

\subsection{Constructors}

\vspace*{10pt}

\methodgrayfalse

\parindent0mm
\hrule
\method{Method}{Description}\hrule
\method{
  \textsf{\textbf{\small Population}}( )}{ 
  Generates an empty population.
}\\
\method{
  \emph{explicit} \textsf{\textbf{\small Population}}(
  \emph{unsigned n} )}{ 
  Generates a new population with {\em n} empty individuals.
}\\
\method{
  \textsf{\textbf{\small Population}}(\\
  \hspace*{1.5em}\emph{const Individual\emph{\&} indiv} )}{ 
  Generates a new population with the individual {\em indiv}.
}\\
\method{
  \textsf{\textbf{\small Population}}(
  \emph{unsigned n,\\ 
  \hspace*{1.5em}const Individual\emph{\&} indiv} )}{ 
  Generates a new population with {\em n} clones of {\em indiv}.
}\\
\method{
  \textsf{\textbf{\small Population}}(
  \emph{unsigned n,\\ 
  \hspace*{1.5em}const Chromosome\emph{\&} chrom0} )}{ 
  Generates a new population with {\em n} individuals with each consisting
  of a clone of {\em chrom0}.
}\\
\method{
  \textsf{\textbf{\small Population}}(
  \emph{unsigned n,\\ 
  \hspace*{1.5em}const Chromosome\emph{\&}\\ 
  \hspace*{2.5em}chrom0, ..., chrom\emph{[1 .. 7]}} )}{ 
  Generates a new population with {\em n} individuals with each consisting
  of a clone of {\em chrom0} to $chrom_i$ (with $i\ =\ 1\ ..\ 7$).
}\\
\method{
  \textsf{\textbf{\small Population}}(
  \emph{unsigned n,\\ 
  \hspace*{1.5em}const vector$<$ Chromosome $\ast$ $>$\emph{\&}\\ 
  \hspace*{2.5em}chrom} )}{ 
  Generates a new population with {\em n} individuals with each 
  consisting of clones of the chromosomes in {\em chrom}.
}\\
\method{
  \textsf{\textbf{\small Population}}(\\
  \hspace*{1.5em}\emph{const Population\emph{\&} pop} )}{ 
  Generates a new population, that consists of a copy of {\em pop}.
}\hrule

\newpage

\subsection{Destructor}

\vspace*{10pt}

\methodgrayfalse

\parindent0mm
\hrule
\method{Method}{Description}\hrule
\method{
  \emph{$\sim$}\textsf{\textbf{\small Population}}( )}{ 
  If the flag {\em subPop} is set to ``false'', the population and all
  its individuals will be destroyed. If {\em subPop} is set to ``true'',
  the population will persist.
}\hrule

\subsection{Operators}

\vspace*{10pt}

\methodgrayfalse

\parindent0mm
\hrule
%\vspace*{1pt}
\method{Method}{Description}\hrule
\method{
  \emph{Population\emph{\&}} \textsf{\textbf{\small operator\ =\ }}(\\
  \hspace*{1.5em}\emph{const Individual\emph{\&} indiv} )}{ 
  Assigns the data of {\em indiv} to all individuals of {\em this}.
}\\
\method{
  \emph{Population\emph{\&}} \textsf{\textbf{\small operator\ =\ }}(\\
  \hspace*{1.5em}\emph{const Population\emph{\&} pop} )}{ 
  Assigns the data of the population {\em pop} to {\em this}.
}\\

\method{
  \emph{bool} \textsf{\textbf{\small operator\ ==\ }}(\\
  \hspace*{1.5em}\emph{const Population\emph{\&} pop} )}{ 
  Checks whether {\em this} and {\em pop} are equal.
}\\
\method{
  \emph{bool} \textsf{\textbf{\small operator\ $<$\ }}(\\
  \hspace*{1.5em}\emph{const Population\emph{\&} pop} )}{ 
  Checks whether {\em this} is less than {\em pop}.
}\\
\method{
  \emph{bool} \textsf{\textbf{\small operator\ $>$\ }}(\\
  \hspace*{1.5em}\emph{const Population\emph{\&} pop} )}{ 
  Checks whether {\em this} is greater than {\em pop}.
}\\
\method{
  \emph{bool} \textsf{\textbf{\small operator\ $<=$\ }}(\\
  \hspace*{1.5em}\emph{const Population\emph{\&} pop} )}{ 
  Checks whether {\em this} is less than or equal to {\em pop}.
}\\
\method{
  \emph{bool} \textsf{\textbf{\small operator\ $>=$\ }}(\\
  \hspace*{1.5em}\emph{const Population\emph{\&} pop} )}{ 
  Checks whether {\em this} is greater than or equal to
  {\em pop}.
}\\
\method{
  \emph{bool} \textsf{\textbf{\small operator\ !=\ }}(\\
  \hspace*{1.5em}\emph{const Population\emph{\&} pop} )}{ 
  Checks whether {\em this} and {\em pop} are different.
}\\
\method{
  \emph{Individual\emph{\&}} \textsf{\textbf{\small operator\ [\ ]\ }}(\\
  \hspace*{1.5em}\emph{unsigned i} )}{ 
  Returns the individual with index {\em i}.
}\\
\method{
  \emph{Population\emph{\&}} \textsf{\textbf{\small operator\ (\ )\ }}(\\
  \hspace*{1.5em}\emph{unsigned from, unsigned to} )}{ 
  Returns a subpopulation of {\em this}, consisting of the individuals with 
  indices {\em from} to {\em to}.
}\hrule

\newpage

\subsection{Information Retrieval Methods}

\vspace*{10pt}

\methodgrayfalse

\parindent0mm
\hrule
%\vspace*{1pt}
\method{Method}{Description}\hrule
\method{
  \emph{int} \textsf{\textbf{\small size}}( ) }{
  Returns the number of individuals in {\em this}.
}\\
\method{
  \emph{bool} \textsf{\textbf{\small ascendingFitness}}( ) }{
  Returns ``true'' if the individuals in {\em this} are sorted
  by ascending fitness values.
}\\
\method{
  \emph{double} \textsf{\textbf{\small minFitness}}( ) }{
  Returns the minimum fitness value of {\em this}.
}\\
\method{
  \emph{double} \textsf{\textbf{\small maxFitness}}( ) }{
  Returns the maximum fitness value of {\em this}.
}\\
\method{
  \emph{double} \textsf{\textbf{\small meanFitness}}( ) }{
  Returns the average fitness value of {\em this}.
}\\
\method{
  \emph{double} \textsf{\textbf{\small stdDevFitness}}( ) }{
  Returns the standard deviation of all fitness values of {\em this}.
}\\
\method{
  \emph{unsigned} \textsf{\textbf{\small bestIndex}}( ) }{
  Returns the index of the individual with the ``best'' fitness value
  (= {\em ascending}-flag == ``true''? $\Longrightarrow$ lowest fitness, 
   {\em ascending}-flag == ``false''? $\Longrightarrow$ highest
   fitness).
}\\
\method{
  \emph{unsigned} \textsf{\textbf{\small worstIndex}}( ) }{
  Returns the index of the individual with the ``worst'' fitness value
  (= {\em ascending}-flag == ``true''? $\Longrightarrow$ highest
   fitness, {\em ascending}-flag == ``false''? $\Longrightarrow$ lowest
   fitness).
}\\
\method{
  \emph{Individual\emph{\&}} \textsf{\textbf{\small best}}( ) }{
  Returns the individual with the ``best'' fitness value according to
  the criteria of the method {\em bestIndex} (see above).
}\\
\method{
  \emph{Individual\emph{\&}} \textsf{\textbf{\small worst}}( ) }{
  Returns the individual with the worst fitness value according to
  the criteria of the method {\em worstIndex} (see above).
}\\\hrule

\vspace*{10pt}

\subsection{Methods for Modifying Internal Variables and Flags}

\vspace*{10pt}

\methodgrayfalse


\parindent0mm
\hrule
\method{Method}{Description}\hrule
\method{
  \emph{void} \textsf{\textbf{\small setMinimize}}( ) }{
  Sets the {\em ascending}-flag to ``true''. 
}\\
\method{
  \emph{void} \textsf{\textbf{\small setMaximize}}( ) }{
  Sets the {\em ascending}-flag to ``false''. 
}\\
\method{
  \emph{void} \textsf{\textbf{\small spinWheelOneTime}}( ) }{
  Sets the {\em spinOnce}-flag to ``true''.
}\\
\method{
  \emph{void} \textsf{\textbf{\small spinWheelMultipleTimes}}( 
  \emph{} ) }{
  Sets the {\em spinOnce}-flag to ``false''.
}\\
\method{
  \emph{void} \textsf{\textbf{\small setAge}}( 
  \emph{unsigned a} ) }{
  Sets the age of all individuals to {\em a}.
}\\
\method{
  \emph{void} \textsf{\textbf{\small incAge}}( 
  \emph{} ) }{
  Increments the age of all individuals by 1.
}\hrule


\newpage

\subsection{Structure Modifying Methods}

\vspace*{10pt}

\methodgrayfalse

\parindent0mm
\hrule
%\vspace*{1pt}
\method{Method}{Description}\hrule
\method{
  \emph{void} \textsf{\textbf{\small resize}}( 
  \emph{unsigned n} ) }{
  Sets the size of {\em this} to {\em n} (individuals).
}\\
\method{
  \emph{void} \textsf{\textbf{\small replace}}( 
  \emph{unsigned i,\\
  \hspace*{1.5em}const Individual\emph{\&} ind} ) }{
  Replaces individual number {\em i} by {\em ind}. 
}\\
\method{
  \emph{void} \textsf{\textbf{\small replace}}( 
  \emph{unsigned i,\\
  \hspace*{1.5em}const Population\emph{\&} pop} ) }{
  Replaces some individuals of {\em this} beginning with
  individual number {\em i} by individuals from {\em pop}.
}\\
\method{
  \emph{void} \textsf{\textbf{\small insert}}( 
  \emph{unsigned i,\\
  \hspace*{1.5em} const Individual\emph{\&} ind} ) }{
  Inserts {\em ind} at position {\em i} into {\em this}.
}\\
\method{
  \emph{void} \textsf{\textbf{\small insert}}( 
  \emph{unsigned i,\\
  \hspace*{1.5em} const Population\emph{\&} pop} ) }{
  Inserts all individuals of {\em pop} at position {\em i}
  into {\em this}. 
}\\
\method{
  \emph{void} \textsf{\textbf{\small append}}( 
  \emph{const Individual\emph{\&} ind} ) }{
  Appends {\em ind} to the end of {\em this}.
}\\
\method{
  \emph{void } \textsf{\textbf{\small append}}( 
  \emph{const Population\emph{\&} pop} ) }{
  Appends all individuals of {\em pop} to the end of {\em this}.
}\\
\method{
  \emph{void} \textsf{\textbf{\small remove}}( 
  \emph{unsigned i} ) }{
  Removes individual number {\em i} from {\em this}.
}\\
\method{
  \emph{void} \textsf{\textbf{\small remove}}( 
  \emph{unsigned from,\\ 
  \hspace*{1.5em}unsigned to} ) }{
  Removes all individuals from index {\em from} to index {\em to} 
  from {\em this}.
}\\
\method{
  \emph{void} \textsf{\textbf{\small exchange}}( 
  \emph{Population\emph{\&} pop} ) }{
  Swaps the contents of the populations {\em this} and {\em pop}.
}\\
\method{
  \emph{void} \textsf{\textbf{\small swap}}( 
  \emph{unsigned i, unsigned j} ) }{
  Swaps the individuals with indices {\em i} and {\em j}. 
}\\
\method{
  \emph{void} \textsf{\textbf{\small sort}}( ) }{
  Sorts all individuals of {\em this} refering to the value
  of the {\em ascending}-flag.
}\\
\method{
  \emph{void} \textsf{\textbf{\small shuffle}}( ) }{
  Randomly chnages the order of the individuals of {\em this}.
}\hrule


\vspace*{10pt}

\subsection{Methods for Extracting Single Individuals}

\vspace*{10pt}

\methodgrayfalse

\parindent0mm
\hrule
%\vspace*{1pt}
\method{Method}{Description}\hrule
\method{
  \emph{Individual\emph{\&}} \textsf{\textbf{\small random}}( ) }{ 
  Returns a randomly chosen individual of {\em this}.
}\\
\method{
  \emph{vector$<$ const Chromosome $\ast$ $>$}\\ 
  \hspace*{1.5em}\textsf{\textbf{\small matingPool}}(
  \emph{unsigned chrom,\\ 
  \hspace*{2.5em}unsigned from = 0,\\ 
  \hspace*{2.5em}unsigned to = 0} ) }{ 
  Extracts all chromosomes with index {\em chrom} out of all
  individuals with indices {\em from} to {\em to} (if {\em from}
  = {\em to} = 0, then the chromosome is extracted from all
  individuals), stores them in a vector and returns this vector.
}\\\hrule

\newpage

\subsection{Scaling Method}

\vspace*{10pt}

\methodgrayfalse

\parindent0mm
\hrule
\method{Method}{Description}\hrule
\method{
  \emph{void} \textsf{\textbf{\small linearDynamicScaling}}(\\
  \emph{\hspace*{1.5em}vector$<$ double $>$\emph{\&} window,\\
  \hspace*{1.5em}unsigned long t} ) }{ 
  Scales all fitness values of {\em this} by using the scaling
  window {\em window} at time {\em t}.\\
}\\\hrule

\vspace*{10pt}

\subsection{Selection Methods}

\vspace*{10pt}

\methodgrayfalse

\parindent0mm
\hrule
\method{Method}{Description}\hrule
\method{
  \emph{void} \textsf{\textbf{\small selectMuLambda}}(\\
  \hspace*{1.5em}\emph{Population\emph{\&} parents,\\
  \hspace*{1.5em}unsigned numElitists = 0} ) }{
  Selects {\em numElitists} elitists from {\em this} and {\em parents} 
  and additional individuals for reproduction using the 
  $(\mu\ [,|+]\ \lambda)$-selection.
}\\
\method{
  \emph{void} \textsf{\textbf{\small selectMuLambdaKappa}}(\\
  \hspace*{1.5em}\emph{Population\emph{\&} parents,\\
  \hspace*{1.5em}unsigned lifespan = 1,\\
  \hspace*{1.5em}unsigned adolescence = 0} ) }{ 
  Selects some individuals from {\em this} and {\em parents} 
  for reproduction by first choosing individuals with age  
  {\em age} $<$ {\em adolescence} and then choosing individuals
  with age {\em age} $<=$ {\em lifespan}. All other individuals
  are discarded.
}\\
\method{
  \emph{Individual\emph{\&}} 
  \textsf{\textbf{\small selectOneIndividual}}( ) }{ 
  Selects one individual by using proportional selection.
}\\
\method{
  \emph{void} \textsf{\textbf{\small selectProportional}}(\\
  \hspace*{1.5em}\emph{Population\emph{\&} parents,\\
  \hspace*{1.5em}unsigned numElitists = 0} ) }{ 
  Selects some individuals from {\em this} and {\em parents}
  by using proportional selection and additionally takes 
  {\em numElitists} elitists over.
}\\
\method{
  \emph{void} \textsf{\textbf{\small selectLinearRanking}}(\\
  \hspace*{1.5em}\emph{Population\emph{\&} parents,\\
  \hspace*{1.5em}double etaMax = 1.1,\\
  \hspace*{1.5em}unsigned numElitists = 0} ) }{ 
  Selects some individuals from {\em this} and {\em parents} 
  by using the linear ranking method and additionally takes 
  {\em numElitists} elitists over. 
  The maximum reproduction rate for a chosen individual is 
  {\em etaMax}.
}\\
\method{
  \emph{void} \textsf{\textbf{\small selectLinearRankingWhitley}}(\\
  \hspace*{1.5em}\emph{Population\emph{\&} parents,\\
  \hspace*{1.5em}double a = 1.1,\\
  \hspace*{1.5em}unsigned numElitists = 0} ) }{ 
  Selects some individuals from {\em this} and {\em parents}
  by using Whitley's linear ranking.
  Additionally {\em numElitists} elitists will be taken
  over. The maximum reproduction rate for a chosen individual is
  {\em a}.
}\hrule

\newpage

\methodgrayfalse

\parindent0mm
\hrule
\method{Method}{Description}\hrule
\method{
  \emph{void} \textsf{\textbf{\small selectUniformRanking}}(\\
  \hspace{1.5em}\emph{Population\emph{\&} parents,\\
  \hspace*{1.5em}unsigned numElitists = 0} ) }{ 
  Selects some individuals from {\em this} and {\em parents}
  by using the uniform ranking method and additionally takes 
  over {\em num\-Elitists} elitists. 
}\\
\method{
  \emph{void} \textsf{\textbf{\small reproduce}}(\\
  \hspace*{1.5em}\emph{Population\emph{\&} parents,\\
  \hspace*{1.5em}unsigned numElitists = 0} ) }{ 
  Equivalent to the selection me\-thod 
  {\em se\-lectUniformRanking}.
}\\
\method{
  \emph{void} \textsf{\textbf{\small selectTournament}}(\\
  \hspace*{1.5em}\emph{Population\emph{\&} parents,\\
  \hspace*{1.5em}unsigned q = 2,\\
  \hspace*{1.5em}unsigned numElitists = 0} ) }{ 
  Chooses {\em n} times the best individual of a
  randomly generated group of {\em q} individuals from
  {\em this} and {\em parents}. Additionally, 
  {\em numElitists} elitists are taken over.
}\\
\method{
  \emph{void} \textsf{\textbf{\small selectEPTournament}}(\\
  \hspace*{1.5em}\emph{Population\emph{\&} parents,\\
  \hspace*{1.5em}unsigned q} ) }{ 
  Selects individuals for reproduction by initiating 
  a competition of each individual from {\em this} and
  {\em parents} against {\em q} other individuals that
  are randomly chosen.
}\hrule

% ######################################################################
\section{Private Methods}
% ######################################################################

\subsection{Constructor and Initialization Method}

\vspace*{10pt}

\methodgrayfalse

\parindent0mm
\hrule
%\vspace*{1pt}
\method{Method}{Description}\hrule
\method{
  \textsf{\textbf{\small Population}}(\\
  \hspace*{1.5em}\emph{const vector$<$ Individual $\ast$ $>$\emph{\&}\\
  \hspace*{2.5em}indvec} ) }{ 
  Generates a new population with the individuals from {\em indvec}.
}\\
\method{
  \emph{void} \textsf{\textbf{\small selectInit}}( ) }{ 
  Initializes the internal variables of all individuals for
  a following selection method.
}\hrule

\newpage

\subsection{Comparison Methods}

\vspace*{10pt}

\methodgrayfalse

\parindent0mm
\hrule
\method{Method}{Description}\hrule
\method{
  \emph{static bool} \textsf{\textbf{\small lessFitness}}(\\ 
  \hspace*{1.5em}\emph{Individual$\ast$const\emph{\&} i1,\\
  \hspace*{1.5em}Individual$\ast$const\emph{\&} i2} ) }{ 
  Returns ``true'' if the fitness of {\em i1} is less than the
  fitness of {\em i2}.
}\\
\method{
  \emph{static bool} \textsf{\textbf{\small greaterFitness}}(\\ 
  \hspace*{1.5em}\emph{Individual$\ast$const\emph{\&} i1,\\
  \hspace*{1.5em}Individual$\ast$const\emph{\&} i2} ) }{ 
  Returns ``true'' if the fitness of {\em i1} is greater than the fitness
  of {\em i2}.
}\\
\method{
  \emph{static bool} \textsf{\textbf{\small greaterScoreAscending}}(\\ 
  \hspace*{1.5em}\emph{Individual$\ast$const\emph{\&} i1,\\
  \hspace*{1.5em}Individual$\ast$const\emph{\&} i2} ) }{ 
  Returns ``true'' if the scaled fitness of {\em i1} is greater or
  the normal fitness of {\em i1} is less than the corresponding values
  of {\em i2}.
}\\
\method{
  \emph{static bool} \textsf{\textbf{\small greaterScoreDescending}}(\\ 
  \hspace*{1.5em}\emph{Individual$\ast$const\emph{\&} i1,\\
  \hspace*{1.5em}Individual$\ast$const\emph{\&} i2} ) }{ 
  Returns ``true'' if the scaled or the normal fitness of {\em i1}
  is greater than the corresponding values of {\em i2}.
}\\
\method{
  \emph{Individual\emph{\&}} \textsf{\textbf{\small best}}(
  \emph{Individual\emph{\&} ind0,\\ 
  \hspace*{1.5em}Individual\emph{\&} ind1} ) }{ 
  Compares {\em ind0} and {\em ind1} and returns the individual with
  the lesser fitness if {\em ascending} is set to ``true'' or the
  individual with the greater fitness, if the {\em ascending}-flag is
  ``false''.
}\\
\method{
  \emph{Individual\emph{\&}} \textsf{\textbf{\small worst}}(
  \emph{Individual\emph{\&} ind0,\\ 
  \hspace*{1.5em}Individual\emph{\&} ind1} ) }{ 
  Compares {\em ind0} and {\em ind1} and returns the individual with
  the greater fitness if {\em ascending} is set to ``true'' or the
  individual with the lesser fitness, if the {\em ascending}-flag is
  ``false''.
}\hrule

\vspace*{10pt}

\subsection{Sorting Method}

\vspace*{10pt}

\methodgrayfalse

\parindent0mm
\hrule
%\vspace*{1pt}
\method{Method}{Description}\hrule
\method{
  \emph{void} \textsf{\textbf{\small sortIndividuals}}(\\ 
  \hspace*{1.5em}\emph{vector$<$ Individual $\ast$ $>$\emph{\&} indvec} ) }{ 
  Sorts all individuals in {\em indvec} by ascending 
  ({\em ascending} = ``true'') or descending 
  ({\em ascending} = ``false'') fitness values.
}\hrule

\newpage

\subsection{Selection Methods}

\vspace*{10pt}

\methodgrayfalse

\parindent0mm
\hrule
%\vspace*{1pt}
\method{Method}{Description}\hrule
\method{
  \emph{void} \textsf{\textbf{\small selectElitists}}(\\ 
  \hspace*{1.5em}\emph{Population\emph{\&} parents,\\
  \hspace*{1.5em}unsigned numElitists} ) }{ 
  Selects {\em numElitists} elitists from {\em this} and 
  {\em parents}.
}\\
\method{
  \emph{void} \textsf{\textbf{\small selectRouletteWheel}}(\\ 
  \hspace*{1.5em}\emph{Population\emph{\&} parents,\\
  \hspace*{1.5em}unsigned numElitists} ) }{ 
  Selects {\em numElitists} elitists from {\em this} and
  {\em parents} by using the Roulette-wheel selection method.
  Depending on the value of the {\em spinOnce}-flag, the
  ``wheel'' spins only one time or several times.
}\hrule

\vspace*{10pt}

\subsection{Input- and Output Methods}

\vspace*{10pt}

\methodgrayfalse

\parindent0mm
\hrule
%\vspace*{1pt}
\method{Method}{Description}\hrule
\method{
  \emph{void} \textsf{\textbf{\small readFrom}}(
  \emph{istream\emph{\&} is} ) }{ 
  Replaces the data of {\em this} with those
  read from input stream {\em is}.
}\\
\method{
  \emph{void} \textsf{\textbf{\small writeTo}}(
  \emph{ostream\emph{\&} os} ) }{ 
  Writes all data of {\em this} to output stream {\em os}.
}\hrule


\end{document}










