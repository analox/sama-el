% 24th, Jan, 2001 Ver.1     Tatsuya Okabe
%                 Ver.2
%                 Ver.3
%                 Ver.4
%                 Ver.5
%
%---------------------------------------------------------------------------%
% Made by Tatsuya Okabe ( HONDA R&D Europe ( Deutschland ) GmbH )           %
% Checked by Bernhard Sendhoff ( HONDA R&D Europe ( Deutschland ) GmbH )    %
%---------------------------------------------------------------------------%
% Class Binomial

\section{Abstract}

\noindent
With the class {\em Binomial}, the ``Binomial distribution'' can be
simulated. To explain this distribution, I will use the example of a
coin toss trial. Here I define the probability of the e.g. right side
as {\em p} and the one of the oposite side as {\em q}, with $p + q =
1$. If we try {\em n} coin tosses, the probability that we can see
{\em x} times the right side is given by the following equation :

\begin{equation}
f(x) = _n \hspace{-1mm} C_x p^x q^{n-x}
\end{equation}

\noindent
This distribution is called the {\em Binomial distribution}.

\vspace*{10mm}

\section{Internal variables}

\begin{itemize}
\item {pN - The variable n in the former equation.}
\item {pP - The probability p in the former equation.}
\end{itemize}

%********************
\index{pN (Variable)}
\index{pP (Variable)}
%********************

\vspace*{10mm}

\section{Public Methods}

\noindent
These methods can be used by all \cpp - programs, that have included the
header file Binomial.h and the library EA. In order to use these
methods, please include Binomial.h. If you declare only Population.h,
we can't use these method in this version. 

\clearpage

\subsection{Constructors}

%---------------------------------------------------------------------------%
% 001
\index{Binomial!( unsigned n, double p )}
\setNormalInstance
\setCorrectWidthThree{8pt}
\setParamOne{n}{unsigned}{The number of trials.} 
\setParamTwo{p}{double}{The probability of the factor we keep our eyes.}
\printMethodWithParamsSaved
{}
{None.}
{Binomial}
{The default constructor. Generates the random generator of binomial distribution.}
{None.}
\setCorrectWidthThree{4pt}
%---------------------------------------------------------------------------%

%---------------------------------------------------------------------------%
% 002
\index{Binomial!( unsigned n, double p, RNG\& rng)}
\setNormalInstance
\setCorrectWidthThree{8pt}
\setParamOne{n}{unsigned}{The number of trials.}
\setParamTwo{p}{double}{The probability of the factor we keep our eyes.}
\setParamThree{rng}{RNG\&}{RNG class.}
\printMethodWithParamsSaved
{}
{None.}
{Bernoulli}
{The constructor. Generates the random generator of binomial distribution.}
{None.}
\setCorrectWidthThree{4pt}
%---------------------------------------------------------------------------%

\vspace*{10mm}

\subsection{Operators}

%---------------------------------------------------------------------------%
% 003
\index{operator( )!( unsigned n, double p )}
\setNormalInstance
\setCorrectWidthThree{8pt}
\setParamOne{n}{unsigned}{The number of trials.}
\setParamTwo{p}{double}{The probability of the factor we keep our eyes.}
\printMethodWithParamsSaved
{long}
{The number of the factors with the probability {\em p}.}
{operator( )}
{Gets the result of the binomial distribution.}
{None.}
\setCorrectWidthThree{4pt}
%---------------------------------------------------------------------------%

\clearpage

%---------------------------------------------------------------------------%
% 004
\index{operator( )!( )} 
\setNormalInstance
\printEmptyMethodReturnSpecial
{long}
{operator( )}
{Gets the result of the binomial distribution using the number {\em pN} and
the probability {\em pP}.}
{The number of the factors with the probability {\em pP}.}
{None.}
%---------------------------------------------------------------------------%

\vspace*{10mm}

\subsection{Information Retrieval Methods}

%---------------------------------------------------------------------------%
% 005
\index{n!( )} 
\setConstInstance
\printEmptyMethodReturnSpecial
{unsigned}
{n}
{Returns the variable {\em pN}.}
{The variable {\em pN}.}
{None.}
%---------------------------------------------------------------------------%

%---------------------------------------------------------------------------%
% 006
\index{p!( )} 
\setConstInstance
\printEmptyMethodReturnSpecial
{double}
{p}
{Returns the probability {\em pP}.}
{The probability {\em pP}.}
{None.}
%---------------------------------------------------------------------------%

%---------------------------------------------------------------------------%
% 007
\index{n!( unsigned newN )} 
\setNormalInstance
\printMethodWithOneParam
{void}
{n}
{unsigned}
{newN}
{New number of trials.}
{Sets the number of trials {\em pN} using new number {\em newN}.}
{None.}
{None.}
%---------------------------------------------------------------------------%

\clearpage

%---------------------------------------------------------------------------%
% 008
\index{p!( double newP )} 
\setNormalInstance
\printMethodWithOneParam
{void}
{p}
{double}
{newP}
{New probability.}
{Sets the probability {\em pP} using new probability {\em newP}.}
{None.}
{None.}
%---------------------------------------------------------------------------%

\vspace*{10mm}

\subsection{The probability}

%---------------------------------------------------------------------------%
% 009
\index{p!( const long\& x )} 
\setConstInstance
\printMethodWithOneParam
{double}
{p}
{const long\&}
{x}
{The number of the factors which were occured.}
{Returns the probability of {\em x}.}
{The probability.}
{None.}
%---------------------------------------------------------------------------%




