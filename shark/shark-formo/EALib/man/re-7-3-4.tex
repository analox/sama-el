\subsection{Mutation Methods}

%---------------------------------------------------------------------------%
\index{mutateNormal!( double stddev )}
\setNormalInstance
\printMethodWithOneParam
{void}
{mutateNormal}
{double}
{stddev}
{Standard deviation for normally distributed random values.}
{Mutates all alleles of {\em this} by adding a normally distributed
 random value between "0" and the standard deviation {\em stddev}
 to the allele value.}
{None.}
{None.}
%---------------------------------------------------------------------------%

\clearpage

%---------------------------------------------------------------------------%
\index{mutateNormal!( const vector$<$ double $>$\& stddev, bool cycle )}
\setNormalInstance
\setCorrectWidthThree{8pt}
\setParamOne{stddev}{const vector$<$ double $>$ \&}{Vector that
contains standard deviation values for the single alleles of {\em this}.
{\em stddev} should contain at most as many values as there are
alleles in {\em this}. Otherwise the method will be aborted with an error
message.}
\setParamTwo{cycle}{bool}{Specifies, whether {\em stddev} can be
used circular ({\em cycle} = "true") or not ({\em cycle} = "false",
this is also the default).}
\printMethodWithParamsSaved
{void}
{}
{mutateNormal}
{Same as above, but here vector {\em stddev} contains a
 standard deviation value for each single allele of {\em this}.
 As {\em stddev} can contain less values than there are alleles in 
 {\em this}, the flag {\em cycle} denotes, whether {\em cycle}
 can be used circular or not.}
{}
\setCorrectWidthThree{4pt}
%---------------------------------------------------------------------------%

\vspace*{4ex}

%---------------------------------------------------------------------------%
\index{mutateNormal!( const Chromosome\& stddev, bool cycle )}
\setNormalInstance
\setCorrectWidthThree{8pt}
\setParamOne{stddev}{const Chromosome \&}{See above.}
\setParamTwo{cycle}{bool}{See above.}
\printMethodWithParamsSaved
{void}
{}
{mutateNormal}
{Same as above, but here a chromosome is used to store the
 standard deviation values.}
{}
\setCorrectWidthThree{4pt}
%---------------------------------------------------------------------------%

\vspace*{4ex}

%---------------------------------------------------------------------------%
\index{mutateNormal!( const ChromosomeT$<$ double $>$\& stddev, bool cycle )}
\setNormalInstance
\setCorrectWidthThree{8pt}
\setParamOne{stddev}{const ChromosomeT$<$ double $>$\&}{See above.}
\setParamTwo{cycle}{bool}{See above.}
\printMethodWithParamsSaved
{void}
{}
{mutateNormal}
{Same as above, but here a {\tt double} chromosome is used to store
 the standard deviation values.}
{}
\setCorrectWidthThree{4pt}
%---------------------------------------------------------------------------%

\clearpage

%---------------------------------------------------------------------------%
\index{mutateCauchy!( double scale )}
\setNormalInstance
\printMethodWithOneParam
{void}
{mutateCauchy}
{double}
{scale}            
{Scaling value for the random numbers.}
{Mutates all alleles of {\em this} by adding random numbers
 produced by the Cauchy method to the allele values. The random
 numbers will be scaled with {\em scale}.}
{None.}
{None.}
%---------------------------------------------------------------------------%

\vspace*{4ex}

%---------------------------------------------------------------------------%
\index{mutateCauchy!( const vector$<$ double $>$\& scale, bool cycle = false )}
\setNormalInstance
\setCorrectWidthThree{8pt}
\setParamOne{scale}{const vector$<$ double $>$\&}{Vector with scaling
values for each allele of {\em this}. {\em scale} must contain at most
as many values as there are alleles in {\em this}. Otherwise the method 
will be
aborted with an error message.}
\setParamTwo{cycle}{bool}{Specifies, whether {\em scale} can be used
circular ({\em cycle} = ``true'') or not ({\em cycle} = ``false'').}
\printMethodWithParamsSaved
{void}
{}
{mutateCauchy}
{Same as above, but here every allele of {\em this} has its own
 corresponding scaling value. These values are stored in vector {\em scale}.
 As {\em scale} can contain less values than there are alleles in {\em this},
 the flag {\em cycle} specifies, whether {\em scale} can be used
 circular or not.} 
{}
\setCorrectWidthThree{4pt}
%---------------------------------------------------------------------------%

\vspace*{4ex}

%---------------------------------------------------------------------------%
\index{mutateCauchy!( const Chromosome\& scale, bool cycle = false )}
\setNormalInstance
\setCorrectWidthThree{8pt}
\setParamOne{scale}{const Chromosome\&}{See above.}
\setParamTwo{cycle}{bool}{See above.}
\printMethodWithParamsSaved
{void}
{}
{mutateCauchy}
{Same as above, but here a chromosome is used to store the scaling
 values for the alleles of {\em this}.}
{}
\setCorrectWidthThree{4pt}
%---------------------------------------------------------------------------%

\clearpage

%---------------------------------------------------------------------------%
\index{mutateCauchy!( const ChromosomeT$<$ double $>$\& scale, bool cycle = false )}
\setNormalInstance
\setCorrectWidthThree{8pt}
\setParamOne{scale}{const ChromosomeT$<$ double $>$\&}{See above.}
\setParamTwo{cycle}{bool}{See above.}
\printMethodWithParamsSaved
{void}
{}
{mutateCauchy}
{Same as above, but here a {\tt double} chromosome is used to
 store the scaling values for the alleles of {\em this}.}
{}
\setCorrectWidthThree{4pt}
%---------------------------------------------------------------------------%









