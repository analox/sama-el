% 24th, Jan, 2001 Ver.1     Tatsuya Okabe
%                 Ver.2
%                 Ver.3
%                 Ver.4
%                 Ver.5
%
%---------------------------------------------------------------------------%
% Made by Tatsuya Okabe ( HONDA R&D Europe ( Deutschland ) GmbH )           %
% Checked by Bernhard Sendhoff ( HONDA R&D Europe ( Deutschland ) GmbH )    %
%---------------------------------------------------------------------------%
% Class RNG

\section{Abstract}

\noindent
In the class {\em RNG}, we can set the random seed value. Also, class
{\em RNG} was instantialized as {\em globalRng}.

\noindent
A computer seems to be able to generate random numbers. But, we should
call them ``pseudo random number'' because we can expect. The reason
why we can do this is that C++ or Fortran uses the following equation
to generate a random number.

\begin{equation}
I_{j+1} = a I_{j} + c \hspace{10mm} ( mod \hspace{5mm} m )
\end{equation}

\noindent
Here, $I$, $a$, $c$ and $m$ are random number, multiplier, increment
and modulus, respectively. Specially, $I_0$ is called ``random seed''.

\noindent
This method is called ``linear congruential method''.

\noindent
This equation shows that generated random numbers are cyclic and the
longest frequency is $m$. However, if we failed to select the correct
values of $a$, $c$ and $m$, the frequency will be shortened. Thus,
many mathematicians researched these values.

\noindent
Park and Miller proposed ``Minimal Standard generater'' in 1988. In
the method, they proposed following values.

\begin{equation}
a = 7^5 = 16807 \hspace{10mm} m = 2^{31} - 1 = 2147483647
\hspace{10mm} c = 0
\end{equation}

\noindent
For more information, please see [\ref{NRC}].

\vspace*{10mm}

\section{Instantialized classes}

\noindent
Instantialized classes are as follows.

\vspace*{3mm}

\begin{tabular}{lll}
RNG       & $\Longrightarrow$ & globalRng   \\
\end{tabular}

\vspace*{3mm}

\section{Internal Variables}

\begin{itemize}
\item initialSx - Initial value of random seed 1.
\item initialSy - Initial value of random seed 2.
\item initialSz - Initial value of random seed 3.
\item sx - Current value of random seed 1.
\item sy - Current value of random seed 2.
\item sz - Current value of random seed 3.
\end{itemize}

%********************
\index{initialSx (Variable)}
\index{initialSy (Variable)}
\index{initialSz (Variable)}
\index{sx (Variable)}
\index{sy (Variable)}
\index{sz (Variable)}
%********************

\vspace*{10mm}

\section{Public Methods}

\noindent
This method can be used by all \cpp - programs, that have included the
header file RNG.h and the library EA.

\subsection{Constructors}

%---------------------------------------------------------------------------%
% 001
\index{RNG!( long s )}
\setNormalInstance
\printMethodWithOneParam
{}
{RNG}
{long}
{s}
{Random seed.}
{The default constructor. Gerenates this class.}
{None.}
{None.}
%---------------------------------------------------------------------------%

\vspace*{10mm}

\subsection{Operators}

%---------------------------------------------------------------------------%
% 002
\index{operator( )!( )}
\setNormalInstance
\printEmptyMethodReturnSpecial
{double}
{operator}
{Gets a random seed for double type.}
{A random seed.}
{None.}
%---------------------------------------------------------------------------%

\clearpage

\subsection{Information Retrieval Methods}

%---------------------------------------------------------------------------%
% 003
\index{seed!( long s )}
\setNormalInstance
\printMethodWithOneParam
{void}
{seed}
{long}
{s}
{Random seed.}
{Sets initial variables and current variables regaring a random seed.}
{None.}
{None.}
%---------------------------------------------------------------------------%

%---------------------------------------------------------------------------%
% 004
\index{reset!( )}
\setNormalInstance
\printEmptyMethod
{reset}
{Initializes current variables.}
%---------------------------------------------------------------------------%

%---------------------------------------------------------------------------%
% 005
\index{genLong!( )}
\setNormalInstance
\printEmptyMethodReturnSpecial
{long}
{genLong}
{Gets a random seed for long type.}
{A random seed.}
{None.}
%---------------------------------------------------------------------------%

%---------------------------------------------------------------------------%
% 006
\index{genDouble!( )}
\setNormalInstance
\printEmptyMethodReturnSpecial
{double}
{genDouble}
{Gets a random seed for double type.}
{A random seed.}
{None.}
%---------------------------------------------------------------------------%

\clearpage

%---------------------------------------------------------------------------%
% 007
\index{getStatus!( unsigned $\&$x, unsigned $\&$y, unsigned $\&$z )}
\setConstInstance
\setCorrectWidthThree{8pt}
\setParamOne{$\&$x}{unsigned}{A random seed 1.} 
\setParamTwo{$\&$y}{unsigned}{A random seed 2.}
\setParamThree{$\&$z}{unsigned}{A random seed 3.}
\printMethodWithParamsSaved
{void}
{None.}
{getStatus}
{Gets the current status {\em sx}, {\em sy} and {\em sz}.}
{None.}
\setCorrectWidthThree{4pt}
%---------------------------------------------------------------------------%

%---------------------------------------------------------------------------%
% 008
\index{setStatus!( unsigned $\&$x, unsigned $\&$y, unsigned $\&$z )}
\setConstInstance
\setCorrectWidthThree{8pt}
\setParamOne{$\&$x}{unsigned}{A random seed 1.} 
\setParamTwo{$\&$y}{unsigned}{A random seed 2.}
\setParamThree{$\&$z}{unsigned}{A random seed 3.}
\printMethodWithParamsSaved
{void}
{None.}
{setStatus}
{Sets the current status {\em sx}, {\em sy} and {\em sz}.}
{None.}
\setCorrectWidthThree{4pt}
%---------------------------------------------------------------------------%



