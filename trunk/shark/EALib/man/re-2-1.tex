\section{Abstract}

Evolutionary algorithms basically work with {\em chromosomes},
that contain a fixed number of data units of any type, called {\em alleles}.
The \cpp\ program uses vectors of type {\em T} to represent chromosomes. 
When a new chromosome is generated it will be administrated by using 
the \cpp\ construct {\tt map}. This construct will use only pointers
to the vector and registrate the chromosome.
The chromosome itself will be stored only one time and many of the
following operations will work only with the pointers. 
Several classes of administration will be built that correspond to
the different types {\em T} of the registered chromosomes.
The type of a chromosome is also used to address the chromosomes of one
administration class. This administration is, however, not visible for
the user of the {\tt Chromosome} class, he only works with instances of
this class.\\
The class {\tt Chromosome} is the base upon which all other classes
work. It contains general methods that are valid for all chromosomes.
The other {\tt Chromosome} classes will contain methods for more special
kinds of chromosomes.
The user cannot use an instance of the class {\tt Chromosome} directly,
because this class is only of declarative character. When using chromosomes,
the user must always declare chromosomes of a special type,
e.g. ``Chromosome$<$ int $>$ {\tt chrom}''.
In the following description of the class methods the current instance
of class {\tt Chromosome}
will always be denoted as {\em this} (as usual in \cpp).
