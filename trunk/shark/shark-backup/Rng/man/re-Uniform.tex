% 24th, Jan, 2001 Ver.1     Tatsuya Okabe
%                 Ver.2
%                 Ver.3
%                 Ver.4
%                 Ver.5
%
%---------------------------------------------------------------------------%
% Made by Tatsuya Okabe ( HONDA R&D Europe ( Deutschland ) GmbH )           %
% Checked by Bernhard Sendhoff ( HONDA R&D Europe ( Deutschland ) GmbH )    %
%---------------------------------------------------------------------------%
% Class Uniform

\section{Abstract}

\noindent
With the class {\em Uniform}, we can simulate the ``Uniform''
distribution. If you want to get real number from $x_{lower}$ to
$x_{upper}$ with the same probability, you can use this class. The
distribution follows the next equation.

\begin{equation}
f(x) = \left\{
\begin{array}{ll}
1 / ( x_{upper} - x_{lower} ) & x_{lower} \le x \le x_{upper} \\
0 & the \hspace{2mm} other
\end{array}
\right.
\end{equation}

\noindent
In this equation, $x_{lower}$, $x$ and $x_{upper}$ mean the lower
boundary, the variable and the upper boundary, respectively.

\vspace*{10mm}

\section{Internal Variables}

\begin{itemize}
\item pLow - The lower boundary.
\item pHigh - The upper boundary.
\end{itemize}

%********************
\index{pLow (Variable)}
\index{pHigh (Variable)}
%********************

\clearpage

\section{Public Methods}

\noindent
These methods can be used by all \cpp - programs, that have included the
header file Uniform.h and the library EA.

\subsection{Constructors}

%---------------------------------------------------------------------------%
% 001
\index{Uniform!( double lo, double hi )}
\setNormalInstance
\setCorrectWidthThree{8pt}
\setParamOne{lo}{double}{The lower boundary.} 
\setParamTwo{hi}{double}{The upper boundary.}
\printMethodWithParamsSaved
{}
{None.}
{Uniform}
{The default constructor. Generates the random generator of Uniform distribution.}
{None.}
\setCorrectWidthThree{4pt}
%---------------------------------------------------------------------------%

%---------------------------------------------------------------------------%
% 002
\index{Uniform!( double lo, double hi, RNG\& rng )}
\setNormalInstance
\setCorrectWidthThree{8pt}
\setParamOne{lo}{double}{The lower boundary.} 
\setParamTwo{hi}{double}{The upper boundary.}
\setParamThree{rng}{RNG\&}{RNG class.}
\printMethodWithParamsSaved
{}
{None.}
{Uniform}
{The constructor. Generates the random generator of Uniform distribution.}
{None.}
\setCorrectWidthThree{4pt}
%---------------------------------------------------------------------------%

\vspace*{10mm}

\subsection{Operators}

%---------------------------------------------------------------------------%
% 003
\index{operator( )!( double lo, double hi )}
\setNormalInstance
\setCorrectWidthThree{8pt}
\setParamOne{lo}{double}{The lower boundary.} 
\setParamTwo{hi}{double}{The upper boundary.}
\printMethodWithParamsSaved
{double}
{The result of uniform distribution.}
{operator( )}
{Gets the result of uniform distribution.}
{None.}
\setCorrectWidthThree{4pt}
%---------------------------------------------------------------------------%

\clearpage

%---------------------------------------------------------------------------%
% 004
\index{operator( )!( )} 
\setNormalInstance
\printEmptyMethodReturnSpecial
{double}
{operator( )}
{Gets the result of uniform distribution.}
{The result of uniform distribution.}
{None.}
%---------------------------------------------------------------------------%

\vspace*{10mm}

\subsection{Information Retrieval Methods}

%---------------------------------------------------------------------------%
% 005
\index{low!( )} 
\setConstInstance
\printEmptyMethodReturnSpecial
{double}
{low}
{Returns the lower boundary {\em pLow}.}
{The lower boundary {\em pLow}.}
{None.}
%---------------------------------------------------------------------------%

%---------------------------------------------------------------------------%
% 006
\index{hign!( )} 
\setConstInstance
\printEmptyMethodReturnSpecial
{double}
{high}
{Returns the upper boundary {\em pHigh}.}
{The upper boundary {\em pHigh}.}
{None.}
%---------------------------------------------------------------------------%

%---------------------------------------------------------------------------%
% 007
\index{low!( double lo )} 
\setNormalInstance
\printMethodWithOneParam
{void}
{low}
{double}
{lo}
{New value of the lower boundary.}
{Sets the lower boundary {\em pLow} using new lower boundary {\em lo}.}
{None.}
{None.}
%---------------------------------------------------------------------------%

\clearpage

%---------------------------------------------------------------------------%
% 008
\index{high!( double hi )} 
\setNormalInstance
\printMethodWithOneParam
{void}
{high}
{double}
{hi}
{New value of the upper boundary.}
{Sets the upper boundary {\em pHigh} using new upper boundary {\em hi}.}
{None.}
{None.}
%---------------------------------------------------------------------------%

\vspace*{10mm}

\subsection{The probability}

%---------------------------------------------------------------------------%
% 009
\index{p!( const double\& x )} 
\setConstInstance
\printMethodWithOneParam
{double}
{p}
{const double\&}
{x}
{The factor which you want to calculate the probability.}
{Returns the probability of {\em x}.}
{The probability.}
{None.}
%---------------------------------------------------------------------------%













