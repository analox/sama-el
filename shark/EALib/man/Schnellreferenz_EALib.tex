\documentclass[12pt]{report}
\usepackage{a4,german}
\usepackage{pstricks}
% =======================================================================
% Dateiname:        Vorlage.tex
% Autor:            Ruediger Alberts
% EMail:            Ruediger.Alberts@neuroinformatik.ruhr-uni-bochum.de
% Erstellt am:      1999-09-07
% Letzte Aenderung: 1999-09-09
%
% Datei wird mit Befehl ``% =======================================================================
% Dateiname:        Vorlage.tex
% Autor:            Ruediger Alberts
% EMail:            Ruediger.Alberts@neuroinformatik.ruhr-uni-bochum.de
% Erstellt am:      1999-09-07
% Letzte Aenderung: 1999-09-09
%
% Datei wird mit Befehl ``% =======================================================================
% Dateiname:        Vorlage.tex
% Autor:            Ruediger Alberts
% EMail:            Ruediger.Alberts@neuroinformatik.ruhr-uni-bochum.de
% Erstellt am:      1999-09-07
% Letzte Aenderung: 1999-09-09
%
% Datei wird mit Befehl ``\input{Vorlage}'' in LaTex-Dokumente 
% (Latex2Epsilon) eingebunden und stellt Befehle zur einfachen Doku- 
% mentation von C-Funktionen/C++-Methoden bereit.
% =======================================================================

\usepackage{ifthen}

    % Definition von Konstanten:
    \newcommand{\ParamDelim}            % Trennungstext zwischen
        { \ - }                         % Parametername und -beschreibung.
    \newcommand{\NormMethodEnd}         % Endtext von Methode bei 
        {\ \ );}                        % veraenderbarer Aufrufinstanz.
    \newcommand{\ConstMethodEnd}        % Endtext von Methode bei
        {\ \ ) const;}                  % nicht veraenderbarer Aufrufinstanz.
    \newcommand{\MaxLabelTxt}           % Komponentenbeschreibungstext
        {{\bf Besonderheiten: }}        % mit maximaler Breite.
    \newcommand{\MethodBegin}           % Beginntext einer Methode.
        {(\ }
    \newcommand{\InBetween}             % Text zwischen Rueckgabewert und
        {\ \ }                          % Methodenname.

    % Definition eigener Variablen:


    \newboolean{ConstInstance}          % Gibt an, ob die Instanz, die
                                        % die Methode aufruft veraendert
                                        % werden darf oder nicht. Im
                                        % 2. Fall endet die Methode mit
                                        % ``) const;'', im 1. Fall 
                                        % mit ``);''.
    \newcounter{SavedParams}            % Anzahl der gespeicherten
                                        % Parameter.
    \setcounter{SavedParams}{0}
    \newcommand{\ParamOne}{}            % Definition der 10 
    \newcommand{\ParamTypeOne}{}        % abspeicherbaren Parameter,
    \newcommand{\ParamDescrOne}{}       % ihrer Typen und Beschreibungen.
    \newcommand{\ParamTwo}{}            
    \newcommand{\ParamTypeTwo}{}
    \newcommand{\ParamDescrTwo}{}
    \newcommand{\ParamThree}{}
    \newcommand{\ParamTypeThree}{}
    \newcommand{\ParamDescrThree}{}
    \newcommand{\ParamFour}{}
    \newcommand{\ParamTypeFour}{}
    \newcommand{\ParamDescrFour}{}
    \newcommand{\ParamFive}{}
    \newcommand{\ParamTypeFive}{}
    \newcommand{\ParamDescrFive}{}
    \newcommand{\ParamSix}{}
    \newcommand{\ParamTypeSix}{}
    \newcommand{\ParamDescrSix}{}
    \newcommand{\ParamSeven}{}
    \newcommand{\ParamTypeSeven}{}
    \newcommand{\ParamDescrSeven}{}
    \newcommand{\ParamEight}{}
    \newcommand{\ParamTypeEight}{}
    \newcommand{\ParamDescrEight}{}
    \newcommand{\ParamNine}{}
    \newcommand{\ParamTypeNine}{}
    \newcommand{\ParamDescrNine}{}
    \newcommand{\ParamTen}{}
    \newcommand{\ParamTypeTen}{}
    \newcommand{\ParamDescrTen}{}
    \newlength{\MaxParamWidth}          % Breite des breitesten 
                                        % Methodenparameters.
    \newlength{\SpecialWidth}           % Breite des Textes 
    \settowidth{\SpecialWidth}          % {\bf Besonderheiten}.
        {\MaxLabelTxt}
    \newlength{\ParamDelimWidth}        % Breite des Textes `` \ - ``.
    \settowidth{\ParamDelimWidth}
        {\ParamDelim}
    \newlength{\ParamDescrWidth}        % Max. Breite des Parametertextes.
    \newlength{\SpecialRetTxtWidth}     % Max. Breite der Texte fuer
                                        % Besonderheiten und Rueckgabewert.
    \newlength{\MethodBeginWidth}       % Breite des Methodenanfangs, d.h.
                                        % ``Rueckgabewert Methodenname ( ``.
    \newlength{\BetweenParsWidth}       % Max. zur Verfuegung stehende
                                        % Breite zwischen den beiden Klammern
                                        % der Methode.
    
    \newlength{\MethodEndWidth}         % Breite des Textes ``  );'' bzw.
    \settowidth{\MethodEndWidth}        % `` ) const;''
        {\NormMethodEnd}
    \newlength{\ParamMaxTypeWidth}      % Breite des breitesten
                                        % Parameterwert-Typs. 
    \newlength{\NoneWidthOne}           % Breite des Textes ``Keine.''.
    \settowidth{\NoneWidthOne}{Keine.}
    \newlength{\NoneWidthTwo}           % Breite des Textes ``Keiner.''.
    \settowidth{\NoneWidthTwo}{Keiner.}
    \newlength{\Temp}                   % Temporaere Hilfsvariable. 
    \newlength{\VertSpace}              % Vertikaler Abstand zwischen 
                                        % zwei Parboxen.
   
    % Trotz der automatischen Berechnung der Textbreiten treten
    % kleine Unstimmigkeiten von wenigen Punkten bei den Texten
    % zur Beschreibung der Parameter, des Rueckgabewertes und
    % der Besonderheiten auf, die mit den folgenden Variablen
    % behoben werden koennen:

    \newlength{\CorrectWidthOne}        % Korrekturbreite fuer die
                                        % Parameterbeschreibung.
    \newlength{\CorrectWidthTwo}        % Korrekturbreite fuer die
                                        % Beschreibungen von Rueckgabewert
                                        % und Besonderheiten.
    \newlength{\CorrectWidthThree}      % Korrekturbreite, da erster
                                        % Parameter einer Methode weiter
                                        % vorrueckt als die anderen.
    

% Uebersicht ueber die Parameter und ihre Bedeutung.


%
% <-- \MethodBeginWidth    -->
%------------------------------------------------------------------------
% Rueckgabewert Methodenname(   Typ_Parameter_1 Parameter_1,
%                            <->Typ_Parameter_2 Parameter_2,
%              \CorrectWideThree...
%                               Typ_Parameter_n Parameter_n   );
%                             <-------------> <--------->
%                          \ParamMaxTypeWidth \MaxParamWidth
%                             <-------------------------->
%                             \BetweenParsWidth
%                                                         <-->
%                                                         \MethodEndWidth
%------------------------------------------------------------------------
% Beschreibung der Methode...
%                             \ParamDelimWidth
%                             <->                        \CorrectWidthOne
% \VertSpace
% Parameter:       Parameter_1  - Beschreibung von Parameter 1.   <----->
%                  Parameter_2  - Beschreibung von Parameter 2.
%                  ...
%                  Parameter_n  - Beschreibung von Parameter n.
% \VertSpace                      \ParamDescrWidth
%                                 <------------------------------>
% Rueckgabewert:   Beschreibung des Rueckgabewertes bzw. der Text <-----> 
%                  Keiner.                   
%                  <----->
%                  \NoneWidthTwo               \SpecialRetTxtWidth
% \VertSpace       <--------------------------------------------->
% Besonderheiten:  Beschreibung von Besonderheiten bzw. der Text  <----->
%                  Keine.     
% <--------------> <---->
%  \SpecialWidth   \NoneWidthOne                         \CorrectWidthTwo


    % Definition eigener Befehle:


    %-----------------------------------------------------------------------
    % Gibt den Begriff ``C++'' in ansprechender Form aus.
    % Makro wurde freundlicherweise zur Verfuegung gestellt von
    % Axel W. Dietrich.
    %
    \newcommand{\cpp}{
        \mbox{\emph{\textrm{C\hspace{-1.5pt}\raisebox{1.75pt}{\scriptsize +}
        \hspace{-6pt}\raisebox{.75pt}{\scriptsize +}}}}%
    }


    %-----------------------------------------------------------------------
    % Gibt die Werte aller Parameter auf dem Bildschirm aus.
    %
    \newcommand{\showAllParams}{
        \noindent
        {\em Werte aller Parameter:}
        \begin{tabbing}
            ParamMaxTypeWidth:\ \ \=\kill
            MaxParamWidth:       \>\the\MaxParamWidth\\
            SpecialWidth:        \>\the\SpecialWidth\\
            ParamDelimWidth:     \>\the\ParamDelimWidth\\
            ParamDescrWidth:     \>\the\ParamDescrWidth\\
            SpecialRetTxtWidth:  \>\the\SpecialRetTxtWidth\\
            MethodBeginWidth:    \>\the\MethodBeginWidth\\
            BetweenParsWidth:    \>\the\BetweenParsWidth\\
            MethodEndWidth:      \>\the\MethodEndWidth\\
            ParamMaxTypeWidth:   \>\the\ParamMaxTypeWidth\\
            NoneWidthOne:        \>\the\NoneWidthOne\\
            NoneWidthTwo:        \>\the\NoneWidthTwo\\
            VertSpace:           \>\the\VertSpace\\
            CorrectWidthOne:     \>\the\CorrectWidthOne\\
            CorrectWidthTwo:     \>\the\CorrectWidthTwo\\
            CorrectWidthThree:   \>\the\CorrectWidthThree\\
            ConstInstance:       
            \ifthenelse{\boolean{ConstInstance}}
                {\>ja\\}
                {\>nein\\}
        \end{tabbing}
    }

    %-----------------------------------------------------------------------
    % Setzt einen Schalter, der angibt, dass die Instanz, die die kommende
    % Methode aufrufen kann, nicht veraendert werden darf. Dies hat fuer
    % die Dokumentation zur Folge, dass die Methode mit ``) const;'' endet.
    %
    \newcommand{\setConstInstance}{    
        \setboolean{ConstInstance}{true}
    }

    %-----------------------------------------------------------------------
    % Setzt einen Schalter, der angibt, dass die Instanz, die die kommende
    % Methode aufrufen kann, veraendert werden darf. Die Methode endet also
    % in der Dokumentation normal mit ``);''.
    %
    \newcommand{\setNormalInstance}{    
        \setboolean{ConstInstance}{false}
    }

    %-----------------------------------------------------------------------
    % Initialisiert die Breite des breitesten Methodenparametertyptextes
    % mit Null.
    %
    \newcommand{\initParamMaxTypeWidth}{
        \setlength{\ParamMaxTypeWidth}{0pt}
    }

    %-----------------------------------------------------------------------
    % Initialisiert die Breite des breitesten Parameternamens mit Null.
    %
    \newcommand{\initMaxParamWidth}{
        \setlength{\MaxParamWidth}{0pt}
    }

    %-----------------------------------------------------------------------
    % Ueberprueft, ob die Breite des uebergebenen Textes des Parametertyps
    % groesser ist als die des bisher als breitester Typtext festgelegten
    % Textes und passt den Wert evtl. an.
    % Parameter #1:  Neuer Typtext, dessen Breite zum Vergleich dient.
    %   
    \newcommand{\setNewParamMaxTypeWidth}[1]{
        \settowidth{\Temp}{#1}
        \ifthenelse{\lengthtest{\ParamMaxTypeWidth < \Temp}}
            {\setlength{\ParamMaxTypeWidth}{\Temp}}{}   
    }

    %-----------------------------------------------------------------------
    % Ueberprueft, ob die Breite des uebergebenen Parameternamens groesser
    % ist als die des bisher als am breitesten geltenden Parameternames
    % und passt den Wert evtl. an. Der Parametername steht dabei in
    % Emphasized.
    % Parameter #1:  Neuer Parametername, dessen Breite zum Vergleich dient.
    %
    \newcommand{\setNewMaxParamWidth}[1]{
        \settowidth{\Temp}{{\em #1}}
        \ifthenelse{\lengthtest{\MaxParamWidth < \Temp}}
            {\setlength{\MaxParamWidth}{\Temp}}{}
    }

    %-----------------------------------------------------------------------
    % Setzt die Breite der Beschreibungstexte fuer Rueckgabewert
    % und Besonderheiten.
    %
    \newcommand{\setSpecialRetTxtWidth}{
        \setlength{\SpecialRetTxtWidth}{\textwidth}
        \addtolength{\SpecialRetTxtWidth}{-1\SpecialWidth} 
    }

    %-----------------------------------------------------------------------
    % Setzt die Breite fuer den Beginn einer Methode, d.h. den Text
    % ``Rueckgabewert Methodenname( ``. Rueckgabewert steht dabei
    % in Emphasized, Methodenname in Bold Font.
    % Parameter #1:  Rueckgabewert der Methode.
    % Parameter #2:  Name der Methode.
    %
    \newcommand{\setMethodBeginWidth}[2]{
        \settowidth{\MethodBeginWidth}
            {{\em #1}\InBetween{\bf #2}\MethodBegin}
    }

    %-----------------------------------------------------------------------
    % Setzt die maximale Breite die an Platz zwischen der oeffnenden
    % und der schliessenden Klammer der Methode zur Verfuegung steht.
    %
    \newcommand{\setBetweenParsWidth}{
        \setlength{\BetweenParsWidth}{\textwidth}
        \addtolength{\BetweenParsWidth}{-1\MethodBeginWidth}
        \addtolength{\BetweenParsWidth}{-1\MethodEndWidth}
    }

    %-----------------------------------------------------------------------
    % Setzt die Breite fuer die Beschreibungstexte
    % der Methodenparameter.
    %
    \newcommand{\setParamDescrWidth}{
        \setlength{\ParamDescrWidth}{\textwidth}
        \addtolength{\ParamDescrWidth}{-1\SpecialWidth} 
        \addtolength{\ParamDescrWidth}{-1\MaxParamWidth}
        \addtolength{\ParamDescrWidth}{-1\ParamDelimWidth}
    }    

    %-----------------------------------------------------------------------
    % Setzt die Korrekturbreite fuer den Beschreibungstext von
    % Methodenparametern auf den uebergebenen Wert.
    % Parameter #1:  Neuer Wert fuer die Korrekturbreite.
    %
    \newcommand{\setCorrectWidthOne}[1]{
        \setlength{\CorrectWidthOne}{#1}
    }    

    %-----------------------------------------------------------------------
    % Setzt die Korrekturbreite fuer den Beschreibungstext von
    % Rueckgabewert und Besonderheiten auf den uebergebenen Wert.
    % Parameter #1:  Neuer Wert fuer die Korrekturbreite.
    %
    \newcommand{\setCorrectWidthTwo}[1]{
        \setlength{\CorrectWidthTwo}{#1}
    }    

    %-----------------------------------------------------------------------
    % Setzt die Korrekturbreite fuer den Abstand des zweiten bis n-ten
    % Parameters vom Anfang auf den uebergebenen Wert.
    % Parameter #1:  Neuer Wert fuer die Korrekturbreite.
    %
    \newcommand{\setCorrectWidthThree}[1]{
        \setlength{\CorrectWidthThree}{#1}
    }    

    %-----------------------------------------------------------------------
    % Setzt den vertikalen Abstand zwischen zwei Parboxen auf den ueber-
    % gebenen Wert.
    % Parameter #1:  Neuer Wert fuer den vertikalen Abstand.
    %
    \newcommand{\setVertSpace}[1]{
        \setlength{\VertSpace}{#1}
    }    

    %-----------------------------------------------------------------------
    % Speichert die uebergebenen Informationen ueber Methodenparameter 1
    % in internen Variablen. Achtung: Die Parameter muessen in der richtigen
    % Reihenfolge gesetzt werden!
    % Parameter #1: Der Typ des Parameters.  
    % Parameter #2: Der Name des Parameters. 
    % Parameter #3: Die Beschreibung des Parameters. 
    % 
    \newcommand{\setParamOne}[3]{
        \renewcommand{\ParamOne}{#1}
        \renewcommand{\ParamTypeOne}{#2}
        \renewcommand{\ParamDescrOne}{#3} 
        \setcounter{SavedParams}{0}
        \stepcounter{SavedParams}
    }

    %-----------------------------------------------------------------------
    % Speichert die uebergebenen Informationen ueber Methodenparameter 2
    % in internen Variablen. Achtung: Die Parameter muessen in der richtigen
    % Reihenfolge gesetzt werden!
    % Parameter #1: Der Typ des Parameters.  
    % Parameter #2: Der Name des Parameters. 
    % Parameter #3: Die Beschreibung des Parameters. 
    % 
    \newcommand{\setParamTwo}[3]{
        \renewcommand{\ParamTwo}{#1}
        \renewcommand{\ParamTypeTwo}{#2}
        \renewcommand{\ParamDescrTwo}{#3} 
        \stepcounter{SavedParams}
    }

    %-----------------------------------------------------------------------
    % Speichert die uebergebenen Informationen ueber Methodenparameter 3
    % in internen Variablen. Achtung: Die Parameter muessen in der richtigen
    % Reihenfolge gesetzt werden!
    % Parameter #1: Der Typ des Parameters.  
    % Parameter #2: Der Name des Parameters. 
    % Parameter #3: Die Beschreibung des Parameters. 
    % 
    \newcommand{\setParamThree}[3]{
        \renewcommand{\ParamThree}{#1}
        \renewcommand{\ParamTypeThree}{#2}
        \renewcommand{\ParamDescrThree}{#3} 
        \stepcounter{SavedParams}
    }

    %-----------------------------------------------------------------------
    % Speichert die uebergebenen Informationen ueber Methodenparameter 4
    % in internen Variablen. Achtung: Die Parameter muessen in der richtigen
    % Reihenfolge gesetzt werden!
    % Parameter #1: Der Typ des Parameters.  
    % Parameter #2: Der Name des Parameters. 
    % Parameter #3: Die Beschreibung des Parameters. 
    % 
    \newcommand{\setParamFour}[3]{
        \renewcommand{\ParamFour}{#1}
        \renewcommand{\ParamTypeFour}{#2}
        \renewcommand{\ParamDescrFour}{#3} 
        \stepcounter{SavedParams}
    }

    %-----------------------------------------------------------------------
    % Speichert die uebergebenen Informationen ueber Methodenparameter 5
    % in internen Variablen. Achtung: Die Parameter muessen in der richtigen
    % Reihenfolge gesetzt werden!
    % Parameter #1: Der Typ des Parameters.  
    % Parameter #2: Der Name des Parameters. 
    % Parameter #3: Die Beschreibung des Parameters. 
    % 
    \newcommand{\setParamFive}[3]{
        \renewcommand{\ParamFive}{#1}
        \renewcommand{\ParamTypeFive}{#2}
        \renewcommand{\ParamDescrFive}{#3} 
        \stepcounter{SavedParams}
    }

    %-----------------------------------------------------------------------
    % Speichert die uebergebenen Informationen ueber Methodenparameter 6
    % in internen Variablen. Achtung: Die Parameter muessen in der richtigen
    % Reihenfolge gesetzt werden!
    % Parameter #1: Der Typ des Parameters.  
    % Parameter #2: Der Name des Parameters. 
    % Parameter #3: Die Beschreibung des Parameters. 
    %     
    \newcommand{\setParamSix}[3]{
        \renewcommand{\ParamSix}{#1}
        \renewcommand{\ParamTypeSix}{#2}
        \renewcommand{\ParamDescrSix}{#3} 
        \stepcounter{SavedParams} 
    }

    %-----------------------------------------------------------------------
    % Speichert die uebergebenen Informationen ueber Methodenparameter 7
    % in internen Variablen. Achtung: Die Parameter muessen in der richtigen
    % Reihenfolge gesetzt werden!
    % Parameter #1: Der Typ des Parameters.  
    % Parameter #2: Der Name des Parameters. 
    % Parameter #3: Die Beschreibung des Parameters. 
    %    
    \newcommand{\setParamSeven}[3]{
        \renewcommand{\ParamSeven}{#1}
        \renewcommand{\ParamTypeSeven}{#2}
        \renewcommand{\ParamDescrSeven}{#3} 
        \stepcounter{SavedParams} 
    }

    %-----------------------------------------------------------------------
    % Speichert die uebergebenen Informationen ueber Methodenparameter 8
    % in internen Variablen. Achtung: Die Parameter muessen in der richtigen
    % Reihenfolge gesetzt werden!
    % Parameter #1: Der Typ des Parameters.  
    % Parameter #2: Der Name des Parameters. 
    % Parameter #3: Die Beschreibung des Parameters. 
    % 
    \newcommand{\setParamEight}[3]{
        \renewcommand{\ParamEight}{#1}
        \renewcommand{\ParamTypeEight}{#2}
        \renewcommand{\ParamDescrEight}{#3} 
        \stepcounter{SavedParams}
    }

    %-----------------------------------------------------------------------
    % Speichert die uebergebenen Informationen ueber Methodenparameter 9
    % in internen Variablen. Achtung: Die Parameter muessen in der richtigen
    % Reihenfolge gesetzt werden!
    % Parameter #1: Der Typ des Parameters.  
    % Parameter #2: Der Name des Parameters. 
    % Parameter #3: Die Beschreibung des Parameters. 
    % 
    \newcommand{\setParamNine}[3]{
        \renewcommand{\ParamNine}{#1}
        \renewcommand{\ParamTypeNine}{#2}
        \renewcommand{\ParamDescrNine}{#3}  
        \stepcounter{SavedParams}
    }

    %-----------------------------------------------------------------------
    % Speichert die uebergebenen Informationen ueber Methodenparameter 10
    % in internen Variablen. Achtung: Die Parameter muessen in der richtigen
    % Reihenfolge gesetzt werden!
    % Parameter #1: Der Typ des Parameters.  
    % Parameter #2: Der Name des Parameters. 
    % Parameter #3: Die Beschreibung des Parameters. 
    % 
    \newcommand{\setParamTen}[3]{
        \renewcommand{\ParamTen}{#1}
        \renewcommand{\ParamTypeTen}{#2}
        \renewcommand{\ParamDescrTen}{#3} 
        \stepcounter{SavedParams}
    }

    %-----------------------------------------------------------------------
    % Gibt den Text ``Rueckgabewert Methodenname(`` aus.
    % Parameter #1:  Rueckgabewert der Methode.
    % Parameter #2:  Name der Methode.
    %
    \newcommand{\printMethodBegin}[2]{
         \ifthenelse{\boolean{ConstInstance}}
             {\settowidth{\MethodEndWidth}{\ConstMethodEnd}}
             {\settowidth{\MethodEndWidth}{\NormMethodEnd}}
         \setMethodBeginWidth{#1}{#2}
         \setBetweenParsWidth
         \noindent
         \hrulefill\\
         \noindent
         \parbox[t]{\MethodBeginWidth}
             {#1\InBetween{\bf #2}\MethodBegin}
     }

    %-----------------------------------------------------------------------
    % Gibt das Ende einer Methode aus, d.h. ``\ );'' oder ``\ ) const;''.
    %
    \newcommand{\printMethodEnd}{
        \ifthenelse{\boolean{ConstInstance}}
            {\ConstMethodEnd\\}
            {\NormMethodEnd\\}
            \mbox{}\hrulefill\\
    }

    %-----------------------------------------------------------------------
    % Gibt den Typen und Namen des einzigen Parameters einer Methode
    % gefolgt von dem Text ``  );'' bzw. `` ) const;'' aus.
    % Parameter #1:  Typ des Parameters.
    % Parameter #2:  Name des Parameters.
    % 
    \newcommand{\printMethodOneParam}[2]{
        \parbox[t]{\ParamMaxTypeWidth}{#1}
        \parbox[t]{\MaxParamWidth}{{\em #2}}
        \printMethodEnd
    }

    %-----------------------------------------------------------------------
    % Gibt den Typen und den Namen des ersten Methodenparameters aus.
    % Parameter #1:  Typ des Parameters.
    % Parameter #2:  Name des Parameters.
    %
    \newcommand{\printMethodFirstParam}[2]{
        \noindent
        \parbox[t]{\ParamMaxTypeWidth}{#1}
        \parbox[t]{\MaxParamWidth}{{\em #2},}\\[\VertSpace]
    }

    %-----------------------------------------------------------------------
    % Gibt den Typen und Namen des letzten Parameters einer Methode aus.
    % Parameter #1: Typ des Parameters.
    % Parameter #2: Name des Parameters.
    %
    \newcommand{\printMethodLastParam}[2]{
        \parbox[t]{\MethodBeginWidth}{\hfill}
        \hspace{\CorrectWidthThree}
        \parbox[t]{\ParamMaxTypeWidth}{#1}
        \parbox[t]{\MaxParamWidth}{{\em #2}}
        \printMethodEnd
    }

    %-----------------------------------------------------------------------
    % Gibt den Typen und Namen eines Methodenparameters aus.
    % Parameter #1:  Typ des Parameters.
    % Parameter #2: Name des Parameters.
    % Achtung: Nicht fuer den ersten oder letzten Parameter der Methode
    % geeignet!
    %
    \newcommand{\printMethodParam}[2]{
        \parbox[t]{\MethodBeginWidth}{\hfill}
        \hspace{\CorrectWidthThree}
        \parbox[t]{\ParamMaxTypeWidth}{#1}
        \parbox[t]{\MaxParamWidth}{{\em #2},}\\[\VertSpace]
    }

    %-----------------------------------------------------------------------
    % Gibt die Beschreibung einer Methode aus.
    % Parameter #1: Beschreibung der Methode.
    %
    \newcommand{\printMethodDescr}[1]{
        \newline
        \noindent
        #1\\[\VertSpace]
    }

    %-----------------------------------------------------------------------
    % Gibt den Text ``Parameter:    `` und danach den Namen des
    % ersten Parameters der Methode, gefolgt von seiner Beschreibung
    % aus.
    % Parameter #1:  Name des Parameters. 
    % Parameter #2:  Beschreibung des Parameters.
    %
    \newcommand{\printFirstParam}[2]{
        \setlength{\Temp}{\ParamDescrWidth}
        \addtolength{\Temp}{1\CorrectWidthOne}
        \parbox[t]{\SpecialWidth}{{\bf Parameter:}}
        \parbox[t]{\MaxParamWidth}{\em #1}
        \parbox[t]{\ParamDelimWidth}{\ParamDelim}
        \parbox[t]{\Temp}{#2}\\[\VertSpace]
    }

    %-----------------------------------------------------------------------
    % Gibt einen Parameter der Methode und die dazugehoerige Beschreibung
    % aus.
    % Parameter #1:  Name des Parameters.
    % Parameter #2:  Beschreibung des Parameters.
    %
    \newcommand{\printParam}[2]{
        \setlength{\Temp}{\ParamDescrWidth}
        \addtolength{\Temp}{1\CorrectWidthOne}
        \parbox[t]{\SpecialWidth}{\hfill}
        \parbox[t]{\MaxParamWidth}{\em #1}
        \parbox[t]{\ParamDelimWidth}{\ParamDelim}
        \parbox[t]{\Temp}{#2}\\[\VertSpace]
    }

    %-----------------------------------------------------------------------
    %
    % Gibt den Text ``Rueckgabewert:  `` gefolgt von einer Beschreibung
    % des Rueckgabewertes aus.
    % Parameter #1:  Beschreibung des Rueckgabewertes.
    %
    \newcommand{\printReturn}[1]{
        \setlength{\Temp}{\SpecialRetTxtWidth}
        \addtolength{\Temp}{1\CorrectWidthTwo}
        \parbox[t]{\SpecialWidth}{{\bf R\"uckgabewert:}}
        \parbox[t]{\Temp}{#1}\\[\VertSpace]
    }

    %-----------------------------------------------------------------------
    % Gibt den Text ``Besonderheiten:  `` gefolgt von einer Beschreibung
    % der Besonderheiten aus.
    % Parameter #1:  Beschreibung der Besonderheiten.
    %
    \newcommand{\printSpecial}[1]{
        \setlength{\Temp}{\SpecialRetTxtWidth}
        \addtolength{\Temp}{1\CorrectWidthTwo}
        \parbox[t]{\SpecialWidth}{\MaxLabelTxt}
        \parbox[t]{\Temp}{#1}\\[\VertSpace]
    }

    %-----------------------------------------------------------------------
    % Gibt den Text ``Parameter:        Keine.'' aus.
    %
    \newcommand{\printNoParams}{
        \parbox[t]{\SpecialWidth}{{\bf Parameter:}}
        \parbox[t]{\NoneWidthOne}{Keine.}\\[\VertSpace]
    }

    %-----------------------------------------------------------------------
    % Gibt den Text ``Rueckgabewert:  Keiner.'' aus.
    %
    \newcommand{\printNoReturn}{
        \parbox[t]{\SpecialWidth}{{\bf R\"uckgabewert:}}
        \parbox[t]{\NoneWidthTwo}{Keiner.}\\[\VertSpace]
    }

    %-----------------------------------------------------------------------
    % Gibt den Text ``Besonderheiten:  Keine.'' aus.
    %
    \newcommand{\printNoSpecial}{
        \parbox[t]{\SpecialWidth}{\MaxLabelTxt}
        \parbox[t]{\NoneWidthOne}{Keine.}\\[\VertSpace]
    }

    %-----------------------------------------------------------------------
    % Gibt den Text:
    % ``Parameter:       Keine.
    %   Rueckgabewert:   Keiner.
    %   Besonderheiten:  Keine.''
    % aus.
    %
    \newcommand{\printNoAll}{
        \printNoParams
        \printNoReturn
        \printNoSpecial
    }

    %-----------------------------------------------------------------------
    % Gibt eine Methode ohne Parameter, Rueckgabewert und Besonderheiten
    % aus.
    % Parameter #1:  Name der Methode.
    % Parameter #2:  Beschreibung der Methode.
    %
    \newcommand{\printEmptyMethod}[2]{
        \setSpecialRetTxtWidth
        \printMethodBegin{void}{#1}
        \printMethodEnd
        \printMethodDescr{#2}
        \printNoAll
    }

    %-----------------------------------------------------------------------
    % Gibt eine Methode ohne Parameter und Besonderheiten, aber mit
    % Rueckgabewert aus.
    % Parameter #1:  Rueckgabewert der Methode.
    % Parameter #2:  Name der Methode.
    % Parameter #3:  Beschreibung der Methode.
    % Parameter #4:  Beschreibung des Rueckgabewertes der Methode.
    %
    \newcommand{\printEmptyMethodReturn}[4]{
        \setSpecialRetTxtWidth
        \printMethodBegin{#1}{#2}
        \printMethodEnd
        \printMethodDescr{#3}
        \printNoParams
        \printReturn{#4}
        \printNoSpecial
    }

    %-----------------------------------------------------------------------
    % Gibt eine Methode ohne Parameter und Rueckgabewert, aber mit
    % Besonderheiten aus.
    % Parameter #1:  Name der Methode.
    % Parameter #2:  Beschreibung der Methode.
    % Parameter #3:  Beschreibung der Besonderheiten der Methode.
    %
    \newcommand{\printEmptyMethodSpecial}[3]{
        \setSpecialRetTxtWidth
        \printMethodBegin{void}{#1}
        \printMethodEnd
        \printMethodDescr{#2}
        \printNoParams
        \printNoReturn
        \printSpecial{#3}
    }

    %-----------------------------------------------------------------------
    % Gibt eine Methode ohne Parameter, aber mit Rueckgabewert
    % und Besonderheiten aus.
    % Parameter #1:  Rueckgabewert der Methode.
    % Parameter #2:  Name der Methode.
    % Parameter #3:  Beschreibung der Methode.
    % Parameter #4:  Beschreibung des Rueckgabewertes der Methode.
    % Parameter #5:  Beschreibung der Besonderheiten der Methode.
    %
    \newcommand{\printEmptyMethodReturnSpecial}[5]{
        \setSpecialRetTxtWidth
        \printMethodBegin{#1}{#2}
        \printMethodEnd
        \printMethodDescr{#3}
        \printNoParams
        \printReturn{#4}
        \printSpecial{#5}
    }

    %-----------------------------------------------------------------------
    % Gibt eine Methode mit einem Parameter komplett mit allen
    % Beschreibungen aus.
    % Parameter #1:  Rueckgabewert der Methode.
    % Parameter #2:  Name der Methode.
    % Parameter #3:  Typ des einzigen Parameters der Methode.
    % Parameter #4:  Name des einzigen Parameters der Methode.
    % Parameter #5:  Beschreibung des einzigen Parameters der Methode.
    % Parameter #6:  Beschreibung der Methode.
    % Parameter #7:  Beschreibung des Rueckgabewertes der Methode.
    % Parameter #8:  Beschreibung der Besonderheiten der Methode.
    %
    \newcommand{\printMethodWithOneParam}[8]{
        \initParamMaxTypeWidth
        \initMaxParamWidth
        \setNewParamMaxTypeWidth{#3}
        \setNewMaxParamWidth{#4}
        \setSpecialRetTxtWidth
        \setParamDescrWidth
        \printMethodBegin{#1}{#2}
        \printMethodOneParam{#3}{#4}
        \printMethodDescr{#6}
        \printFirstParam{#4}{#5}
        \ifthenelse{\equal{#7}{Keiner.}}
            {\printNoReturn}
            {\printReturn{#7}}
        \ifthenelse{\equal{#8}{Keine.}}
            {\printNoSpecial}
            {\printSpecial{#8}}
    }

    %-----------------------------------------------------------------------
    % Gibt eine Methode mit einem Parameter komplett mit allen
    % Beschreibungen aus.
    % Parameter #1:  Rueckgabewert der Methode.
    % Parameter #2:  Beschreibung des Rueckgabewertes der Methode oder 
    %                Aufruf mit leerer Klammer. 
    % Parameter #3:  Name der Methode.
    % Parameter #4:  Beschreibung der Methode.
    % Parameter #5:  Beschreibung der Besonderheiten der Methode oder 
    %                Aufruf mit leerer Klammer.
    %
    \newcommand{\printMethodWithParamsSaved}[5]{
        \ifthenelse{\value{SavedParams} > 0}
            {\initParamMaxTypeWidth
             \initMaxParamWidth
             \setNewParamMaxTypeWidth{\ParamTypeOne}
             \setNewMaxParamWidth{\ParamOne}
             \ifthenelse{\value{SavedParams} = 2 \or \value{SavedParams} > 2}
                 {\setNewParamMaxTypeWidth{\ParamTypeTwo}
                  \setNewMaxParamWidth{\ParamTwo}}{}   
             \ifthenelse{\value{SavedParams} = 3 \or \value{SavedParams} > 3}
                 {\setNewParamMaxTypeWidth{\ParamTypeThree}
                  \setNewMaxParamWidth{\ParamThree}}{}    
             \ifthenelse{\value{SavedParams} = 4 \or \value{SavedParams} > 4}
                 {\setNewParamMaxTypeWidth{\ParamTypeFour}
                  \setNewMaxParamWidth{\ParamFour}}{}
             \ifthenelse{\value{SavedParams} = 5 \or \value{SavedParams} > 5}
                 {\setNewParamMaxTypeWidth{\ParamTypeFive}
                  \setNewMaxParamWidth{\ParamFive}}{}      
             \ifthenelse{\value{SavedParams} = 6 \or \value{SavedParams} > 6}
                 {\setNewParamMaxTypeWidth{\ParamTypeSix}
                  \setNewMaxParamWidth{\ParamSix}}{}
             \ifthenelse{\value{SavedParams} = 7 \or \value{SavedParams} > 7}
                 {\setNewParamMaxTypeWidth{\ParamTypeSeven}
                  \setNewMaxParamWidth{\ParamSeven}}{}       
             \ifthenelse{\value{SavedParams} = 8 \or \value{SavedParams} > 8}
                 {\setNewParamMaxTypeWidth{\ParamTypeEight}
                  \setNewMaxParamWidth{\ParamEight}}{}
             \ifthenelse{\value{SavedParams} = 9 \or \value{SavedParams} > 9}
                 {\setNewParamMaxTypeWidth{\ParamTypeNine}
                  \setNewMaxParamWidth{\ParamNine}}{}
             \ifthenelse{\value{SavedParams} = 10 \or 
                         \value{SavedParams} > 10}
                 {\setNewParamMaxTypeWidth{\ParamTypeTen}
                  \setNewMaxParamWidth{\ParamTen}}{}         
             \setSpecialRetTxtWidth
             \setParamDescrWidth
             \printMethodBegin{#1}{#3}
             \ifthenelse{\value{SavedParams} = 1} 
                 {\printMethodOneParam{\ParamTypeOne}{\ParamOne}}{}
                 {\printMethodFirstParam{\ParamTypeOne}{\ParamOne}}{}
             \ifthenelse{\value{SavedParams} = 2} 
                 {\printMethodLastParam{\ParamTypeTwo}{\ParamTwo}}{}
             \ifthenelse{\value{SavedParams} > 2}  
                 {\printMethodParam{\ParamTypeTwo}{\ParamTwo}}{} 
             \ifthenelse{\value{SavedParams} = 3} 
                 {\printMethodLastParam{\ParamTypeThree}{\ParamThree}}{}
             \ifthenelse{\value{SavedParams} > 3}  
                 {\printMethodParam{\ParamTypeThree}{\ParamThree}}{} 
             \ifthenelse{\value{SavedParams} = 4} 
                 {\printMethodLastParam{\ParamTypeFour}{\ParamFour}}{}
             \ifthenelse{\value{SavedParams} > 4}  
                 {\printMethodParam{\ParamTypeFour}{\ParamFour}}{} 
             \ifthenelse{\value{SavedParams} = 5} 
                 {\printMethodLastParam{\ParamTypeFive}{\ParamFive}}{}
             \ifthenelse{\value{SavedParams} > 5}  
                 {\printMethodParam{\ParamTypeFive}{\ParamFive}}{} 
             \ifthenelse{\value{SavedParams} = 6} 
                 {\printMethodLastParam{\ParamTypeSix}{\ParamSix}}{}
             \ifthenelse{\value{SavedParams} > 6}  
                 {\printMethodParam{\ParamTypeSix}{\ParamSix}}{} 
             \ifthenelse{\value{SavedParams} = 7} 
                 {\printMethodLastParam{\ParamTypeSeven}{\ParamSeven}}{}
             \ifthenelse{\value{SavedParams} > 7}  
                 {\printMethodParam{\ParamTypeSeven}{\ParamSeven}}{} 
             \ifthenelse{\value{SavedParams} = 8} 
                 {\printMethodLastParam{\ParamTypeEight}{\ParamEight}}{}
             \ifthenelse{\value{SavedParams} > 8}  
                 {\printMethodParam{\ParamTypeEight}{\ParamEight}}{}  
             \ifthenelse{\value{SavedParams} = 9} 
                 {\printMethodLastParam{\ParamTypeNine}{\ParamNine}}{}
             \ifthenelse{\value{SavedParams} > 9}  
                 {\printMethodParam{\ParamTypeNine}{\ParamNine}}{} 
             \ifthenelse{\value{SavedParams} = 10} 
                 {\printMethodLastParam{\ParamTypeTen}{\ParamTen}}{}
             \printMethodDescr{#4}
             \printFirstParam{\ParamOne}{\ParamDescrOne}
             \ifthenelse{\value{SavedParams} = 2 \or \value{SavedParams} > 2}
                  {\printParam{\ParamTwo}{\ParamDescrTwo}}{}
             \ifthenelse{\value{SavedParams} = 3 \or \value{SavedParams} > 3}
                  {\printParam{\ParamThree}{\ParamDescrThree}}{}
             \ifthenelse{\value{SavedParams} = 4 \or \value{SavedParams} > 4}
                  {\printParam{\ParamFour}{\ParamDescrFour}}{}
             \ifthenelse{\value{SavedParams} = 5 \or \value{SavedParams} > 5}
                  {\printParam{\ParamFive}{\ParamDescrFive}}{}
             \ifthenelse{\value{SavedParams} = 6 \or \value{SavedParams} > 6}
                  {\printParam{\ParamSix}{\ParamDescrSix}}{}
             \ifthenelse{\value{SavedParams} = 7 \or \value{SavedParams} > 7}
                  {\printParam{\ParamSeven}{\ParamDescrSeven}}{}
             \ifthenelse{\value{SavedParams} = 8 \or \value{SavedParams} > 8}
                  {\printParam{\ParamEight}{\ParamDescrEight}}{} 
             \ifthenelse{\value{SavedParams} = 9 \or \value{SavedParams} > 9}
                  {\printParam{\ParamNine}{\ParamDescrNine}}{}
             \ifthenelse{\value{SavedParams} = 10}
                  {\printParam{\ParamTen}{\ParamDescrTen}}{}
             \ifthenelse{\equal{#2}{}}
                 {\printNoReturn}
                 {\printReturn{#2}}
             \ifthenelse{\equal{#5}{}}
                 {\printNoSpecial}
                 {\printSpecial{#5}}}
        {\noindent
         Fehler Methode {\em printMethodWithParamsSaved}: Es sind noch keine 
         Parameter gesetzt worden!}
    }

%-----------------------------------------------------------------------
% Ende der Befehlsdefinitionen.
%-----------------------------------------------------------------------





%%% Local Variables: 
%%% mode: latex
%%% TeX-master: t
%%% End: 
'' in LaTex-Dokumente 
% (Latex2Epsilon) eingebunden und stellt Befehle zur einfachen Doku- 
% mentation von C-Funktionen/C++-Methoden bereit.
% =======================================================================

\usepackage{ifthen}

    % Definition von Konstanten:
    \newcommand{\ParamDelim}            % Trennungstext zwischen
        { \ - }                         % Parametername und -beschreibung.
    \newcommand{\NormMethodEnd}         % Endtext von Methode bei 
        {\ \ );}                        % veraenderbarer Aufrufinstanz.
    \newcommand{\ConstMethodEnd}        % Endtext von Methode bei
        {\ \ ) const;}                  % nicht veraenderbarer Aufrufinstanz.
    \newcommand{\MaxLabelTxt}           % Komponentenbeschreibungstext
        {{\bf Besonderheiten: }}        % mit maximaler Breite.
    \newcommand{\MethodBegin}           % Beginntext einer Methode.
        {(\ }
    \newcommand{\InBetween}             % Text zwischen Rueckgabewert und
        {\ \ }                          % Methodenname.

    % Definition eigener Variablen:


    \newboolean{ConstInstance}          % Gibt an, ob die Instanz, die
                                        % die Methode aufruft veraendert
                                        % werden darf oder nicht. Im
                                        % 2. Fall endet die Methode mit
                                        % ``) const;'', im 1. Fall 
                                        % mit ``);''.
    \newcounter{SavedParams}            % Anzahl der gespeicherten
                                        % Parameter.
    \setcounter{SavedParams}{0}
    \newcommand{\ParamOne}{}            % Definition der 10 
    \newcommand{\ParamTypeOne}{}        % abspeicherbaren Parameter,
    \newcommand{\ParamDescrOne}{}       % ihrer Typen und Beschreibungen.
    \newcommand{\ParamTwo}{}            
    \newcommand{\ParamTypeTwo}{}
    \newcommand{\ParamDescrTwo}{}
    \newcommand{\ParamThree}{}
    \newcommand{\ParamTypeThree}{}
    \newcommand{\ParamDescrThree}{}
    \newcommand{\ParamFour}{}
    \newcommand{\ParamTypeFour}{}
    \newcommand{\ParamDescrFour}{}
    \newcommand{\ParamFive}{}
    \newcommand{\ParamTypeFive}{}
    \newcommand{\ParamDescrFive}{}
    \newcommand{\ParamSix}{}
    \newcommand{\ParamTypeSix}{}
    \newcommand{\ParamDescrSix}{}
    \newcommand{\ParamSeven}{}
    \newcommand{\ParamTypeSeven}{}
    \newcommand{\ParamDescrSeven}{}
    \newcommand{\ParamEight}{}
    \newcommand{\ParamTypeEight}{}
    \newcommand{\ParamDescrEight}{}
    \newcommand{\ParamNine}{}
    \newcommand{\ParamTypeNine}{}
    \newcommand{\ParamDescrNine}{}
    \newcommand{\ParamTen}{}
    \newcommand{\ParamTypeTen}{}
    \newcommand{\ParamDescrTen}{}
    \newlength{\MaxParamWidth}          % Breite des breitesten 
                                        % Methodenparameters.
    \newlength{\SpecialWidth}           % Breite des Textes 
    \settowidth{\SpecialWidth}          % {\bf Besonderheiten}.
        {\MaxLabelTxt}
    \newlength{\ParamDelimWidth}        % Breite des Textes `` \ - ``.
    \settowidth{\ParamDelimWidth}
        {\ParamDelim}
    \newlength{\ParamDescrWidth}        % Max. Breite des Parametertextes.
    \newlength{\SpecialRetTxtWidth}     % Max. Breite der Texte fuer
                                        % Besonderheiten und Rueckgabewert.
    \newlength{\MethodBeginWidth}       % Breite des Methodenanfangs, d.h.
                                        % ``Rueckgabewert Methodenname ( ``.
    \newlength{\BetweenParsWidth}       % Max. zur Verfuegung stehende
                                        % Breite zwischen den beiden Klammern
                                        % der Methode.
    
    \newlength{\MethodEndWidth}         % Breite des Textes ``  );'' bzw.
    \settowidth{\MethodEndWidth}        % `` ) const;''
        {\NormMethodEnd}
    \newlength{\ParamMaxTypeWidth}      % Breite des breitesten
                                        % Parameterwert-Typs. 
    \newlength{\NoneWidthOne}           % Breite des Textes ``Keine.''.
    \settowidth{\NoneWidthOne}{Keine.}
    \newlength{\NoneWidthTwo}           % Breite des Textes ``Keiner.''.
    \settowidth{\NoneWidthTwo}{Keiner.}
    \newlength{\Temp}                   % Temporaere Hilfsvariable. 
    \newlength{\VertSpace}              % Vertikaler Abstand zwischen 
                                        % zwei Parboxen.
   
    % Trotz der automatischen Berechnung der Textbreiten treten
    % kleine Unstimmigkeiten von wenigen Punkten bei den Texten
    % zur Beschreibung der Parameter, des Rueckgabewertes und
    % der Besonderheiten auf, die mit den folgenden Variablen
    % behoben werden koennen:

    \newlength{\CorrectWidthOne}        % Korrekturbreite fuer die
                                        % Parameterbeschreibung.
    \newlength{\CorrectWidthTwo}        % Korrekturbreite fuer die
                                        % Beschreibungen von Rueckgabewert
                                        % und Besonderheiten.
    \newlength{\CorrectWidthThree}      % Korrekturbreite, da erster
                                        % Parameter einer Methode weiter
                                        % vorrueckt als die anderen.
    

% Uebersicht ueber die Parameter und ihre Bedeutung.


%
% <-- \MethodBeginWidth    -->
%------------------------------------------------------------------------
% Rueckgabewert Methodenname(   Typ_Parameter_1 Parameter_1,
%                            <->Typ_Parameter_2 Parameter_2,
%              \CorrectWideThree...
%                               Typ_Parameter_n Parameter_n   );
%                             <-------------> <--------->
%                          \ParamMaxTypeWidth \MaxParamWidth
%                             <-------------------------->
%                             \BetweenParsWidth
%                                                         <-->
%                                                         \MethodEndWidth
%------------------------------------------------------------------------
% Beschreibung der Methode...
%                             \ParamDelimWidth
%                             <->                        \CorrectWidthOne
% \VertSpace
% Parameter:       Parameter_1  - Beschreibung von Parameter 1.   <----->
%                  Parameter_2  - Beschreibung von Parameter 2.
%                  ...
%                  Parameter_n  - Beschreibung von Parameter n.
% \VertSpace                      \ParamDescrWidth
%                                 <------------------------------>
% Rueckgabewert:   Beschreibung des Rueckgabewertes bzw. der Text <-----> 
%                  Keiner.                   
%                  <----->
%                  \NoneWidthTwo               \SpecialRetTxtWidth
% \VertSpace       <--------------------------------------------->
% Besonderheiten:  Beschreibung von Besonderheiten bzw. der Text  <----->
%                  Keine.     
% <--------------> <---->
%  \SpecialWidth   \NoneWidthOne                         \CorrectWidthTwo


    % Definition eigener Befehle:


    %-----------------------------------------------------------------------
    % Gibt den Begriff ``C++'' in ansprechender Form aus.
    % Makro wurde freundlicherweise zur Verfuegung gestellt von
    % Axel W. Dietrich.
    %
    \newcommand{\cpp}{
        \mbox{\emph{\textrm{C\hspace{-1.5pt}\raisebox{1.75pt}{\scriptsize +}
        \hspace{-6pt}\raisebox{.75pt}{\scriptsize +}}}}%
    }


    %-----------------------------------------------------------------------
    % Gibt die Werte aller Parameter auf dem Bildschirm aus.
    %
    \newcommand{\showAllParams}{
        \noindent
        {\em Werte aller Parameter:}
        \begin{tabbing}
            ParamMaxTypeWidth:\ \ \=\kill
            MaxParamWidth:       \>\the\MaxParamWidth\\
            SpecialWidth:        \>\the\SpecialWidth\\
            ParamDelimWidth:     \>\the\ParamDelimWidth\\
            ParamDescrWidth:     \>\the\ParamDescrWidth\\
            SpecialRetTxtWidth:  \>\the\SpecialRetTxtWidth\\
            MethodBeginWidth:    \>\the\MethodBeginWidth\\
            BetweenParsWidth:    \>\the\BetweenParsWidth\\
            MethodEndWidth:      \>\the\MethodEndWidth\\
            ParamMaxTypeWidth:   \>\the\ParamMaxTypeWidth\\
            NoneWidthOne:        \>\the\NoneWidthOne\\
            NoneWidthTwo:        \>\the\NoneWidthTwo\\
            VertSpace:           \>\the\VertSpace\\
            CorrectWidthOne:     \>\the\CorrectWidthOne\\
            CorrectWidthTwo:     \>\the\CorrectWidthTwo\\
            CorrectWidthThree:   \>\the\CorrectWidthThree\\
            ConstInstance:       
            \ifthenelse{\boolean{ConstInstance}}
                {\>ja\\}
                {\>nein\\}
        \end{tabbing}
    }

    %-----------------------------------------------------------------------
    % Setzt einen Schalter, der angibt, dass die Instanz, die die kommende
    % Methode aufrufen kann, nicht veraendert werden darf. Dies hat fuer
    % die Dokumentation zur Folge, dass die Methode mit ``) const;'' endet.
    %
    \newcommand{\setConstInstance}{    
        \setboolean{ConstInstance}{true}
    }

    %-----------------------------------------------------------------------
    % Setzt einen Schalter, der angibt, dass die Instanz, die die kommende
    % Methode aufrufen kann, veraendert werden darf. Die Methode endet also
    % in der Dokumentation normal mit ``);''.
    %
    \newcommand{\setNormalInstance}{    
        \setboolean{ConstInstance}{false}
    }

    %-----------------------------------------------------------------------
    % Initialisiert die Breite des breitesten Methodenparametertyptextes
    % mit Null.
    %
    \newcommand{\initParamMaxTypeWidth}{
        \setlength{\ParamMaxTypeWidth}{0pt}
    }

    %-----------------------------------------------------------------------
    % Initialisiert die Breite des breitesten Parameternamens mit Null.
    %
    \newcommand{\initMaxParamWidth}{
        \setlength{\MaxParamWidth}{0pt}
    }

    %-----------------------------------------------------------------------
    % Ueberprueft, ob die Breite des uebergebenen Textes des Parametertyps
    % groesser ist als die des bisher als breitester Typtext festgelegten
    % Textes und passt den Wert evtl. an.
    % Parameter #1:  Neuer Typtext, dessen Breite zum Vergleich dient.
    %   
    \newcommand{\setNewParamMaxTypeWidth}[1]{
        \settowidth{\Temp}{#1}
        \ifthenelse{\lengthtest{\ParamMaxTypeWidth < \Temp}}
            {\setlength{\ParamMaxTypeWidth}{\Temp}}{}   
    }

    %-----------------------------------------------------------------------
    % Ueberprueft, ob die Breite des uebergebenen Parameternamens groesser
    % ist als die des bisher als am breitesten geltenden Parameternames
    % und passt den Wert evtl. an. Der Parametername steht dabei in
    % Emphasized.
    % Parameter #1:  Neuer Parametername, dessen Breite zum Vergleich dient.
    %
    \newcommand{\setNewMaxParamWidth}[1]{
        \settowidth{\Temp}{{\em #1}}
        \ifthenelse{\lengthtest{\MaxParamWidth < \Temp}}
            {\setlength{\MaxParamWidth}{\Temp}}{}
    }

    %-----------------------------------------------------------------------
    % Setzt die Breite der Beschreibungstexte fuer Rueckgabewert
    % und Besonderheiten.
    %
    \newcommand{\setSpecialRetTxtWidth}{
        \setlength{\SpecialRetTxtWidth}{\textwidth}
        \addtolength{\SpecialRetTxtWidth}{-1\SpecialWidth} 
    }

    %-----------------------------------------------------------------------
    % Setzt die Breite fuer den Beginn einer Methode, d.h. den Text
    % ``Rueckgabewert Methodenname( ``. Rueckgabewert steht dabei
    % in Emphasized, Methodenname in Bold Font.
    % Parameter #1:  Rueckgabewert der Methode.
    % Parameter #2:  Name der Methode.
    %
    \newcommand{\setMethodBeginWidth}[2]{
        \settowidth{\MethodBeginWidth}
            {{\em #1}\InBetween{\bf #2}\MethodBegin}
    }

    %-----------------------------------------------------------------------
    % Setzt die maximale Breite die an Platz zwischen der oeffnenden
    % und der schliessenden Klammer der Methode zur Verfuegung steht.
    %
    \newcommand{\setBetweenParsWidth}{
        \setlength{\BetweenParsWidth}{\textwidth}
        \addtolength{\BetweenParsWidth}{-1\MethodBeginWidth}
        \addtolength{\BetweenParsWidth}{-1\MethodEndWidth}
    }

    %-----------------------------------------------------------------------
    % Setzt die Breite fuer die Beschreibungstexte
    % der Methodenparameter.
    %
    \newcommand{\setParamDescrWidth}{
        \setlength{\ParamDescrWidth}{\textwidth}
        \addtolength{\ParamDescrWidth}{-1\SpecialWidth} 
        \addtolength{\ParamDescrWidth}{-1\MaxParamWidth}
        \addtolength{\ParamDescrWidth}{-1\ParamDelimWidth}
    }    

    %-----------------------------------------------------------------------
    % Setzt die Korrekturbreite fuer den Beschreibungstext von
    % Methodenparametern auf den uebergebenen Wert.
    % Parameter #1:  Neuer Wert fuer die Korrekturbreite.
    %
    \newcommand{\setCorrectWidthOne}[1]{
        \setlength{\CorrectWidthOne}{#1}
    }    

    %-----------------------------------------------------------------------
    % Setzt die Korrekturbreite fuer den Beschreibungstext von
    % Rueckgabewert und Besonderheiten auf den uebergebenen Wert.
    % Parameter #1:  Neuer Wert fuer die Korrekturbreite.
    %
    \newcommand{\setCorrectWidthTwo}[1]{
        \setlength{\CorrectWidthTwo}{#1}
    }    

    %-----------------------------------------------------------------------
    % Setzt die Korrekturbreite fuer den Abstand des zweiten bis n-ten
    % Parameters vom Anfang auf den uebergebenen Wert.
    % Parameter #1:  Neuer Wert fuer die Korrekturbreite.
    %
    \newcommand{\setCorrectWidthThree}[1]{
        \setlength{\CorrectWidthThree}{#1}
    }    

    %-----------------------------------------------------------------------
    % Setzt den vertikalen Abstand zwischen zwei Parboxen auf den ueber-
    % gebenen Wert.
    % Parameter #1:  Neuer Wert fuer den vertikalen Abstand.
    %
    \newcommand{\setVertSpace}[1]{
        \setlength{\VertSpace}{#1}
    }    

    %-----------------------------------------------------------------------
    % Speichert die uebergebenen Informationen ueber Methodenparameter 1
    % in internen Variablen. Achtung: Die Parameter muessen in der richtigen
    % Reihenfolge gesetzt werden!
    % Parameter #1: Der Typ des Parameters.  
    % Parameter #2: Der Name des Parameters. 
    % Parameter #3: Die Beschreibung des Parameters. 
    % 
    \newcommand{\setParamOne}[3]{
        \renewcommand{\ParamOne}{#1}
        \renewcommand{\ParamTypeOne}{#2}
        \renewcommand{\ParamDescrOne}{#3} 
        \setcounter{SavedParams}{0}
        \stepcounter{SavedParams}
    }

    %-----------------------------------------------------------------------
    % Speichert die uebergebenen Informationen ueber Methodenparameter 2
    % in internen Variablen. Achtung: Die Parameter muessen in der richtigen
    % Reihenfolge gesetzt werden!
    % Parameter #1: Der Typ des Parameters.  
    % Parameter #2: Der Name des Parameters. 
    % Parameter #3: Die Beschreibung des Parameters. 
    % 
    \newcommand{\setParamTwo}[3]{
        \renewcommand{\ParamTwo}{#1}
        \renewcommand{\ParamTypeTwo}{#2}
        \renewcommand{\ParamDescrTwo}{#3} 
        \stepcounter{SavedParams}
    }

    %-----------------------------------------------------------------------
    % Speichert die uebergebenen Informationen ueber Methodenparameter 3
    % in internen Variablen. Achtung: Die Parameter muessen in der richtigen
    % Reihenfolge gesetzt werden!
    % Parameter #1: Der Typ des Parameters.  
    % Parameter #2: Der Name des Parameters. 
    % Parameter #3: Die Beschreibung des Parameters. 
    % 
    \newcommand{\setParamThree}[3]{
        \renewcommand{\ParamThree}{#1}
        \renewcommand{\ParamTypeThree}{#2}
        \renewcommand{\ParamDescrThree}{#3} 
        \stepcounter{SavedParams}
    }

    %-----------------------------------------------------------------------
    % Speichert die uebergebenen Informationen ueber Methodenparameter 4
    % in internen Variablen. Achtung: Die Parameter muessen in der richtigen
    % Reihenfolge gesetzt werden!
    % Parameter #1: Der Typ des Parameters.  
    % Parameter #2: Der Name des Parameters. 
    % Parameter #3: Die Beschreibung des Parameters. 
    % 
    \newcommand{\setParamFour}[3]{
        \renewcommand{\ParamFour}{#1}
        \renewcommand{\ParamTypeFour}{#2}
        \renewcommand{\ParamDescrFour}{#3} 
        \stepcounter{SavedParams}
    }

    %-----------------------------------------------------------------------
    % Speichert die uebergebenen Informationen ueber Methodenparameter 5
    % in internen Variablen. Achtung: Die Parameter muessen in der richtigen
    % Reihenfolge gesetzt werden!
    % Parameter #1: Der Typ des Parameters.  
    % Parameter #2: Der Name des Parameters. 
    % Parameter #3: Die Beschreibung des Parameters. 
    % 
    \newcommand{\setParamFive}[3]{
        \renewcommand{\ParamFive}{#1}
        \renewcommand{\ParamTypeFive}{#2}
        \renewcommand{\ParamDescrFive}{#3} 
        \stepcounter{SavedParams}
    }

    %-----------------------------------------------------------------------
    % Speichert die uebergebenen Informationen ueber Methodenparameter 6
    % in internen Variablen. Achtung: Die Parameter muessen in der richtigen
    % Reihenfolge gesetzt werden!
    % Parameter #1: Der Typ des Parameters.  
    % Parameter #2: Der Name des Parameters. 
    % Parameter #3: Die Beschreibung des Parameters. 
    %     
    \newcommand{\setParamSix}[3]{
        \renewcommand{\ParamSix}{#1}
        \renewcommand{\ParamTypeSix}{#2}
        \renewcommand{\ParamDescrSix}{#3} 
        \stepcounter{SavedParams} 
    }

    %-----------------------------------------------------------------------
    % Speichert die uebergebenen Informationen ueber Methodenparameter 7
    % in internen Variablen. Achtung: Die Parameter muessen in der richtigen
    % Reihenfolge gesetzt werden!
    % Parameter #1: Der Typ des Parameters.  
    % Parameter #2: Der Name des Parameters. 
    % Parameter #3: Die Beschreibung des Parameters. 
    %    
    \newcommand{\setParamSeven}[3]{
        \renewcommand{\ParamSeven}{#1}
        \renewcommand{\ParamTypeSeven}{#2}
        \renewcommand{\ParamDescrSeven}{#3} 
        \stepcounter{SavedParams} 
    }

    %-----------------------------------------------------------------------
    % Speichert die uebergebenen Informationen ueber Methodenparameter 8
    % in internen Variablen. Achtung: Die Parameter muessen in der richtigen
    % Reihenfolge gesetzt werden!
    % Parameter #1: Der Typ des Parameters.  
    % Parameter #2: Der Name des Parameters. 
    % Parameter #3: Die Beschreibung des Parameters. 
    % 
    \newcommand{\setParamEight}[3]{
        \renewcommand{\ParamEight}{#1}
        \renewcommand{\ParamTypeEight}{#2}
        \renewcommand{\ParamDescrEight}{#3} 
        \stepcounter{SavedParams}
    }

    %-----------------------------------------------------------------------
    % Speichert die uebergebenen Informationen ueber Methodenparameter 9
    % in internen Variablen. Achtung: Die Parameter muessen in der richtigen
    % Reihenfolge gesetzt werden!
    % Parameter #1: Der Typ des Parameters.  
    % Parameter #2: Der Name des Parameters. 
    % Parameter #3: Die Beschreibung des Parameters. 
    % 
    \newcommand{\setParamNine}[3]{
        \renewcommand{\ParamNine}{#1}
        \renewcommand{\ParamTypeNine}{#2}
        \renewcommand{\ParamDescrNine}{#3}  
        \stepcounter{SavedParams}
    }

    %-----------------------------------------------------------------------
    % Speichert die uebergebenen Informationen ueber Methodenparameter 10
    % in internen Variablen. Achtung: Die Parameter muessen in der richtigen
    % Reihenfolge gesetzt werden!
    % Parameter #1: Der Typ des Parameters.  
    % Parameter #2: Der Name des Parameters. 
    % Parameter #3: Die Beschreibung des Parameters. 
    % 
    \newcommand{\setParamTen}[3]{
        \renewcommand{\ParamTen}{#1}
        \renewcommand{\ParamTypeTen}{#2}
        \renewcommand{\ParamDescrTen}{#3} 
        \stepcounter{SavedParams}
    }

    %-----------------------------------------------------------------------
    % Gibt den Text ``Rueckgabewert Methodenname(`` aus.
    % Parameter #1:  Rueckgabewert der Methode.
    % Parameter #2:  Name der Methode.
    %
    \newcommand{\printMethodBegin}[2]{
         \ifthenelse{\boolean{ConstInstance}}
             {\settowidth{\MethodEndWidth}{\ConstMethodEnd}}
             {\settowidth{\MethodEndWidth}{\NormMethodEnd}}
         \setMethodBeginWidth{#1}{#2}
         \setBetweenParsWidth
         \noindent
         \hrulefill\\
         \noindent
         \parbox[t]{\MethodBeginWidth}
             {#1\InBetween{\bf #2}\MethodBegin}
     }

    %-----------------------------------------------------------------------
    % Gibt das Ende einer Methode aus, d.h. ``\ );'' oder ``\ ) const;''.
    %
    \newcommand{\printMethodEnd}{
        \ifthenelse{\boolean{ConstInstance}}
            {\ConstMethodEnd\\}
            {\NormMethodEnd\\}
            \mbox{}\hrulefill\\
    }

    %-----------------------------------------------------------------------
    % Gibt den Typen und Namen des einzigen Parameters einer Methode
    % gefolgt von dem Text ``  );'' bzw. `` ) const;'' aus.
    % Parameter #1:  Typ des Parameters.
    % Parameter #2:  Name des Parameters.
    % 
    \newcommand{\printMethodOneParam}[2]{
        \parbox[t]{\ParamMaxTypeWidth}{#1}
        \parbox[t]{\MaxParamWidth}{{\em #2}}
        \printMethodEnd
    }

    %-----------------------------------------------------------------------
    % Gibt den Typen und den Namen des ersten Methodenparameters aus.
    % Parameter #1:  Typ des Parameters.
    % Parameter #2:  Name des Parameters.
    %
    \newcommand{\printMethodFirstParam}[2]{
        \noindent
        \parbox[t]{\ParamMaxTypeWidth}{#1}
        \parbox[t]{\MaxParamWidth}{{\em #2},}\\[\VertSpace]
    }

    %-----------------------------------------------------------------------
    % Gibt den Typen und Namen des letzten Parameters einer Methode aus.
    % Parameter #1: Typ des Parameters.
    % Parameter #2: Name des Parameters.
    %
    \newcommand{\printMethodLastParam}[2]{
        \parbox[t]{\MethodBeginWidth}{\hfill}
        \hspace{\CorrectWidthThree}
        \parbox[t]{\ParamMaxTypeWidth}{#1}
        \parbox[t]{\MaxParamWidth}{{\em #2}}
        \printMethodEnd
    }

    %-----------------------------------------------------------------------
    % Gibt den Typen und Namen eines Methodenparameters aus.
    % Parameter #1:  Typ des Parameters.
    % Parameter #2: Name des Parameters.
    % Achtung: Nicht fuer den ersten oder letzten Parameter der Methode
    % geeignet!
    %
    \newcommand{\printMethodParam}[2]{
        \parbox[t]{\MethodBeginWidth}{\hfill}
        \hspace{\CorrectWidthThree}
        \parbox[t]{\ParamMaxTypeWidth}{#1}
        \parbox[t]{\MaxParamWidth}{{\em #2},}\\[\VertSpace]
    }

    %-----------------------------------------------------------------------
    % Gibt die Beschreibung einer Methode aus.
    % Parameter #1: Beschreibung der Methode.
    %
    \newcommand{\printMethodDescr}[1]{
        \newline
        \noindent
        #1\\[\VertSpace]
    }

    %-----------------------------------------------------------------------
    % Gibt den Text ``Parameter:    `` und danach den Namen des
    % ersten Parameters der Methode, gefolgt von seiner Beschreibung
    % aus.
    % Parameter #1:  Name des Parameters. 
    % Parameter #2:  Beschreibung des Parameters.
    %
    \newcommand{\printFirstParam}[2]{
        \setlength{\Temp}{\ParamDescrWidth}
        \addtolength{\Temp}{1\CorrectWidthOne}
        \parbox[t]{\SpecialWidth}{{\bf Parameter:}}
        \parbox[t]{\MaxParamWidth}{\em #1}
        \parbox[t]{\ParamDelimWidth}{\ParamDelim}
        \parbox[t]{\Temp}{#2}\\[\VertSpace]
    }

    %-----------------------------------------------------------------------
    % Gibt einen Parameter der Methode und die dazugehoerige Beschreibung
    % aus.
    % Parameter #1:  Name des Parameters.
    % Parameter #2:  Beschreibung des Parameters.
    %
    \newcommand{\printParam}[2]{
        \setlength{\Temp}{\ParamDescrWidth}
        \addtolength{\Temp}{1\CorrectWidthOne}
        \parbox[t]{\SpecialWidth}{\hfill}
        \parbox[t]{\MaxParamWidth}{\em #1}
        \parbox[t]{\ParamDelimWidth}{\ParamDelim}
        \parbox[t]{\Temp}{#2}\\[\VertSpace]
    }

    %-----------------------------------------------------------------------
    %
    % Gibt den Text ``Rueckgabewert:  `` gefolgt von einer Beschreibung
    % des Rueckgabewertes aus.
    % Parameter #1:  Beschreibung des Rueckgabewertes.
    %
    \newcommand{\printReturn}[1]{
        \setlength{\Temp}{\SpecialRetTxtWidth}
        \addtolength{\Temp}{1\CorrectWidthTwo}
        \parbox[t]{\SpecialWidth}{{\bf R\"uckgabewert:}}
        \parbox[t]{\Temp}{#1}\\[\VertSpace]
    }

    %-----------------------------------------------------------------------
    % Gibt den Text ``Besonderheiten:  `` gefolgt von einer Beschreibung
    % der Besonderheiten aus.
    % Parameter #1:  Beschreibung der Besonderheiten.
    %
    \newcommand{\printSpecial}[1]{
        \setlength{\Temp}{\SpecialRetTxtWidth}
        \addtolength{\Temp}{1\CorrectWidthTwo}
        \parbox[t]{\SpecialWidth}{\MaxLabelTxt}
        \parbox[t]{\Temp}{#1}\\[\VertSpace]
    }

    %-----------------------------------------------------------------------
    % Gibt den Text ``Parameter:        Keine.'' aus.
    %
    \newcommand{\printNoParams}{
        \parbox[t]{\SpecialWidth}{{\bf Parameter:}}
        \parbox[t]{\NoneWidthOne}{Keine.}\\[\VertSpace]
    }

    %-----------------------------------------------------------------------
    % Gibt den Text ``Rueckgabewert:  Keiner.'' aus.
    %
    \newcommand{\printNoReturn}{
        \parbox[t]{\SpecialWidth}{{\bf R\"uckgabewert:}}
        \parbox[t]{\NoneWidthTwo}{Keiner.}\\[\VertSpace]
    }

    %-----------------------------------------------------------------------
    % Gibt den Text ``Besonderheiten:  Keine.'' aus.
    %
    \newcommand{\printNoSpecial}{
        \parbox[t]{\SpecialWidth}{\MaxLabelTxt}
        \parbox[t]{\NoneWidthOne}{Keine.}\\[\VertSpace]
    }

    %-----------------------------------------------------------------------
    % Gibt den Text:
    % ``Parameter:       Keine.
    %   Rueckgabewert:   Keiner.
    %   Besonderheiten:  Keine.''
    % aus.
    %
    \newcommand{\printNoAll}{
        \printNoParams
        \printNoReturn
        \printNoSpecial
    }

    %-----------------------------------------------------------------------
    % Gibt eine Methode ohne Parameter, Rueckgabewert und Besonderheiten
    % aus.
    % Parameter #1:  Name der Methode.
    % Parameter #2:  Beschreibung der Methode.
    %
    \newcommand{\printEmptyMethod}[2]{
        \setSpecialRetTxtWidth
        \printMethodBegin{void}{#1}
        \printMethodEnd
        \printMethodDescr{#2}
        \printNoAll
    }

    %-----------------------------------------------------------------------
    % Gibt eine Methode ohne Parameter und Besonderheiten, aber mit
    % Rueckgabewert aus.
    % Parameter #1:  Rueckgabewert der Methode.
    % Parameter #2:  Name der Methode.
    % Parameter #3:  Beschreibung der Methode.
    % Parameter #4:  Beschreibung des Rueckgabewertes der Methode.
    %
    \newcommand{\printEmptyMethodReturn}[4]{
        \setSpecialRetTxtWidth
        \printMethodBegin{#1}{#2}
        \printMethodEnd
        \printMethodDescr{#3}
        \printNoParams
        \printReturn{#4}
        \printNoSpecial
    }

    %-----------------------------------------------------------------------
    % Gibt eine Methode ohne Parameter und Rueckgabewert, aber mit
    % Besonderheiten aus.
    % Parameter #1:  Name der Methode.
    % Parameter #2:  Beschreibung der Methode.
    % Parameter #3:  Beschreibung der Besonderheiten der Methode.
    %
    \newcommand{\printEmptyMethodSpecial}[3]{
        \setSpecialRetTxtWidth
        \printMethodBegin{void}{#1}
        \printMethodEnd
        \printMethodDescr{#2}
        \printNoParams
        \printNoReturn
        \printSpecial{#3}
    }

    %-----------------------------------------------------------------------
    % Gibt eine Methode ohne Parameter, aber mit Rueckgabewert
    % und Besonderheiten aus.
    % Parameter #1:  Rueckgabewert der Methode.
    % Parameter #2:  Name der Methode.
    % Parameter #3:  Beschreibung der Methode.
    % Parameter #4:  Beschreibung des Rueckgabewertes der Methode.
    % Parameter #5:  Beschreibung der Besonderheiten der Methode.
    %
    \newcommand{\printEmptyMethodReturnSpecial}[5]{
        \setSpecialRetTxtWidth
        \printMethodBegin{#1}{#2}
        \printMethodEnd
        \printMethodDescr{#3}
        \printNoParams
        \printReturn{#4}
        \printSpecial{#5}
    }

    %-----------------------------------------------------------------------
    % Gibt eine Methode mit einem Parameter komplett mit allen
    % Beschreibungen aus.
    % Parameter #1:  Rueckgabewert der Methode.
    % Parameter #2:  Name der Methode.
    % Parameter #3:  Typ des einzigen Parameters der Methode.
    % Parameter #4:  Name des einzigen Parameters der Methode.
    % Parameter #5:  Beschreibung des einzigen Parameters der Methode.
    % Parameter #6:  Beschreibung der Methode.
    % Parameter #7:  Beschreibung des Rueckgabewertes der Methode.
    % Parameter #8:  Beschreibung der Besonderheiten der Methode.
    %
    \newcommand{\printMethodWithOneParam}[8]{
        \initParamMaxTypeWidth
        \initMaxParamWidth
        \setNewParamMaxTypeWidth{#3}
        \setNewMaxParamWidth{#4}
        \setSpecialRetTxtWidth
        \setParamDescrWidth
        \printMethodBegin{#1}{#2}
        \printMethodOneParam{#3}{#4}
        \printMethodDescr{#6}
        \printFirstParam{#4}{#5}
        \ifthenelse{\equal{#7}{Keiner.}}
            {\printNoReturn}
            {\printReturn{#7}}
        \ifthenelse{\equal{#8}{Keine.}}
            {\printNoSpecial}
            {\printSpecial{#8}}
    }

    %-----------------------------------------------------------------------
    % Gibt eine Methode mit einem Parameter komplett mit allen
    % Beschreibungen aus.
    % Parameter #1:  Rueckgabewert der Methode.
    % Parameter #2:  Beschreibung des Rueckgabewertes der Methode oder 
    %                Aufruf mit leerer Klammer. 
    % Parameter #3:  Name der Methode.
    % Parameter #4:  Beschreibung der Methode.
    % Parameter #5:  Beschreibung der Besonderheiten der Methode oder 
    %                Aufruf mit leerer Klammer.
    %
    \newcommand{\printMethodWithParamsSaved}[5]{
        \ifthenelse{\value{SavedParams} > 0}
            {\initParamMaxTypeWidth
             \initMaxParamWidth
             \setNewParamMaxTypeWidth{\ParamTypeOne}
             \setNewMaxParamWidth{\ParamOne}
             \ifthenelse{\value{SavedParams} = 2 \or \value{SavedParams} > 2}
                 {\setNewParamMaxTypeWidth{\ParamTypeTwo}
                  \setNewMaxParamWidth{\ParamTwo}}{}   
             \ifthenelse{\value{SavedParams} = 3 \or \value{SavedParams} > 3}
                 {\setNewParamMaxTypeWidth{\ParamTypeThree}
                  \setNewMaxParamWidth{\ParamThree}}{}    
             \ifthenelse{\value{SavedParams} = 4 \or \value{SavedParams} > 4}
                 {\setNewParamMaxTypeWidth{\ParamTypeFour}
                  \setNewMaxParamWidth{\ParamFour}}{}
             \ifthenelse{\value{SavedParams} = 5 \or \value{SavedParams} > 5}
                 {\setNewParamMaxTypeWidth{\ParamTypeFive}
                  \setNewMaxParamWidth{\ParamFive}}{}      
             \ifthenelse{\value{SavedParams} = 6 \or \value{SavedParams} > 6}
                 {\setNewParamMaxTypeWidth{\ParamTypeSix}
                  \setNewMaxParamWidth{\ParamSix}}{}
             \ifthenelse{\value{SavedParams} = 7 \or \value{SavedParams} > 7}
                 {\setNewParamMaxTypeWidth{\ParamTypeSeven}
                  \setNewMaxParamWidth{\ParamSeven}}{}       
             \ifthenelse{\value{SavedParams} = 8 \or \value{SavedParams} > 8}
                 {\setNewParamMaxTypeWidth{\ParamTypeEight}
                  \setNewMaxParamWidth{\ParamEight}}{}
             \ifthenelse{\value{SavedParams} = 9 \or \value{SavedParams} > 9}
                 {\setNewParamMaxTypeWidth{\ParamTypeNine}
                  \setNewMaxParamWidth{\ParamNine}}{}
             \ifthenelse{\value{SavedParams} = 10 \or 
                         \value{SavedParams} > 10}
                 {\setNewParamMaxTypeWidth{\ParamTypeTen}
                  \setNewMaxParamWidth{\ParamTen}}{}         
             \setSpecialRetTxtWidth
             \setParamDescrWidth
             \printMethodBegin{#1}{#3}
             \ifthenelse{\value{SavedParams} = 1} 
                 {\printMethodOneParam{\ParamTypeOne}{\ParamOne}}{}
                 {\printMethodFirstParam{\ParamTypeOne}{\ParamOne}}{}
             \ifthenelse{\value{SavedParams} = 2} 
                 {\printMethodLastParam{\ParamTypeTwo}{\ParamTwo}}{}
             \ifthenelse{\value{SavedParams} > 2}  
                 {\printMethodParam{\ParamTypeTwo}{\ParamTwo}}{} 
             \ifthenelse{\value{SavedParams} = 3} 
                 {\printMethodLastParam{\ParamTypeThree}{\ParamThree}}{}
             \ifthenelse{\value{SavedParams} > 3}  
                 {\printMethodParam{\ParamTypeThree}{\ParamThree}}{} 
             \ifthenelse{\value{SavedParams} = 4} 
                 {\printMethodLastParam{\ParamTypeFour}{\ParamFour}}{}
             \ifthenelse{\value{SavedParams} > 4}  
                 {\printMethodParam{\ParamTypeFour}{\ParamFour}}{} 
             \ifthenelse{\value{SavedParams} = 5} 
                 {\printMethodLastParam{\ParamTypeFive}{\ParamFive}}{}
             \ifthenelse{\value{SavedParams} > 5}  
                 {\printMethodParam{\ParamTypeFive}{\ParamFive}}{} 
             \ifthenelse{\value{SavedParams} = 6} 
                 {\printMethodLastParam{\ParamTypeSix}{\ParamSix}}{}
             \ifthenelse{\value{SavedParams} > 6}  
                 {\printMethodParam{\ParamTypeSix}{\ParamSix}}{} 
             \ifthenelse{\value{SavedParams} = 7} 
                 {\printMethodLastParam{\ParamTypeSeven}{\ParamSeven}}{}
             \ifthenelse{\value{SavedParams} > 7}  
                 {\printMethodParam{\ParamTypeSeven}{\ParamSeven}}{} 
             \ifthenelse{\value{SavedParams} = 8} 
                 {\printMethodLastParam{\ParamTypeEight}{\ParamEight}}{}
             \ifthenelse{\value{SavedParams} > 8}  
                 {\printMethodParam{\ParamTypeEight}{\ParamEight}}{}  
             \ifthenelse{\value{SavedParams} = 9} 
                 {\printMethodLastParam{\ParamTypeNine}{\ParamNine}}{}
             \ifthenelse{\value{SavedParams} > 9}  
                 {\printMethodParam{\ParamTypeNine}{\ParamNine}}{} 
             \ifthenelse{\value{SavedParams} = 10} 
                 {\printMethodLastParam{\ParamTypeTen}{\ParamTen}}{}
             \printMethodDescr{#4}
             \printFirstParam{\ParamOne}{\ParamDescrOne}
             \ifthenelse{\value{SavedParams} = 2 \or \value{SavedParams} > 2}
                  {\printParam{\ParamTwo}{\ParamDescrTwo}}{}
             \ifthenelse{\value{SavedParams} = 3 \or \value{SavedParams} > 3}
                  {\printParam{\ParamThree}{\ParamDescrThree}}{}
             \ifthenelse{\value{SavedParams} = 4 \or \value{SavedParams} > 4}
                  {\printParam{\ParamFour}{\ParamDescrFour}}{}
             \ifthenelse{\value{SavedParams} = 5 \or \value{SavedParams} > 5}
                  {\printParam{\ParamFive}{\ParamDescrFive}}{}
             \ifthenelse{\value{SavedParams} = 6 \or \value{SavedParams} > 6}
                  {\printParam{\ParamSix}{\ParamDescrSix}}{}
             \ifthenelse{\value{SavedParams} = 7 \or \value{SavedParams} > 7}
                  {\printParam{\ParamSeven}{\ParamDescrSeven}}{}
             \ifthenelse{\value{SavedParams} = 8 \or \value{SavedParams} > 8}
                  {\printParam{\ParamEight}{\ParamDescrEight}}{} 
             \ifthenelse{\value{SavedParams} = 9 \or \value{SavedParams} > 9}
                  {\printParam{\ParamNine}{\ParamDescrNine}}{}
             \ifthenelse{\value{SavedParams} = 10}
                  {\printParam{\ParamTen}{\ParamDescrTen}}{}
             \ifthenelse{\equal{#2}{}}
                 {\printNoReturn}
                 {\printReturn{#2}}
             \ifthenelse{\equal{#5}{}}
                 {\printNoSpecial}
                 {\printSpecial{#5}}}
        {\noindent
         Fehler Methode {\em printMethodWithParamsSaved}: Es sind noch keine 
         Parameter gesetzt worden!}
    }

%-----------------------------------------------------------------------
% Ende der Befehlsdefinitionen.
%-----------------------------------------------------------------------





%%% Local Variables: 
%%% mode: latex
%%% TeX-master: t
%%% End: 
'' in LaTex-Dokumente 
% (Latex2Epsilon) eingebunden und stellt Befehle zur einfachen Doku- 
% mentation von C-Funktionen/C++-Methoden bereit.
% =======================================================================

\usepackage{ifthen}

    % Definition von Konstanten:
    \newcommand{\ParamDelim}            % Trennungstext zwischen
        { \ - }                         % Parametername und -beschreibung.
    \newcommand{\NormMethodEnd}         % Endtext von Methode bei 
        {\ \ );}                        % veraenderbarer Aufrufinstanz.
    \newcommand{\ConstMethodEnd}        % Endtext von Methode bei
        {\ \ ) const;}                  % nicht veraenderbarer Aufrufinstanz.
    \newcommand{\MaxLabelTxt}           % Komponentenbeschreibungstext
        {{\bf Besonderheiten: }}        % mit maximaler Breite.
    \newcommand{\MethodBegin}           % Beginntext einer Methode.
        {(\ }
    \newcommand{\InBetween}             % Text zwischen Rueckgabewert und
        {\ \ }                          % Methodenname.

    % Definition eigener Variablen:


    \newboolean{ConstInstance}          % Gibt an, ob die Instanz, die
                                        % die Methode aufruft veraendert
                                        % werden darf oder nicht. Im
                                        % 2. Fall endet die Methode mit
                                        % ``) const;'', im 1. Fall 
                                        % mit ``);''.
    \newcounter{SavedParams}            % Anzahl der gespeicherten
                                        % Parameter.
    \setcounter{SavedParams}{0}
    \newcommand{\ParamOne}{}            % Definition der 10 
    \newcommand{\ParamTypeOne}{}        % abspeicherbaren Parameter,
    \newcommand{\ParamDescrOne}{}       % ihrer Typen und Beschreibungen.
    \newcommand{\ParamTwo}{}            
    \newcommand{\ParamTypeTwo}{}
    \newcommand{\ParamDescrTwo}{}
    \newcommand{\ParamThree}{}
    \newcommand{\ParamTypeThree}{}
    \newcommand{\ParamDescrThree}{}
    \newcommand{\ParamFour}{}
    \newcommand{\ParamTypeFour}{}
    \newcommand{\ParamDescrFour}{}
    \newcommand{\ParamFive}{}
    \newcommand{\ParamTypeFive}{}
    \newcommand{\ParamDescrFive}{}
    \newcommand{\ParamSix}{}
    \newcommand{\ParamTypeSix}{}
    \newcommand{\ParamDescrSix}{}
    \newcommand{\ParamSeven}{}
    \newcommand{\ParamTypeSeven}{}
    \newcommand{\ParamDescrSeven}{}
    \newcommand{\ParamEight}{}
    \newcommand{\ParamTypeEight}{}
    \newcommand{\ParamDescrEight}{}
    \newcommand{\ParamNine}{}
    \newcommand{\ParamTypeNine}{}
    \newcommand{\ParamDescrNine}{}
    \newcommand{\ParamTen}{}
    \newcommand{\ParamTypeTen}{}
    \newcommand{\ParamDescrTen}{}
    \newlength{\MaxParamWidth}          % Breite des breitesten 
                                        % Methodenparameters.
    \newlength{\SpecialWidth}           % Breite des Textes 
    \settowidth{\SpecialWidth}          % {\bf Besonderheiten}.
        {\MaxLabelTxt}
    \newlength{\ParamDelimWidth}        % Breite des Textes `` \ - ``.
    \settowidth{\ParamDelimWidth}
        {\ParamDelim}
    \newlength{\ParamDescrWidth}        % Max. Breite des Parametertextes.
    \newlength{\SpecialRetTxtWidth}     % Max. Breite der Texte fuer
                                        % Besonderheiten und Rueckgabewert.
    \newlength{\MethodBeginWidth}       % Breite des Methodenanfangs, d.h.
                                        % ``Rueckgabewert Methodenname ( ``.
    \newlength{\BetweenParsWidth}       % Max. zur Verfuegung stehende
                                        % Breite zwischen den beiden Klammern
                                        % der Methode.
    
    \newlength{\MethodEndWidth}         % Breite des Textes ``  );'' bzw.
    \settowidth{\MethodEndWidth}        % `` ) const;''
        {\NormMethodEnd}
    \newlength{\ParamMaxTypeWidth}      % Breite des breitesten
                                        % Parameterwert-Typs. 
    \newlength{\NoneWidthOne}           % Breite des Textes ``Keine.''.
    \settowidth{\NoneWidthOne}{Keine.}
    \newlength{\NoneWidthTwo}           % Breite des Textes ``Keiner.''.
    \settowidth{\NoneWidthTwo}{Keiner.}
    \newlength{\Temp}                   % Temporaere Hilfsvariable. 
    \newlength{\VertSpace}              % Vertikaler Abstand zwischen 
                                        % zwei Parboxen.
   
    % Trotz der automatischen Berechnung der Textbreiten treten
    % kleine Unstimmigkeiten von wenigen Punkten bei den Texten
    % zur Beschreibung der Parameter, des Rueckgabewertes und
    % der Besonderheiten auf, die mit den folgenden Variablen
    % behoben werden koennen:

    \newlength{\CorrectWidthOne}        % Korrekturbreite fuer die
                                        % Parameterbeschreibung.
    \newlength{\CorrectWidthTwo}        % Korrekturbreite fuer die
                                        % Beschreibungen von Rueckgabewert
                                        % und Besonderheiten.
    \newlength{\CorrectWidthThree}      % Korrekturbreite, da erster
                                        % Parameter einer Methode weiter
                                        % vorrueckt als die anderen.
    

% Uebersicht ueber die Parameter und ihre Bedeutung.


%
% <-- \MethodBeginWidth    -->
%------------------------------------------------------------------------
% Rueckgabewert Methodenname(   Typ_Parameter_1 Parameter_1,
%                            <->Typ_Parameter_2 Parameter_2,
%              \CorrectWideThree...
%                               Typ_Parameter_n Parameter_n   );
%                             <-------------> <--------->
%                          \ParamMaxTypeWidth \MaxParamWidth
%                             <-------------------------->
%                             \BetweenParsWidth
%                                                         <-->
%                                                         \MethodEndWidth
%------------------------------------------------------------------------
% Beschreibung der Methode...
%                             \ParamDelimWidth
%                             <->                        \CorrectWidthOne
% \VertSpace
% Parameter:       Parameter_1  - Beschreibung von Parameter 1.   <----->
%                  Parameter_2  - Beschreibung von Parameter 2.
%                  ...
%                  Parameter_n  - Beschreibung von Parameter n.
% \VertSpace                      \ParamDescrWidth
%                                 <------------------------------>
% Rueckgabewert:   Beschreibung des Rueckgabewertes bzw. der Text <-----> 
%                  Keiner.                   
%                  <----->
%                  \NoneWidthTwo               \SpecialRetTxtWidth
% \VertSpace       <--------------------------------------------->
% Besonderheiten:  Beschreibung von Besonderheiten bzw. der Text  <----->
%                  Keine.     
% <--------------> <---->
%  \SpecialWidth   \NoneWidthOne                         \CorrectWidthTwo


    % Definition eigener Befehle:


    %-----------------------------------------------------------------------
    % Gibt den Begriff ``C++'' in ansprechender Form aus.
    % Makro wurde freundlicherweise zur Verfuegung gestellt von
    % Axel W. Dietrich.
    %
    \newcommand{\cpp}{
        \mbox{\emph{\textrm{C\hspace{-1.5pt}\raisebox{1.75pt}{\scriptsize +}
        \hspace{-6pt}\raisebox{.75pt}{\scriptsize +}}}}%
    }


    %-----------------------------------------------------------------------
    % Gibt die Werte aller Parameter auf dem Bildschirm aus.
    %
    \newcommand{\showAllParams}{
        \noindent
        {\em Werte aller Parameter:}
        \begin{tabbing}
            ParamMaxTypeWidth:\ \ \=\kill
            MaxParamWidth:       \>\the\MaxParamWidth\\
            SpecialWidth:        \>\the\SpecialWidth\\
            ParamDelimWidth:     \>\the\ParamDelimWidth\\
            ParamDescrWidth:     \>\the\ParamDescrWidth\\
            SpecialRetTxtWidth:  \>\the\SpecialRetTxtWidth\\
            MethodBeginWidth:    \>\the\MethodBeginWidth\\
            BetweenParsWidth:    \>\the\BetweenParsWidth\\
            MethodEndWidth:      \>\the\MethodEndWidth\\
            ParamMaxTypeWidth:   \>\the\ParamMaxTypeWidth\\
            NoneWidthOne:        \>\the\NoneWidthOne\\
            NoneWidthTwo:        \>\the\NoneWidthTwo\\
            VertSpace:           \>\the\VertSpace\\
            CorrectWidthOne:     \>\the\CorrectWidthOne\\
            CorrectWidthTwo:     \>\the\CorrectWidthTwo\\
            CorrectWidthThree:   \>\the\CorrectWidthThree\\
            ConstInstance:       
            \ifthenelse{\boolean{ConstInstance}}
                {\>ja\\}
                {\>nein\\}
        \end{tabbing}
    }

    %-----------------------------------------------------------------------
    % Setzt einen Schalter, der angibt, dass die Instanz, die die kommende
    % Methode aufrufen kann, nicht veraendert werden darf. Dies hat fuer
    % die Dokumentation zur Folge, dass die Methode mit ``) const;'' endet.
    %
    \newcommand{\setConstInstance}{    
        \setboolean{ConstInstance}{true}
    }

    %-----------------------------------------------------------------------
    % Setzt einen Schalter, der angibt, dass die Instanz, die die kommende
    % Methode aufrufen kann, veraendert werden darf. Die Methode endet also
    % in der Dokumentation normal mit ``);''.
    %
    \newcommand{\setNormalInstance}{    
        \setboolean{ConstInstance}{false}
    }

    %-----------------------------------------------------------------------
    % Initialisiert die Breite des breitesten Methodenparametertyptextes
    % mit Null.
    %
    \newcommand{\initParamMaxTypeWidth}{
        \setlength{\ParamMaxTypeWidth}{0pt}
    }

    %-----------------------------------------------------------------------
    % Initialisiert die Breite des breitesten Parameternamens mit Null.
    %
    \newcommand{\initMaxParamWidth}{
        \setlength{\MaxParamWidth}{0pt}
    }

    %-----------------------------------------------------------------------
    % Ueberprueft, ob die Breite des uebergebenen Textes des Parametertyps
    % groesser ist als die des bisher als breitester Typtext festgelegten
    % Textes und passt den Wert evtl. an.
    % Parameter #1:  Neuer Typtext, dessen Breite zum Vergleich dient.
    %   
    \newcommand{\setNewParamMaxTypeWidth}[1]{
        \settowidth{\Temp}{#1}
        \ifthenelse{\lengthtest{\ParamMaxTypeWidth < \Temp}}
            {\setlength{\ParamMaxTypeWidth}{\Temp}}{}   
    }

    %-----------------------------------------------------------------------
    % Ueberprueft, ob die Breite des uebergebenen Parameternamens groesser
    % ist als die des bisher als am breitesten geltenden Parameternames
    % und passt den Wert evtl. an. Der Parametername steht dabei in
    % Emphasized.
    % Parameter #1:  Neuer Parametername, dessen Breite zum Vergleich dient.
    %
    \newcommand{\setNewMaxParamWidth}[1]{
        \settowidth{\Temp}{{\em #1}}
        \ifthenelse{\lengthtest{\MaxParamWidth < \Temp}}
            {\setlength{\MaxParamWidth}{\Temp}}{}
    }

    %-----------------------------------------------------------------------
    % Setzt die Breite der Beschreibungstexte fuer Rueckgabewert
    % und Besonderheiten.
    %
    \newcommand{\setSpecialRetTxtWidth}{
        \setlength{\SpecialRetTxtWidth}{\textwidth}
        \addtolength{\SpecialRetTxtWidth}{-1\SpecialWidth} 
    }

    %-----------------------------------------------------------------------
    % Setzt die Breite fuer den Beginn einer Methode, d.h. den Text
    % ``Rueckgabewert Methodenname( ``. Rueckgabewert steht dabei
    % in Emphasized, Methodenname in Bold Font.
    % Parameter #1:  Rueckgabewert der Methode.
    % Parameter #2:  Name der Methode.
    %
    \newcommand{\setMethodBeginWidth}[2]{
        \settowidth{\MethodBeginWidth}
            {{\em #1}\InBetween{\bf #2}\MethodBegin}
    }

    %-----------------------------------------------------------------------
    % Setzt die maximale Breite die an Platz zwischen der oeffnenden
    % und der schliessenden Klammer der Methode zur Verfuegung steht.
    %
    \newcommand{\setBetweenParsWidth}{
        \setlength{\BetweenParsWidth}{\textwidth}
        \addtolength{\BetweenParsWidth}{-1\MethodBeginWidth}
        \addtolength{\BetweenParsWidth}{-1\MethodEndWidth}
    }

    %-----------------------------------------------------------------------
    % Setzt die Breite fuer die Beschreibungstexte
    % der Methodenparameter.
    %
    \newcommand{\setParamDescrWidth}{
        \setlength{\ParamDescrWidth}{\textwidth}
        \addtolength{\ParamDescrWidth}{-1\SpecialWidth} 
        \addtolength{\ParamDescrWidth}{-1\MaxParamWidth}
        \addtolength{\ParamDescrWidth}{-1\ParamDelimWidth}
    }    

    %-----------------------------------------------------------------------
    % Setzt die Korrekturbreite fuer den Beschreibungstext von
    % Methodenparametern auf den uebergebenen Wert.
    % Parameter #1:  Neuer Wert fuer die Korrekturbreite.
    %
    \newcommand{\setCorrectWidthOne}[1]{
        \setlength{\CorrectWidthOne}{#1}
    }    

    %-----------------------------------------------------------------------
    % Setzt die Korrekturbreite fuer den Beschreibungstext von
    % Rueckgabewert und Besonderheiten auf den uebergebenen Wert.
    % Parameter #1:  Neuer Wert fuer die Korrekturbreite.
    %
    \newcommand{\setCorrectWidthTwo}[1]{
        \setlength{\CorrectWidthTwo}{#1}
    }    

    %-----------------------------------------------------------------------
    % Setzt die Korrekturbreite fuer den Abstand des zweiten bis n-ten
    % Parameters vom Anfang auf den uebergebenen Wert.
    % Parameter #1:  Neuer Wert fuer die Korrekturbreite.
    %
    \newcommand{\setCorrectWidthThree}[1]{
        \setlength{\CorrectWidthThree}{#1}
    }    

    %-----------------------------------------------------------------------
    % Setzt den vertikalen Abstand zwischen zwei Parboxen auf den ueber-
    % gebenen Wert.
    % Parameter #1:  Neuer Wert fuer den vertikalen Abstand.
    %
    \newcommand{\setVertSpace}[1]{
        \setlength{\VertSpace}{#1}
    }    

    %-----------------------------------------------------------------------
    % Speichert die uebergebenen Informationen ueber Methodenparameter 1
    % in internen Variablen. Achtung: Die Parameter muessen in der richtigen
    % Reihenfolge gesetzt werden!
    % Parameter #1: Der Typ des Parameters.  
    % Parameter #2: Der Name des Parameters. 
    % Parameter #3: Die Beschreibung des Parameters. 
    % 
    \newcommand{\setParamOne}[3]{
        \renewcommand{\ParamOne}{#1}
        \renewcommand{\ParamTypeOne}{#2}
        \renewcommand{\ParamDescrOne}{#3} 
        \setcounter{SavedParams}{0}
        \stepcounter{SavedParams}
    }

    %-----------------------------------------------------------------------
    % Speichert die uebergebenen Informationen ueber Methodenparameter 2
    % in internen Variablen. Achtung: Die Parameter muessen in der richtigen
    % Reihenfolge gesetzt werden!
    % Parameter #1: Der Typ des Parameters.  
    % Parameter #2: Der Name des Parameters. 
    % Parameter #3: Die Beschreibung des Parameters. 
    % 
    \newcommand{\setParamTwo}[3]{
        \renewcommand{\ParamTwo}{#1}
        \renewcommand{\ParamTypeTwo}{#2}
        \renewcommand{\ParamDescrTwo}{#3} 
        \stepcounter{SavedParams}
    }

    %-----------------------------------------------------------------------
    % Speichert die uebergebenen Informationen ueber Methodenparameter 3
    % in internen Variablen. Achtung: Die Parameter muessen in der richtigen
    % Reihenfolge gesetzt werden!
    % Parameter #1: Der Typ des Parameters.  
    % Parameter #2: Der Name des Parameters. 
    % Parameter #3: Die Beschreibung des Parameters. 
    % 
    \newcommand{\setParamThree}[3]{
        \renewcommand{\ParamThree}{#1}
        \renewcommand{\ParamTypeThree}{#2}
        \renewcommand{\ParamDescrThree}{#3} 
        \stepcounter{SavedParams}
    }

    %-----------------------------------------------------------------------
    % Speichert die uebergebenen Informationen ueber Methodenparameter 4
    % in internen Variablen. Achtung: Die Parameter muessen in der richtigen
    % Reihenfolge gesetzt werden!
    % Parameter #1: Der Typ des Parameters.  
    % Parameter #2: Der Name des Parameters. 
    % Parameter #3: Die Beschreibung des Parameters. 
    % 
    \newcommand{\setParamFour}[3]{
        \renewcommand{\ParamFour}{#1}
        \renewcommand{\ParamTypeFour}{#2}
        \renewcommand{\ParamDescrFour}{#3} 
        \stepcounter{SavedParams}
    }

    %-----------------------------------------------------------------------
    % Speichert die uebergebenen Informationen ueber Methodenparameter 5
    % in internen Variablen. Achtung: Die Parameter muessen in der richtigen
    % Reihenfolge gesetzt werden!
    % Parameter #1: Der Typ des Parameters.  
    % Parameter #2: Der Name des Parameters. 
    % Parameter #3: Die Beschreibung des Parameters. 
    % 
    \newcommand{\setParamFive}[3]{
        \renewcommand{\ParamFive}{#1}
        \renewcommand{\ParamTypeFive}{#2}
        \renewcommand{\ParamDescrFive}{#3} 
        \stepcounter{SavedParams}
    }

    %-----------------------------------------------------------------------
    % Speichert die uebergebenen Informationen ueber Methodenparameter 6
    % in internen Variablen. Achtung: Die Parameter muessen in der richtigen
    % Reihenfolge gesetzt werden!
    % Parameter #1: Der Typ des Parameters.  
    % Parameter #2: Der Name des Parameters. 
    % Parameter #3: Die Beschreibung des Parameters. 
    %     
    \newcommand{\setParamSix}[3]{
        \renewcommand{\ParamSix}{#1}
        \renewcommand{\ParamTypeSix}{#2}
        \renewcommand{\ParamDescrSix}{#3} 
        \stepcounter{SavedParams} 
    }

    %-----------------------------------------------------------------------
    % Speichert die uebergebenen Informationen ueber Methodenparameter 7
    % in internen Variablen. Achtung: Die Parameter muessen in der richtigen
    % Reihenfolge gesetzt werden!
    % Parameter #1: Der Typ des Parameters.  
    % Parameter #2: Der Name des Parameters. 
    % Parameter #3: Die Beschreibung des Parameters. 
    %    
    \newcommand{\setParamSeven}[3]{
        \renewcommand{\ParamSeven}{#1}
        \renewcommand{\ParamTypeSeven}{#2}
        \renewcommand{\ParamDescrSeven}{#3} 
        \stepcounter{SavedParams} 
    }

    %-----------------------------------------------------------------------
    % Speichert die uebergebenen Informationen ueber Methodenparameter 8
    % in internen Variablen. Achtung: Die Parameter muessen in der richtigen
    % Reihenfolge gesetzt werden!
    % Parameter #1: Der Typ des Parameters.  
    % Parameter #2: Der Name des Parameters. 
    % Parameter #3: Die Beschreibung des Parameters. 
    % 
    \newcommand{\setParamEight}[3]{
        \renewcommand{\ParamEight}{#1}
        \renewcommand{\ParamTypeEight}{#2}
        \renewcommand{\ParamDescrEight}{#3} 
        \stepcounter{SavedParams}
    }

    %-----------------------------------------------------------------------
    % Speichert die uebergebenen Informationen ueber Methodenparameter 9
    % in internen Variablen. Achtung: Die Parameter muessen in der richtigen
    % Reihenfolge gesetzt werden!
    % Parameter #1: Der Typ des Parameters.  
    % Parameter #2: Der Name des Parameters. 
    % Parameter #3: Die Beschreibung des Parameters. 
    % 
    \newcommand{\setParamNine}[3]{
        \renewcommand{\ParamNine}{#1}
        \renewcommand{\ParamTypeNine}{#2}
        \renewcommand{\ParamDescrNine}{#3}  
        \stepcounter{SavedParams}
    }

    %-----------------------------------------------------------------------
    % Speichert die uebergebenen Informationen ueber Methodenparameter 10
    % in internen Variablen. Achtung: Die Parameter muessen in der richtigen
    % Reihenfolge gesetzt werden!
    % Parameter #1: Der Typ des Parameters.  
    % Parameter #2: Der Name des Parameters. 
    % Parameter #3: Die Beschreibung des Parameters. 
    % 
    \newcommand{\setParamTen}[3]{
        \renewcommand{\ParamTen}{#1}
        \renewcommand{\ParamTypeTen}{#2}
        \renewcommand{\ParamDescrTen}{#3} 
        \stepcounter{SavedParams}
    }

    %-----------------------------------------------------------------------
    % Gibt den Text ``Rueckgabewert Methodenname(`` aus.
    % Parameter #1:  Rueckgabewert der Methode.
    % Parameter #2:  Name der Methode.
    %
    \newcommand{\printMethodBegin}[2]{
         \ifthenelse{\boolean{ConstInstance}}
             {\settowidth{\MethodEndWidth}{\ConstMethodEnd}}
             {\settowidth{\MethodEndWidth}{\NormMethodEnd}}
         \setMethodBeginWidth{#1}{#2}
         \setBetweenParsWidth
         \noindent
         \hrulefill\\
         \noindent
         \parbox[t]{\MethodBeginWidth}
             {#1\InBetween{\bf #2}\MethodBegin}
     }

    %-----------------------------------------------------------------------
    % Gibt das Ende einer Methode aus, d.h. ``\ );'' oder ``\ ) const;''.
    %
    \newcommand{\printMethodEnd}{
        \ifthenelse{\boolean{ConstInstance}}
            {\ConstMethodEnd\\}
            {\NormMethodEnd\\}
            \mbox{}\hrulefill\\
    }

    %-----------------------------------------------------------------------
    % Gibt den Typen und Namen des einzigen Parameters einer Methode
    % gefolgt von dem Text ``  );'' bzw. `` ) const;'' aus.
    % Parameter #1:  Typ des Parameters.
    % Parameter #2:  Name des Parameters.
    % 
    \newcommand{\printMethodOneParam}[2]{
        \parbox[t]{\ParamMaxTypeWidth}{#1}
        \parbox[t]{\MaxParamWidth}{{\em #2}}
        \printMethodEnd
    }

    %-----------------------------------------------------------------------
    % Gibt den Typen und den Namen des ersten Methodenparameters aus.
    % Parameter #1:  Typ des Parameters.
    % Parameter #2:  Name des Parameters.
    %
    \newcommand{\printMethodFirstParam}[2]{
        \noindent
        \parbox[t]{\ParamMaxTypeWidth}{#1}
        \parbox[t]{\MaxParamWidth}{{\em #2},}\\[\VertSpace]
    }

    %-----------------------------------------------------------------------
    % Gibt den Typen und Namen des letzten Parameters einer Methode aus.
    % Parameter #1: Typ des Parameters.
    % Parameter #2: Name des Parameters.
    %
    \newcommand{\printMethodLastParam}[2]{
        \parbox[t]{\MethodBeginWidth}{\hfill}
        \hspace{\CorrectWidthThree}
        \parbox[t]{\ParamMaxTypeWidth}{#1}
        \parbox[t]{\MaxParamWidth}{{\em #2}}
        \printMethodEnd
    }

    %-----------------------------------------------------------------------
    % Gibt den Typen und Namen eines Methodenparameters aus.
    % Parameter #1:  Typ des Parameters.
    % Parameter #2: Name des Parameters.
    % Achtung: Nicht fuer den ersten oder letzten Parameter der Methode
    % geeignet!
    %
    \newcommand{\printMethodParam}[2]{
        \parbox[t]{\MethodBeginWidth}{\hfill}
        \hspace{\CorrectWidthThree}
        \parbox[t]{\ParamMaxTypeWidth}{#1}
        \parbox[t]{\MaxParamWidth}{{\em #2},}\\[\VertSpace]
    }

    %-----------------------------------------------------------------------
    % Gibt die Beschreibung einer Methode aus.
    % Parameter #1: Beschreibung der Methode.
    %
    \newcommand{\printMethodDescr}[1]{
        \newline
        \noindent
        #1\\[\VertSpace]
    }

    %-----------------------------------------------------------------------
    % Gibt den Text ``Parameter:    `` und danach den Namen des
    % ersten Parameters der Methode, gefolgt von seiner Beschreibung
    % aus.
    % Parameter #1:  Name des Parameters. 
    % Parameter #2:  Beschreibung des Parameters.
    %
    \newcommand{\printFirstParam}[2]{
        \setlength{\Temp}{\ParamDescrWidth}
        \addtolength{\Temp}{1\CorrectWidthOne}
        \parbox[t]{\SpecialWidth}{{\bf Parameter:}}
        \parbox[t]{\MaxParamWidth}{\em #1}
        \parbox[t]{\ParamDelimWidth}{\ParamDelim}
        \parbox[t]{\Temp}{#2}\\[\VertSpace]
    }

    %-----------------------------------------------------------------------
    % Gibt einen Parameter der Methode und die dazugehoerige Beschreibung
    % aus.
    % Parameter #1:  Name des Parameters.
    % Parameter #2:  Beschreibung des Parameters.
    %
    \newcommand{\printParam}[2]{
        \setlength{\Temp}{\ParamDescrWidth}
        \addtolength{\Temp}{1\CorrectWidthOne}
        \parbox[t]{\SpecialWidth}{\hfill}
        \parbox[t]{\MaxParamWidth}{\em #1}
        \parbox[t]{\ParamDelimWidth}{\ParamDelim}
        \parbox[t]{\Temp}{#2}\\[\VertSpace]
    }

    %-----------------------------------------------------------------------
    %
    % Gibt den Text ``Rueckgabewert:  `` gefolgt von einer Beschreibung
    % des Rueckgabewertes aus.
    % Parameter #1:  Beschreibung des Rueckgabewertes.
    %
    \newcommand{\printReturn}[1]{
        \setlength{\Temp}{\SpecialRetTxtWidth}
        \addtolength{\Temp}{1\CorrectWidthTwo}
        \parbox[t]{\SpecialWidth}{{\bf R\"uckgabewert:}}
        \parbox[t]{\Temp}{#1}\\[\VertSpace]
    }

    %-----------------------------------------------------------------------
    % Gibt den Text ``Besonderheiten:  `` gefolgt von einer Beschreibung
    % der Besonderheiten aus.
    % Parameter #1:  Beschreibung der Besonderheiten.
    %
    \newcommand{\printSpecial}[1]{
        \setlength{\Temp}{\SpecialRetTxtWidth}
        \addtolength{\Temp}{1\CorrectWidthTwo}
        \parbox[t]{\SpecialWidth}{\MaxLabelTxt}
        \parbox[t]{\Temp}{#1}\\[\VertSpace]
    }

    %-----------------------------------------------------------------------
    % Gibt den Text ``Parameter:        Keine.'' aus.
    %
    \newcommand{\printNoParams}{
        \parbox[t]{\SpecialWidth}{{\bf Parameter:}}
        \parbox[t]{\NoneWidthOne}{Keine.}\\[\VertSpace]
    }

    %-----------------------------------------------------------------------
    % Gibt den Text ``Rueckgabewert:  Keiner.'' aus.
    %
    \newcommand{\printNoReturn}{
        \parbox[t]{\SpecialWidth}{{\bf R\"uckgabewert:}}
        \parbox[t]{\NoneWidthTwo}{Keiner.}\\[\VertSpace]
    }

    %-----------------------------------------------------------------------
    % Gibt den Text ``Besonderheiten:  Keine.'' aus.
    %
    \newcommand{\printNoSpecial}{
        \parbox[t]{\SpecialWidth}{\MaxLabelTxt}
        \parbox[t]{\NoneWidthOne}{Keine.}\\[\VertSpace]
    }

    %-----------------------------------------------------------------------
    % Gibt den Text:
    % ``Parameter:       Keine.
    %   Rueckgabewert:   Keiner.
    %   Besonderheiten:  Keine.''
    % aus.
    %
    \newcommand{\printNoAll}{
        \printNoParams
        \printNoReturn
        \printNoSpecial
    }

    %-----------------------------------------------------------------------
    % Gibt eine Methode ohne Parameter, Rueckgabewert und Besonderheiten
    % aus.
    % Parameter #1:  Name der Methode.
    % Parameter #2:  Beschreibung der Methode.
    %
    \newcommand{\printEmptyMethod}[2]{
        \setSpecialRetTxtWidth
        \printMethodBegin{void}{#1}
        \printMethodEnd
        \printMethodDescr{#2}
        \printNoAll
    }

    %-----------------------------------------------------------------------
    % Gibt eine Methode ohne Parameter und Besonderheiten, aber mit
    % Rueckgabewert aus.
    % Parameter #1:  Rueckgabewert der Methode.
    % Parameter #2:  Name der Methode.
    % Parameter #3:  Beschreibung der Methode.
    % Parameter #4:  Beschreibung des Rueckgabewertes der Methode.
    %
    \newcommand{\printEmptyMethodReturn}[4]{
        \setSpecialRetTxtWidth
        \printMethodBegin{#1}{#2}
        \printMethodEnd
        \printMethodDescr{#3}
        \printNoParams
        \printReturn{#4}
        \printNoSpecial
    }

    %-----------------------------------------------------------------------
    % Gibt eine Methode ohne Parameter und Rueckgabewert, aber mit
    % Besonderheiten aus.
    % Parameter #1:  Name der Methode.
    % Parameter #2:  Beschreibung der Methode.
    % Parameter #3:  Beschreibung der Besonderheiten der Methode.
    %
    \newcommand{\printEmptyMethodSpecial}[3]{
        \setSpecialRetTxtWidth
        \printMethodBegin{void}{#1}
        \printMethodEnd
        \printMethodDescr{#2}
        \printNoParams
        \printNoReturn
        \printSpecial{#3}
    }

    %-----------------------------------------------------------------------
    % Gibt eine Methode ohne Parameter, aber mit Rueckgabewert
    % und Besonderheiten aus.
    % Parameter #1:  Rueckgabewert der Methode.
    % Parameter #2:  Name der Methode.
    % Parameter #3:  Beschreibung der Methode.
    % Parameter #4:  Beschreibung des Rueckgabewertes der Methode.
    % Parameter #5:  Beschreibung der Besonderheiten der Methode.
    %
    \newcommand{\printEmptyMethodReturnSpecial}[5]{
        \setSpecialRetTxtWidth
        \printMethodBegin{#1}{#2}
        \printMethodEnd
        \printMethodDescr{#3}
        \printNoParams
        \printReturn{#4}
        \printSpecial{#5}
    }

    %-----------------------------------------------------------------------
    % Gibt eine Methode mit einem Parameter komplett mit allen
    % Beschreibungen aus.
    % Parameter #1:  Rueckgabewert der Methode.
    % Parameter #2:  Name der Methode.
    % Parameter #3:  Typ des einzigen Parameters der Methode.
    % Parameter #4:  Name des einzigen Parameters der Methode.
    % Parameter #5:  Beschreibung des einzigen Parameters der Methode.
    % Parameter #6:  Beschreibung der Methode.
    % Parameter #7:  Beschreibung des Rueckgabewertes der Methode.
    % Parameter #8:  Beschreibung der Besonderheiten der Methode.
    %
    \newcommand{\printMethodWithOneParam}[8]{
        \initParamMaxTypeWidth
        \initMaxParamWidth
        \setNewParamMaxTypeWidth{#3}
        \setNewMaxParamWidth{#4}
        \setSpecialRetTxtWidth
        \setParamDescrWidth
        \printMethodBegin{#1}{#2}
        \printMethodOneParam{#3}{#4}
        \printMethodDescr{#6}
        \printFirstParam{#4}{#5}
        \ifthenelse{\equal{#7}{Keiner.}}
            {\printNoReturn}
            {\printReturn{#7}}
        \ifthenelse{\equal{#8}{Keine.}}
            {\printNoSpecial}
            {\printSpecial{#8}}
    }

    %-----------------------------------------------------------------------
    % Gibt eine Methode mit einem Parameter komplett mit allen
    % Beschreibungen aus.
    % Parameter #1:  Rueckgabewert der Methode.
    % Parameter #2:  Beschreibung des Rueckgabewertes der Methode oder 
    %                Aufruf mit leerer Klammer. 
    % Parameter #3:  Name der Methode.
    % Parameter #4:  Beschreibung der Methode.
    % Parameter #5:  Beschreibung der Besonderheiten der Methode oder 
    %                Aufruf mit leerer Klammer.
    %
    \newcommand{\printMethodWithParamsSaved}[5]{
        \ifthenelse{\value{SavedParams} > 0}
            {\initParamMaxTypeWidth
             \initMaxParamWidth
             \setNewParamMaxTypeWidth{\ParamTypeOne}
             \setNewMaxParamWidth{\ParamOne}
             \ifthenelse{\value{SavedParams} = 2 \or \value{SavedParams} > 2}
                 {\setNewParamMaxTypeWidth{\ParamTypeTwo}
                  \setNewMaxParamWidth{\ParamTwo}}{}   
             \ifthenelse{\value{SavedParams} = 3 \or \value{SavedParams} > 3}
                 {\setNewParamMaxTypeWidth{\ParamTypeThree}
                  \setNewMaxParamWidth{\ParamThree}}{}    
             \ifthenelse{\value{SavedParams} = 4 \or \value{SavedParams} > 4}
                 {\setNewParamMaxTypeWidth{\ParamTypeFour}
                  \setNewMaxParamWidth{\ParamFour}}{}
             \ifthenelse{\value{SavedParams} = 5 \or \value{SavedParams} > 5}
                 {\setNewParamMaxTypeWidth{\ParamTypeFive}
                  \setNewMaxParamWidth{\ParamFive}}{}      
             \ifthenelse{\value{SavedParams} = 6 \or \value{SavedParams} > 6}
                 {\setNewParamMaxTypeWidth{\ParamTypeSix}
                  \setNewMaxParamWidth{\ParamSix}}{}
             \ifthenelse{\value{SavedParams} = 7 \or \value{SavedParams} > 7}
                 {\setNewParamMaxTypeWidth{\ParamTypeSeven}
                  \setNewMaxParamWidth{\ParamSeven}}{}       
             \ifthenelse{\value{SavedParams} = 8 \or \value{SavedParams} > 8}
                 {\setNewParamMaxTypeWidth{\ParamTypeEight}
                  \setNewMaxParamWidth{\ParamEight}}{}
             \ifthenelse{\value{SavedParams} = 9 \or \value{SavedParams} > 9}
                 {\setNewParamMaxTypeWidth{\ParamTypeNine}
                  \setNewMaxParamWidth{\ParamNine}}{}
             \ifthenelse{\value{SavedParams} = 10 \or 
                         \value{SavedParams} > 10}
                 {\setNewParamMaxTypeWidth{\ParamTypeTen}
                  \setNewMaxParamWidth{\ParamTen}}{}         
             \setSpecialRetTxtWidth
             \setParamDescrWidth
             \printMethodBegin{#1}{#3}
             \ifthenelse{\value{SavedParams} = 1} 
                 {\printMethodOneParam{\ParamTypeOne}{\ParamOne}}{}
                 {\printMethodFirstParam{\ParamTypeOne}{\ParamOne}}{}
             \ifthenelse{\value{SavedParams} = 2} 
                 {\printMethodLastParam{\ParamTypeTwo}{\ParamTwo}}{}
             \ifthenelse{\value{SavedParams} > 2}  
                 {\printMethodParam{\ParamTypeTwo}{\ParamTwo}}{} 
             \ifthenelse{\value{SavedParams} = 3} 
                 {\printMethodLastParam{\ParamTypeThree}{\ParamThree}}{}
             \ifthenelse{\value{SavedParams} > 3}  
                 {\printMethodParam{\ParamTypeThree}{\ParamThree}}{} 
             \ifthenelse{\value{SavedParams} = 4} 
                 {\printMethodLastParam{\ParamTypeFour}{\ParamFour}}{}
             \ifthenelse{\value{SavedParams} > 4}  
                 {\printMethodParam{\ParamTypeFour}{\ParamFour}}{} 
             \ifthenelse{\value{SavedParams} = 5} 
                 {\printMethodLastParam{\ParamTypeFive}{\ParamFive}}{}
             \ifthenelse{\value{SavedParams} > 5}  
                 {\printMethodParam{\ParamTypeFive}{\ParamFive}}{} 
             \ifthenelse{\value{SavedParams} = 6} 
                 {\printMethodLastParam{\ParamTypeSix}{\ParamSix}}{}
             \ifthenelse{\value{SavedParams} > 6}  
                 {\printMethodParam{\ParamTypeSix}{\ParamSix}}{} 
             \ifthenelse{\value{SavedParams} = 7} 
                 {\printMethodLastParam{\ParamTypeSeven}{\ParamSeven}}{}
             \ifthenelse{\value{SavedParams} > 7}  
                 {\printMethodParam{\ParamTypeSeven}{\ParamSeven}}{} 
             \ifthenelse{\value{SavedParams} = 8} 
                 {\printMethodLastParam{\ParamTypeEight}{\ParamEight}}{}
             \ifthenelse{\value{SavedParams} > 8}  
                 {\printMethodParam{\ParamTypeEight}{\ParamEight}}{}  
             \ifthenelse{\value{SavedParams} = 9} 
                 {\printMethodLastParam{\ParamTypeNine}{\ParamNine}}{}
             \ifthenelse{\value{SavedParams} > 9}  
                 {\printMethodParam{\ParamTypeNine}{\ParamNine}}{} 
             \ifthenelse{\value{SavedParams} = 10} 
                 {\printMethodLastParam{\ParamTypeTen}{\ParamTen}}{}
             \printMethodDescr{#4}
             \printFirstParam{\ParamOne}{\ParamDescrOne}
             \ifthenelse{\value{SavedParams} = 2 \or \value{SavedParams} > 2}
                  {\printParam{\ParamTwo}{\ParamDescrTwo}}{}
             \ifthenelse{\value{SavedParams} = 3 \or \value{SavedParams} > 3}
                  {\printParam{\ParamThree}{\ParamDescrThree}}{}
             \ifthenelse{\value{SavedParams} = 4 \or \value{SavedParams} > 4}
                  {\printParam{\ParamFour}{\ParamDescrFour}}{}
             \ifthenelse{\value{SavedParams} = 5 \or \value{SavedParams} > 5}
                  {\printParam{\ParamFive}{\ParamDescrFive}}{}
             \ifthenelse{\value{SavedParams} = 6 \or \value{SavedParams} > 6}
                  {\printParam{\ParamSix}{\ParamDescrSix}}{}
             \ifthenelse{\value{SavedParams} = 7 \or \value{SavedParams} > 7}
                  {\printParam{\ParamSeven}{\ParamDescrSeven}}{}
             \ifthenelse{\value{SavedParams} = 8 \or \value{SavedParams} > 8}
                  {\printParam{\ParamEight}{\ParamDescrEight}}{} 
             \ifthenelse{\value{SavedParams} = 9 \or \value{SavedParams} > 9}
                  {\printParam{\ParamNine}{\ParamDescrNine}}{}
             \ifthenelse{\value{SavedParams} = 10}
                  {\printParam{\ParamTen}{\ParamDescrTen}}{}
             \ifthenelse{\equal{#2}{}}
                 {\printNoReturn}
                 {\printReturn{#2}}
             \ifthenelse{\equal{#5}{}}
                 {\printNoSpecial}
                 {\printSpecial{#5}}}
        {\noindent
         Fehler Methode {\em printMethodWithParamsSaved}: Es sind noch keine 
         Parameter gesetzt worden!}
    }

%-----------------------------------------------------------------------
% Ende der Befehlsdefinitionen.
%-----------------------------------------------------------------------





%%% Local Variables: 
%%% mode: latex
%%% TeX-master: t
%%% End: 

\begin{document}

\textheight 23cm
\topmargin -0.54cm
\textwidth 16.5cm
\oddsidemargin -0.04cm
\evensidemargin -0.54cm


\newgray{lightgray}{.97}
\newif\ifmethodgray
\methodgrayfalse
\def\method#1#2{%
  \ifmethodgray
    \psframebox*[framesep=1pt,fillcolor=lightgray,fillstyle=solid]{%
      %\begin{minipage}{.996\textwidth}
      \begin{minipage}{.873\textwidth}
        \begin{minipage}{17em} #1 \end{minipage}
        %\begin{minipage}{22.5em} #2 \end{minipage}
        \begin{minipage}{17.7em} #2 \end{minipage}
      \end{minipage}}
    \methodgrayfalse
  \else
    \psframebox*[framesep=1pt]{%
      %\begin{minipage}{.996\textwidth}
      \begin{minipage}{.873\textwidth}
        \begin{minipage}{17em} #1 \end{minipage}
        %\begin{minipage}{22.5em} #2 \end{minipage}
        \begin{minipage}{17.7em} #2 \end{minipage}
      \end{minipage}}
    \methodgraytrue
  \fi
}


\begin{center}
{\Large {\bf Schnellreferenz zur Bibliothek EALib}}\\
{\bf R\"udiger Alberts}\\
{\bf 2000-01-07}
\end{center}

\newpage
\tableofcontents
\newpage
\chapter{Einleitung}
Bei diesem Dokument handelt es sich um eine deutsche Fassung der
Schnellreferenz f\"ur die {\em EALib}-Bibliothek zur
Programmierung evolution\"arer Algorithmen von Martin Kreutz.\\
Diese Referenz beschreibt die Methoden und Variablen der Klassen
{\tt Chromosome}, {\tt ChromosomeT\_num}, {\tt ChromosomeT\_int},
{\tt ChromosomeT\_double}, {\tt ChromosomeT\_char}, \\
{\tt ChromosomeT\_bool}, {\tt Individual}, {\tt Population} und
{\tt EvoAlg} und entspricht der Schnellreferenz in \cite{EALib}.

\chapter{Klasse {\tt Chromosome}}

% ######################################################################
\section{Frei verf\"ugbare Methoden}
% ######################################################################

% -----------------------------------------------------------------------
\subsection{Konstruktoren und Destruktor}
% -----------------------------------------------------------------------

\vspace*{10pt}

\methodgrayfalse

\parindent0mm
\hrule
\method{Methode}{Beschreibung}\hrule
\method{
  \textsf{\textbf{\small ChromosomeT\_base}}( ) }{ 
  Erzeugt ein leeres Chromosom be\-lie\-bi\-gen Typs 
  und registriert es.
}\\
\method{
  \textsf{\textbf{\small ChromosomeT\_base}}( \emph{unsigned l} ) }{ 
  Erzeugt und registriert ein leeres Chromosom beliebigen Typs mit 
  Platz f\"ur {\em l} Allele. 
}\\
\method{
  \textsf{\textbf{\small ChromosomeT\_base}}( \emph{unsigned l,\\ 
  \hspace*{1.5em}const T\emph{\&} v} ) }{ 
  Erzeugt und registriert ein Chromosom des Typs {\sl T} mit
  {\em l} Allelen vom Wert {\em v}.  
}\\
\method{
  \textsf{\textbf{\small ChromosomeT\_base}}(\\ 
  \hspace*{1.5em}\emph{const vector$<$ T $>$\emph{\&} v} ) }
  {Erzeugt und registriert ein Chromosom des Typs {\sl T} mit 
   den Allelwerten {\em v}.
}\\
\method{
  \textsf{\textbf{\small $\sim$Chromosome}}( ) }{
  Zerst\"ort das aktuelle Chromosom. 
}\\\hrule

\newpage


% -----------------------------------------------------------------------
\subsection{Operatoren}
% -----------------------------------------------------------------------

\vspace*{10pt}

\methodgrayfalse
\parindent0mm
\hrule
\method{Methode}{Beschreibung}\hrule
\method{
  \emph{Chromosome\emph{\&}} \textsf{\textbf{\small operator =\ }}(\\ 
  \hspace*{1.5em}\emph{const Chromosome\emph{\&} c} ) }{
  Weist dem aktuellen Chromosom {\em this} die 
  Werte des Chromosoms {\em c} zu.
}\\
\method{
  \emph{Chromosome\emph{\&}} \textsf{\textbf{\small operator =\ }}(\\ 
  \hspace*{1.5em}\emph{const vector$<$ T $>$\emph{\&} c} ) }{
  Weist dem aktuellen Chromosom {\em this} die Werte von
  Vektor {\em c} als neue Allelwerte zu.
}\\
\method{
  \emph{Chromosome\emph{\&}} \textsf{\textbf{\small operator =\ }}(\\ 
  \hspace*{1.5em}\emph{const T\emph{\&} c} ) }{
  Weist allen Allelen des aktuellen Chromosoms {\em this} den Wert 
  {\em c} vom Typ {\sl T} zu.
}\\
\method{
  \emph{bool} \textsf{\textbf{\small operator ==\ }}(\\ 
  \hspace*{1.5em}\emph{const Chromosome\emph{\&} c} ) }{
  Pr\"uft, ob {\em this} gleich {\em c} ist.
}\\
\method{
  \emph{bool} \textsf{\textbf{\small operator $<$\ }}(\\ 
  \hspace*{1.5em}\emph{const Chromosome\emph{\&} c} ) }{
  Pr\"uft, ob {\em this} kleiner als {\em c} ist.
}\\
\method{
  \emph{bool} \textsf{\textbf{\small operator $<=$\ }}(\\ 
  \hspace*{1.5em}\emph{const Chromosome\emph{\&} c} ) }{
  Pr\"uft, ob {\em this} kleiner oder gleich {\em c} ist.
}\\
\method{
  \emph{bool} \textsf{\textbf{\small operator $>$\ }}(\\ 
  \hspace*{1.5em}\emph{const Chromosome\emph{\&} c} ) }{
  Pr\"uft, ob {\em this} gr\"o{\ss}er als {\em c} ist.
}\\
\method{
  \emph{bool} \textsf{\textbf{\small operator $>=$\ }}(\\
  \hspace*{1.5em}\emph{const Chromosome\emph{\&} c} ) }{
  Pr\"uft, ob {\em this} gr\"o{\ss}er oder gleich {\em c} ist.
}\\
\method{
  \emph{bool} \textsf{\textbf{\small operator !$=$\ }}(\\ 
  \hspace*{1.5em}\emph{const Chromosome\emph{\&} c} ) }{
  Pr\"uft, ob {\em this} ungleich {\em c} ist.
}\\
\method{
  \emph{T} \textsf{\textbf{\small operator [\ ]\ }}(\\
  \hspace*{1.5em}\emph{unsigned allele} ) }{
  Gibt Allel Nummer {\em allele} vom Typ {\sl T} zur\"uck.
}\\\hrule


\vspace{4ex}

% -----------------------------------------------------------------------
\subsection{Informationsmethoden}
% -----------------------------------------------------------------------

\vspace*{10pt}

\methodgrayfalse
\parindent0mm
\hrule
\vspace*{1pt}
\method{Methode}{Beschreibung}\hrule
\method{
  \emph{T} \textsf{\textbf{\small min}}( 
  \emph{T a, T b} ) }{
  Vergleicht {\em a} und {\em b}, gibt den kleineren Wert zur\"uck.
}\\
\method{
  \emph{T} \textsf{\textbf{\small max}}( 
  \emph{T a, T b} ) }{
  Vergleicht {\em a} und {\em b}, gibt den gr\"o{\ss}eren Wert zur\"uck.
}\\
\method{
  \emph{bool} \textsf{\textbf{\small sameType}}(\\ 
  \hspace*{1.5em}\emph{const Chromosome\emph{\&} c} ) }{
  Pr\"uft, ob {\em this} und {\em c} denselben Typ haben.
}\\
\method{
  \emph{const char$\ast$} \textsf{\textbf{\small typeOfAlleles}}( ) }{
  Gibt den Typ des aktuellen Chromosoms zur\"uck.
}\\
\method{
  \emph{unsigned} \textsf{\textbf{\small sizeOfAlleles}}( ) }{
  Gibt die Gr\"o{\ss}e eines Chromosomenallels in Bytes zur\"uck.
}\\
\method{
  \emph{unsigned} \textsf{\textbf{\small size}}( ) }{
  Gibt die Anzahl der im Chromosom enthaltenen Allele zur\"uck.
}\\\hrule

\newpage


% -----------------------------------------------------------------------
\subsection{Strukturver\"anderungsmethoden}
% -----------------------------------------------------------------------

\vspace*{10pt}

\methodgrayfalse
\parindent0mm
\hrule
\method{Methode}{Beschreibung}\hrule
\method{
  \emph{void} \textsf{\textbf{\small resize}}( 
  \emph{unsigned n} ) }{
  Anzahl der Chromosomenallele wird auf neue Zahl {\em n} gesetzt.
}\\
\method{
  \emph{void} \textsf{\textbf{\small duplicate}}( 
  \emph{unsigned start,\\ 
  \hspace*{1.5em}unsigned stop, unsigned dest} ) }{
  Kopiert Allele von {\em start} bis {\em stop} nach {\em dest}.
}\\
\method{
  \emph{void} \textsf{\textbf{\small invert}}( 
  \emph{unsigned start,\\
  \hspace*{1.5em}unsigned stop,\\ 
  \hspace*{1.5em}unsigned granularity} ) }{
  Invertiert Reihenf\"olge der Allelbl\"ocke von {\em start} bis {\em stop}
  mit Blockgr\"o{\ss}e {\em granularity}.
}\\
\method{
  \emph{void} \textsf{\textbf{\small invert}}( 
  \emph{unsigned granularity} ) }{
  Wie zuvor, jedoch werden alle Allele invertiert.
}\\
\method{
  \emph{void} \textsf{\textbf{\small transcribe}}( 
  \emph{unsigned start,\\ 
  \hspace*{1.5em}unsigned stop,\\
  \hspace*{1.5em}const Chromosome\emph{\&} chrom} ) }{
  Kopiert Allelsequenz von {\em start} bis {\em stop} aus {\em chrom}
  in {\em this}. 
}\\
\method{
  \emph{void} \textsf{\textbf{\small swap}}( 
  \emph{unsigned i, unsigned j} ) }{
  Vertauscht die Allele {\em i} und {\em j} miteinander.
}\\
\method{
  \emph{void} \textsf{\textbf{\small shuffle}}( ) }{
  \"Andert willk\"urlich die Reihenfolge der Allele im Chromosom.
}\\
\method{
  \emph{void} \textsf{\textbf{\small replace}}( 
  \emph{unsigned i, const T\emph{\&} v} ) }{
  Allel {\em i} erh\"alt den neuen Wert {\em v}.
}\\
\method{
  \emph{void} \textsf{\textbf{\small replace}}( 
  \emph{unsigned i,\\ 
  \hspace*{1.5em}const Chromosome\emph{\&} chrom} ) }{
  Ersetzt Allele ab Position {\em i} mit Allelwerten in {\em chrom}.
}\\
\method{
  \emph{void} \textsf{\textbf{\small insert}}( 
  \emph{unsigned i,\\ 
  \hspace*{1.5em}const T\emph{\&} allele} ) }{
  F\"ugt Allel {\em allele} an Position {\em i} ein.
}\\
\method{
  \emph{void} \textsf{\textbf{\small insert}}( 
  \emph{unsigned i,\\ 
  \hspace*{1.5em}const Chromosome\emph{\&} chrom} ) }{
  F\"ugt Allele in {\em chrom} an Position {\em i} in {\em this} ein.
}\\
\method{
  \emph{void} \textsf{\textbf{\small append}}( 
  \emph{const T\emph{\&} allele} ) }{
  H\"angt Allel {\em allele} am Ende von {\em this} an.
}\\
\method{
  \emph{void} \textsf{\textbf{\small append}}(\\ 
  \hspace*{1.5em}\emph{const Chromosome\emph{\&} chrom} ) }{
  H\"angt die Allele in {\em chrom} am Ende von {\em this} an.
}\\
\method{
  \emph{void} \textsf{\textbf{\small remove}}( 
  \emph{unsigned i} ) }{
  Entfernt das Allel an Position {\em i}.
}\\
\method{
  \emph{void} \textsf{\textbf{\small remove}}( 
  \emph{unsigned i, unsigned k} ) }{
  Entfernt die Allele {\em i} bis {\em k}.
}\\
\method{
  \emph{void} \textsf{\textbf{\small rotateRight}}( 
  \emph{unsigned n} ) }{
  Rotiert alle Allele um {\em n} Positionen nach rechts.
}\\
\method{
  \emph{void} \textsf{\textbf{\small rotateLeft}}( 
  \emph{unsigned n} ) }{
  Rotiert alle Allele um {\em n} Positionen nach links.
}\\\hrule


\newpage


% -----------------------------------------------------------------------
\subsection{Crossover-Methoden}
% -----------------------------------------------------------------------

\vspace*{10pt}

\methodgrayfalse
\parindent0mm
\hrule
\method{Methode}{Beschreibung}\hrule
\method{
  \emph{void} \textsf{\textbf{\small crossover}}(\\
  \hspace*{1.5em}\emph{const Chromosome\emph{\&} dadChrom,\\ 
  \hspace*{1.5em}const Chromosome\emph{\&} momChrom,\\
  \hspace*{1.5em}const vector$<$unsigned$>$\emph{\&} points} ) }{
  Neues {\em this} entsteht durch Crossover von {\em dadChrom} und
  {\em momChrom} an den Positionen {\em points}.
}\\
\method{
  \emph{void} \textsf{\textbf{\small crossover}}( 
  \emph{Chromosome\emph{\&} mate,\\ 
  \hspace*{1.5em}const vector$<$unsigned$>$\emph{\&} points} ) }{
  Crossover von {\em this} und {\em mate} 
  an den Positionen {\em points}.
}\\
\method{%
  \emph{void} \textsf{\textbf{\small crossover}}(\\ 
  \hspace*{1.5em}\emph{const Chromosome\emph{\&} dadChrom,\\
  \hspace*{1.5em}const Chromosome\emph{\&} momChrom,\\ 
  \hspace*{1.5em}const vector$<$ bool $>$\emph{\&} pos} ) }{
  Neues {\em this} entsteht durch Crossover von {\em dadChrom} und
  {\em momChrom} an den durch ``true'' in {\em pos} gekennzeichneten 
  Positionen.
}\\
\method{
  \emph{void} \textsf{\textbf{\small crossover}}(\\
  \hspace*{1.5em}\emph{Chromosome\emph{\&} mate,\\
  \hspace*{1.7em}const vector$<$ bool $>$\emph{\&} pos} ) }{
  Crossover von {\em this} und {\em mate} 
  an den durch ``true'' in {\em pos} gekennzeichneten Positionen.
}\\
\method{
  \emph{void} \textsf{\textbf{\small crossover}}(\\
  \hspace*{1.5em}\emph{const Chromosome\emph{\&} dadChrom,\\
  \hspace*{1.5em}const Chromosome\emph{\&} momChrom,\\
  \hspace*{1.5em}unsigned npoints, unsigned align} ) }{
  Zuf\"alliges Crossover von {\em dadChrom} und {\em momChrom} in {\em this}
  an max. {\em npoints} m\"oglichen Punkten mit Mindestabstand {\em align}.
}\\
\method{
  \emph{void} \textsf{\textbf{\small crossover}}(\\ 
  \hspace*{1.5em}\emph{Chromosome\emph{\&} mate,\\
  \hspace*{1.5em}unsigned npoints, unsigned align} ) }{
  Zuf\"alliges Crossover von {\em this} und {\em mate}
  an max. {\em npoints} m\"oglichen Punkten mit Mindestabstand {\em align}.
}\\
\method{%
  \emph{void} \textsf{\textbf{\small crossover}}(\\
  \hspace*{1.5em}\emph{const Chromosome\emph{\&} dadChrom,\\ 
  \hspace*{1.5em}const Chromosome\emph{\&} momChrom,\\
  \hspace*{1.5em}const Chromosome\emph{\&} posChrom} ) }{
  Crossover von {\em dadChrom} und {\em momChrom} in {\em this}
  an den durch in {\em posChrom} mit ``1'' bezeichneten Punkten.
}\\
\method{
  \emph{void} \textsf{\textbf{\small crossover}}(\\
  \hspace*{1.5em}\emph{Chromosome\emph{\&} mate,\\ 
  \hspace*{1.5em}const Chromosome\emph{\&} posChrom} ) }{
  Crossover von {\em this} und {\em mate}
  an den durch in {\em posChrom} mit ``1'' bezeichneten Punkten.
}\\
\method{
  \emph{void} \textsf{\textbf{\small crossoverUniform}}(\\
  \hspace*{1.5em}\emph{const Chromosome\emph{\&} dadChrom,\\ 
  \hspace*{1.5em}const Chromosome\emph{\&} momChrom,\\ 
  \hspace*{1.5em}const vector$<$ bool $>$\emph{\&} pos} ) }{
  Crossover von {\em dadChrom} und {\em momChrom} in {\em this}.\\
  Position $pos_i=$``true'' $\rightarrow this_i= 
  momChrom_i$, $pos_i=$``false'' $\rightarrow this_i= 
  dadChrom_i$.
}\\
\method{
  \emph{void} \textsf{\textbf{\small crossoverUniform}}(\\
  \hspace*{1.5em}\emph{Chromosome\emph{\&} mateChrom,\\
  \hspace*{1.5em}const vector$<$ bool $>$\emph{\&} pos} ) }{
  Crossover von {\em this} und {\em mate}.\\
  Position $pos_i=$``true'' $\rightarrow this_i= 
  mateChrom_i$, $pos_i=$``false'' $\rightarrow this_i= 
  this_i$.
}\\
\method{
  \emph{void} \textsf{\textbf{\small crossoverUniform}}(\\
  \hspace*{1.5em}\emph{const Chromosome\emph{\&} dadChrom,\\ 
  \hspace*{1.5em}const Chromosome\emph{\&} momChrom} ) }{
  Zuf\"alliges Crossover von {\em dadChrom} und {\em momChrom} in {\em this}
  mit Wahrscheinlichkeit von $0.5$ pro Elternteil und Allel. 
}\\
\method{
  \emph{void} \textsf{\textbf{\small crossoverUniform}}(\\
  \hspace*{1.5em}\emph{Chromosome\emph{\&} mateChrom} ) }{
  Zuf\"alliges Crossover von {\em this} und {\em mate}
  mit Wahrscheinlichkeit von $0.5$ pro Elternteil und Allel. 
}\\\hrule

\newpage


\methodgrayfalse
\parindent0mm
\hrule
\vspace*{1pt}
\method{Methode}{Beschreibung}\hrule
\method{
  \emph{void} \textsf{\textbf{\small crossoverUniform}}(\\
  \hspace*{1.5em}\emph{const Chromosome\emph{\&} dadChrom,\\ 
  \hspace*{1.5em}const Chromosome\emph{\&} momChrom,\\
  \hspace*{1.5em}const Chromosome\emph{\&} posChrom} ) }{
  Crossover von {\em dadChrom} und {\em momChrom} in {\em this}.\\
  Position $posChrom_i=$``0'' $\rightarrow this_i= 
  dadChrom_i$, $posChrom_i<>$``0'' $\rightarrow this_i= 
  momChrom_i$.
}\\
\method{
  \emph{void} \textsf{\textbf{\small crossoverUniform}}(\\
  \hspace*{1.5em}\emph{Chromosome\emph{\&} mateChrom,\\ 
  \hspace*{1.5em}const Chromosome\emph{\&} posChrom} ) }{
  Crossover von {\em this} und {\em mate}.\\
  Position $posChrom_i=$``0'' $\rightarrow this_i= 
  mateChrom_i$, $posChrom_i<>$``0'' $\rightarrow this_i= 
  this_i$.
}\\
\method{
  \emph{void} \textsf{\textbf{\small recombineDiscrete}}(\\
  \hspace*{1.5em}\emph{const Chromosome\emph{\&} dadChrom,\\ 
  \hspace*{1.5em}const Chromosome\emph{\&} momChrom} ) }{
  Zuf\"allige Rekombination von {\em dadChrom} und {\em momChrom}
  in {\em this} mit Wahrscheinlichkeit von $0.5$ pro Elternteil
  und Allel.
}\\
\method{
  \emph{void} \textsf{\textbf{\small recombineDiscrete}}(\\
  \hspace*{1.5em}\emph{Chromosome\emph{\&} mate} ) }{
  Zuf\"allige Rekombination von {\em mate} und {\em this}
  mit Wahrscheinlichkeit von $0.5$ pro Elternteil
  und Allel.
}\\\hrule

\vspace{4ex}


% -----------------------------------------------------------------------
\subsection{Eingabemethode}
% -----------------------------------------------------------------------

\vspace*{10pt}

\methodgrayfalse
\parindent0mm
\hrule
\method{Methode}{Beschreibung}\hrule{\hspace{\linewidth}}\\
\method{%
  \emph{static Chromosome$\ast$} \textsf{\textbf{\small readNewFrom}}(\\
  \hspace*{1.5em}\emph{istream\emph{\&} is} ) }{
  Liest neue Daten f\"ur {\em this} aus {\em is} und registriert es.
}\\\hrule


\newpage

% ######################################################################
\section{Gesch\"utzte Methoden}
% ######################################################################

% -----------------------------------------------------------------------
\subsection{Erzeugungsmethoden}
% -----------------------------------------------------------------------



\methodgrayfalse
\parindent0mm
\hrule
\vspace*{1pt}
\method{Methode}{Beschreibung}\hrule{\hspace{\linewidth}}\\
\method{
  \emph{Chromosome$\ast$} \textsf{\textbf{\small clone}}( ) }{
  Gibt eine Kopie des aktuellen Chromosoms zur\"uck.
}\\
\method{
  \emph{Chromosome$\ast$} \textsf{\textbf{\small empty}}( ) }{
  Gibt ein neues leeres Chromosom zur\"uck.
}\\\hrule

\vspace{4ex}

% -----------------------------------------------------------------------
\subsection{Ein-/Ausgabemethoden}
% -----------------------------------------------------------------------

\vspace*{10pt}

\methodgrayfalse
\parindent0mm
\hrule
\vspace*{1pt}
\method{Methode}{Beschreibung}\hrule
\method{
  \emph{void} \textsf{\textbf{\small writeDataTo}}( 
  \emph{ostream\emph{\&} os} ) }{
  Schreibt das Chromosom {\em this} in den Ausgabestrom
  {\em os}.
}\\
\method{
  \emph{void} \textsf{\textbf{\small writeTo}}( 
  \emph{ostream\emph{\&} os} ) }{
  Schreibt das Chromosom {\em this} inkl. Typ und Gr\"o{\ss}e in den 
  Ausgabestrom {\em os}.
}\\
\method{
  \emph{void} \textsf{\textbf{\small readDataFrom}}( 
  \emph{istream\emph{\&} is} ) }{
  Neue Allele f\"ur {\em this} werden aus {\em is} eingelesen.
}\\
\method{
  \emph{void} \textsf{\textbf{\small readFrom}}( 
  \emph{istream\emph{\&} is} ) }{
  Neue Allele f\"ur {\em this} werden inkl. Typ und 
  Gr\"o{\ss}e des Chromosoms aus {\em is} eingelesen.
}\\\hrule


\newpage
\chapter{Klasse {\tt ChromosomeT\_num}}

% ######################################################################
\section{Frei verf\"ugbare Methoden}
% ######################################################################

% -----------------------------------------------------------------------
\subsection{Initialisierungs- und Anpassungsmethoden}
% -----------------------------------------------------------------------

\vspace*{10pt}

\methodgrayfalse

\parindent0mm
\hrule
\method{Methode}{Beschreibung}\hrule
\method{
  \emph{void} \textsf{\textbf{\small initialize}}( \emph{T min, T max} ) }{ 
  Initialisiert die Allele mit zuf\"alligen Werten zwischen
  {\em min} und {\em max}.
}\\
\method{
  \emph{void} \textsf{\textbf{\small initialize}}(\\
  \hspace*{1.5em}\emph{const Chromosome\emph{\&} minChrom,\\
  \hspace*{1.5em}const Chromosome\emph{\&} maxChrom} ) }{ 
  Initialisiert $this_i$ mit \nolinebreak $i\ =\ 0\ ..\ n-1$ 
  \linebreak mit einem zuf\"alligen Wert
  zwischen $minChrom_i$ und $maxChrom_i$.
}\\
\method{
  \emph{void} \textsf{\textbf{\small cutOff}}(
  \emph{T min, T max} ) }{
  Passt alle Allelwerte au{\ss}erhalb des Intervalls {\em min} bis {\em max}
  an die n\"achstliegende Intervallgrenze an.  
}\\
\method{
  \emph{void} \textsf{\textbf{\small cutOff}}(\\
  \hspace*{1.5em}\emph{const Chromosome\emph{\&} minChrom,\\
  \hspace*{1.5em}const Chromosome\emph{\&} maxChrom} ) }{
  Passt $this_i$ mit $i\ =\ 0\ ..\ n-1, this_i < minChrom_i$ oder
  $this_i > maxChrom_i$ an die n\"achstliegende Intervallgrenze an.
}\\\hrule

\vspace{4ex}

\newpage
% -----------------------------------------------------------------------
\subsection{Mutationsmethoden}
% -----------------------------------------------------------------------

\vspace*{10pt}

\methodgrayfalse
\parindent0mm
\hrule
\vspace*{1pt}
\method{Methode}{Beschreibung}\hrule
\method{
  \emph{void} \textsf{\textbf{\small mutateUniform}}( 
  \emph{T min,\\
  \hspace*{1.5em}T max,\\
  \hspace*{1.5em}const vector$<$ double $>$\emph{\&} p,\\
  \hspace*{1.5em}bool cycle = ``false''} ) }{
  Ersetzt $this_i$ ($i\ =\ 0\ ..\ n-1$) mit Wahrscheinlichkeit $p_i$
  durch Zufallswert zwischen $min_i$ und $max_i$. 
  Bei {\em cycle} = ``true'' mehrfaches Durchlaufen von {\em p} 
  m\"oglich.
}\\
\method{
  \emph{void} \textsf{\textbf{\small mutateUniform}}( 
  \emph{T min,\\
  \hspace*{1.5em}T max,\\
  \hspace*{1.5em}const Chromosome\emph{\&} p,\\
  \hspace*{1.5em}bool cycle = ``false''} ) }{
  Ersetzt $this_i$ ($i\ =\ 0\ ..\ n-1$) mit Wahrscheinlichkeit $p_i$
  durch Zufallswert zwischen $min_i$ und $max_i$. 
  Bei {\em cycle} = ``true'' mehrfaches Durchlaufen von {\em p} 
  m\"oglich.
}\\
\method{
  \emph{void} \textsf{\textbf{\small mutateUniform}}(\\
  \hspace*{1.5em}\emph{const Chromosome\emph{\&} minChrom,\\
  \hspace*{1.5em}const Chromosome\emph{\&} maxChrom,\\
  \hspace*{1.5em}double p} ) }{
  Ersetzt $this_i$ mit \nolinebreak $i\ =\ 0\ ..\ n-1$ 
  \linebreak mit Wahrscheinlichkeit {\em p} durch einen zuf\"alligen 
  Wert zwischen $minChrom_i$ und $maxChrom_i$.
}\\
\method{
  \emph{void} \textsf{\textbf{\small mutateUniform}}(\\
  \hspace*{1.5em}\emph{const Chromosome\emph{\&} minChrom,\\
  \hspace*{1.5em}const Chromosome\emph{\&} maxChrom,\\
  \hspace*{1.5em}const vector$<$ double $>$\emph{\&} p,\\
  \hspace*{1.5em}bool cycle = ``false''} ) }{
  Ersetzt $this_i$ ($i\ =\ 0\ ..\ n-1$) mit Wahrscheinlichkeit $p_i$
  durch Zufallswert zwischen $minChrom_i$ und $maxChrom_i$. 
  Bei {\em cycle} = ``true'' mehrfaches Durchlaufen von {\em p} 
  m\"oglich.
}\\
\method{
  \emph{void} \textsf{\textbf{\small mutateUniform}}(\\
  \hspace*{1.5em}\emph{const Chromosome\emph{\&} min,\\
  \hspace*{1.5em}const Chromosome\emph{\&} max,\\
  \hspace*{1.5em}const Chromosome\emph{\&} p,\\
  \hspace*{1.5em}bool cycle = ``false''} ) }{
  Ersetzt $this_i$ ($i\ =\ 0\ ..\ n-1$) mit Wahrscheinlichkeit $p_i$
  durch Zufallswert zwischen $min_i$ und $max_i$. 
  Bei {\em cycle} = ``true'' mehrfaches Durchlaufen von {\em p} 
  m\"oglich.
}\\\hrule

\newpage
\chapter{Klasse {\tt ChromosomeT$<$ int $>$}}

% ######################################################################
\section{Frei verf\"ugbare Methoden}
% ######################################################################

% -----------------------------------------------------------------------
\subsection{Mutationsmethoden}
% -----------------------------------------------------------------------

\vspace*{10pt}

\methodgrayfalse

\parindent0mm
\hrule
\vspace{1pt}
\method{Methode}{Beschreibung}\hrule
\method{
  \emph{void} \textsf{\textbf{\small mutateDiffGeom}}( 
  \emph{double s}) }{ 
  Zum Wert jedes Allels wird eine Zufallszahl addiert.
  {\em s} gibt die Schrittweite f\"ur die spezielle
  Zufallszahlenverteilung (s. ausf\"uhrliche Referenz) an.
}\\
\method{
  \emph{void} \textsf{\textbf{\small mutateDiffGeom}}(\\ 
  \hspace*{1.5em}\emph{const vector$<$ double $>$\emph{\&} s,\\
  \hspace*{1.5em}bool cycle = ``false''} ) }{ 
  Zu jedem Wert der Allele $this_i$ $(i\ =\ 0..n-1)$ wird eine
  Zufallszahl mit spezieller
  Verteilung (s. ausf\"uhrliche Referenz) mit Schrittweite
  $s_i$ addiert. Bei {\em cycle} = ``true'' kann
  {\em s} mehrmals durchlaufen werden.
}\\
\method{
  \emph{void} \textsf{\textbf{\small mutateDiffGeom}}(\\ 
  \hspace*{1.5em}\emph{const ChromosomeT$<$ double $>$\emph{\&} s,\\
  \hspace*{1.5em}bool cycle = ``false''} ) }{ 
  Zu jedem Wert der Allele $this_i$ $(i\ =\ 0..n-1)$ wird eine
  Zufallszahl nach spezieller
  Verteilung (s. ausf\"uhrliche Referenz) mit Schrittweite
  $s_i$ addiert. Bei {\em cycle} = ``true'' kann
  {\em s} mehrmals durchlaufen werden.
}\\
\method{
  \emph{void} \textsf{\textbf{\small mutateDiffGeom}}(\\ 
  \hspace*{1.5em}\emph{const Chromosome\emph{\&} s,\\
  \hspace*{1.5em}bool cycle = ``false''} ) }{ 
  Zu jedem Wert der Allele $this_i$ $(i\ =\ 0..n-1)$ wird eine
  Zufallszahl nach spezieller
  Verteilung (s. ausf\"uhrliche Referenz) mit Schrittweite
  $s_i$ addiert. Bei {\em cycle} = ``true'' kann
  {\em s} mehrmals durchlaufen werden.
}\\\hrule


\newpage
\chapter{Klasse {\tt ChromosomeT$<$ double $>$}}

% ######################################################################
\section{Interne Klasse {\tt DerandomConst}}
% ######################################################################

% -----------------------------------------------------------------------
\subsection{Variablen}
% -----------------------------------------------------------------------

\vspace*{10pt}

\methodgrayfalse

\parindent0mm
\hrule
\vspace*{1pt}
\method{Variable}{Beschreibung}\hrule
\method{
  \textsf{\textbf{\small dim}}}{ 
  Dimension von {\em this}.
}\\
\method{
  \textsf{\textbf{\small C}}}{ 
  Kumulationszeitraum, initialisiert mit $\frac{1.0}{\sqrt{n}}$.
}\\
\method{
  \textsf{\textbf{\small Cu}}}{ 
  Normalisierungsfaktor f\"ur den Summierungsvektor {\em s}, 
  initialisiert mit $\sqrt{\frac{2.0 - C}{C}}$.
}\\
\method{
  \textsf{\textbf{\small Crr}}}{ 
  Kumulationszeitraum f\"ur die Vorzugsrichtung, initialisiert mit  
  $\frac{3.0}{n}$. 
}\\
\method{
  \textsf{\textbf{\small Beta}}}{ 
  D\"ampfung der globalen Schrittweitenanpassung, initialisiert mit  
  $\frac{1.0}{\sqrt{n}}$.
}\\
\method{
  \textsf{\textbf{\small BetaI}}}{ 
  D\"ampfung der individuellen Schrittweitenanpassung, initialisiert mit 
  $\frac{1}{n}$.
}\\
\method{
  \textsf{\textbf{\small BetaR}}}{ 
  D\"ampfung der Richtung, initialisiert mit
  $\frac{1.0}{\sqrt{4n}}$. 
}\\
\method{
  \textsf{\textbf{\small ChiN}}}{ 
  Erwartungswert der $\chi_n$-Verteilung, initialisiert mit 
  $\sqrt{n}(1 - \frac{1}{4n} + \frac{1}{21n^2})$.
}\\
\method{
  \textsf{\textbf{\small Chi1}}}{ 
  Erwartungswert der $\chi_1$-Verteilung, initialisiert mit 
  $\sqrt{\frac{2}{\pi}}$.
}\\\hrule

\vspace{10ex}

\newpage
% -----------------------------------------------------------------------
\subsection{Methoden}
% -----------------------------------------------------------------------

\vspace*{10pt}

\methodgrayfalse
\parindent0mm
\hrule
\method{Methode}{Beschreibung}\hrule
\method{
  \textsf{\textbf{\small DerandomConst}}( 
  \emph{unsigned n} ) }{
  Initialisiert alle internen Variablen. {\em dim} wird auf
  {\em n} gesetzt.
}\\
\method{
  \emph{unsigned} \textsf{\textbf{\small nobj}}( ) }{
  Gibt den Dimensionswert {\em dim} zur\"uck.
}\\
\method{
  \emph{unsigned} \textsf{\textbf{\small npar}}( ) }{
  Gibt die Anzahl der ben\"otigten Parameter f\"ur die {\em Derandom}-
  Mutationsmethode zur\"uck.
}\\\hrule


\newpage

% ######################################################################
\section{Frei verf\"ugbare Methoden}
% ######################################################################


% -----------------------------------------------------------------------
\subsection{Initialisierungsmethoden}
% -----------------------------------------------------------------------

\vspace*{10pt}

\methodgrayfalse
\parindent0mm
\hrule
\vspace*{1pt}
\method{Methode}{Beschreibung}\hrule
\method{
  \emph{void} \textsf{\textbf{\small initializeRotate}}(\\ 
  \hspace*{1.5em}\emph{double SigmaMin,\\
  \hspace*{1.5em}double SigmaMax} ) }{
  Initialisiert die Allele von {\em this} f\"ur die 
  Methode {\em mutateRotate} (s.u.) mit Zufallszahlen
  zwischen {\em SigmaMin} und {\em SigmaMax}.   
}\\
\method{
  \emph{void} \textsf{\textbf{\small initializeRotate}}(\\ 
  \hspace*{1.5em}\emph{const vector$<$ double $>$ \emph{\&} sigma} ) }{
  Initialisiert die Allele von {\em this} f\"ur die 
  Methode {\em mutateRotate} (s.u.) mit den in {\em sigma}
  stehenden Werten.
}\\
\method{
  \emph{void} \textsf{\textbf{\small initializeRotate}}(\\ 
  \hspace*{1.5em}\emph{const ChromosomeT$<$ double $>$\emph{\&}\\
  \hspace*{2.5em}sigma} ) }{
  Initialisiert die Allele von {\em this} f\"ur die 
  Methode {\em mutateRotate} (s.u.) mit den in {\em sigma}
  stehenden Werten.
}\\
\method{
  \emph{void} \textsf{\textbf{\small initializeDerandom}}(\\
  \hspace*{1.5em}\emph{double minVal,\\ 
  \hspace*{1.5em}double maxVal} ) }{
  Initialisiert die Allele von {\em this} f\"ur die Methode 
  {\em mutateDerandom} (s.u.) mit Zufallszahlen zwischen {\em minVal}
  und \nolinebreak {\em maxVal}. \linebreak
}\\
\method{
  \emph{void} \textsf{\textbf{\small initializeCMA}}(\\
  \hspace*{1.5em}\emph{double SigmaMin,\\
  \hspace*{1.5em}double SigmaMax} ) }{
  Initialisiert die Allele von {\em this} f\"ur die Methode {\em mutateCMA} 
  (s.u.) mit Zufallszahlen. $this_0$ erh\"alt dabei einen Wert
  zwischen {\em SigmaMin} und {\em SigmaMax}.
}\\
\method{
  \emph{void} \textsf{\textbf{\small initializeCMA}}(\\ 
  \hspace*{1.5em}\emph{const vector$<$ double $>$ \emph{\&} sigma} ) }{
  Initialisiert die Allele von {\em this} f\"ur
  die Methode {\em mutateCMA} (s.u.) mit den in {\em sigma} 
  stehenden Werten.
}\\
\method{
  \emph{void} \textsf{\textbf{\small initializeCMA}}(\\ 
  \hspace*{1.5em}\emph{const ChromosomeT$<$ double $>$\emph{\&}\\
  \hspace*{2.5em}sigma} ) }{
  Initialisiert die Allele von {\em this} f\"ur
  die Methode {\em mutateCMA} (s.u.) mit den in {\em sigma} 
  stehenden Werten.
}\\
\method{
  \emph{void} \textsf{\textbf{\small initializeGSA}}(\\
  \hspace*{1.5em}\emph{double SigmaMin,\\
  \hspace*{1.5em}double SigmaMax} ) }{
  Initialisiert die Allele von {\em this} f\"ur die Methode {\em mutateGSA} 
  (s.u.) mit Zufallszahlen. $this_0$ erh\"alt dabei einen
  Wert zwischen {\em SigmaMin} und {\em SigmaMax}.
}\\
\method{
  \emph{void} \textsf{\textbf{\small initializeGSA}}(\\ 
  \hspace*{1.5em}\emph{const vector$<$ double $>$ \emph{\&} sigma,\\
  \hspace*{1.5em}int baseSize} ) }{
  Initialisiert die Allele von {\em this} mit den in {\em sigma} stehenden 
  Werten f\"ur die Methode {\em mutateGSA} (s.u.) mit 
  {\em baseSize} Vektoren der generierenden
  Menge.
}\\
\method{
  \emph{void} \textsf{\textbf{\small initializeGSA}}(\\ 
  \hspace*{1.5em}\emph{const ChromosomeT$<$ double $>$\emph{\&}\\
  \hspace*{2.5em}sigma,\\
  \hspace*{1.5em}int baseSize} ) }{
  Initialisiert die Allele von {\em this} mit den in {\em sigma} stehenden 
  Werten f\"ur die Methode {\em mutateGSA} (s.u.) mit 
  {\em baseSize} Vektoren der generierenden
  Menge.
}\\\hrule

\newpage

\methodgrayfalse
\parindent0mm
\hrule
\vspace*{1pt}
\method{Methode}{Beschreibung}\hrule
\method{
  \emph{void} \textsf{\textbf{\small initializeIDA}}(\\
  \hspace*{1.5em}\emph{double SigmaMin,\\
  \hspace*{1.5em}double SigmaMax} ) }{
  Initialisiert die Allele von {\em this}
  f\"ur die Methode {\em mutateIDA} (s.u.) mit Zufallszahlen. Diverse
  Allele werden dabeit mit Werten zwischen {\em SigmaMin} und {\em SigmaMax}
  belegt.
}\\
\method{
  \emph{void} \textsf{\textbf{\small initializeIDA}}(\\
  \hspace*{1.5em}\emph{const vector$<$ double $>$ \emph{\&} sigma} ) }{
  Initialisiert die Allele von {\em this} f\"ur
  die Methode {\em mutateIDA} (s.u.)
  mit den in {\em sigma} stehenden Werten.
}\\
\method{
  \emph{void} \textsf{\textbf{\small initializeIDA}}(\\
  \hspace*{1.5em}\emph{const ChromosomeT$<$ double $>$\emph{\&}\\
  \hspace*{2.5em}sigma} ) }{
  Initialisiert die Allele von {\em this} f\"ur
  die Methode {\em mutateIDA} (s.u.)
  mit den in {\em sigma} stehenden Werten.
}\\
\method{
  \emph{void} \textsf{\textbf{\small initializeIDAiso}}(\\
  \hspace*{1.5em}\emph{double SigmaMin,\\
  \hspace*{1.5em}double SigmaMax} ) }{
  Initialisiert die Allele von {\em this} f\"ur die Methode 
  {\em mutateIDAiso} (s.u.) mit Zufallszahlen. Diverse Allele
  werden dabei mit Werten zwischen {\em SigmaMin} und {\em SigmaMax}
  belegt.
}\\
\method{
  \emph{void} \textsf{\textbf{\small initializeIDAiso}}(\\
  \hspace*{1.5em}\emph{const vector$<$ double $>$ \emph{\&} sigma} ) }{
  Initialisiert die Allele von {\em this} f\"ur
  die Methode {\em mutateIDAiso} (s.u.) mit den in {\em sigma} 
  stehenden Werten.
}\\
\method{
  \emph{void} \textsf{\textbf{\small initializeIDAiso}}(\\
  \hspace*{1.5em}\emph{const ChromosomeT$<$ double $>$\emph{\&}\\
  \hspace*{2.5em}sigma} ) }{
  Initialisiert die Allele von {\em this} f\"ur
  die Methode {\em mutateIDAiso} (s.u.) mit den in {\em sigma} 
  stehenden Werten.
}\\\hrule

\vspace*{10ex}

% -----------------------------------------------------------------------
\subsection{Dekodierungsmethode}
% -----------------------------------------------------------------------

\vspace*{10pt}

\methodgrayfalse
\parindent0mm
\hrule
\method{Methode}{Beschreibung}\hrule
\method{
  \emph{void} \textsf{\textbf{\small decodeBinary}}(\\
  \hspace*{1.5em}\emph{const Chromosome\emph{\&} chrom,\\
  \hspace*{1.5em}const Interval\emph{\&} range,\\
  \hspace*{1.5em}bool useGray = ``false''} ) }{
  Dekodiert die als Bitstring in {\em chrom} (bei {\em useGray} = ``true''
  per Gray-Kodierung) mit je {\em nbits} Bits kodierten Zahlen 
  aus dem Intervall {\em range} und legt sie in {\em this} ab.
}\\\hrule

\newpage

% -----------------------------------------------------------------------
\subsection{Kumulationsmethoden}
% -----------------------------------------------------------------------

\vspace*{10pt}

\methodgrayfalse
\parindent0mm
\hrule
\method{Methode}{Beschreibung}\hrule
\method{
  \emph{void} \textsf{\textbf{\small accumulate}}(\\
  \hspace*{1.5em}\emph{const vector$<$ double $>$ \emph{\&} acc,\\
  \hspace*{1.5em}double c} ) }{
  Addiert $acc_i$ (mit $i\ =\ 0\ ..\ n-1$) mit Gewichtung $c_i$ zu 
  $this_i$.
}\\
\method{
  \emph{void} \textsf{\textbf{\small accumulate}}(\\
  \hspace*{1.5em}\emph{const Chromosome \emph{\&} acc,\\
  \hspace*{1.5em}double c} ) }{
  Addiert $acc_i$ (mit $i\ =\ 0\ ..\ n-1$) mit Gewichtung $c_i$ zu 
  $this_i$.
}\\\hrule

\vspace*{10ex}


% -----------------------------------------------------------------------
\subsection{Mutationsmethoden}
% -----------------------------------------------------------------------

\vspace*{10pt}

\methodgrayfalse
\parindent0mm
\hrule
\vspace*{1pt}
\method{Methode}{Beschreibung}\hrule
\method{
  \emph{void} \textsf{\textbf{\small mutateNormal}}(
  \emph{double stddev} ) }{
  Eine normalverteilte Zufallszahl zwischen ``0'' und {\em stddev} wird
  zu allen Allelen addiert.
}\\
\method{
  \emph{void} \textsf{\textbf{\small mutateNormal}}(\\
  \hspace*{1.5em}\emph{const vector$<$ double $>$ \emph{\&} stddev,\\
  \hspace*{1.5em}bool cycle} ) }{
  Eine normalverteilte Zufallszahl zwischen ``0'' und $stddev_i$ 
  (mit $i\ =\ 0..n-1$) wird zu $this_i$ addiert. Bei {\em cycle} = ``true''
  kann {\em stddev} mehrmals durchlaufen werden.
}\\
\method{
  \emph{void} \textsf{\textbf{\small mutateNormal}}(\\
  \hspace*{1.5em}\emph{const Chromosome \emph{\&} stddev,\\
  \hspace*{1.5em}bool cycle} ) }{
  Eine normalverteilte Zufallszahl zwischen ``0'' und $stddev_i$ 
  (mit $i\ =\ 0..n-1$) wird zu $this_i$ addiert. Bei {\em cycle} = ``true''
  kann {\em stddev} mehrmals durchlaufen werden.
}\\
\method{
  \emph{void} \textsf{\textbf{\small mutateNormal}}(\\
  \hspace*{1.5em}\emph{const ChromosomeT$<$ double $>$\emph{\&}\\
  \hspace*{2.5em} stddev,\\
  \hspace*{1.5em}bool cycle} ) }{
  Eine normalverteilte Zufallszahl zwischen ``0'' und $stddev_i$ 
  (mit $i\ =\ 0..n-1$) wird zu $this_i$ addiert. Bei {\em cycle} = ``true''
  kann {\em stddev} mehrmals durchlaufen werden.
}\\
\method{
  \emph{void} \textsf{\textbf{\small mutateCauchy}}(
  \emph{double scale} ) }{
  Zu allen Allelen werden mit {\em scale} skalierte Cauchy-Zufallszahlen
  addiert.
}\\
\method{
  \emph{void} \textsf{\textbf{\small mutateCauchy}}(\\
  \hspace*{1.5em}\emph{const vector$<$ double $>$\emph{\&} scale,\\
  \hspace*{1.5em}bool cycle = ``false''} ) }{
  Zu $this_i$ (mit i\ =\ 0..n-1) werden mit $scale_i$ skalierte 
  Cauchy-Zufallszahlen addiert. Bei {\em cycle} = ``true'' darf
  {\em scale} mehrmals durchlaufen werden.
}\\
\method{
  \emph{void} \textsf{\textbf{\small mutateCauchy}}(\\
  \hspace*{1.5em}\emph{const Chromosome\emph{\&} scale,\\
  \hspace*{1.5em}bool cycle = ``false''} ) }{
  Zu $this_i$ (mit i\ =\ 0..n-1) werden mit $scale_i$ skalierte 
  Cauchy-Zufallszahlen addiert. Bei {\em cycle} = ``true'' darf
  {\em scale} mehrmals durchlaufen werden.
}\\\hrule

\newpage

\methodgrayfalse
\parindent0mm
\hrule
\method{Methode}{Beschreibung}\hrule
\method{
  \emph{void} \textsf{\textbf{\small mutateCauchy}}(\\
  \hspace*{1.5em}\emph{const ChromosomeT$<$ double $>$\emph{\&}\\
  \hspace*{2.5em}scale,\\
  \hspace*{1.5em}bool cycle = ``false''} ) }{
  Zu $this_i$ (mit i\ =\ 0..n-1) werden mit $scale_i$ skalierte 
  Cauchy-Zufallszahlen addiert. Bei {\em cycle} = ``true'' darf
  {\em scale} mehrmals durchlaufen werden.
}\\\hrule

\vspace*{10ex}

% -----------------------------------------------------------------------
\subsection{Rekombinationsmethoden}
% -----------------------------------------------------------------------

\vspace*{10pt}

\methodgrayfalse
\parindent0mm
\hrule
\vspace*{1pt}
\method{Methode}{Beschreibung}\hrule
\method{
  \emph{void} \textsf{\textbf{\small recombineIntermediate}}(\\
  \hspace*{1.5em}\emph{const Chromosome\emph{\&} dadChrom,\\
  \hspace*{1.5em}const Chromosome\emph{\&} momChrom} ) }{
  $this_i$ (mit $i\ =\ 0\ ..\ n-1$) erh\"alt den Durchschnitt von
  $dadChrom_i$ und $momChrom_i$ zugewiesen.
}\\
\method{
  \emph{void} \textsf{\textbf{\small recombineGenIntermediate}}(\\
  \hspace*{1.5em}\emph{const Chromosome\emph{\&} dadChrom,\\
  \hspace*{1.5em}const Chromosome\emph{\&} momChrom} ) }{
  $this_i$ (mit $i\ =\ 0\ ..\ n-1$) erh\"alt gleichverteilte Zufallswerte
  zwischen $dadChrom_i$ und $momChrom_i$ zugewiesen.
}\\
\method{
  \emph{void} \textsf{\textbf{\small recombineGeomIntermediate}}(\\
  \hspace*{1.5em}\emph{const Chromosome\emph{\&} dadChrom,\\
  \hspace*{1.5em}const Chromosome\emph{\&} momChrom} ) }{
  $this_i$ (mit $i\ =\ 0\ ..\ n-1$) erh\"alt den Wert
  $\sqrt{dadChrom_i \cdot momChrom_i}$ zugewiesen.
}\\
\method{
  \emph{void} \textsf{\textbf{\small recombineIntermediate}}(\\
  \hspace*{1.5em}\emph{Chromosome\emph{\&} mate} ) }{
  $this_i$ (mit $i\ =\ 0\ ..\ n-1$) erh\"alt den Durchschnitt von
  $mate_i$ und $this_{i_{alt}}$ zugewiesen.  
}\\
\method{
  \emph{void} \textsf{\textbf{\small recombineGenIntermediate}}(\\
  \hspace*{1.5em}\emph{Chromosome\emph{\&} mate} ) }{
  $this_i$ und $mate_i$ (mit $i\ =\ 0\ ..\ n-1$) erhalten gleichverteilte
  Zufallszahlen zwischen $this_{i_{alt}}$ und $mate_{i_{alt}}$ zugewiesen.  
}\\
\method{
  \emph{void} \textsf{\textbf{\small recombineGeomIntermediate}}(\\
  \hspace*{1.5em}\emph{Chromosome\emph{\&} mate} ) }{
  $this_i$ und $mate_i$ (mit $i\ =\ 0\ ..\ n-1$) erhalten die Werte
  $\sqrt{this_{i_{alt}} \cdot mate_{i_{alt}}}$ zugewiesen.  
}\\\hrule

\newpage

% -----------------------------------------------------------------------
\subsection{Selbstadaptationsmethoden}
% -----------------------------------------------------------------------

\vspace*{10pt}

\methodgrayfalse
\parindent0mm
\hrule
\method{Methode}{Beschreibung}\hrule
\method{
  \emph{void} \textsf{\textbf{\small mutateLogNormal}}(\\
  \hspace*{1.5em}\emph{double overallStdDev,\\
  \hspace*{1.5em}double indivStdDev} ) }{
  Zu $this_i$ (mit $i\ =\ 0\ ..\ n-1$) wird eine normalverteilte Zufallszahl
  zwischen ``0'' und $sigma_i^2$ addiert, $sigma_i$ wird f\"ur jedes
  Allel einzeln ({\em indivStdDev}) und f\"ur alle Allele gemeinsam
  ({\em overallStdDev}) mutiert. N\"aheres siehe in ausf\"uhrlicher Referenz.
}\\
\method{
  \emph{void} \textsf{\textbf{\small mutateRotate}}(\\
  \hspace*{1.5em}\emph{ChromosomeT$<$ double $>$\emph{\&}\\
  \hspace*{2.5em}sigma} ) }{
  F\"uhrt eine Mutation von {\em this} und den Stategievariablen auf der
  Basis einer Rotationsmatrix mit den Standardabweichungen in {\em sigma}
  aus. N\"aheres siehe in ausf\"uhrlicher Referenz.
}\\
\method{
  \emph{void} \textsf{\textbf{\small mutateRotate}}(\\
  \hspace*{1.5em}\emph{ChromosomeT$<$ double $>$\emph{\&} sigma,\\
  \hspace*{1.5em}double tau1, double tau2,\\
  \hspace*{1.5em}double beta, int sigmaCheck\\
  \hspace*{1.5em}double epsi} ) }{
  F\"uhrt eine Mutation von {\em this} und den Stategievariablen auf der
  Basis einer Rotationsmatrix mit den Standardabweichungen in
  {\em sigma} f\"ur die Normalverteilung, der Schrittweitenanpassung 
  f\"ur alle Individuen ({\em tau1}) und f\"ur einzelne Individuen
  ({\em tau2}), der D\"ampfung {\em beta} der Schrittweitenvariation zwischen 
  aufeinanderfolgenden Generationen und der unteren Grenze {\em epsi}
  f\"ur die $\sigma$-Werte aus. {\em sigmaCheck} gibt an, ob {\em sigma}
  auf zu kleine Werte \"uberpr\"uft werden soll (``true'') oder nicht
  (``false''). N\"aheres siehe in ausf\"uhrlicher Referenz.
}\\
\method{
  \emph{void} \textsf{\textbf{\small mutateNormalRotAngles}}(\\
  \hspace*{1.5em}\emph{const Chromosome\emph{\&} sigma,\\
  \hspace*{1.5em}const Chromosome\emph{\&} alpha} ) }{
  F\"uhrt eine Mutation von {\em this} und den Stategievariablen auf der
  Basis einer Rotationsmatrix mit den Standardabweichungen in
  {\em sigma} f\"ur die Normalverteilung und den Angelpunkten
  {\em alpha} f\"ur die Rotationen aus. N\"aheres siehe in ausf\"uhrlicher 
  Referenz.
}\\
\method{
  \emph{void} \textsf{\textbf{\small mutateDerandom}}(\\
  \hspace*{1.5em}\emph{vector$<$ double $>$\emph{\&} v,\\
  \hspace*{1.5em}const DerandomConst\emph{\&} K} ) }{
  F\"uhrt eine Mutation von {\em this} nach der 
  {\em Derandom}-Methode mit den Parameterwerten {\em v} und
  den Konstanten {\em K} aus. N\"aheres siehe in ausf\"uhrlicher Referenz.
}\\
\method{
  \emph{void} \textsf{\textbf{\small mutateDerandom}}(\\
  \hspace*{1.5em}\emph{Chromosome\emph{\&} chrom,\\
  \hspace*{1.5em}const DerandomConst\emph{\&} K} ) }{
  F\"uhrt eine Mutation von {\em this} nach der 
  {\em Derandom}-Methode mit den Parameterwerten {\em chrom} und
  den Konstanten {\em K} aus. N\"aheres siehe in ausf\"uhrlicher Referenz.
}\\\hrule

\newpage

\methodgrayfalse
\parindent0mm
\hrule
\method{Methode}{Beschreibung}\hrule
\method{
  \emph{void} \textsf{\textbf{\small mutateCMA}}(\\
  \hspace*{1.5em}\emph{ChromosomeT$<$ double $>$\emph{\&}\\
  \hspace*{2.5em}sigma} ) }{
  Mutiert {\em this} und die Strategieparameter anhand der 
  Mutationsmethode mit Kovarianzmatrix ({\em Covariance Matrix Adaptation})
  und den Standardabweichungen in {\em sigma} f\"ur die Normalverteilung.
  N\"aheres siehe in ausf\"uhrlicher Referenz.
}\\
\method{
  \emph{void} \textsf{\textbf{\small mutateCMA}}(\\
  \hspace*{1.5em}\emph{ChromosomeT$<$ double $>$ \emph{\&}\\
  \hspace*{2.5em}sigma,\\
  \hspace*{1.5em}double c, double cu,\\
  \hspace*{1.5em}double ccov, double beta} ) }{
  Mutiert {\em this} und die Strategieparameter anhand der 
  Mutationsmethode mit Kovarianzmatrix ({\em Covariance Matrix Adaptation})
  und den Standardabweichungen in {\em sigma} f\"ur die Normalverteilung,
  dem Wert {\em c} zur Bestimmung der Kumulationszeit f\"ur den 
  Summierungsvektor {\em s}, der Normalisierung {\em cu} der Varianz des 
  Summierungsvektors, der Zeit {\em ccov} zur Durchschnittsbestimmung
  der Verteilung $ss^t$ \"uber der Generationssequenz und dem
  Parameter {\em beta} zur D\"ampfung der Schrittweitenvariation zwischen 
  aufeinanderfolgenden Generationen. N\"aheres siehe in ausf\"uhrlicher 
  Referenz.
}\\
\method{
  \emph{void} \textsf{\textbf{\small mutateMSR}}(
  \emph{double xi\_prob} ) }{
  Mutiert {\em this} und die Strategieparameter anhand der Methode der 
  {\em mutativen Schrittweitenregelung} mit Wahrscheinlichkeit {\em xi\_prob},
  da{\ss} der generelle Schrittweiten\"anderungsfaktor $\xi$ den Wert 
  $\alpha = 1.5$ erh\"alt. N\"aheres siehe in ausf\"uhrlicher Referenz.
}\\
\method{
  \emph{void} \textsf{\textbf{\small mutateGSA}}(\\
  \hspace*{1.5em}\emph{ChromosomeT$<$ double $>$\emph{\&}\\
  \hspace*{2.5em}sigma} ) }{
  Mutiert {\em this} und die Strategieparameter anhand der 
  mengengenerierenden Adaptation ({\em Generating Set Adaptation}) mit den
  Standardabweichungen in {\em sigma} f\"ur die Normalverteilung.
  N\"aheres siehe in ausf\"uhrlicher Referenz.
}\\\hrule

\newpage

\methodgrayfalse
\parindent0mm
\hrule
\method{Methode}{Beschreibung}\hrule
\method{
  \emph{void} \textsf{\textbf{\small mutateGSA}}(\\
  \hspace*{1.5em}\emph{ChromosomeT$<$ double $>$\emph{\&}\\
  \hspace*{2.5em}sigma\\
  \hspace*{1.5em}double beta, double xi\_const,\\
  \hspace*{1.5em}double cu, double cm} ) }{
  Mutiert {\em this} und die Strategieparameter anhand der 
  mengengenerierenden Adaptation ({\em Generating Set Adaptation}) mit den
  Standardabweichungen in {\em sigma} f\"ur die Normalverteilung, dem
  Parameter {\em beta} zur D\"ampfung der Schrittweitenvariation zwischen 
  aufeinanderfolgenden Generationen, dem Faktor {\em xi\_const} zur 
  Schrittweitenanpassung, dem Normalisierungsfaktor {\em cu} f\"ur Varianzen
  und dem L\"angenanpassungsfaktor {\em cm}.
  N\"aheres siehe in ausf\"uhrlicher Referenz.
}\\
\method{
  \emph{void} \textsf{\textbf{\small mutateIDA}}(\\
  \hspace*{1.5em}\emph{ChromosomeT$<$ double $>$\emph{\&}\\
  \hspace*{2.5em}sigma} ) }{
  Mutiert {\em this} und die Strategieparameter anhand der 
  Adaptation in eine Richtung mit individuellen Schrittweiten
  ({\em Individual step sizes and one Direction Adaptation}) mit den
  Standardabweichungen in {\em sigma} f\"ur die Normalverteilung.
  N\"aheres siehe in ausf\"uhrlicher Referenz.
}\\
\method{
  \emph{void} \textsf{\textbf{\small mutateIDA}}(\\
  \hspace*{1.5em}\emph{ChromosomeT$<$ double $>$\emph{\&}\\
  \hspace*{2.5em}sigma\\
  \hspace*{1.5em}double c, double c\_r\\
  \hspace*{1.5em}double beta, double beta\_ind,\\
  \hspace*{1.5em}double beta\_r, double cu, double xi} ) }{
  Mutiert {\em this} und die Strategieparameter anhand der 
  Adaptation in eine Richtung mit individuellen Schrittweiten
  ({\em Individual step sizes and one Direction Adaptation}) mit den
  Standardabweichungen in {\em sigma} f\"ur die Normalverteilung,
  der Kumulationszeit {\em c\_r}, dem Parameter zur D\"ampfung der 
  Schrittweitenvariation zwischen aufeinanderfolgenden Generationen
  f\"ur alle Individuen ({\em beta}), f\"ur ein Individuum ({\em beta\_ind})
  und f\"ur die Vorzugsrichtung ({\em beta\_r}), sowie dem 
  Normalisierungsfaktor {\em cu} f\"ur Varianzen und dem Faktor
  {\em xi} zur Schrittweitenanpassung.
  N\"aheres siehe in ausf\"uhrlicher Referenz.
}\\
\method{
  \emph{void} \textsf{\textbf{\small mutateIDAiso}}(\\
  \hspace*{1.5em}\emph{ChromosomeT$<$ double $>$\emph{\&}\\
  \hspace*{2.5em}sigma} ) }{
  Mutiert {\em this} und die Strategieparameter anhand der 
  isotrophen Adaptation in eine Richtung mit individuellen Schrittweiten 
  ({\em Individual step sizes and one Direction Adaptation isotropic}) mit
  den Standardabweichungen in {\em sigma} f\"ur die Normalverteilung.
  N\"aheres siehe in ausf\"uhrlicher Referenz.
}\\\hrule

\newpage

\methodgrayfalse
\parindent0mm
\hrule
\method{Methode}{Beschreibung}\hrule
\method{
  \emph{void} \textsf{\textbf{\small mutateIDAiso}}(\\
  \hspace*{1.5em}\emph{ChromosomeT$<$ double $>$\emph{\&}\\
  \hspace*{2.5em}sigma\\
  \hspace*{1.5em}double c, double beta,\\
  \hspace*{1.5em}double beta\_ind, double cu} ) }{
  Mutiert {\em this} und die Strategieparameter anhand der 
  isotrophen Adaptation in eine Richtung mit individuellen Schrittweiten 
  ({\em Individual step sizes and one Direction Adaptation isotropic}) mit
  den Standardabweichungen in {\em sigma} f\"ur die Normalverteilung, dem
  Parameter zur D\"ampfung der Schrittweitenvariation zwischen 
  aufeinanderfolgenden Generationen f\"ur alle Individuen
  ({\em beta}) und f\"ur ein Individuum ({\em beta\_ind}), sowie
  dem Normalisierungsfaktor {\em cu} f\"ur Varianzen.
  N\"aheres siehe in ausf\"uhrlicher Referenz.
}\\\hrule

\vspace*{10ex}

% -----------------------------------------------------------------------
\subsection{Methoden zur Anzeige von Parametern}
% -----------------------------------------------------------------------

\vspace*{10pt}

\methodgrayfalse
\parindent0mm
\hrule
\method{Methode}{Beschreibung}\hrule
\method{
  \emph{void} \textsf{\textbf{\small showRotate}}( ) }{
  Gibt die Parameter {\em sigma} und {\em alpha} der Selbstadaptationsmethode 
  {\em mutateRotate} (s.o.) aus. 
}\\
\method{
  \emph{void} \textsf{\textbf{\small showCMA}}( ) }{
  Gibt die Parameter der Selbstadaptationsmethode {\em mutateCMA} (s.o.)
  aus.
}\\
\method{
  \emph{void} \textsf{\textbf{\small showIDA}}( ) }{
  Gibt die Parameter der Selbstadaptationsmethode {\em mutateIDA} (s.o.)
  aus. 
}\\
\method{
  \emph{void} \textsf{\textbf{\small showGSA}}( 
  \emph{int baseSize} ) }{
  Gibt die Parameter der Selbstadaptationsmethode {\em mutateGSA} (s.o.)
  mit {\em baseSize} Vektoren der generierenden Menge aus. 
}\\
\method{
  \emph{void} \textsf{\textbf{\small dumpDerandom}}( 
  \emph{ostream\emph{\&} os} ) }{
  Gibt die Werte der Parameter f\"ur die 
  {\em Derandom}-Selbstadaptationsmethode (s.o.) auf dem 
  Ausgabestrom {\em os} aus. 
}\\\hrule


\newpage
\chapter{Klasse {\tt ChromosomeT$<$ bool $>$}}


% ######################################################################
\section{Frei verf\"ugbare Methoden}
% ######################################################################

% -----------------------------------------------------------------------
\subsection{Initialisierungsmethode}
% -----------------------------------------------------------------------

\vspace*{10pt}

\methodgrayfalse

\parindent0mm
\hrule
\method{Methode}{Beschreibung}\hrule
\method{
  \emph{void} \textsf{\textbf{\small initialize}}( ) }{ 
  Initialisiert alle Allele zuf\"allig mit ``true'' oder
  ``false''.
}\\\hrule

\vspace*{10pt}


% -----------------------------------------------------------------------
\subsection{(De-)Kodierungsmethoden}
% -----------------------------------------------------------------------

\vspace*{10pt}

\methodgrayfalse

\parindent0mm
\hrule
\method{Methode}{Beschreibung}\hrule
\method{
  \emph{void} \textsf{\textbf{\small encode}}( \emph{double val,\\
  \hspace*{1.5em}const Interval\emph{\&} range,\\
  \hspace*{1.5em}unsigned nbits,\\
  \hspace*{1.5em}bool useGray = ``false''} ) }{ 
  Kodiert {\em val} aus dem Intervall {\em range} (bei {\em useGray} = ``true''
  per Gray-Kodierung) in den Bitstring {\em this} der L\"ange {\em nbits}. 
}\\
\method{
  \emph{void} \textsf{\textbf{\small encodeBinary}}(\\
  \hspace*{1.5em}\emph{const vector$<$ double $>$\emph{\&} chrom,\\
  \hspace*{1.5em}const Interval\emph{\&} range,\\
  \hspace*{1.5em}unsigned nbits,\\
  \hspace*{1.5em}bool useGray = ``false''} ) }{ 
  Kodiert alle Zahlen in {\em chrom} aus dem Intervall {\em range} 
  (bei {\em useGray} = ``true'' per Gray-Kodierung) in den Bitstring 
  {\em this}.
  Jede Zahl wird in {\em nbits} Bits kodiert. 
}\\
\method{
  \emph{void} \textsf{\textbf{\small encodeBinary}}(\\
  \hspace*{1.5em}\emph{const Chromosome\emph{\&} chrom,\\
  \hspace*{1.5em}const Interval\emph{\&} range,\\
  \hspace*{1.5em}unsigned nbits,\\
  \hspace*{1.5em}bool useGray = ``false''} ) }{ 
  Kodiert alle Zahlen in {\em chrom} aus dem Intervall {\em range} 
  (bei {\em useGray} = ``true'' per Gray-Kodierung) in den Bitstring 
  {\em this}.
  Jede Zahl wird in {\em nbits} Bits kodiert. 
}\\
\method{
  \emph{double} \textsf{\textbf{\small decode}}(\\
  \hspace*{1.5em}\emph{const Interval\emph{\&} range,\\
  \hspace*{1.5em}bool useGray = ``false''} ) }{ 
  Dekodiert die in {\em this} (bei {\em useGray} = ``true'' per
  Gray-Kodierung) in {\em nbits} Bits kodierte Zahl und gibt sie
  zur\"uck.
}\\\hrule

\vspace*{10pt}


% -----------------------------------------------------------------------
\subsection{Mutationsmethoden}
% -----------------------------------------------------------------------

\vspace*{10pt}

\methodgrayfalse

\parindent0mm
\hrule
\vspace*{1pt}
\method{Methode}{Beschreibung}\hrule
\method{
  \emph{void} \textsf{\textbf{\small flip}}(
  \emph{double p} ) }{ 
  Jeder Allelwert wird mit Wahrscheinlichkeit {\em p} invertiert.
}\\
\method{
  \emph{void} \textsf{\textbf{\small flip}}(
  \emph{const vector$<$ double $>$\emph{\&} p\\
  \hspace*{1.5em}bool cycle = ``false''} ) }{ 
  $this_i$ ($i\ =\ 0..n-1$) wird mit Wahrscheinlichkeit
  $p_i$ invertiert. Bei {\em cycle} = ``true'' darf {\em p}
  mehrmals durchlaufen werden.
}\\
\method{
  \emph{void} \textsf{\textbf{\small flip}}(
  \emph{const Chromosome\emph{\&} p\\
  \hspace*{1.5em}bool cycle = ``false''} ) }{ 
  $this_i$ ($i\ =\ 0..n-1$) wird mit Wahrscheinlichkeit
  $p_i$ invertiert. Bei {\em cycle} = ``true'' darf {\em p}
  mehrmals durchlaufen werden.
}\\\hrule


\newpage
\chapter{Klasse {\tt Individual}}


% ######################################################################
\section{Interne Variablen und Flags}
% ######################################################################

\vspace*{10pt}

\methodgrayfalse

\parindent0mm
\hrule
\vspace*{1pt}
\method{Variable/Flag}{Beschreibung}\hrule
\method{
  \textsf{\textbf{\small fitness}}}{ 
  Ergebnis der Bewertung des Individuums, mit ``0'' initialisiert.
}\\
\method{
  \textsf{\textbf{\small scaledFitness}}}{ 
  Skalierte Fitne{\ss} - positiver Wert, der zusammen mit allen
  anderen Fitne{\ss}werten die Summe ``1.0'' ergibt. Wird mit
  ``0'' initialisiert.
}\\
\method{
  \textsf{\textbf{\small age}}}{ 
  Alter des Individuums, wird mit ``0'' initialisiert.
}\\
\method{
  \textsf{\textbf{\small selProb}}}{ 
  Selektionswahrscheinlichkeit des Individuums, wird 
  mit ``0'' initialisiert.
}\\
\method{
  \textsf{\textbf{\small numCopies}}}{ 
  Anzahl der Reproduktionen des Individuums bei letzter
  Selektion, wird mit ``0'' initialisiert.
}\\
\method{
  \textsf{\textbf{\small evalFlg}}}{ 
  Ist das Flag gesetzt, mu{\ss} die FItne{\ss} des Individuums
  bewertet werden. Das Flag wird mit ``false'' initialisiert.
}\\
\method{
  \textsf{\textbf{\small feasible}}}{ 
  Ist das Flag gesetzt, so stellt das Individuum eine m\"ogliche
  L\"osung des Optimierungsproblems dar. Das Flag wird mit ``false''
  intialisiert.
}\\
\method{
  \textsf{\textbf{\small elitist}}}{ 
  Ist das Flag gesetzt, so wurde das Individuum bei der letzten Selektion
  als Elitist ausgew\"ahlt. Das Flag wird mit ``false''
  intialisiert.
}\\\hrule

\newpage

% ######################################################################
\section{Frei verf\"ugbare Methoden}
% ######################################################################

\subsection{Konstruktoren und Destruktor}

\vspace*{10pt}

\methodgrayfalse

\parindent0mm
\hrule
\method{Methode}{Beschreibung}\hrule
\method{
  \textsf{\textbf{\small Individual}}( )}{ 
  Erzeugt ein leeres Individuum.
}\\
\method{
  \emph{explicit} \textsf{\textbf{\small Individual}}( \emph{unsigned n} )}{ 
  Erzeugt ein Individuum mit {\em n} Chromosomen des Typs {\tt char}.
}\\
\method{
  \textsf{\textbf{\small Individual}}( \emph{unsigned n,\\
  \hspace*{1.5em}const Chromosome\emph{\&} chrom} )}{ 
  Erzeugt ein Individuum, mit {\em n} Klonen von {\em chrom}.
}\\
\method{
  \textsf{\textbf{\small Individual}}(\\
  \hspace*{1.5em}\emph{const Chromosome\emph{\&} chrom0} )}{ 
  Erzeugt ein Individuum mit Chromosom {\em chrom0}.
}\\
\method{
  \textsf{\textbf{\small Individual}}(\\
  \hspace*{1.5em}\emph{const Chromosome\emph{\&}\\
  \hspace*{2.5em}chrom0, ..., chrom\emph{[1 .. 7]}} )}{ 
  Erzeugt ein Individuum mit Klonen von {\em chrom0} bis 
  $chrom_i$ (mit $i\ =\ 1\ ..\ 7$).
}\\
\method{
  \textsf{\textbf{\small Individual}}(\\
  \hspace*{1.5em}\emph{const vector$<$Chromosome$\ast>$\emph{\&}\\
  \hspace*{2.5em}chrom} 
  )}{ 
  Erzeugt ein Individuum mit Klonen aus {\em chrom}.
}\\
\method{
  \textsf{\textbf{\small Individual}}(\\
  \hspace*{1.5em}\emph{const Individual\emph{\&} indiv} )}{ 
  Erzeugt ein Individuum mit Inhalt von {\em indiv}.
}\\
\method{
  \emph{$\sim$} \textsf{\textbf{\small Individual}}( )}{ 
  L\"oscht alle Individuen in {\em this} und zerst\"ort dann die Instanz.
}\\\hrule

\vspace*{10pt}

\subsection{Operatoren}

\vspace*{10pt}

\methodgrayfalse

\parindent0mm
\hrule
\vspace*{1pt}
\method{Methode}{Beschreibung}\hrule
\method{
  \emph{Individual\emph{\&}} \textsf{\textbf{\small operator\ =\ }}(\\
  \hspace*{1.5em}\emph{const Individual\emph{\&} indiv} )}{ 
  Weist {\em this} die Daten vom Individuum {\em indiv} zu.
}\\
\method{
  \emph{bool} \textsf{\textbf{\small operator\ ==\ }}(\\
  \hspace*{1.5em}\emph{const Individual\emph{\&} ind} )}{ 
  Pr\"uft, ob {\em this} und {\em ind} gleich sind.
}\\
\method{
  \emph{bool} \textsf{\textbf{\small operator\ $<$\ }}(\\
  \hspace*{1.5em}\emph{const Individual\emph{\&} ind} )}{ 
  Pr\"uft, ob {\em this} kleiner als {\em ind} ist.
}\\
\method{
  \emph{bool} \textsf{\textbf{\small operator\ $>$\ }}(\\
  \hspace*{1.5em}\emph{const Individual\emph{\&} ind} )}{ 
  Pr\"uft, ob {\em this} gr\"o{\ss}er als {\em ind} ist.
}\\
\method{
  \emph{bool} \textsf{\textbf{\small operator\ $<=$\ }}(\\
  \hspace*{1.5em}\emph{const Individual\emph{\&} ind} )}{ 
  Pr\"uft, ob {\em this} kleiner oder gleich {\em ind} ist.
}\\
\method{
  \emph{bool} \textsf{\textbf{\small operator\ $>=$\ }}(\\
  \hspace*{1.5em}\emph{const Individual\emph{\&} ind} )}{ 
  Pr\"uft, ob {\em this} gr\"o{\ss}er oder gleich {\em ind} ist.
}\\
\method{
  \emph{bool} \textsf{\textbf{\small operator\ !=\ }}(\\
  \hspace*{1.5em}\emph{const Individual\emph{\&} ind} )}{ 
  Pr\"uft, ob {\em this} und {\em ind} verschieden sind.
}\\
\method{
  \emph{Chromosome\emph{\&}} \textsf{\textbf{\small operator\ [\ ]\ }}(\\
  \hspace*{1.5em}\emph{unsigned i} )}{ 
  Gibt das Chromosom mit Index {\em i} zur\"uck.
}\\\hrule

\newpage

\subsection{Informationsmethoden}

\vspace*{10pt}

\methodgrayfalse

\parindent0mm
\hrule
\vspace*{1pt}
\method{Methode}{Beschreibung}\hrule
\method{
  \emph{unsigned} \textsf{\textbf{\small size}}( ) }{
  Gibt die Anzahl der Chromosomen in {\em this} zur\"uck. 
}\\
\method{
  \emph{unsigned} \textsf{\textbf{\small totalSize}}( ) }{
  Gibt die Gesamtzahl aller Chromosomenallele in {\em this} zur\"uck. 
}\\
\method{
  \emph{double} \textsf{\textbf{\small fitnessValue}}( ) }{
  Gibt den Fitne{\ss}wert von {\em this} zur\"uck.
}\\
\method{
  \emph{unsigned} \textsf{\textbf{\small getAge}}( ) }{
  Gibt das Alter von {\em this} zur\"uck.
}\\
\method{
  \emph{bool} \textsf{\textbf{\small needEvaluation}}( ) }{
  Gibt ``true'' zur\"uck, wenn {\em this} bewertet werden mu{\ss}.
}\\
\method{
  \emph{double} \textsf{\textbf{\small selectionProbability}}( ) }{
  Gibt die Selektionswahrscheinlichkeit von {\em this} zur\"uck.
}\\
\method{
  \emph{bool} \textsf{\textbf{\small isFeasible}}( ) }{
  Gibt ``true'' zur\"uck, wenn {\em this} eine m\"ogliche L\"osung des
  Optimierungsproblems darstellt.
}\\
\method{
  \emph{unsigned} \textsf{\textbf{\small numberOfCopies}}( ) }{
  Gibt an, wie oft {\em this} bei der letzten Selektion
  reproduziert wurde.
}\\
\method{
  \emph{bool} \textsf{\textbf{\small isElitist}}( ) }{
  Gibt ``true'' zur\"uck, wenn {\em this} bei der letzten Selektion als 
  Elitist ausgew\"ahlt wurde.
}\\\hrule

\vspace*{10pt}

\subsection{Methoden zur Ver\"anderung interner Variablen und Flags}

\vspace*{10pt}

\methodgrayfalse


\parindent0mm
\hrule
\vspace*{1pt}
\method{Methode}{Beschreibung}\hrule
\method{
  \emph{void} \textsf{\textbf{\small setFitness}}( 
  \emph{double fit} ) }{
  Setzt normale und skalierte Fitne{\ss} von {\em this}
  auf {\em fit}. 
}\\
\method{
  \emph{void} \textsf{\textbf{\small setAge}}(
  \emph{unsigned a} ) }{
  Setzt das Alter von {\em this} auf {\em a}.
}\\
\method{
  \emph{void} \textsf{\textbf{\small incAge}}( ) }{
  Erh\"oht das Alter von {\em this} um den Wert ``1''.
}\\
\method{
  \emph{void} \textsf{\textbf{\small setEvaluationFlag}}( ) }{
  Setzt {\em evalFlg} auf den Wert ``true''.
}\\
\method{
  \emph{void} \textsf{\textbf{\small clearEvaluationFlag}}( ) }{
  Setzt {\em evalFlg} auf den Wert ``false''.
}\\
\method{
  \emph{void} \textsf{\textbf{\small setFeasible}}(
  \emph{bool f} ) }{
  Setzt das {\em feasible}-Flag auf den neuen Wert {\em f}.
}\\
\method{
  \emph{void} \textsf{\textbf{\small setSelectionProbability}}(\\
  \hspace*{1.5em}\emph{double ps} ) }{
  Setzt die Selektionswahrscheinlichkeit des Individuums auf {\em ps}.
}\\\hrule

\newpage

\subsection{Strukturver\"anderungsmethoden}

\vspace*{10pt}

\methodgrayfalse


\parindent0mm
\hrule
\vspace*{1pt}
\method{Methode}{Beschreibung}\hrule
\method{
  \emph{void} \textsf{\textbf{\small replace}}( 
  \emph{unsigned i,\\ 
  \hspace*{1.5em}const Chromosome\emph{\&} chrom} ) }{
  Ersetzt Chromosom Nummer {\em i} von {\em this} durch 
  {\em chrom}.
}\\
\method{
  \emph{void} \textsf{\textbf{\small insert}}( 
  \emph{unsigned i,\\ 
  \hspace*{1.5em}const Chromosome\emph{\&} chrom} ) }{
  F\"ugt {\em chrom} an Position {\em i} in {\em this}
  ein.
}\\
\method{
  \emph{void} \textsf{\textbf{\small append}}(\\
  \hspace*{1.5em}\emph{const Chromosome\emph{\&} chrom} ) }{
  H\"angt {\em chrom} am Ende von {\em this} an.
}\\
\method{
  \emph{void} \textsf{\textbf{\small remove}}( 
  \emph{unsigned i} ) }{
  L\"oscht das Chromosom mit Index {\em i} aus {\em this}.
}\\
\method{
  \emph{void} \textsf{\textbf{\small remove}}( 
  \emph{unsigned from,\\ 
  \hspace*{1.5em}unsigned to} ) }{
  L\"oscht die Chromosomem von Index {\em from} bis {\em to} aus {\em this}.
}\\\hrule

\vspace*{10pt}

\subsection{Ein- und Ausgabemethoden}

\vspace*{10pt}

\methodgrayfalse


\parindent0mm
\hrule
\vspace*{1pt}
\method{Methode}{Beschreibung}\hrule
\method{
  \emph{void} \textsf{\textbf{\small readFrom}}(
  \emph{istream\emph{\&} is} ) }{ 
  Die Daten von {\em this} werden durch die aus {\em is} gelesenen
  Daten ersetzt.
}\\
\method{
  \emph{void} \textsf{\textbf{\small writeTo}}(
  \emph{ostream\emph{\&} os} ) }{ 
  Schreibt die Daten von {\em this} in {\em os}.
}\\\hrule


\newpage
\chapter{Klasse {\tt Population}}


% ######################################################################
\section{Interne Variablen und Flags}
% ######################################################################

\vspace*{10pt}

\methodgrayfalse

\parindent0mm
\hrule
\vspace*{1pt}
\method{Variable/Flag}{Beschreibung}\hrule
\method{
  \textsf{\textbf{\small index}}}{ 
  Index des Individuums mit bestem Fitne{\ss}wert.
}\\
\method{
  \textsf{\textbf{\small subPop}}}{ 
  ``true'' zeigt an, da{\ss} es sich bei der Population
  um eine Subpopulation handelt. Das Flag wird mit ``false''
  initialisiert.
}\\
\method{
  \textsf{\textbf{\small ascending}}}{ 
  ``true'' zeigt an, da{\ss} die Individuen nach aufsteigenden
  Fitne{\ss}werten sortiert werden. Bei ``false'' (= Initialwert)
  wird nach absteigender Fitne{\ss} sortiert.
}\\
\method{
  \textsf{\textbf{\small spinOnce}}}{ 
  Flag f\"ur Selektionsmethode {\em selectRouletteWheel} (s.u.).
  Bei ``true'' (= Initialwert) dreht sich das Rad nur einmal, bei ``false''
  mehrmals.
}\\\hrule


\newpage
% ######################################################################
\section{Frei verf\"ugbare Methoden}
% ######################################################################

\subsection{Konstruktoren und Destruktor}

\vspace*{10pt}

\methodgrayfalse

\parindent0mm
\hrule
\method{Methode}{Beschreibung}\hrule
\method{
  \textsf{\textbf{\small Population}}( )}{ 
  Erzeugt eine leere Population.
}\\
\method{
  \emph{explicit} \textsf{\textbf{\small Population}}(
  \emph{unsigned n} )}{ 
  Erzeugt eine neue Population mit {\em n}
  leeren Individuen.
}\\
\method{
  \textsf{\textbf{\small Population}}(\\
  \hspace*{1.5em}\emph{const Individual\emph{\&} indiv} )}{ 
  Erzeugt eine neue Population mit Individuum {\em indiv}.
}\\
\method{
  \textsf{\textbf{\small Population}}(
  \emph{unsigned n,\\ 
  \hspace*{1.5em}const Individual\emph{\&} indiv} )}{ 
  Erzeugt eine neue Population mit {\em n} Klonen von {\em indiv}.
}\\
\method{
  \textsf{\textbf{\small Population}}(
  \emph{unsigned n,\\ 
  \hspace*{1.5em}const Chromosome\emph{\&} chrom0} )}{ 
  Erzeugt eine neue Population mit {\em n} Individuen, die jeweils aus
  einem Klon von {\em chrom0} bestehen.
}\\
\method{
  \textsf{\textbf{\small Population}}(
  \emph{unsigned n,\\ 
  \hspace*{1.5em}const Chromosome\emph{\&}\\ 
  \hspace*{2.5em}chrom0, ..., chrom\emph{[1 .. 7]}} )}{ 
  Erzeugt eine neue Population mit {\em n} Individuen, die jeweils aus
  den Klonen {\em chrom0} bis $chrom_i$ (mit $i\ =\ 1\ ..\ 7$)
  bestehen.
}\\
\method{
  \textsf{\textbf{\small Population}}(
  \emph{unsigned n,\\ 
  \hspace*{1.5em}const vector$<$ Chromosome $\ast$ $>$\emph{\&}\\ 
  \hspace*{2.5em}chrom} )}{ 
  Erzeugt eine neue Population mit {\em n} Individuen, die jeweils aus
  den Klonen in {\em chrom} bestehen.
}\\
\method{
  \textsf{\textbf{\small Population}}(\\
  \hspace*{1.5em}\emph{const Population\emph{\&} pop} )}{ 
  Erzeugt eine neue Population, die aus einer Kopie von {\em pop}
  besteht.
}\\
\method{
  \emph{$\sim$} \textsf{\textbf{\small Population}}( )}{ 
  Ist das Flag {\em subPop} auf ``false'' gesetzt, wird die
  Population inklusive aller Individuen zerst\"ort.
  {\em subPop} = ``true'', so bleibt die Population erhalten.
}\\\hrule

\newpage

\subsection{Operatoren}

\vspace*{10pt}

\methodgrayfalse

\parindent0mm
\hrule
\vspace*{1pt}
\method{Methode}{Beschreibung}\hrule
\method{
  \emph{Population\emph{\&}} \textsf{\textbf{\small operator\ =\ }}(\\
  \hspace*{1.5em}\emph{const Individual\emph{\&} indiv} )}{ 
  Weist allen Individuen von {\em this} die Daten des Individuums 
  {\em indiv} zu.
}\\
\method{
  \emph{Population\emph{\&}} \textsf{\textbf{\small operator\ =\ }}(\\
  \hspace*{1.5em}\emph{const Population\emph{\&} pop} )}{ 
  Weist {\em this} die Daten der Population {\em pop} zu.
}\\

\method{
  \emph{bool} \textsf{\textbf{\small operator\ ==\ }}(\\
  \hspace*{1.5em}\emph{const Population\emph{\&} pop} )}{ 
  Pr\"uft, ob {\em this} und {\em pop} gleich sind.
}\\
\method{
  \emph{bool} \textsf{\textbf{\small operator\ $<$\ }}(\\
  \hspace*{1.5em}\emph{const Population\emph{\&} pop} )}{ 
  Pr\"uft, ob {\em this} kleiner als {\em pop} ist.
}\\
\method{
  \emph{bool} \textsf{\textbf{\small operator\ $>$\ }}(\\
  \hspace*{1.5em}\emph{const Population\emph{\&} pop} )}{ 
  Pr\"uft, ob {\em this} gr\"o{\ss}er als {\em pop} ist.
}\\
\method{
  \emph{bool} \textsf{\textbf{\small operator\ $<=$\ }}(\\
  \hspace*{1.5em}\emph{const Population\emph{\&} pop} )}{ 
  Pr\"uft, ob {\em this} kleiner oder gleich {\em pop} ist.
}\\
\method{
  \emph{bool} \textsf{\textbf{\small operator\ $>=$\ }}(\\
  \hspace*{1.5em}\emph{const Population\emph{\&} pop} )}{ 
  Pr\"uft, ob {\em this} gr\"o{\ss}er oder gleich {\em pop} ist.
}\\
\method{
  \emph{bool} \textsf{\textbf{\small operator\ !=\ }}(\\
  \hspace*{1.5em}\emph{const Population\emph{\&} pop} )}{ 
  Pr\"uft, ob {\em this} und {\em pop} verschieden sind.
}\\
\method{
  \emph{Individual\emph{\&}} \textsf{\textbf{\small operator\ [\ ]\ }}(\\
  \hspace*{1.5em}\emph{unsigned i} )}{ 
  Gibt das Individuum mit Index {\em i} zur\"uck.
}\\
\method{
  \emph{Population\emph{\&}} \textsf{\textbf{\small operator\ (\ )\ }}(\\
  \hspace*{1.5em}\emph{unsigned from, unsigned to} )}{ 
  Gibt eine Subpopulation, bestehend aus den Individuen {\em from}
  bis {\em to} aus {\em this} zur\"uck.
}\\\hrule

\newpage

\subsection{Informationsmethoden}

\vspace*{10pt}

\methodgrayfalse

\parindent0mm
\hrule
\vspace*{1pt}
\method{Methode}{Beschreibung}\hrule
\method{
  \emph{int} \textsf{\textbf{\small size}}( ) }{
  Gibt die Anzahl der in {\em this} enthaltenen Individuen zur\"uck.
}\\
\method{
  \emph{bool} \textsf{\textbf{\small ascendingFitness}}( ) }{
  Gibt ``true'' zur\"uck, wenn die Individuen in {\em this} nach
  aufsteigenden Fitne{\ss}werten sortiert sind.
}\\
\method{
  \emph{double} \textsf{\textbf{\small minFitness}}( ) }{
  Gibt den minimalen Fitne{\ss}wert in {\em this} zur\"uck.
}\\
\method{
  \emph{double} \textsf{\textbf{\small maxFitness}}( ) }{
  Gibt den maximalen Fitne{\ss}wert in {\em this} zur\"uck.
}\\
\method{
  \emph{double} \textsf{\textbf{\small meanFitness}}( ) }{
  Gibt den durchschnittlichen Fitne{\ss}wert in {\em this} zur\"uck.
}\\
\method{
  \emph{double} \textsf{\textbf{\small stdDevFitness}}( ) }{
  Gibt die Standardabweichung aller Fitne{\ss}werte in {\em this} zur\"uck.
}\\
\method{
  \emph{unsigned} \textsf{\textbf{\small bestIndex}}( ) }{
  Gibt den Index des Individuums mit dem besten Fitne{\ss}wert 
  (= {\em ascending}-Flag = ``true''? $\Longrightarrow$ niedrigste
   Fitne{\ss}, {\em ascending}-Flag = ``false''? $\Longrightarrow$ h\"ochste
   Fitne{\ss}) zur\"uck.
}\\
\method{
  \emph{unsigned} \textsf{\textbf{\small worstIndex}}( ) }{
  Gibt den Index des Individuums mit dem schlechtesten Fitne{\ss}wert 
  (= {\em ascending}-Flag = ``true''? $\Longrightarrow$ h\"ochste
   Fitne{\ss}, {\em ascending}-Flag = ``false''? $\Longrightarrow$ niedrigste
   Fitne{\ss}) zur\"uck.
}\\
\method{
  \emph{Individual\emph{\&}} \textsf{\textbf{\small best}}( ) }{
  Gibt das Individuum mit dem besten Fitne{\ss}wert nach den
  Kriterien von {\em bestIndex} (s.o.) zur\"uck.
}\\
\method{
  \emph{Individual\emph{\&}} \textsf{\textbf{\small worst}}( ) }{
  Gibt das Individuum mit dem schlechtesten Fitne{\ss}wert nach den
  Kriterien von {\em worstIndex} (s.o.) zur\"uck.
}\\\hrule

\vspace*{10pt}

\subsection{Methoden zur Ver\"anderung interner Variablen und Flags}

\vspace*{10pt}

\methodgrayfalse


\parindent0mm
\hrule
\method{Methode}{Beschreibung}\hrule
\method{
  \emph{void} \textsf{\textbf{\small setMinimize}}( ) }{
  Setzt das {\em ascending}-Flag auf ``true''. 
}\\
\method{
  \emph{void} \textsf{\textbf{\small setMaximize}}( ) }{
  Setzt das {\em ascending}-Flag auf ``false''. 
}\\
\method{
  \emph{void} \textsf{\textbf{\small spinWheelOneTime}}( ) }{
  Setzt das {\em spinOnce}-Flag auf ``true''.
}\\
\method{
  \emph{void} \textsf{\textbf{\small spinWheelMultipleTimes}}( 
  \emph{} ) }{
  Setzt das {\em spinOnce}-Flag auf ``false''.
}\\
\method{
  \emph{void} \textsf{\textbf{\small setAge}}( 
  \emph{unsigned a} ) }{
  Setzt das Alter aller Individuen auf {\em a}.
}\\
\method{
  \emph{void} \textsf{\textbf{\small incAge}}( 
  \emph{} ) }{
  Erh\"oht das Alter aller Individuen um den Wert ``1''.
}\\\hrule


\newpage

\subsection{Strukturver\"anderungsmethoden}

\vspace*{10pt}

\methodgrayfalse

\parindent0mm
\hrule
\vspace*{1pt}
\method{Methode}{Beschreibung}\hrule
\method{
  \emph{void} \textsf{\textbf{\small resize}}( 
  \emph{unsigned n} ) }{
  Setzt die Gr\"o{\ss}e von {\em this} auf {\em n} Individuen.
}\\
\method{
  \emph{void} \textsf{\textbf{\small replace}}( 
  \emph{unsigned i,\\
  \hspace*{1.5em}const Individual\emph{\&} ind} ) }{
  Ersetzt das Individuum an Position {\em i} durch {\em ind}. 
}\\
\method{
  \emph{void} \textsf{\textbf{\small replace}}( 
  \emph{unsigned i,\\
  \hspace*{1.5em}const Population\emph{\&} pop} ) }{
  Ersetzt die Individuen ab Position {\em i} mit denen
  aus {\em pop}.
}\\
\method{
  \emph{void} \textsf{\textbf{\small insert}}( 
  \emph{unsigned i,\\
  \hspace*{1.5em} const Individual\emph{\&} ind} ) }{
  F\"ugt {\em ind} an Position {\em i} in {\em this} ein.
}\\
\method{
  \emph{void} \textsf{\textbf{\small insert}}( 
  \emph{unsigned i,\\
  \hspace*{1.5em} const Population\emph{\&} pop} ) }{
  F\"ugt alle Individuen aus {\em pop} ab Position {\em i}
  in {\em this} ein. 
}\\
\method{
  \emph{void} \textsf{\textbf{\small append}}( 
  \emph{const Individual\emph{\&} ind} ) }{
  H\"angt {\em ind} am Ende von {\em this} an.
}\\
\method{
  \emph{void } \textsf{\textbf{\small append}}( 
  \emph{const Population\emph{\&} pop} ) }{
  H\"angt {\em pop} am Ende von {\em this} an.
}\\
\method{
  \emph{void} \textsf{\textbf{\small remove}}( 
  \emph{unsigned i} ) }{
  Entfernt das Individuum an Position {\em i} aus {\em this}.
}\\
\method{
  \emph{void} \textsf{\textbf{\small remove}}( 
  \emph{unsigned from,\\ 
  \hspace*{1.5em}unsigned to} ) }{
  Entfernt die Individuen von Position {\em from} bis {\em to} 
  aus {\em this}.
}\\
\method{
  \emph{void} \textsf{\textbf{\small exchange}}( 
  \emph{Population\emph{\&} pop} ) }{
  Vertauscht die Inhalte der Populationen {\em this} und {\em pop}
  miteinander.
}\\
\method{
  \emph{void} \textsf{\textbf{\small swap}}( 
  \emph{unsigned i, unsigned j} ) }{
  Vertauscht die Individuen an den Positionen {\em i} und {\em j}
  miteinander.
}\\
\method{
  \emph{void} \textsf{\textbf{\small sort}}( ) }{
  Sortiert die Individuen in {\em this} entsprechend
  dem Wert von {\em ascending}.
}\\
\method{
  \emph{void} \textsf{\textbf{\small shuffle}}( ) }{
  \"Andert die Reihenfolge der Individuen in {\em this}
  auf zuf\"alliger Basis.
}\\\hrule


\vspace*{10pt}

\subsection{Methoden zur Extraktion einzelner Individuen}

\vspace*{15pt}

\methodgrayfalse

\parindent0mm
\hrule
\vspace*{1pt}
\method{Methode}{Beschreibung}\hrule
\method{
  \emph{Individual\emph{\&}} \textsf{\textbf{\small random}}( ) }{ 
  Gibt ein zuf\"allig ausgew\"ahltes Individuum zur\"uck.
}\\
\method{
  \emph{vector$<$ const Chromosome $\ast$ $>$}\\ 
  \hspace*{1.5em}\textsf{\textbf{\small matingPool}}(
  \emph{unsigned chrom,\\ 
  \hspace*{2.5em}unsigned from = 0,\\ 
  \hspace*{2.5em}unsigned to = 0} ) }{ 
  Gibt alle Chromosomen mit Index {\em chrom} aus den
  Individuen {\em from} bis {\em to} (bei {\em from} = {\em to} = 0
  aus allen Individuen) zur\"uck.
}\\\hrule

\newpage

\subsection{Skalierungsmethode}

\vspace*{10pt}

\methodgrayfalse

\parindent0mm
\hrule
\method{Methode}{Beschreibung}\hrule
\method{
  \emph{void} \textsf{\textbf{\small linearDynamicScaling}}(\\
  \emph{\hspace*{1.5em}vector$<$ double $>$\emph{\&} window,\\
  \hspace*{1.5em}unsigned long t} ) }{ 
  Skaliert alle Fitne{\ss}werte mittels
  Skalierungsfenster {\em window} zum Zeitpunkt {\em t}.\\
}\\\hrule

\vspace*{1.5em}

\subsection{Selektionsmethoden}

\vspace*{10pt}

\methodgrayfalse

\parindent0mm
\hrule
\method{Methode}{Beschreibung}\hrule
\method{
  \emph{void} \textsf{\textbf{\small selectMuLambda}}(\\
  \hspace*{1.5em}\emph{Population\emph{\&} parents,\\
  \hspace*{1.5em}unsigned numElitists = 0} ) }{
  W\"ahlt aus {\em this} und {\em parents} {\em numElitists}
  Elitisten und weitere Individuen zur Reproduktion nach der 
  $(\mu\ [,|+]\ \lambda)$-Selektionsmethode aus.
  N\"aheres siehe in ausf\"uhrlicher Referenz.
}\\
\method{
  \emph{void} \textsf{\textbf{\small selectMuLambdaKappa}}(\\
  \hspace*{1.5em}\emph{Population\emph{\&} parents,\\
  \hspace*{1.5em}unsigned lifespan = 1,\\
  \hspace*{1.5em}unsigned adolescence = 0} ) }{ 
  Selektiert aus {\em this} und {\em parents} zuerst Individuen
  mit Alter {\em age} $<$ {\em adolescence} und dann jene mit
  {\em age} $<=$ {\em lifespan} zur Reproduktion.
  N\"aheres siehe in ausf\"uhrlicher Referenz.
}\\
\method{
  \emph{Individual\emph{\&}} 
  \textsf{\textbf{\small selectOneIndividual}}( ) }{ 
  Auswahl eines Individuums durch proportionale Selektion.
  N\"aheres siehe in ausf\"uhrlicher Referenz.
}\\
\method{
  \emph{void} \textsf{\textbf{\small selectProportional}}(\\
  \hspace*{1.5em}\emph{Population\emph{\&} parents,\\
  \hspace*{1.5em}unsigned numElitists = 0} ) }{ 
  Selektiert aus {\em this} und {\em parents} Individuen
  mittels proportionaler Selektion, \"ubernimmt dabei
  {\em numElitists} Elitisten.
  N\"aheres siehe in ausf\"uhrlicher Referenz.
}\\
\method{
  \emph{void} \textsf{\textbf{\small selectLinearRanking}}(\\
  \hspace*{1.5em}\emph{Population\emph{\&} parents,\\
  \hspace*{1.5em}double etaMax = 1.1,\\
  \hspace*{1.5em}unsigned numElitists = 0} ) }{ 
  Selektiert aus {\em this} und {\em parents} Individuen nach
  der Methode des linearen Ranges und \"ubernimmt dabei 
  {\em numElitists} Elitisten. Die maximale Reproduktionsrate
  von selektierten Individuen betr\"agt {\em etaMax}.
  N\"aheres siehe in ausf\"uhrlicher Referenz.
}\\
\method{
  \emph{void} \textsf{\textbf{\small selectLinearRankingWhitley}}(\\
  \hspace*{1.5em}\emph{Population\emph{\&} parents,\\
  \hspace*{1.5em}double a = 1.1,\\
  \hspace*{1.5em}unsigned numElitists = 0} ) }{ 
  Selektiert aus {\em this} und {\em parents} Individuen nach
  der Methode des linearen Ranges nach Whitley und \"ubernimmt dabei 
  {\em numElitists} Elitisten. Die maximale Reproduktionsrate
  von selektierten Individuen betr\"agt {\em a}.
  N\"aheres siehe in ausf\"uhrlicher Referenz.
}\\\hrule

\newpage

\methodgrayfalse

\parindent0mm
\hrule
\method{Methode}{Beschreibung}\hrule
\method{
  \emph{void} \textsf{\textbf{\small selectUniformRanking}}(\\
  \hspace{1.5em}\emph{Population\emph{\&} parents,\\
  \hspace*{1.5em}unsigned numElitists = 0} ) }{ 
  Selektiert aus {\em this} und {\em parents} Individuen nach
  der Methode des einheitlichen Ranges und \"ubernimmt dabei 
  {\em numElitists} Elitisten. 
  N\"aheres siehe in ausf\"uhrlicher Referenz.
}\\
\method{
  \emph{void} \textsf{\textbf{\small reproduce}}(\\
  \hspace*{1.5em}\emph{Population\emph{\&} parents,\\
  \hspace*{1.5em}unsigned numElitists = 0} ) }{ 
  Entspricht der Selektion anhand der
  Einheitsrang-Methode (s. {\em selectUniformRanking}).
}\\
\method{
  \emph{void} \textsf{\textbf{\small selectTournament}}(\\
  \hspace*{1.5em}\emph{Population\emph{\&} parents,\\
  \hspace*{1.5em}unsigned q = 2,\\
  \hspace*{1.5em}unsigned numElitists = 0} ) }{ 
  W\"ahlt in {\em n} Schritten jeweils das beste Individuum
  aus einer per Zufall zusammengestellten Gruppe von {\em q}
  Individuen aus {\em this} und {\em parents} aus und
  \"ubernimmt zus\"atzlich {\em numElitists} Elitisten.
  N\"aheres siehe in ausf\"uhrlicher Referenz.
}\\
\method{
  \emph{void} \textsf{\textbf{\small selectEPTournament}}(\\
  \hspace*{1.5em}\emph{Population\emph{\&} parents,\\
  \hspace*{1.5em}unsigned q} ) }{ 
  Selektiert Individuen zur Reproduktion, indem jedes
  Individuum aus {\em this} und {\em parents} gegen
  {\em q} andere, zuf\"allig ausgew\"ahlte Individuen
  bestehen mu{\ss}.
  N\"aheres siehe in ausf\"uhrlicher Referenz.
}\\\hrule

% ######################################################################
\section{Gesch\"utzte Methoden}
% ######################################################################

\subsection{Konstruktor und Initialisierungsmethode}

\vspace*{15pt}

\methodgrayfalse

\parindent0mm
\hrule
\vspace*{1pt}
\method{Methode}{Beschreibung}\hrule
\method{
  \textsf{\textbf{\small Population}}(\\
  \hspace*{1.5em}\emph{const vector$<$ Individual $\ast$ $>$\emph{\&}\\
  \hspace*{2.5em}indvec} ) }{ 
  Erzeugt eine neue Population mit Individuen aus {\em indvec}.
}\\
\method{
  \emph{void} \textsf{\textbf{\small selectInit}}( ) }{ 
  Initialisiert die internen Variablen aller Individuen f\"ur 
  eine sp\"atere Selektionsmethode.
}\\\hrule

\newpage

\subsection{Vergleichsmethoden}

\vspace*{10pt}

\methodgrayfalse

\parindent0mm
\hrule
\method{Methode}{Beschreibung}\hrule
\method{
  \emph{static bool} \textsf{\textbf{\small lessFitness}}(\\ 
  \hspace*{1.5em}\emph{Individual$\ast$const\emph{\&} i1,\\
  \hspace*{1.5em}Individual$\ast$const\emph{\&} i2} ) }{ 
  Gibt ``true'' zur\"uck, wenn {\em i1} eine geringere Fitne{\ss}
  hat als {\em i2}.
}\\
\method{
  \emph{static bool} \textsf{\textbf{\small greaterFitness}}(\\ 
  \hspace*{1.5em}\emph{Individual$\ast$const\emph{\&} i1,\\
  \hspace*{1.5em}Individual$\ast$const\emph{\&} i2} ) }{ 
  Gibt ``true'' zur\"uck, wenn {\em i1} eine h\"ohere Fitne{\ss}
  hat als {\em i2}.
}\\
\method{
  \emph{static bool} \textsf{\textbf{\small greaterScoreAscending}}(\\ 
  \hspace*{1.5em}\emph{Individual$\ast$const\emph{\&} i1,\\
  \hspace*{1.5em}Individual$\ast$const\emph{\&} i2} ) }{ 
  Gibt ``true'' zur\"uck, wenn die skalierte Fitne{\ss} von {\em i1} 
  gr\"o{\ss}er oder die normale Fitne{\ss} kleiner ist als die 
  entsprechenden Werte von {\em i2}.
}\\
\method{
  \emph{static bool} \textsf{\textbf{\small greaterScoreDescending}}(\\ 
  \hspace*{1.5em}\emph{Individual$\ast$const\emph{\&} i1,\\
  \hspace*{1.5em}Individual$\ast$const\emph{\&} i2} ) }{ 
  Gibt ``true'' zur\"uck, wenn die skalierte oder die normale Fitne{\ss} 
  von {\em i1} gr\"o{\ss}er sind als jene von {\em i2}.
}\\
\method{
  \emph{Individual\emph{\&}} \textsf{\textbf{\small best}}(
  \emph{Individual\emph{\&} ind0,\\ 
  \hspace*{1.5em}Individual\emph{\&} ind1} ) }{ 
  Vergleicht {\em ind0} und {\em ind1} und gibt bei {\em ascending}
  = ``true'' das Individuum mit der kleineren Fitne{\ss}, bei
  {\em ascending} = ``false'' jenes mit der gr\"o{\ss}eren
  Fitne{\ss} zur\"uck.
}\\
\method{
  \emph{Individual\emph{\&}} \textsf{\textbf{\small worst}}(
  \emph{Individual\emph{\&} ind0,\\ 
  \hspace*{1.5em}Individual\emph{\&} ind1} ) }{ 
  Vergleicht {\em ind0} und {\em ind1} und gibt bei {\em ascending}
  = ``true'' das Individuum mit der gr\"o{\ss}eren Fitne{\ss}, bei
  {\em ascending} = ``false'' jenes mit der kleineren
  Fitne{\ss} zur\"uck.
}\\\hrule

\vspace*{10pt}

\subsection{Sortiermethode}

\vspace*{10pt}

\methodgrayfalse

\parindent0mm
\hrule
\vspace*{1pt}
\method{Methode}{Beschreibung}\hrule
\method{
  \emph{void} \textsf{\textbf{\small sortIndividuals}}(\\ 
  \hspace*{1.5em}\emph{vector$<$ Individual $\ast$ $>$\emph{\&} indvec} ) }{ 
  Sortiert die Individuen in {\em indvec} bei {\em ascending} = ``true''
  nach aufsteigender, bei {\em ascending} = ``false'' nach absteigender
  Fitne{\ss}.
}\\\hrule

\newpage

\subsection{Selektionsmethoden}

\vspace*{10pt}

\methodgrayfalse

\parindent0mm
\hrule
\vspace*{1pt}
\method{Methode}{Beschreibung}\hrule
\method{
  \emph{void} \textsf{\textbf{\small selectElitists}}(\\ 
  \hspace*{1.5em}\emph{Population\emph{\&} parents,\\
  \hspace*{1.5em}unsigned numElitists} ) }{ 
  Selektiert aus {\em this} und {\em parents} {\em numElitists}
  Elitisten.
  N\"aheres siehe in ausf\"uhrlicher Referenz.
}\\
\method{
  \emph{void} \textsf{\textbf{\small selectRouletteWheel}}(\\ 
  \hspace*{1.5em}\emph{Population\emph{\&} parents,\\
  \hspace*{1.5em}unsigned numElitists} ) }{ 
  Selektiert aus {\em this} und {\em parents} {\em numElitists}
  Elitisten nach der Roulette-Rad-Methode. Je nach Wert von
  {\em spinOnce} dreht sich das Rad ein- oder mehrmals.
  N\"aheres siehe in ausf\"uhrlicher Referenz.
}\\\hrule

\vspace*{10pt}

\subsection{Ein- und Ausgabemethoden}

\vspace*{10pt}

\methodgrayfalse

\parindent0mm
\hrule
\vspace*{1pt}
\method{Methode}{Beschreibung}\hrule
\method{
  \emph{void} \textsf{\textbf{\small readFrom}}(
  \emph{istream\emph{\&} is} ) }{ 
  Ersetzt die Daten von {\em this} durch jene, die aus
  {\em is} gelesen werden.
}\\
\method{
  \emph{void} \textsf{\textbf{\small writeTo}}(
  \emph{ostream\emph{\&} os} ) }{ 
  Schreibt die Daten von {\em this} in {\em os}.
}\\\hrule



\begin{thebibliography}{55}
  
    \bibitem{EALib}
    {\em EALib: A \cpp\ class library for evolutionary
    algorithms}, Version 1.5, von M. Kreutz, B. Sendhoff, C. Igel,
    Institut f\"ur Neuroinformatik, Ruhr-Universit\"at Bochum, 
    2000.

\end{thebibliography}

\end{document}




%%% Local Variables: 
%%% mode: latex
%%% TeX-master: t
%%% End: 













