\section{Interpolate\-RBF Class Reference}
\label{classInterpolateRBF}\index{InterpolateRBF@{InterpolateRBF}}
This class provides different functions for the approximation using an Interpolation\-RBF model.  


{\tt \#include $<$Interpolate\-RBF.h$>$}

\subsection*{Public Member Functions}
\begin{CompactItemize}
\item 
{\bf Interpolate\-RBF} ()\label{classInterpolateRBF_a0}

\begin{CompactList}\small\item\em constructor \item\end{CompactList}\item 
{\bf Interpolate\-RBF} (unsigned in\-Dim, kernel\-Type a\-Kernel, Array$<$ double $>$ kernel\-Params)
\begin{CompactList}\small\item\em This constructor generates an Interpolation RBF model. \item\end{CompactList}\item 
{\bf $\sim$Interpolate\-RBF} ()\label{classInterpolateRBF_a2}

\begin{CompactList}\small\item\em destructor \item\end{CompactList}\item 
void {\bf evaluate} (Individual \&offspring)
\begin{CompactList}\small\item\em This function evaluates the individual $<$offspring$>$ with the RBF model and stores the result as the fitness value of the offspring. \item\end{CompactList}\item 
void {\bf evaluate} (Population \&offsprings)
\begin{CompactList}\small\item\em This function evaluates the population $<$offsprings$>$ with the RBF model and stores the results as the fitness value of each member of offsprings. \item\end{CompactList}\item 
void {\bf evaluate} (Array$<$ double $>$ input\-Data, Array$<$ double $>$ \&output\-Data)
\begin{CompactList}\small\item\em This function evaluates the data in $<$inputdata$>$ with the RBF model and stores the results in $<$outputdata$>$. \item\end{CompactList}\item 
double {\bf mse} (Array$<$ double $>$ Input, Array$<$ double $>$ Target)
\begin{CompactList}\small\item\em This function calculates the mean square error of the approximation considering the given arrays $<$input$>$ and $<$target$>$. \item\end{CompactList}\item 
double {\bf mse} (Population offsprings)
\begin{CompactList}\small\item\em This function calculates the mean square error of the approximation considering a given population $<$offsprings$>$. \item\end{CompactList}\item 
void {\bf train} (Array$<$ double $>$ Input\-Data, Array$<$ double $>$ Target\-Data)
\begin{CompactList}\small\item\em This function approximates the datas which are stored in the arrays $<$inputdata$>$ and $<$targetdata$>$ using the RBF model. \item\end{CompactList}\item 
void {\bf set\-Design\-Matrix} (Array$<$ double $>$ \&input\-Data)
\begin{CompactList}\small\item\em This function set the design(hidden) matrix of the RBF model based on $<$inputdata$>$ provided. \item\end{CompactList}\item 
void {\bf set\-Weight\-Matrix} (Array$<$ double $>$ \&target\-Data)
\begin{CompactList}\small\item\em This function set the weight matrix of the RBF model based on $<$targetdata$>$ provided. \item\end{CompactList}\item 
Array$<$ double $>$ {\bf get\-Design\-Matrix} ()
\begin{CompactList}\small\item\em This function returns the design(hidden) matrix of the RBF model. \item\end{CompactList}\item 
Array$<$ double $>$ {\bf get\-Weight\-Matrix} ()
\begin{CompactList}\small\item\em This function returns the weight matrix of the RBF model. \item\end{CompactList}\item 
Array$<$ double $>$ {\bf get\-Base\-Centres} ()
\begin{CompactList}\small\item\em This function returns the kernel function centres of the RBF model. \item\end{CompactList}\end{CompactItemize}
\subsection*{Static Public Member Functions}
\begin{CompactItemize}
\item 
double {\bf get\-Distance} (Array$<$ double $>$ \&input, unsigned i, Array$<$ double $>$ \&centre, unsigned j)
\begin{CompactList}\small\item\em This function calculates the distance between {\em -th $<$input$>$ and j-th $<$centre$>$. \/}\item\end{CompactList}\item 
double {\bf gauss\-Func} (Array$<$ double $>$ input, unsigned i, Array$<$ double $>$ centre, unsigned j, Array$<$ double $>$ params)
\begin{CompactList}\small\item\em This function calculates the gaussian kernel output between {\em -th $<$input$>$ and j-th $<$centre$>$. \/}\item\end{CompactList}\item 
double {\bf cubic\-Func} (Array$<$ double $>$ input, unsigned i, Array$<$ double $>$ centre, unsigned j, Array$<$ double $>$ params)
\begin{CompactList}\small\item\em This function calculates the cubic spline kernel output between {\em -th $<$input$>$ and j-th $<$centre$>$. \/}\item\end{CompactList}\item 
double {\bf linear\-Func} (Array$<$ double $>$ input, unsigned i, Array$<$ double $>$ centre, unsigned j, Array$<$ double $>$ params)
\begin{CompactList}\small\item\em This function calculates the linear spline kernel output between {\em -th $<$input$>$ and j-th $<$centre$>$. \/}\item\end{CompactList}\item 
double {\bf multiquadric\-Func} (Array$<$ double $>$ input, unsigned i, Array$<$ double $>$ centre, unsigned j, Array$<$ double $>$ params)
\begin{CompactList}\small\item\em This function calculates the multiquadrics kernel output between {\em -th $<$input$>$ and j-th $<$centre$>$. \/}\item\end{CompactList}\item 
double {\bf inverse\-Multiquadric\-Func} (Array$<$ double $>$ input, unsigned i, Array$<$ double $>$ centre, unsigned j, Array$<$ double $>$ params)
\begin{CompactList}\small\item\em This function calculates the inverse multiquadrics kernel output between {\em -th $<$input$>$ and j-th $<$centre$>$. \/}\item\end{CompactList}\item 
double {\bf cauchy\-Func} (Array$<$ double $>$ input, unsigned i, Array$<$ double $>$ centre, unsigned j, Array$<$ double $>$ params)
\begin{CompactList}\small\item\em This function calculates the cauchy kernel output between {\em -th $<$input$>$ and j-th $<$centre$>$. \/}\item\end{CompactList}\item 
void {\bf KMean} (unsigned n\-Cluster, unsigned max\-Iter, Array$<$ double $>$ input, Array$<$ double $>$ \&centres\-Found, Array$<$ unsigned $>$ \&clust\-Map)
\begin{CompactList}\small\item\em This function performs the K-Mean clustering algorithm to provide $<$ncluster$>$ clusters based on a set of inputs $<$input$>$. \item\end{CompactList}\item 
template$<$class Type$>$ void {\bf init\-Array\-To\-Zero} (Array$<$ Type $>$ \&arr)
\begin{CompactList}\small\item\em This function initializes an array to zero. \item\end{CompactList}\end{CompactItemize}


\subsection{Detailed Description}
This class provides different functions for the approximation using an Interpolation\-RBF model. 

This class is based on the RBF approximation models. Therefore in the next step the model should be trained and after this the evaluation functions can be used. 



\subsection{Constructor \& Destructor Documentation}
\index{InterpolateRBF@{Interpolate\-RBF}!InterpolateRBF@{InterpolateRBF}}
\index{InterpolateRBF@{InterpolateRBF}!InterpolateRBF@{Interpolate\-RBF}}
\subsubsection{\setlength{\rightskip}{0pt plus 5cm}Interpolate\-RBF::Interpolate\-RBF (unsigned {\em in\-Dim}, kernel\-Type {\em a\-Kernel}, Array$<$ double $>$ {\em kernel\-Params})}\label{classInterpolateRBF_a1}


This constructor generates an Interpolation RBF model. 

\begin{Desc}
\item[Parameters:]
\begin{description}
\item[{\em in\-Dim}]dimensionality of the problem. \item[{\em a\-Kernel}]kernel function used. \item[{\em kernel\-Params}]parameters for the kernel function.\end{description}
\end{Desc}


\subsection{Member Function Documentation}
\index{InterpolateRBF@{Interpolate\-RBF}!cauchyFunc@{cauchyFunc}}
\index{cauchyFunc@{cauchyFunc}!InterpolateRBF@{Interpolate\-RBF}}
\subsubsection{\setlength{\rightskip}{0pt plus 5cm}double Interpolate\-RBF::cauchy\-Func (Array$<$ double $>$ {\em input}, unsigned {\em i}, Array$<$ double $>$ {\em centre}, unsigned {\em j}, Array$<$ double $>$ {\em params})\hspace{0.3cm}{\tt  [static]}}\label{classInterpolateRBF_e6}


This function calculates the cauchy kernel output between {\em -th $<$input$>$ and j-th $<$centre$>$. \/}

\begin{Desc}
\item[Parameters:]
\begin{description}
\item[{\em input}]array containing the input which are used for training. \item[{\em i}]index of the input which distance to be calculated. \item[{\em centre}]array containing the centres of the RBF model. \item[{\em j}]index of the centre to which the distance from input is to be calculated. \item[{\em params}]array containing parameter(s) for the cauchy kernel function. \end{description}
\end{Desc}
\begin{Desc}
\item[Return values:]
\begin{description}
\item[{\em kernel\-Output}]value of kernel output.\end{description}
\end{Desc}
\index{InterpolateRBF@{Interpolate\-RBF}!cubicFunc@{cubicFunc}}
\index{cubicFunc@{cubicFunc}!InterpolateRBF@{Interpolate\-RBF}}
\subsubsection{\setlength{\rightskip}{0pt plus 5cm}double Interpolate\-RBF::cubic\-Func (Array$<$ double $>$ {\em input}, unsigned {\em i}, Array$<$ double $>$ {\em centre}, unsigned {\em j}, Array$<$ double $>$ {\em params})\hspace{0.3cm}{\tt  [static]}}\label{classInterpolateRBF_e2}


This function calculates the cubic spline kernel output between {\em -th $<$input$>$ and j-th $<$centre$>$. \/}

\begin{Desc}
\item[Parameters:]
\begin{description}
\item[{\em input}]array containing the input which are used for training. \item[{\em i}]index of the input which distance to be calculated. \item[{\em centre}]array containing the centres of the RBF model. \item[{\em j}]index of the centre to which the distance from input is to be calculated. \item[{\em params}]array containing parameter(s) for the cubic spline kernel function. \end{description}
\end{Desc}
\begin{Desc}
\item[Return values:]
\begin{description}
\item[{\em kernel\-Output}]value of kernel output.\end{description}
\end{Desc}
\index{InterpolateRBF@{Interpolate\-RBF}!evaluate@{evaluate}}
\index{evaluate@{evaluate}!InterpolateRBF@{Interpolate\-RBF}}
\subsubsection{\setlength{\rightskip}{0pt plus 5cm}void Interpolate\-RBF::evaluate (Array$<$ double $>$ {\em input\-Data}, Array$<$ double $>$ \& {\em output\-Data})}\label{classInterpolateRBF_a5}


This function evaluates the data in $<$inputdata$>$ with the RBF model and stores the results in $<$outputdata$>$. 

\begin{Desc}
\item[Parameters:]
\begin{description}
\item[{\em input\-Data}]array containing the input parameters. \item[{\em output\-Data}]reference to array containing the evaluation results.\end{description}
\end{Desc}
\index{InterpolateRBF@{Interpolate\-RBF}!evaluate@{evaluate}}
\index{evaluate@{evaluate}!InterpolateRBF@{Interpolate\-RBF}}
\subsubsection{\setlength{\rightskip}{0pt plus 5cm}void Interpolate\-RBF::evaluate (Population \& {\em offsprings})}\label{classInterpolateRBF_a4}


This function evaluates the population $<$offsprings$>$ with the RBF model and stores the results as the fitness value of each member of offsprings. 

\begin{Desc}
\item[Parameters:]
\begin{description}
\item[{\em offsprings}]population of offspring to be evaluated.\end{description}
\end{Desc}
\index{InterpolateRBF@{Interpolate\-RBF}!evaluate@{evaluate}}
\index{evaluate@{evaluate}!InterpolateRBF@{Interpolate\-RBF}}
\subsubsection{\setlength{\rightskip}{0pt plus 5cm}void Interpolate\-RBF::evaluate (Individual \& {\em offspring})}\label{classInterpolateRBF_a3}


This function evaluates the individual $<$offspring$>$ with the RBF model and stores the result as the fitness value of the offspring. 

\begin{Desc}
\item[Parameters:]
\begin{description}
\item[{\em offspring}]individual of offspring to be evaluated.\end{description}
\end{Desc}
\index{InterpolateRBF@{Interpolate\-RBF}!gaussFunc@{gaussFunc}}
\index{gaussFunc@{gaussFunc}!InterpolateRBF@{Interpolate\-RBF}}
\subsubsection{\setlength{\rightskip}{0pt plus 5cm}double Interpolate\-RBF::gauss\-Func (Array$<$ double $>$ {\em input}, unsigned {\em i}, Array$<$ double $>$ {\em centre}, unsigned {\em j}, Array$<$ double $>$ {\em params})\hspace{0.3cm}{\tt  [static]}}\label{classInterpolateRBF_e1}


This function calculates the gaussian kernel output between {\em -th $<$input$>$ and j-th $<$centre$>$. \/}

\begin{Desc}
\item[Parameters:]
\begin{description}
\item[{\em input}]array containing the input which are used for training. \item[{\em i}]index of the input which distance to be calculated. \item[{\em centre}]array containing the centres of the RBF model. \item[{\em j}]index of the centre to which the distance from input is to be calculated. \item[{\em params}]array containing parameter(s) for the gaussian kernel function. \end{description}
\end{Desc}
\begin{Desc}
\item[Return values:]
\begin{description}
\item[{\em kernel\-Output}]value of kernel output.\end{description}
\end{Desc}
\index{InterpolateRBF@{Interpolate\-RBF}!getBaseCentres@{getBaseCentres}}
\index{getBaseCentres@{getBaseCentres}!InterpolateRBF@{Interpolate\-RBF}}
\subsubsection{\setlength{\rightskip}{0pt plus 5cm}Array$<$double$>$ Interpolate\-RBF::get\-Base\-Centres ()}\label{classInterpolateRBF_a13}


This function returns the kernel function centres of the RBF model. 

\begin{Desc}
\item[Return values:]
\begin{description}
\item[{\em base\-Centres}]array containing the $<$basecentres$>$ of the RBF model.\end{description}
\end{Desc}
\index{InterpolateRBF@{Interpolate\-RBF}!getDesignMatrix@{getDesignMatrix}}
\index{getDesignMatrix@{getDesignMatrix}!InterpolateRBF@{Interpolate\-RBF}}
\subsubsection{\setlength{\rightskip}{0pt plus 5cm}Array$<$double$>$ Interpolate\-RBF::get\-Design\-Matrix ()}\label{classInterpolateRBF_a11}


This function returns the design(hidden) matrix of the RBF model. 

\begin{Desc}
\item[Return values:]
\begin{description}
\item[{\em design\-Matrix}]array containing the $<$designmatrix$>$ of the RBF model.\end{description}
\end{Desc}
\index{InterpolateRBF@{Interpolate\-RBF}!getDistance@{getDistance}}
\index{getDistance@{getDistance}!InterpolateRBF@{Interpolate\-RBF}}
\subsubsection{\setlength{\rightskip}{0pt plus 5cm}double Interpolate\-RBF::get\-Distance (Array$<$ double $>$ \& {\em input}, unsigned {\em i}, Array$<$ double $>$ \& {\em centre}, unsigned {\em j})\hspace{0.3cm}{\tt  [static]}}\label{classInterpolateRBF_e0}


This function calculates the distance between {\em -th $<$input$>$ and j-th $<$centre$>$. \/}

\begin{Desc}
\item[Parameters:]
\begin{description}
\item[{\em input}]array containing the input which are used for training. \item[{\em i}]index of the input which distance to be calculated. \item[{\em centre}]array containing the centres of the RBF model. \item[{\em j}]index of the centre to which the distance from input is to be calculated. \end{description}
\end{Desc}
\begin{Desc}
\item[Return values:]
\begin{description}
\item[{\em distance}]value of distance.\end{description}
\end{Desc}
\index{InterpolateRBF@{Interpolate\-RBF}!getWeightMatrix@{getWeightMatrix}}
\index{getWeightMatrix@{getWeightMatrix}!InterpolateRBF@{Interpolate\-RBF}}
\subsubsection{\setlength{\rightskip}{0pt plus 5cm}Array$<$double$>$ Interpolate\-RBF::get\-Weight\-Matrix ()}\label{classInterpolateRBF_a12}


This function returns the weight matrix of the RBF model. 

\begin{Desc}
\item[Return values:]
\begin{description}
\item[{\em weight\-Matrix}]array containing the $<$weightmatrix$>$ of the RBF model.\end{description}
\end{Desc}
\index{InterpolateRBF@{Interpolate\-RBF}!initArrayToZero@{initArrayToZero}}
\index{initArrayToZero@{initArrayToZero}!InterpolateRBF@{Interpolate\-RBF}}
\subsubsection{\setlength{\rightskip}{0pt plus 5cm}template$<$class Type$>$ void Interpolate\-RBF::init\-Array\-To\-Zero (Array$<$ Type $>$ \& {\em arr})\hspace{0.3cm}{\tt  [inline, static]}}\label{classInterpolateRBF_e8}


This function initializes an array to zero. 

\begin{Desc}
\item[Parameters:]
\begin{description}
\item[{\em arr}]array to be initialized to zero.\end{description}
\end{Desc}
\index{InterpolateRBF@{Interpolate\-RBF}!inverseMultiquadricFunc@{inverseMultiquadricFunc}}
\index{inverseMultiquadricFunc@{inverseMultiquadricFunc}!InterpolateRBF@{Interpolate\-RBF}}
\subsubsection{\setlength{\rightskip}{0pt plus 5cm}double Interpolate\-RBF::inverse\-Multiquadric\-Func (Array$<$ double $>$ {\em input}, unsigned {\em i}, Array$<$ double $>$ {\em centre}, unsigned {\em j}, Array$<$ double $>$ {\em params})\hspace{0.3cm}{\tt  [static]}}\label{classInterpolateRBF_e5}


This function calculates the inverse multiquadrics kernel output between {\em -th $<$input$>$ and j-th $<$centre$>$. \/}

\begin{Desc}
\item[Parameters:]
\begin{description}
\item[{\em input}]array containing the input which are used for training. \item[{\em i}]index of the input which distance to be calculated. \item[{\em centre}]array containing the centres of the RBF model. \item[{\em j}]index of the centre to which the distance from input is to be calculated. \item[{\em params}]array containing parameter(s) for the inverse multiquadrics kernel function. \end{description}
\end{Desc}
\begin{Desc}
\item[Return values:]
\begin{description}
\item[{\em kernel\-Output}]value of kernel output.\end{description}
\end{Desc}
\index{InterpolateRBF@{Interpolate\-RBF}!KMean@{KMean}}
\index{KMean@{KMean}!InterpolateRBF@{Interpolate\-RBF}}
\subsubsection{\setlength{\rightskip}{0pt plus 5cm}void Interpolate\-RBF::KMean (unsigned {\em n\-Cluster}, unsigned {\em max\-Iter}, Array$<$ double $>$ {\em input}, Array$<$ double $>$ \& {\em centres\-Found}, Array$<$ unsigned $>$ \& {\em clust\-Map})\hspace{0.3cm}{\tt  [static]}}\label{classInterpolateRBF_e7}


This function performs the K-Mean clustering algorithm to provide $<$ncluster$>$ clusters based on a set of inputs $<$input$>$. 

\begin{Desc}
\item[Parameters:]
\begin{description}
\item[{\em n\-Cluster}]number of desired clusters. \item[{\em max\-Iter}]maximum iteration count. \item[{\em input}]array containing the inputs. \item[{\em centres\-Found}]reference to array containing the cluster centres found. \item[{\em clust\-Map}]reference to array containing the pair of inputs and cluster numbers to which each of them are assigned.\end{description}
\end{Desc}
\index{InterpolateRBF@{Interpolate\-RBF}!linearFunc@{linearFunc}}
\index{linearFunc@{linearFunc}!InterpolateRBF@{Interpolate\-RBF}}
\subsubsection{\setlength{\rightskip}{0pt plus 5cm}double Interpolate\-RBF::linear\-Func (Array$<$ double $>$ {\em input}, unsigned {\em i}, Array$<$ double $>$ {\em centre}, unsigned {\em j}, Array$<$ double $>$ {\em params})\hspace{0.3cm}{\tt  [static]}}\label{classInterpolateRBF_e3}


This function calculates the linear spline kernel output between {\em -th $<$input$>$ and j-th $<$centre$>$. \/}

\begin{Desc}
\item[Parameters:]
\begin{description}
\item[{\em input}]array containing the input which are used for training. \item[{\em i}]index of the input which distance to be calculated. \item[{\em centre}]array containing the centres of the RBF model. \item[{\em j}]index of the centre to which the distance from input is to be calculated. \item[{\em params}]array containing parameter(s) for the linear spline kernel function. \end{description}
\end{Desc}
\begin{Desc}
\item[Return values:]
\begin{description}
\item[{\em kernel\-Output}]value of kernel output.\end{description}
\end{Desc}
\index{InterpolateRBF@{Interpolate\-RBF}!mse@{mse}}
\index{mse@{mse}!InterpolateRBF@{Interpolate\-RBF}}
\subsubsection{\setlength{\rightskip}{0pt plus 5cm}double Interpolate\-RBF::mse (Population {\em offsprings})}\label{classInterpolateRBF_a7}


This function calculates the mean square error of the approximation considering a given population $<$offsprings$>$. 

\begin{Desc}
\item[Parameters:]
\begin{description}
\item[{\em offsprings}]population containing the parameters in the first Chromosome and the fitness value as target value \end{description}
\end{Desc}
\begin{Desc}
\item[Return values:]
\begin{description}
\item[{\em MSE}]mean square error of the approximation\end{description}
\end{Desc}
\index{InterpolateRBF@{Interpolate\-RBF}!mse@{mse}}
\index{mse@{mse}!InterpolateRBF@{Interpolate\-RBF}}
\subsubsection{\setlength{\rightskip}{0pt plus 5cm}double Interpolate\-RBF::mse (Array$<$ double $>$ {\em Input}, Array$<$ double $>$ {\em Target})}\label{classInterpolateRBF_a6}


This function calculates the mean square error of the approximation considering the given arrays $<$input$>$ and $<$target$>$. 

\begin{Desc}
\item[Parameters:]
\begin{description}
\item[{\em Input}]array containing the input data. \item[{\em Target}]array containing the target data. \end{description}
\end{Desc}
\begin{Desc}
\item[Return values:]
\begin{description}
\item[{\em MSE}]mean square error of the approximation.\end{description}
\end{Desc}
\index{InterpolateRBF@{Interpolate\-RBF}!multiquadricFunc@{multiquadricFunc}}
\index{multiquadricFunc@{multiquadricFunc}!InterpolateRBF@{Interpolate\-RBF}}
\subsubsection{\setlength{\rightskip}{0pt plus 5cm}double Interpolate\-RBF::multiquadric\-Func (Array$<$ double $>$ {\em input}, unsigned {\em i}, Array$<$ double $>$ {\em centre}, unsigned {\em j}, Array$<$ double $>$ {\em params})\hspace{0.3cm}{\tt  [static]}}\label{classInterpolateRBF_e4}


This function calculates the multiquadrics kernel output between {\em -th $<$input$>$ and j-th $<$centre$>$. \/}

\begin{Desc}
\item[Parameters:]
\begin{description}
\item[{\em input}]array containing the input which are used for training. \item[{\em i}]index of the input which distance to be calculated. \item[{\em centre}]array containing the centres of the RBF model. \item[{\em j}]index of the centre to which the distance from input is to be calculated. \item[{\em params}]array containing parameter(s) for the multiquadrics kernel function. \end{description}
\end{Desc}
\begin{Desc}
\item[Return values:]
\begin{description}
\item[{\em kernel\-Output}]value of kernel output.\end{description}
\end{Desc}
\index{InterpolateRBF@{Interpolate\-RBF}!setDesignMatrix@{setDesignMatrix}}
\index{setDesignMatrix@{setDesignMatrix}!InterpolateRBF@{Interpolate\-RBF}}
\subsubsection{\setlength{\rightskip}{0pt plus 5cm}void Interpolate\-RBF::set\-Design\-Matrix (Array$<$ double $>$ \& {\em input\-Data})}\label{classInterpolateRBF_a9}


This function set the design(hidden) matrix of the RBF model based on $<$inputdata$>$ provided. 

\begin{Desc}
\item[Parameters:]
\begin{description}
\item[{\em input\-Data}]array containing the inputdata which are used for approximation.\end{description}
\end{Desc}
\index{InterpolateRBF@{Interpolate\-RBF}!setWeightMatrix@{setWeightMatrix}}
\index{setWeightMatrix@{setWeightMatrix}!InterpolateRBF@{Interpolate\-RBF}}
\subsubsection{\setlength{\rightskip}{0pt plus 5cm}void Interpolate\-RBF::set\-Weight\-Matrix (Array$<$ double $>$ \& {\em target\-Data})}\label{classInterpolateRBF_a10}


This function set the weight matrix of the RBF model based on $<$targetdata$>$ provided. 

\begin{Desc}
\item[Parameters:]
\begin{description}
\item[{\em input\-Data}]array containing the inputdata which are used for approximation. \item[{\em target\-Data}]array containing the target data used for approximation.\end{description}
\end{Desc}
\index{InterpolateRBF@{Interpolate\-RBF}!train@{train}}
\index{train@{train}!InterpolateRBF@{Interpolate\-RBF}}
\subsubsection{\setlength{\rightskip}{0pt plus 5cm}void Interpolate\-RBF::train (Array$<$ double $>$ {\em Input\-Data}, Array$<$ double $>$ {\em Target\-Data})}\label{classInterpolateRBF_a8}


This function approximates the datas which are stored in the arrays $<$inputdata$>$ and $<$targetdata$>$ using the RBF model. 

\begin{Desc}
\item[Parameters:]
\begin{description}
\item[{\em Input\-Data}]array containing the inputdata which are used for approximation. \item[{\em Target\-Data}]array containing the targetdata which are used for approximation.\end{description}
\end{Desc}


The documentation for this class was generated from the following file:\begin{CompactItemize}
\item 
{\bf Interpolate\-RBF.h}\end{CompactItemize}
