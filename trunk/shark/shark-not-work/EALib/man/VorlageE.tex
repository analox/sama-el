% =======================================================================
% Dateiname:        Vorlage.tex
% Autor:            Ruediger Alberts
% EMail:            Ruediger.Alberts@neuroinformatik.ruhr-uni-bochum.de
% Erstellt am:      1999-09-07
% Letzte Aenderung: 1999-09-09
%
% Datei wird mit Befehl ``% =======================================================================
% Dateiname:        Vorlage.tex
% Autor:            Ruediger Alberts
% EMail:            Ruediger.Alberts@neuroinformatik.ruhr-uni-bochum.de
% Erstellt am:      1999-09-07
% Letzte Aenderung: 1999-09-09
%
% Datei wird mit Befehl ``% =======================================================================
% Dateiname:        Vorlage.tex
% Autor:            Ruediger Alberts
% EMail:            Ruediger.Alberts@neuroinformatik.ruhr-uni-bochum.de
% Erstellt am:      1999-09-07
% Letzte Aenderung: 1999-09-09
%
% Datei wird mit Befehl ``% =======================================================================
% Dateiname:        Vorlage.tex
% Autor:            Ruediger Alberts
% EMail:            Ruediger.Alberts@neuroinformatik.ruhr-uni-bochum.de
% Erstellt am:      1999-09-07
% Letzte Aenderung: 1999-09-09
%
% Datei wird mit Befehl ``\input{Vorlage}'' in LaTex-Dokumente 
% (Latex2Epsilon) eingebunden und stellt Befehle zur einfachen Doku- 
% mentation von C-Funktionen/C++-Methoden bereit.
% =======================================================================

\usepackage{ifthen}

    % Definition von Konstanten:
    \newcommand{\ParamDelim}            % Trennungstext zwischen
        { \ - }                         % Parametername und -beschreibung.
    \newcommand{\NormMethodEnd}         % Endtext von Methode bei 
        {\ \ );}                        % veraenderbarer Aufrufinstanz.
    \newcommand{\ConstMethodEnd}        % Endtext von Methode bei
        {\ \ ) const;}                  % nicht veraenderbarer Aufrufinstanz.
    \newcommand{\MaxLabelTxt}           % Komponentenbeschreibungstext
        {{\bf Besonderheiten: }}        % mit maximaler Breite.
    \newcommand{\MethodBegin}           % Beginntext einer Methode.
        {(\ }
    \newcommand{\InBetween}             % Text zwischen Rueckgabewert und
        {\ \ }                          % Methodenname.

    % Definition eigener Variablen:


    \newboolean{ConstInstance}          % Gibt an, ob die Instanz, die
                                        % die Methode aufruft veraendert
                                        % werden darf oder nicht. Im
                                        % 2. Fall endet die Methode mit
                                        % ``) const;'', im 1. Fall 
                                        % mit ``);''.
    \newcounter{SavedParams}            % Anzahl der gespeicherten
                                        % Parameter.
    \setcounter{SavedParams}{0}
    \newcommand{\ParamOne}{}            % Definition der 10 
    \newcommand{\ParamTypeOne}{}        % abspeicherbaren Parameter,
    \newcommand{\ParamDescrOne}{}       % ihrer Typen und Beschreibungen.
    \newcommand{\ParamTwo}{}            
    \newcommand{\ParamTypeTwo}{}
    \newcommand{\ParamDescrTwo}{}
    \newcommand{\ParamThree}{}
    \newcommand{\ParamTypeThree}{}
    \newcommand{\ParamDescrThree}{}
    \newcommand{\ParamFour}{}
    \newcommand{\ParamTypeFour}{}
    \newcommand{\ParamDescrFour}{}
    \newcommand{\ParamFive}{}
    \newcommand{\ParamTypeFive}{}
    \newcommand{\ParamDescrFive}{}
    \newcommand{\ParamSix}{}
    \newcommand{\ParamTypeSix}{}
    \newcommand{\ParamDescrSix}{}
    \newcommand{\ParamSeven}{}
    \newcommand{\ParamTypeSeven}{}
    \newcommand{\ParamDescrSeven}{}
    \newcommand{\ParamEight}{}
    \newcommand{\ParamTypeEight}{}
    \newcommand{\ParamDescrEight}{}
    \newcommand{\ParamNine}{}
    \newcommand{\ParamTypeNine}{}
    \newcommand{\ParamDescrNine}{}
    \newcommand{\ParamTen}{}
    \newcommand{\ParamTypeTen}{}
    \newcommand{\ParamDescrTen}{}
    \newlength{\MaxParamWidth}          % Breite des breitesten 
                                        % Methodenparameters.
    \newlength{\SpecialWidth}           % Breite des Textes 
    \settowidth{\SpecialWidth}          % {\bf Besonderheiten}.
        {\MaxLabelTxt}
    \newlength{\ParamDelimWidth}        % Breite des Textes `` \ - ``.
    \settowidth{\ParamDelimWidth}
        {\ParamDelim}
    \newlength{\ParamDescrWidth}        % Max. Breite des Parametertextes.
    \newlength{\SpecialRetTxtWidth}     % Max. Breite der Texte fuer
                                        % Besonderheiten und Rueckgabewert.
    \newlength{\MethodBeginWidth}       % Breite des Methodenanfangs, d.h.
                                        % ``Rueckgabewert Methodenname ( ``.
    \newlength{\BetweenParsWidth}       % Max. zur Verfuegung stehende
                                        % Breite zwischen den beiden Klammern
                                        % der Methode.
    
    \newlength{\MethodEndWidth}         % Breite des Textes ``  );'' bzw.
    \settowidth{\MethodEndWidth}        % `` ) const;''
        {\NormMethodEnd}
    \newlength{\ParamMaxTypeWidth}      % Breite des breitesten
                                        % Parameterwert-Typs. 
    \newlength{\NoneWidthOne}           % Breite des Textes ``Keine.''.
    \settowidth{\NoneWidthOne}{Keine.}
    \newlength{\NoneWidthTwo}           % Breite des Textes ``Keiner.''.
    \settowidth{\NoneWidthTwo}{Keiner.}
    \newlength{\Temp}                   % Temporaere Hilfsvariable. 
    \newlength{\VertSpace}              % Vertikaler Abstand zwischen 
                                        % zwei Parboxen.
   
    % Trotz der automatischen Berechnung der Textbreiten treten
    % kleine Unstimmigkeiten von wenigen Punkten bei den Texten
    % zur Beschreibung der Parameter, des Rueckgabewertes und
    % der Besonderheiten auf, die mit den folgenden Variablen
    % behoben werden koennen:

    \newlength{\CorrectWidthOne}        % Korrekturbreite fuer die
                                        % Parameterbeschreibung.
    \newlength{\CorrectWidthTwo}        % Korrekturbreite fuer die
                                        % Beschreibungen von Rueckgabewert
                                        % und Besonderheiten.
    \newlength{\CorrectWidthThree}      % Korrekturbreite, da erster
                                        % Parameter einer Methode weiter
                                        % vorrueckt als die anderen.
    

% Uebersicht ueber die Parameter und ihre Bedeutung.


%
% <-- \MethodBeginWidth    -->
%------------------------------------------------------------------------
% Rueckgabewert Methodenname(   Typ_Parameter_1 Parameter_1,
%                            <->Typ_Parameter_2 Parameter_2,
%              \CorrectWideThree...
%                               Typ_Parameter_n Parameter_n   );
%                             <-------------> <--------->
%                          \ParamMaxTypeWidth \MaxParamWidth
%                             <-------------------------->
%                             \BetweenParsWidth
%                                                         <-->
%                                                         \MethodEndWidth
%------------------------------------------------------------------------
% Beschreibung der Methode...
%                             \ParamDelimWidth
%                             <->                        \CorrectWidthOne
% \VertSpace
% Parameter:       Parameter_1  - Beschreibung von Parameter 1.   <----->
%                  Parameter_2  - Beschreibung von Parameter 2.
%                  ...
%                  Parameter_n  - Beschreibung von Parameter n.
% \VertSpace                      \ParamDescrWidth
%                                 <------------------------------>
% Rueckgabewert:   Beschreibung des Rueckgabewertes bzw. der Text <-----> 
%                  Keiner.                   
%                  <----->
%                  \NoneWidthTwo               \SpecialRetTxtWidth
% \VertSpace       <--------------------------------------------->
% Besonderheiten:  Beschreibung von Besonderheiten bzw. der Text  <----->
%                  Keine.     
% <--------------> <---->
%  \SpecialWidth   \NoneWidthOne                         \CorrectWidthTwo


    % Definition eigener Befehle:


    %-----------------------------------------------------------------------
    % Gibt den Begriff ``C++'' in ansprechender Form aus.
    % Makro wurde freundlicherweise zur Verfuegung gestellt von
    % Axel W. Dietrich.
    %
    \newcommand{\cpp}{
        \mbox{\emph{\textrm{C\hspace{-1.5pt}\raisebox{1.75pt}{\scriptsize +}
        \hspace{-6pt}\raisebox{.75pt}{\scriptsize +}}}}%
    }


    %-----------------------------------------------------------------------
    % Gibt die Werte aller Parameter auf dem Bildschirm aus.
    %
    \newcommand{\showAllParams}{
        \noindent
        {\em Werte aller Parameter:}
        \begin{tabbing}
            ParamMaxTypeWidth:\ \ \=\kill
            MaxParamWidth:       \>\the\MaxParamWidth\\
            SpecialWidth:        \>\the\SpecialWidth\\
            ParamDelimWidth:     \>\the\ParamDelimWidth\\
            ParamDescrWidth:     \>\the\ParamDescrWidth\\
            SpecialRetTxtWidth:  \>\the\SpecialRetTxtWidth\\
            MethodBeginWidth:    \>\the\MethodBeginWidth\\
            BetweenParsWidth:    \>\the\BetweenParsWidth\\
            MethodEndWidth:      \>\the\MethodEndWidth\\
            ParamMaxTypeWidth:   \>\the\ParamMaxTypeWidth\\
            NoneWidthOne:        \>\the\NoneWidthOne\\
            NoneWidthTwo:        \>\the\NoneWidthTwo\\
            VertSpace:           \>\the\VertSpace\\
            CorrectWidthOne:     \>\the\CorrectWidthOne\\
            CorrectWidthTwo:     \>\the\CorrectWidthTwo\\
            CorrectWidthThree:   \>\the\CorrectWidthThree\\
            ConstInstance:       
            \ifthenelse{\boolean{ConstInstance}}
                {\>ja\\}
                {\>nein\\}
        \end{tabbing}
    }

    %-----------------------------------------------------------------------
    % Setzt einen Schalter, der angibt, dass die Instanz, die die kommende
    % Methode aufrufen kann, nicht veraendert werden darf. Dies hat fuer
    % die Dokumentation zur Folge, dass die Methode mit ``) const;'' endet.
    %
    \newcommand{\setConstInstance}{    
        \setboolean{ConstInstance}{true}
    }

    %-----------------------------------------------------------------------
    % Setzt einen Schalter, der angibt, dass die Instanz, die die kommende
    % Methode aufrufen kann, veraendert werden darf. Die Methode endet also
    % in der Dokumentation normal mit ``);''.
    %
    \newcommand{\setNormalInstance}{    
        \setboolean{ConstInstance}{false}
    }

    %-----------------------------------------------------------------------
    % Initialisiert die Breite des breitesten Methodenparametertyptextes
    % mit Null.
    %
    \newcommand{\initParamMaxTypeWidth}{
        \setlength{\ParamMaxTypeWidth}{0pt}
    }

    %-----------------------------------------------------------------------
    % Initialisiert die Breite des breitesten Parameternamens mit Null.
    %
    \newcommand{\initMaxParamWidth}{
        \setlength{\MaxParamWidth}{0pt}
    }

    %-----------------------------------------------------------------------
    % Ueberprueft, ob die Breite des uebergebenen Textes des Parametertyps
    % groesser ist als die des bisher als breitester Typtext festgelegten
    % Textes und passt den Wert evtl. an.
    % Parameter #1:  Neuer Typtext, dessen Breite zum Vergleich dient.
    %   
    \newcommand{\setNewParamMaxTypeWidth}[1]{
        \settowidth{\Temp}{#1}
        \ifthenelse{\lengthtest{\ParamMaxTypeWidth < \Temp}}
            {\setlength{\ParamMaxTypeWidth}{\Temp}}{}   
    }

    %-----------------------------------------------------------------------
    % Ueberprueft, ob die Breite des uebergebenen Parameternamens groesser
    % ist als die des bisher als am breitesten geltenden Parameternames
    % und passt den Wert evtl. an. Der Parametername steht dabei in
    % Emphasized.
    % Parameter #1:  Neuer Parametername, dessen Breite zum Vergleich dient.
    %
    \newcommand{\setNewMaxParamWidth}[1]{
        \settowidth{\Temp}{{\em #1}}
        \ifthenelse{\lengthtest{\MaxParamWidth < \Temp}}
            {\setlength{\MaxParamWidth}{\Temp}}{}
    }

    %-----------------------------------------------------------------------
    % Setzt die Breite der Beschreibungstexte fuer Rueckgabewert
    % und Besonderheiten.
    %
    \newcommand{\setSpecialRetTxtWidth}{
        \setlength{\SpecialRetTxtWidth}{\textwidth}
        \addtolength{\SpecialRetTxtWidth}{-1\SpecialWidth} 
    }

    %-----------------------------------------------------------------------
    % Setzt die Breite fuer den Beginn einer Methode, d.h. den Text
    % ``Rueckgabewert Methodenname( ``. Rueckgabewert steht dabei
    % in Emphasized, Methodenname in Bold Font.
    % Parameter #1:  Rueckgabewert der Methode.
    % Parameter #2:  Name der Methode.
    %
    \newcommand{\setMethodBeginWidth}[2]{
        \settowidth{\MethodBeginWidth}
            {{\em #1}\InBetween{\bf #2}\MethodBegin}
    }

    %-----------------------------------------------------------------------
    % Setzt die maximale Breite die an Platz zwischen der oeffnenden
    % und der schliessenden Klammer der Methode zur Verfuegung steht.
    %
    \newcommand{\setBetweenParsWidth}{
        \setlength{\BetweenParsWidth}{\textwidth}
        \addtolength{\BetweenParsWidth}{-1\MethodBeginWidth}
        \addtolength{\BetweenParsWidth}{-1\MethodEndWidth}
    }

    %-----------------------------------------------------------------------
    % Setzt die Breite fuer die Beschreibungstexte
    % der Methodenparameter.
    %
    \newcommand{\setParamDescrWidth}{
        \setlength{\ParamDescrWidth}{\textwidth}
        \addtolength{\ParamDescrWidth}{-1\SpecialWidth} 
        \addtolength{\ParamDescrWidth}{-1\MaxParamWidth}
        \addtolength{\ParamDescrWidth}{-1\ParamDelimWidth}
    }    

    %-----------------------------------------------------------------------
    % Setzt die Korrekturbreite fuer den Beschreibungstext von
    % Methodenparametern auf den uebergebenen Wert.
    % Parameter #1:  Neuer Wert fuer die Korrekturbreite.
    %
    \newcommand{\setCorrectWidthOne}[1]{
        \setlength{\CorrectWidthOne}{#1}
    }    

    %-----------------------------------------------------------------------
    % Setzt die Korrekturbreite fuer den Beschreibungstext von
    % Rueckgabewert und Besonderheiten auf den uebergebenen Wert.
    % Parameter #1:  Neuer Wert fuer die Korrekturbreite.
    %
    \newcommand{\setCorrectWidthTwo}[1]{
        \setlength{\CorrectWidthTwo}{#1}
    }    

    %-----------------------------------------------------------------------
    % Setzt die Korrekturbreite fuer den Abstand des zweiten bis n-ten
    % Parameters vom Anfang auf den uebergebenen Wert.
    % Parameter #1:  Neuer Wert fuer die Korrekturbreite.
    %
    \newcommand{\setCorrectWidthThree}[1]{
        \setlength{\CorrectWidthThree}{#1}
    }    

    %-----------------------------------------------------------------------
    % Setzt den vertikalen Abstand zwischen zwei Parboxen auf den ueber-
    % gebenen Wert.
    % Parameter #1:  Neuer Wert fuer den vertikalen Abstand.
    %
    \newcommand{\setVertSpace}[1]{
        \setlength{\VertSpace}{#1}
    }    

    %-----------------------------------------------------------------------
    % Speichert die uebergebenen Informationen ueber Methodenparameter 1
    % in internen Variablen. Achtung: Die Parameter muessen in der richtigen
    % Reihenfolge gesetzt werden!
    % Parameter #1: Der Typ des Parameters.  
    % Parameter #2: Der Name des Parameters. 
    % Parameter #3: Die Beschreibung des Parameters. 
    % 
    \newcommand{\setParamOne}[3]{
        \renewcommand{\ParamOne}{#1}
        \renewcommand{\ParamTypeOne}{#2}
        \renewcommand{\ParamDescrOne}{#3} 
        \setcounter{SavedParams}{0}
        \stepcounter{SavedParams}
    }

    %-----------------------------------------------------------------------
    % Speichert die uebergebenen Informationen ueber Methodenparameter 2
    % in internen Variablen. Achtung: Die Parameter muessen in der richtigen
    % Reihenfolge gesetzt werden!
    % Parameter #1: Der Typ des Parameters.  
    % Parameter #2: Der Name des Parameters. 
    % Parameter #3: Die Beschreibung des Parameters. 
    % 
    \newcommand{\setParamTwo}[3]{
        \renewcommand{\ParamTwo}{#1}
        \renewcommand{\ParamTypeTwo}{#2}
        \renewcommand{\ParamDescrTwo}{#3} 
        \stepcounter{SavedParams}
    }

    %-----------------------------------------------------------------------
    % Speichert die uebergebenen Informationen ueber Methodenparameter 3
    % in internen Variablen. Achtung: Die Parameter muessen in der richtigen
    % Reihenfolge gesetzt werden!
    % Parameter #1: Der Typ des Parameters.  
    % Parameter #2: Der Name des Parameters. 
    % Parameter #3: Die Beschreibung des Parameters. 
    % 
    \newcommand{\setParamThree}[3]{
        \renewcommand{\ParamThree}{#1}
        \renewcommand{\ParamTypeThree}{#2}
        \renewcommand{\ParamDescrThree}{#3} 
        \stepcounter{SavedParams}
    }

    %-----------------------------------------------------------------------
    % Speichert die uebergebenen Informationen ueber Methodenparameter 4
    % in internen Variablen. Achtung: Die Parameter muessen in der richtigen
    % Reihenfolge gesetzt werden!
    % Parameter #1: Der Typ des Parameters.  
    % Parameter #2: Der Name des Parameters. 
    % Parameter #3: Die Beschreibung des Parameters. 
    % 
    \newcommand{\setParamFour}[3]{
        \renewcommand{\ParamFour}{#1}
        \renewcommand{\ParamTypeFour}{#2}
        \renewcommand{\ParamDescrFour}{#3} 
        \stepcounter{SavedParams}
    }

    %-----------------------------------------------------------------------
    % Speichert die uebergebenen Informationen ueber Methodenparameter 5
    % in internen Variablen. Achtung: Die Parameter muessen in der richtigen
    % Reihenfolge gesetzt werden!
    % Parameter #1: Der Typ des Parameters.  
    % Parameter #2: Der Name des Parameters. 
    % Parameter #3: Die Beschreibung des Parameters. 
    % 
    \newcommand{\setParamFive}[3]{
        \renewcommand{\ParamFive}{#1}
        \renewcommand{\ParamTypeFive}{#2}
        \renewcommand{\ParamDescrFive}{#3} 
        \stepcounter{SavedParams}
    }

    %-----------------------------------------------------------------------
    % Speichert die uebergebenen Informationen ueber Methodenparameter 6
    % in internen Variablen. Achtung: Die Parameter muessen in der richtigen
    % Reihenfolge gesetzt werden!
    % Parameter #1: Der Typ des Parameters.  
    % Parameter #2: Der Name des Parameters. 
    % Parameter #3: Die Beschreibung des Parameters. 
    %     
    \newcommand{\setParamSix}[3]{
        \renewcommand{\ParamSix}{#1}
        \renewcommand{\ParamTypeSix}{#2}
        \renewcommand{\ParamDescrSix}{#3} 
        \stepcounter{SavedParams} 
    }

    %-----------------------------------------------------------------------
    % Speichert die uebergebenen Informationen ueber Methodenparameter 7
    % in internen Variablen. Achtung: Die Parameter muessen in der richtigen
    % Reihenfolge gesetzt werden!
    % Parameter #1: Der Typ des Parameters.  
    % Parameter #2: Der Name des Parameters. 
    % Parameter #3: Die Beschreibung des Parameters. 
    %    
    \newcommand{\setParamSeven}[3]{
        \renewcommand{\ParamSeven}{#1}
        \renewcommand{\ParamTypeSeven}{#2}
        \renewcommand{\ParamDescrSeven}{#3} 
        \stepcounter{SavedParams} 
    }

    %-----------------------------------------------------------------------
    % Speichert die uebergebenen Informationen ueber Methodenparameter 8
    % in internen Variablen. Achtung: Die Parameter muessen in der richtigen
    % Reihenfolge gesetzt werden!
    % Parameter #1: Der Typ des Parameters.  
    % Parameter #2: Der Name des Parameters. 
    % Parameter #3: Die Beschreibung des Parameters. 
    % 
    \newcommand{\setParamEight}[3]{
        \renewcommand{\ParamEight}{#1}
        \renewcommand{\ParamTypeEight}{#2}
        \renewcommand{\ParamDescrEight}{#3} 
        \stepcounter{SavedParams}
    }

    %-----------------------------------------------------------------------
    % Speichert die uebergebenen Informationen ueber Methodenparameter 9
    % in internen Variablen. Achtung: Die Parameter muessen in der richtigen
    % Reihenfolge gesetzt werden!
    % Parameter #1: Der Typ des Parameters.  
    % Parameter #2: Der Name des Parameters. 
    % Parameter #3: Die Beschreibung des Parameters. 
    % 
    \newcommand{\setParamNine}[3]{
        \renewcommand{\ParamNine}{#1}
        \renewcommand{\ParamTypeNine}{#2}
        \renewcommand{\ParamDescrNine}{#3}  
        \stepcounter{SavedParams}
    }

    %-----------------------------------------------------------------------
    % Speichert die uebergebenen Informationen ueber Methodenparameter 10
    % in internen Variablen. Achtung: Die Parameter muessen in der richtigen
    % Reihenfolge gesetzt werden!
    % Parameter #1: Der Typ des Parameters.  
    % Parameter #2: Der Name des Parameters. 
    % Parameter #3: Die Beschreibung des Parameters. 
    % 
    \newcommand{\setParamTen}[3]{
        \renewcommand{\ParamTen}{#1}
        \renewcommand{\ParamTypeTen}{#2}
        \renewcommand{\ParamDescrTen}{#3} 
        \stepcounter{SavedParams}
    }

    %-----------------------------------------------------------------------
    % Gibt den Text ``Rueckgabewert Methodenname(`` aus.
    % Parameter #1:  Rueckgabewert der Methode.
    % Parameter #2:  Name der Methode.
    %
    \newcommand{\printMethodBegin}[2]{
         \ifthenelse{\boolean{ConstInstance}}
             {\settowidth{\MethodEndWidth}{\ConstMethodEnd}}
             {\settowidth{\MethodEndWidth}{\NormMethodEnd}}
         \setMethodBeginWidth{#1}{#2}
         \setBetweenParsWidth
         \noindent
         \hrulefill\\
         \noindent
         \parbox[t]{\MethodBeginWidth}
             {#1\InBetween{\bf #2}\MethodBegin}
     }

    %-----------------------------------------------------------------------
    % Gibt das Ende einer Methode aus, d.h. ``\ );'' oder ``\ ) const;''.
    %
    \newcommand{\printMethodEnd}{
        \ifthenelse{\boolean{ConstInstance}}
            {\ConstMethodEnd\\}
            {\NormMethodEnd\\}
            \mbox{}\hrulefill\\
    }

    %-----------------------------------------------------------------------
    % Gibt den Typen und Namen des einzigen Parameters einer Methode
    % gefolgt von dem Text ``  );'' bzw. `` ) const;'' aus.
    % Parameter #1:  Typ des Parameters.
    % Parameter #2:  Name des Parameters.
    % 
    \newcommand{\printMethodOneParam}[2]{
        \parbox[t]{\ParamMaxTypeWidth}{#1}
        \parbox[t]{\MaxParamWidth}{{\em #2}}
        \printMethodEnd
    }

    %-----------------------------------------------------------------------
    % Gibt den Typen und den Namen des ersten Methodenparameters aus.
    % Parameter #1:  Typ des Parameters.
    % Parameter #2:  Name des Parameters.
    %
    \newcommand{\printMethodFirstParam}[2]{
        \noindent
        \parbox[t]{\ParamMaxTypeWidth}{#1}
        \parbox[t]{\MaxParamWidth}{{\em #2},}\\[\VertSpace]
    }

    %-----------------------------------------------------------------------
    % Gibt den Typen und Namen des letzten Parameters einer Methode aus.
    % Parameter #1: Typ des Parameters.
    % Parameter #2: Name des Parameters.
    %
    \newcommand{\printMethodLastParam}[2]{
        \parbox[t]{\MethodBeginWidth}{\hfill}
        \hspace{\CorrectWidthThree}
        \parbox[t]{\ParamMaxTypeWidth}{#1}
        \parbox[t]{\MaxParamWidth}{{\em #2}}
        \printMethodEnd
    }

    %-----------------------------------------------------------------------
    % Gibt den Typen und Namen eines Methodenparameters aus.
    % Parameter #1:  Typ des Parameters.
    % Parameter #2: Name des Parameters.
    % Achtung: Nicht fuer den ersten oder letzten Parameter der Methode
    % geeignet!
    %
    \newcommand{\printMethodParam}[2]{
        \parbox[t]{\MethodBeginWidth}{\hfill}
        \hspace{\CorrectWidthThree}
        \parbox[t]{\ParamMaxTypeWidth}{#1}
        \parbox[t]{\MaxParamWidth}{{\em #2},}\\[\VertSpace]
    }

    %-----------------------------------------------------------------------
    % Gibt die Beschreibung einer Methode aus.
    % Parameter #1: Beschreibung der Methode.
    %
    \newcommand{\printMethodDescr}[1]{
        \newline
        \noindent
        #1\\[\VertSpace]
    }

    %-----------------------------------------------------------------------
    % Gibt den Text ``Parameter:    `` und danach den Namen des
    % ersten Parameters der Methode, gefolgt von seiner Beschreibung
    % aus.
    % Parameter #1:  Name des Parameters. 
    % Parameter #2:  Beschreibung des Parameters.
    %
    \newcommand{\printFirstParam}[2]{
        \setlength{\Temp}{\ParamDescrWidth}
        \addtolength{\Temp}{1\CorrectWidthOne}
        \parbox[t]{\SpecialWidth}{{\bf Parameter:}}
        \parbox[t]{\MaxParamWidth}{\em #1}
        \parbox[t]{\ParamDelimWidth}{\ParamDelim}
        \parbox[t]{\Temp}{#2}\\[\VertSpace]
    }

    %-----------------------------------------------------------------------
    % Gibt einen Parameter der Methode und die dazugehoerige Beschreibung
    % aus.
    % Parameter #1:  Name des Parameters.
    % Parameter #2:  Beschreibung des Parameters.
    %
    \newcommand{\printParam}[2]{
        \setlength{\Temp}{\ParamDescrWidth}
        \addtolength{\Temp}{1\CorrectWidthOne}
        \parbox[t]{\SpecialWidth}{\hfill}
        \parbox[t]{\MaxParamWidth}{\em #1}
        \parbox[t]{\ParamDelimWidth}{\ParamDelim}
        \parbox[t]{\Temp}{#2}\\[\VertSpace]
    }

    %-----------------------------------------------------------------------
    %
    % Gibt den Text ``Rueckgabewert:  `` gefolgt von einer Beschreibung
    % des Rueckgabewertes aus.
    % Parameter #1:  Beschreibung des Rueckgabewertes.
    %
    \newcommand{\printReturn}[1]{
        \setlength{\Temp}{\SpecialRetTxtWidth}
        \addtolength{\Temp}{1\CorrectWidthTwo}
        \parbox[t]{\SpecialWidth}{{\bf R\"uckgabewert:}}
        \parbox[t]{\Temp}{#1}\\[\VertSpace]
    }

    %-----------------------------------------------------------------------
    % Gibt den Text ``Besonderheiten:  `` gefolgt von einer Beschreibung
    % der Besonderheiten aus.
    % Parameter #1:  Beschreibung der Besonderheiten.
    %
    \newcommand{\printSpecial}[1]{
        \setlength{\Temp}{\SpecialRetTxtWidth}
        \addtolength{\Temp}{1\CorrectWidthTwo}
        \parbox[t]{\SpecialWidth}{\MaxLabelTxt}
        \parbox[t]{\Temp}{#1}\\[\VertSpace]
    }

    %-----------------------------------------------------------------------
    % Gibt den Text ``Parameter:        Keine.'' aus.
    %
    \newcommand{\printNoParams}{
        \parbox[t]{\SpecialWidth}{{\bf Parameter:}}
        \parbox[t]{\NoneWidthOne}{Keine.}\\[\VertSpace]
    }

    %-----------------------------------------------------------------------
    % Gibt den Text ``Rueckgabewert:  Keiner.'' aus.
    %
    \newcommand{\printNoReturn}{
        \parbox[t]{\SpecialWidth}{{\bf R\"uckgabewert:}}
        \parbox[t]{\NoneWidthTwo}{Keiner.}\\[\VertSpace]
    }

    %-----------------------------------------------------------------------
    % Gibt den Text ``Besonderheiten:  Keine.'' aus.
    %
    \newcommand{\printNoSpecial}{
        \parbox[t]{\SpecialWidth}{\MaxLabelTxt}
        \parbox[t]{\NoneWidthOne}{Keine.}\\[\VertSpace]
    }

    %-----------------------------------------------------------------------
    % Gibt den Text:
    % ``Parameter:       Keine.
    %   Rueckgabewert:   Keiner.
    %   Besonderheiten:  Keine.''
    % aus.
    %
    \newcommand{\printNoAll}{
        \printNoParams
        \printNoReturn
        \printNoSpecial
    }

    %-----------------------------------------------------------------------
    % Gibt eine Methode ohne Parameter, Rueckgabewert und Besonderheiten
    % aus.
    % Parameter #1:  Name der Methode.
    % Parameter #2:  Beschreibung der Methode.
    %
    \newcommand{\printEmptyMethod}[2]{
        \setSpecialRetTxtWidth
        \printMethodBegin{void}{#1}
        \printMethodEnd
        \printMethodDescr{#2}
        \printNoAll
    }

    %-----------------------------------------------------------------------
    % Gibt eine Methode ohne Parameter und Besonderheiten, aber mit
    % Rueckgabewert aus.
    % Parameter #1:  Rueckgabewert der Methode.
    % Parameter #2:  Name der Methode.
    % Parameter #3:  Beschreibung der Methode.
    % Parameter #4:  Beschreibung des Rueckgabewertes der Methode.
    %
    \newcommand{\printEmptyMethodReturn}[4]{
        \setSpecialRetTxtWidth
        \printMethodBegin{#1}{#2}
        \printMethodEnd
        \printMethodDescr{#3}
        \printNoParams
        \printReturn{#4}
        \printNoSpecial
    }

    %-----------------------------------------------------------------------
    % Gibt eine Methode ohne Parameter und Rueckgabewert, aber mit
    % Besonderheiten aus.
    % Parameter #1:  Name der Methode.
    % Parameter #2:  Beschreibung der Methode.
    % Parameter #3:  Beschreibung der Besonderheiten der Methode.
    %
    \newcommand{\printEmptyMethodSpecial}[3]{
        \setSpecialRetTxtWidth
        \printMethodBegin{void}{#1}
        \printMethodEnd
        \printMethodDescr{#2}
        \printNoParams
        \printNoReturn
        \printSpecial{#3}
    }

    %-----------------------------------------------------------------------
    % Gibt eine Methode ohne Parameter, aber mit Rueckgabewert
    % und Besonderheiten aus.
    % Parameter #1:  Rueckgabewert der Methode.
    % Parameter #2:  Name der Methode.
    % Parameter #3:  Beschreibung der Methode.
    % Parameter #4:  Beschreibung des Rueckgabewertes der Methode.
    % Parameter #5:  Beschreibung der Besonderheiten der Methode.
    %
    \newcommand{\printEmptyMethodReturnSpecial}[5]{
        \setSpecialRetTxtWidth
        \printMethodBegin{#1}{#2}
        \printMethodEnd
        \printMethodDescr{#3}
        \printNoParams
        \printReturn{#4}
        \printSpecial{#5}
    }

    %-----------------------------------------------------------------------
    % Gibt eine Methode mit einem Parameter komplett mit allen
    % Beschreibungen aus.
    % Parameter #1:  Rueckgabewert der Methode.
    % Parameter #2:  Name der Methode.
    % Parameter #3:  Typ des einzigen Parameters der Methode.
    % Parameter #4:  Name des einzigen Parameters der Methode.
    % Parameter #5:  Beschreibung des einzigen Parameters der Methode.
    % Parameter #6:  Beschreibung der Methode.
    % Parameter #7:  Beschreibung des Rueckgabewertes der Methode.
    % Parameter #8:  Beschreibung der Besonderheiten der Methode.
    %
    \newcommand{\printMethodWithOneParam}[8]{
        \initParamMaxTypeWidth
        \initMaxParamWidth
        \setNewParamMaxTypeWidth{#3}
        \setNewMaxParamWidth{#4}
        \setSpecialRetTxtWidth
        \setParamDescrWidth
        \printMethodBegin{#1}{#2}
        \printMethodOneParam{#3}{#4}
        \printMethodDescr{#6}
        \printFirstParam{#4}{#5}
        \ifthenelse{\equal{#7}{Keiner.}}
            {\printNoReturn}
            {\printReturn{#7}}
        \ifthenelse{\equal{#8}{Keine.}}
            {\printNoSpecial}
            {\printSpecial{#8}}
    }

    %-----------------------------------------------------------------------
    % Gibt eine Methode mit einem Parameter komplett mit allen
    % Beschreibungen aus.
    % Parameter #1:  Rueckgabewert der Methode.
    % Parameter #2:  Beschreibung des Rueckgabewertes der Methode oder 
    %                Aufruf mit leerer Klammer. 
    % Parameter #3:  Name der Methode.
    % Parameter #4:  Beschreibung der Methode.
    % Parameter #5:  Beschreibung der Besonderheiten der Methode oder 
    %                Aufruf mit leerer Klammer.
    %
    \newcommand{\printMethodWithParamsSaved}[5]{
        \ifthenelse{\value{SavedParams} > 0}
            {\initParamMaxTypeWidth
             \initMaxParamWidth
             \setNewParamMaxTypeWidth{\ParamTypeOne}
             \setNewMaxParamWidth{\ParamOne}
             \ifthenelse{\value{SavedParams} = 2 \or \value{SavedParams} > 2}
                 {\setNewParamMaxTypeWidth{\ParamTypeTwo}
                  \setNewMaxParamWidth{\ParamTwo}}{}   
             \ifthenelse{\value{SavedParams} = 3 \or \value{SavedParams} > 3}
                 {\setNewParamMaxTypeWidth{\ParamTypeThree}
                  \setNewMaxParamWidth{\ParamThree}}{}    
             \ifthenelse{\value{SavedParams} = 4 \or \value{SavedParams} > 4}
                 {\setNewParamMaxTypeWidth{\ParamTypeFour}
                  \setNewMaxParamWidth{\ParamFour}}{}
             \ifthenelse{\value{SavedParams} = 5 \or \value{SavedParams} > 5}
                 {\setNewParamMaxTypeWidth{\ParamTypeFive}
                  \setNewMaxParamWidth{\ParamFive}}{}      
             \ifthenelse{\value{SavedParams} = 6 \or \value{SavedParams} > 6}
                 {\setNewParamMaxTypeWidth{\ParamTypeSix}
                  \setNewMaxParamWidth{\ParamSix}}{}
             \ifthenelse{\value{SavedParams} = 7 \or \value{SavedParams} > 7}
                 {\setNewParamMaxTypeWidth{\ParamTypeSeven}
                  \setNewMaxParamWidth{\ParamSeven}}{}       
             \ifthenelse{\value{SavedParams} = 8 \or \value{SavedParams} > 8}
                 {\setNewParamMaxTypeWidth{\ParamTypeEight}
                  \setNewMaxParamWidth{\ParamEight}}{}
             \ifthenelse{\value{SavedParams} = 9 \or \value{SavedParams} > 9}
                 {\setNewParamMaxTypeWidth{\ParamTypeNine}
                  \setNewMaxParamWidth{\ParamNine}}{}
             \ifthenelse{\value{SavedParams} = 10 \or 
                         \value{SavedParams} > 10}
                 {\setNewParamMaxTypeWidth{\ParamTypeTen}
                  \setNewMaxParamWidth{\ParamTen}}{}         
             \setSpecialRetTxtWidth
             \setParamDescrWidth
             \printMethodBegin{#1}{#3}
             \ifthenelse{\value{SavedParams} = 1} 
                 {\printMethodOneParam{\ParamTypeOne}{\ParamOne}}{}
                 {\printMethodFirstParam{\ParamTypeOne}{\ParamOne}}{}
             \ifthenelse{\value{SavedParams} = 2} 
                 {\printMethodLastParam{\ParamTypeTwo}{\ParamTwo}}{}
             \ifthenelse{\value{SavedParams} > 2}  
                 {\printMethodParam{\ParamTypeTwo}{\ParamTwo}}{} 
             \ifthenelse{\value{SavedParams} = 3} 
                 {\printMethodLastParam{\ParamTypeThree}{\ParamThree}}{}
             \ifthenelse{\value{SavedParams} > 3}  
                 {\printMethodParam{\ParamTypeThree}{\ParamThree}}{} 
             \ifthenelse{\value{SavedParams} = 4} 
                 {\printMethodLastParam{\ParamTypeFour}{\ParamFour}}{}
             \ifthenelse{\value{SavedParams} > 4}  
                 {\printMethodParam{\ParamTypeFour}{\ParamFour}}{} 
             \ifthenelse{\value{SavedParams} = 5} 
                 {\printMethodLastParam{\ParamTypeFive}{\ParamFive}}{}
             \ifthenelse{\value{SavedParams} > 5}  
                 {\printMethodParam{\ParamTypeFive}{\ParamFive}}{} 
             \ifthenelse{\value{SavedParams} = 6} 
                 {\printMethodLastParam{\ParamTypeSix}{\ParamSix}}{}
             \ifthenelse{\value{SavedParams} > 6}  
                 {\printMethodParam{\ParamTypeSix}{\ParamSix}}{} 
             \ifthenelse{\value{SavedParams} = 7} 
                 {\printMethodLastParam{\ParamTypeSeven}{\ParamSeven}}{}
             \ifthenelse{\value{SavedParams} > 7}  
                 {\printMethodParam{\ParamTypeSeven}{\ParamSeven}}{} 
             \ifthenelse{\value{SavedParams} = 8} 
                 {\printMethodLastParam{\ParamTypeEight}{\ParamEight}}{}
             \ifthenelse{\value{SavedParams} > 8}  
                 {\printMethodParam{\ParamTypeEight}{\ParamEight}}{}  
             \ifthenelse{\value{SavedParams} = 9} 
                 {\printMethodLastParam{\ParamTypeNine}{\ParamNine}}{}
             \ifthenelse{\value{SavedParams} > 9}  
                 {\printMethodParam{\ParamTypeNine}{\ParamNine}}{} 
             \ifthenelse{\value{SavedParams} = 10} 
                 {\printMethodLastParam{\ParamTypeTen}{\ParamTen}}{}
             \printMethodDescr{#4}
             \printFirstParam{\ParamOne}{\ParamDescrOne}
             \ifthenelse{\value{SavedParams} = 2 \or \value{SavedParams} > 2}
                  {\printParam{\ParamTwo}{\ParamDescrTwo}}{}
             \ifthenelse{\value{SavedParams} = 3 \or \value{SavedParams} > 3}
                  {\printParam{\ParamThree}{\ParamDescrThree}}{}
             \ifthenelse{\value{SavedParams} = 4 \or \value{SavedParams} > 4}
                  {\printParam{\ParamFour}{\ParamDescrFour}}{}
             \ifthenelse{\value{SavedParams} = 5 \or \value{SavedParams} > 5}
                  {\printParam{\ParamFive}{\ParamDescrFive}}{}
             \ifthenelse{\value{SavedParams} = 6 \or \value{SavedParams} > 6}
                  {\printParam{\ParamSix}{\ParamDescrSix}}{}
             \ifthenelse{\value{SavedParams} = 7 \or \value{SavedParams} > 7}
                  {\printParam{\ParamSeven}{\ParamDescrSeven}}{}
             \ifthenelse{\value{SavedParams} = 8 \or \value{SavedParams} > 8}
                  {\printParam{\ParamEight}{\ParamDescrEight}}{} 
             \ifthenelse{\value{SavedParams} = 9 \or \value{SavedParams} > 9}
                  {\printParam{\ParamNine}{\ParamDescrNine}}{}
             \ifthenelse{\value{SavedParams} = 10}
                  {\printParam{\ParamTen}{\ParamDescrTen}}{}
             \ifthenelse{\equal{#2}{}}
                 {\printNoReturn}
                 {\printReturn{#2}}
             \ifthenelse{\equal{#5}{}}
                 {\printNoSpecial}
                 {\printSpecial{#5}}}
        {\noindent
         Fehler Methode {\em printMethodWithParamsSaved}: Es sind noch keine 
         Parameter gesetzt worden!}
    }

%-----------------------------------------------------------------------
% Ende der Befehlsdefinitionen.
%-----------------------------------------------------------------------





%%% Local Variables: 
%%% mode: latex
%%% TeX-master: t
%%% End: 
'' in LaTex-Dokumente 
% (Latex2Epsilon) eingebunden und stellt Befehle zur einfachen Doku- 
% mentation von C-Funktionen/C++-Methoden bereit.
% =======================================================================

\usepackage{ifthen}

    % Definition von Konstanten:
    \newcommand{\ParamDelim}            % Trennungstext zwischen
        { \ - }                         % Parametername und -beschreibung.
    \newcommand{\NormMethodEnd}         % Endtext von Methode bei 
        {\ \ );}                        % veraenderbarer Aufrufinstanz.
    \newcommand{\ConstMethodEnd}        % Endtext von Methode bei
        {\ \ ) const;}                  % nicht veraenderbarer Aufrufinstanz.
    \newcommand{\MaxLabelTxt}           % Komponentenbeschreibungstext
        {{\bf Besonderheiten: }}        % mit maximaler Breite.
    \newcommand{\MethodBegin}           % Beginntext einer Methode.
        {(\ }
    \newcommand{\InBetween}             % Text zwischen Rueckgabewert und
        {\ \ }                          % Methodenname.

    % Definition eigener Variablen:


    \newboolean{ConstInstance}          % Gibt an, ob die Instanz, die
                                        % die Methode aufruft veraendert
                                        % werden darf oder nicht. Im
                                        % 2. Fall endet die Methode mit
                                        % ``) const;'', im 1. Fall 
                                        % mit ``);''.
    \newcounter{SavedParams}            % Anzahl der gespeicherten
                                        % Parameter.
    \setcounter{SavedParams}{0}
    \newcommand{\ParamOne}{}            % Definition der 10 
    \newcommand{\ParamTypeOne}{}        % abspeicherbaren Parameter,
    \newcommand{\ParamDescrOne}{}       % ihrer Typen und Beschreibungen.
    \newcommand{\ParamTwo}{}            
    \newcommand{\ParamTypeTwo}{}
    \newcommand{\ParamDescrTwo}{}
    \newcommand{\ParamThree}{}
    \newcommand{\ParamTypeThree}{}
    \newcommand{\ParamDescrThree}{}
    \newcommand{\ParamFour}{}
    \newcommand{\ParamTypeFour}{}
    \newcommand{\ParamDescrFour}{}
    \newcommand{\ParamFive}{}
    \newcommand{\ParamTypeFive}{}
    \newcommand{\ParamDescrFive}{}
    \newcommand{\ParamSix}{}
    \newcommand{\ParamTypeSix}{}
    \newcommand{\ParamDescrSix}{}
    \newcommand{\ParamSeven}{}
    \newcommand{\ParamTypeSeven}{}
    \newcommand{\ParamDescrSeven}{}
    \newcommand{\ParamEight}{}
    \newcommand{\ParamTypeEight}{}
    \newcommand{\ParamDescrEight}{}
    \newcommand{\ParamNine}{}
    \newcommand{\ParamTypeNine}{}
    \newcommand{\ParamDescrNine}{}
    \newcommand{\ParamTen}{}
    \newcommand{\ParamTypeTen}{}
    \newcommand{\ParamDescrTen}{}
    \newlength{\MaxParamWidth}          % Breite des breitesten 
                                        % Methodenparameters.
    \newlength{\SpecialWidth}           % Breite des Textes 
    \settowidth{\SpecialWidth}          % {\bf Besonderheiten}.
        {\MaxLabelTxt}
    \newlength{\ParamDelimWidth}        % Breite des Textes `` \ - ``.
    \settowidth{\ParamDelimWidth}
        {\ParamDelim}
    \newlength{\ParamDescrWidth}        % Max. Breite des Parametertextes.
    \newlength{\SpecialRetTxtWidth}     % Max. Breite der Texte fuer
                                        % Besonderheiten und Rueckgabewert.
    \newlength{\MethodBeginWidth}       % Breite des Methodenanfangs, d.h.
                                        % ``Rueckgabewert Methodenname ( ``.
    \newlength{\BetweenParsWidth}       % Max. zur Verfuegung stehende
                                        % Breite zwischen den beiden Klammern
                                        % der Methode.
    
    \newlength{\MethodEndWidth}         % Breite des Textes ``  );'' bzw.
    \settowidth{\MethodEndWidth}        % `` ) const;''
        {\NormMethodEnd}
    \newlength{\ParamMaxTypeWidth}      % Breite des breitesten
                                        % Parameterwert-Typs. 
    \newlength{\NoneWidthOne}           % Breite des Textes ``Keine.''.
    \settowidth{\NoneWidthOne}{Keine.}
    \newlength{\NoneWidthTwo}           % Breite des Textes ``Keiner.''.
    \settowidth{\NoneWidthTwo}{Keiner.}
    \newlength{\Temp}                   % Temporaere Hilfsvariable. 
    \newlength{\VertSpace}              % Vertikaler Abstand zwischen 
                                        % zwei Parboxen.
   
    % Trotz der automatischen Berechnung der Textbreiten treten
    % kleine Unstimmigkeiten von wenigen Punkten bei den Texten
    % zur Beschreibung der Parameter, des Rueckgabewertes und
    % der Besonderheiten auf, die mit den folgenden Variablen
    % behoben werden koennen:

    \newlength{\CorrectWidthOne}        % Korrekturbreite fuer die
                                        % Parameterbeschreibung.
    \newlength{\CorrectWidthTwo}        % Korrekturbreite fuer die
                                        % Beschreibungen von Rueckgabewert
                                        % und Besonderheiten.
    \newlength{\CorrectWidthThree}      % Korrekturbreite, da erster
                                        % Parameter einer Methode weiter
                                        % vorrueckt als die anderen.
    

% Uebersicht ueber die Parameter und ihre Bedeutung.


%
% <-- \MethodBeginWidth    -->
%------------------------------------------------------------------------
% Rueckgabewert Methodenname(   Typ_Parameter_1 Parameter_1,
%                            <->Typ_Parameter_2 Parameter_2,
%              \CorrectWideThree...
%                               Typ_Parameter_n Parameter_n   );
%                             <-------------> <--------->
%                          \ParamMaxTypeWidth \MaxParamWidth
%                             <-------------------------->
%                             \BetweenParsWidth
%                                                         <-->
%                                                         \MethodEndWidth
%------------------------------------------------------------------------
% Beschreibung der Methode...
%                             \ParamDelimWidth
%                             <->                        \CorrectWidthOne
% \VertSpace
% Parameter:       Parameter_1  - Beschreibung von Parameter 1.   <----->
%                  Parameter_2  - Beschreibung von Parameter 2.
%                  ...
%                  Parameter_n  - Beschreibung von Parameter n.
% \VertSpace                      \ParamDescrWidth
%                                 <------------------------------>
% Rueckgabewert:   Beschreibung des Rueckgabewertes bzw. der Text <-----> 
%                  Keiner.                   
%                  <----->
%                  \NoneWidthTwo               \SpecialRetTxtWidth
% \VertSpace       <--------------------------------------------->
% Besonderheiten:  Beschreibung von Besonderheiten bzw. der Text  <----->
%                  Keine.     
% <--------------> <---->
%  \SpecialWidth   \NoneWidthOne                         \CorrectWidthTwo


    % Definition eigener Befehle:


    %-----------------------------------------------------------------------
    % Gibt den Begriff ``C++'' in ansprechender Form aus.
    % Makro wurde freundlicherweise zur Verfuegung gestellt von
    % Axel W. Dietrich.
    %
    \newcommand{\cpp}{
        \mbox{\emph{\textrm{C\hspace{-1.5pt}\raisebox{1.75pt}{\scriptsize +}
        \hspace{-6pt}\raisebox{.75pt}{\scriptsize +}}}}%
    }


    %-----------------------------------------------------------------------
    % Gibt die Werte aller Parameter auf dem Bildschirm aus.
    %
    \newcommand{\showAllParams}{
        \noindent
        {\em Werte aller Parameter:}
        \begin{tabbing}
            ParamMaxTypeWidth:\ \ \=\kill
            MaxParamWidth:       \>\the\MaxParamWidth\\
            SpecialWidth:        \>\the\SpecialWidth\\
            ParamDelimWidth:     \>\the\ParamDelimWidth\\
            ParamDescrWidth:     \>\the\ParamDescrWidth\\
            SpecialRetTxtWidth:  \>\the\SpecialRetTxtWidth\\
            MethodBeginWidth:    \>\the\MethodBeginWidth\\
            BetweenParsWidth:    \>\the\BetweenParsWidth\\
            MethodEndWidth:      \>\the\MethodEndWidth\\
            ParamMaxTypeWidth:   \>\the\ParamMaxTypeWidth\\
            NoneWidthOne:        \>\the\NoneWidthOne\\
            NoneWidthTwo:        \>\the\NoneWidthTwo\\
            VertSpace:           \>\the\VertSpace\\
            CorrectWidthOne:     \>\the\CorrectWidthOne\\
            CorrectWidthTwo:     \>\the\CorrectWidthTwo\\
            CorrectWidthThree:   \>\the\CorrectWidthThree\\
            ConstInstance:       
            \ifthenelse{\boolean{ConstInstance}}
                {\>ja\\}
                {\>nein\\}
        \end{tabbing}
    }

    %-----------------------------------------------------------------------
    % Setzt einen Schalter, der angibt, dass die Instanz, die die kommende
    % Methode aufrufen kann, nicht veraendert werden darf. Dies hat fuer
    % die Dokumentation zur Folge, dass die Methode mit ``) const;'' endet.
    %
    \newcommand{\setConstInstance}{    
        \setboolean{ConstInstance}{true}
    }

    %-----------------------------------------------------------------------
    % Setzt einen Schalter, der angibt, dass die Instanz, die die kommende
    % Methode aufrufen kann, veraendert werden darf. Die Methode endet also
    % in der Dokumentation normal mit ``);''.
    %
    \newcommand{\setNormalInstance}{    
        \setboolean{ConstInstance}{false}
    }

    %-----------------------------------------------------------------------
    % Initialisiert die Breite des breitesten Methodenparametertyptextes
    % mit Null.
    %
    \newcommand{\initParamMaxTypeWidth}{
        \setlength{\ParamMaxTypeWidth}{0pt}
    }

    %-----------------------------------------------------------------------
    % Initialisiert die Breite des breitesten Parameternamens mit Null.
    %
    \newcommand{\initMaxParamWidth}{
        \setlength{\MaxParamWidth}{0pt}
    }

    %-----------------------------------------------------------------------
    % Ueberprueft, ob die Breite des uebergebenen Textes des Parametertyps
    % groesser ist als die des bisher als breitester Typtext festgelegten
    % Textes und passt den Wert evtl. an.
    % Parameter #1:  Neuer Typtext, dessen Breite zum Vergleich dient.
    %   
    \newcommand{\setNewParamMaxTypeWidth}[1]{
        \settowidth{\Temp}{#1}
        \ifthenelse{\lengthtest{\ParamMaxTypeWidth < \Temp}}
            {\setlength{\ParamMaxTypeWidth}{\Temp}}{}   
    }

    %-----------------------------------------------------------------------
    % Ueberprueft, ob die Breite des uebergebenen Parameternamens groesser
    % ist als die des bisher als am breitesten geltenden Parameternames
    % und passt den Wert evtl. an. Der Parametername steht dabei in
    % Emphasized.
    % Parameter #1:  Neuer Parametername, dessen Breite zum Vergleich dient.
    %
    \newcommand{\setNewMaxParamWidth}[1]{
        \settowidth{\Temp}{{\em #1}}
        \ifthenelse{\lengthtest{\MaxParamWidth < \Temp}}
            {\setlength{\MaxParamWidth}{\Temp}}{}
    }

    %-----------------------------------------------------------------------
    % Setzt die Breite der Beschreibungstexte fuer Rueckgabewert
    % und Besonderheiten.
    %
    \newcommand{\setSpecialRetTxtWidth}{
        \setlength{\SpecialRetTxtWidth}{\textwidth}
        \addtolength{\SpecialRetTxtWidth}{-1\SpecialWidth} 
    }

    %-----------------------------------------------------------------------
    % Setzt die Breite fuer den Beginn einer Methode, d.h. den Text
    % ``Rueckgabewert Methodenname( ``. Rueckgabewert steht dabei
    % in Emphasized, Methodenname in Bold Font.
    % Parameter #1:  Rueckgabewert der Methode.
    % Parameter #2:  Name der Methode.
    %
    \newcommand{\setMethodBeginWidth}[2]{
        \settowidth{\MethodBeginWidth}
            {{\em #1}\InBetween{\bf #2}\MethodBegin}
    }

    %-----------------------------------------------------------------------
    % Setzt die maximale Breite die an Platz zwischen der oeffnenden
    % und der schliessenden Klammer der Methode zur Verfuegung steht.
    %
    \newcommand{\setBetweenParsWidth}{
        \setlength{\BetweenParsWidth}{\textwidth}
        \addtolength{\BetweenParsWidth}{-1\MethodBeginWidth}
        \addtolength{\BetweenParsWidth}{-1\MethodEndWidth}
    }

    %-----------------------------------------------------------------------
    % Setzt die Breite fuer die Beschreibungstexte
    % der Methodenparameter.
    %
    \newcommand{\setParamDescrWidth}{
        \setlength{\ParamDescrWidth}{\textwidth}
        \addtolength{\ParamDescrWidth}{-1\SpecialWidth} 
        \addtolength{\ParamDescrWidth}{-1\MaxParamWidth}
        \addtolength{\ParamDescrWidth}{-1\ParamDelimWidth}
    }    

    %-----------------------------------------------------------------------
    % Setzt die Korrekturbreite fuer den Beschreibungstext von
    % Methodenparametern auf den uebergebenen Wert.
    % Parameter #1:  Neuer Wert fuer die Korrekturbreite.
    %
    \newcommand{\setCorrectWidthOne}[1]{
        \setlength{\CorrectWidthOne}{#1}
    }    

    %-----------------------------------------------------------------------
    % Setzt die Korrekturbreite fuer den Beschreibungstext von
    % Rueckgabewert und Besonderheiten auf den uebergebenen Wert.
    % Parameter #1:  Neuer Wert fuer die Korrekturbreite.
    %
    \newcommand{\setCorrectWidthTwo}[1]{
        \setlength{\CorrectWidthTwo}{#1}
    }    

    %-----------------------------------------------------------------------
    % Setzt die Korrekturbreite fuer den Abstand des zweiten bis n-ten
    % Parameters vom Anfang auf den uebergebenen Wert.
    % Parameter #1:  Neuer Wert fuer die Korrekturbreite.
    %
    \newcommand{\setCorrectWidthThree}[1]{
        \setlength{\CorrectWidthThree}{#1}
    }    

    %-----------------------------------------------------------------------
    % Setzt den vertikalen Abstand zwischen zwei Parboxen auf den ueber-
    % gebenen Wert.
    % Parameter #1:  Neuer Wert fuer den vertikalen Abstand.
    %
    \newcommand{\setVertSpace}[1]{
        \setlength{\VertSpace}{#1}
    }    

    %-----------------------------------------------------------------------
    % Speichert die uebergebenen Informationen ueber Methodenparameter 1
    % in internen Variablen. Achtung: Die Parameter muessen in der richtigen
    % Reihenfolge gesetzt werden!
    % Parameter #1: Der Typ des Parameters.  
    % Parameter #2: Der Name des Parameters. 
    % Parameter #3: Die Beschreibung des Parameters. 
    % 
    \newcommand{\setParamOne}[3]{
        \renewcommand{\ParamOne}{#1}
        \renewcommand{\ParamTypeOne}{#2}
        \renewcommand{\ParamDescrOne}{#3} 
        \setcounter{SavedParams}{0}
        \stepcounter{SavedParams}
    }

    %-----------------------------------------------------------------------
    % Speichert die uebergebenen Informationen ueber Methodenparameter 2
    % in internen Variablen. Achtung: Die Parameter muessen in der richtigen
    % Reihenfolge gesetzt werden!
    % Parameter #1: Der Typ des Parameters.  
    % Parameter #2: Der Name des Parameters. 
    % Parameter #3: Die Beschreibung des Parameters. 
    % 
    \newcommand{\setParamTwo}[3]{
        \renewcommand{\ParamTwo}{#1}
        \renewcommand{\ParamTypeTwo}{#2}
        \renewcommand{\ParamDescrTwo}{#3} 
        \stepcounter{SavedParams}
    }

    %-----------------------------------------------------------------------
    % Speichert die uebergebenen Informationen ueber Methodenparameter 3
    % in internen Variablen. Achtung: Die Parameter muessen in der richtigen
    % Reihenfolge gesetzt werden!
    % Parameter #1: Der Typ des Parameters.  
    % Parameter #2: Der Name des Parameters. 
    % Parameter #3: Die Beschreibung des Parameters. 
    % 
    \newcommand{\setParamThree}[3]{
        \renewcommand{\ParamThree}{#1}
        \renewcommand{\ParamTypeThree}{#2}
        \renewcommand{\ParamDescrThree}{#3} 
        \stepcounter{SavedParams}
    }

    %-----------------------------------------------------------------------
    % Speichert die uebergebenen Informationen ueber Methodenparameter 4
    % in internen Variablen. Achtung: Die Parameter muessen in der richtigen
    % Reihenfolge gesetzt werden!
    % Parameter #1: Der Typ des Parameters.  
    % Parameter #2: Der Name des Parameters. 
    % Parameter #3: Die Beschreibung des Parameters. 
    % 
    \newcommand{\setParamFour}[3]{
        \renewcommand{\ParamFour}{#1}
        \renewcommand{\ParamTypeFour}{#2}
        \renewcommand{\ParamDescrFour}{#3} 
        \stepcounter{SavedParams}
    }

    %-----------------------------------------------------------------------
    % Speichert die uebergebenen Informationen ueber Methodenparameter 5
    % in internen Variablen. Achtung: Die Parameter muessen in der richtigen
    % Reihenfolge gesetzt werden!
    % Parameter #1: Der Typ des Parameters.  
    % Parameter #2: Der Name des Parameters. 
    % Parameter #3: Die Beschreibung des Parameters. 
    % 
    \newcommand{\setParamFive}[3]{
        \renewcommand{\ParamFive}{#1}
        \renewcommand{\ParamTypeFive}{#2}
        \renewcommand{\ParamDescrFive}{#3} 
        \stepcounter{SavedParams}
    }

    %-----------------------------------------------------------------------
    % Speichert die uebergebenen Informationen ueber Methodenparameter 6
    % in internen Variablen. Achtung: Die Parameter muessen in der richtigen
    % Reihenfolge gesetzt werden!
    % Parameter #1: Der Typ des Parameters.  
    % Parameter #2: Der Name des Parameters. 
    % Parameter #3: Die Beschreibung des Parameters. 
    %     
    \newcommand{\setParamSix}[3]{
        \renewcommand{\ParamSix}{#1}
        \renewcommand{\ParamTypeSix}{#2}
        \renewcommand{\ParamDescrSix}{#3} 
        \stepcounter{SavedParams} 
    }

    %-----------------------------------------------------------------------
    % Speichert die uebergebenen Informationen ueber Methodenparameter 7
    % in internen Variablen. Achtung: Die Parameter muessen in der richtigen
    % Reihenfolge gesetzt werden!
    % Parameter #1: Der Typ des Parameters.  
    % Parameter #2: Der Name des Parameters. 
    % Parameter #3: Die Beschreibung des Parameters. 
    %    
    \newcommand{\setParamSeven}[3]{
        \renewcommand{\ParamSeven}{#1}
        \renewcommand{\ParamTypeSeven}{#2}
        \renewcommand{\ParamDescrSeven}{#3} 
        \stepcounter{SavedParams} 
    }

    %-----------------------------------------------------------------------
    % Speichert die uebergebenen Informationen ueber Methodenparameter 8
    % in internen Variablen. Achtung: Die Parameter muessen in der richtigen
    % Reihenfolge gesetzt werden!
    % Parameter #1: Der Typ des Parameters.  
    % Parameter #2: Der Name des Parameters. 
    % Parameter #3: Die Beschreibung des Parameters. 
    % 
    \newcommand{\setParamEight}[3]{
        \renewcommand{\ParamEight}{#1}
        \renewcommand{\ParamTypeEight}{#2}
        \renewcommand{\ParamDescrEight}{#3} 
        \stepcounter{SavedParams}
    }

    %-----------------------------------------------------------------------
    % Speichert die uebergebenen Informationen ueber Methodenparameter 9
    % in internen Variablen. Achtung: Die Parameter muessen in der richtigen
    % Reihenfolge gesetzt werden!
    % Parameter #1: Der Typ des Parameters.  
    % Parameter #2: Der Name des Parameters. 
    % Parameter #3: Die Beschreibung des Parameters. 
    % 
    \newcommand{\setParamNine}[3]{
        \renewcommand{\ParamNine}{#1}
        \renewcommand{\ParamTypeNine}{#2}
        \renewcommand{\ParamDescrNine}{#3}  
        \stepcounter{SavedParams}
    }

    %-----------------------------------------------------------------------
    % Speichert die uebergebenen Informationen ueber Methodenparameter 10
    % in internen Variablen. Achtung: Die Parameter muessen in der richtigen
    % Reihenfolge gesetzt werden!
    % Parameter #1: Der Typ des Parameters.  
    % Parameter #2: Der Name des Parameters. 
    % Parameter #3: Die Beschreibung des Parameters. 
    % 
    \newcommand{\setParamTen}[3]{
        \renewcommand{\ParamTen}{#1}
        \renewcommand{\ParamTypeTen}{#2}
        \renewcommand{\ParamDescrTen}{#3} 
        \stepcounter{SavedParams}
    }

    %-----------------------------------------------------------------------
    % Gibt den Text ``Rueckgabewert Methodenname(`` aus.
    % Parameter #1:  Rueckgabewert der Methode.
    % Parameter #2:  Name der Methode.
    %
    \newcommand{\printMethodBegin}[2]{
         \ifthenelse{\boolean{ConstInstance}}
             {\settowidth{\MethodEndWidth}{\ConstMethodEnd}}
             {\settowidth{\MethodEndWidth}{\NormMethodEnd}}
         \setMethodBeginWidth{#1}{#2}
         \setBetweenParsWidth
         \noindent
         \hrulefill\\
         \noindent
         \parbox[t]{\MethodBeginWidth}
             {#1\InBetween{\bf #2}\MethodBegin}
     }

    %-----------------------------------------------------------------------
    % Gibt das Ende einer Methode aus, d.h. ``\ );'' oder ``\ ) const;''.
    %
    \newcommand{\printMethodEnd}{
        \ifthenelse{\boolean{ConstInstance}}
            {\ConstMethodEnd\\}
            {\NormMethodEnd\\}
            \mbox{}\hrulefill\\
    }

    %-----------------------------------------------------------------------
    % Gibt den Typen und Namen des einzigen Parameters einer Methode
    % gefolgt von dem Text ``  );'' bzw. `` ) const;'' aus.
    % Parameter #1:  Typ des Parameters.
    % Parameter #2:  Name des Parameters.
    % 
    \newcommand{\printMethodOneParam}[2]{
        \parbox[t]{\ParamMaxTypeWidth}{#1}
        \parbox[t]{\MaxParamWidth}{{\em #2}}
        \printMethodEnd
    }

    %-----------------------------------------------------------------------
    % Gibt den Typen und den Namen des ersten Methodenparameters aus.
    % Parameter #1:  Typ des Parameters.
    % Parameter #2:  Name des Parameters.
    %
    \newcommand{\printMethodFirstParam}[2]{
        \noindent
        \parbox[t]{\ParamMaxTypeWidth}{#1}
        \parbox[t]{\MaxParamWidth}{{\em #2},}\\[\VertSpace]
    }

    %-----------------------------------------------------------------------
    % Gibt den Typen und Namen des letzten Parameters einer Methode aus.
    % Parameter #1: Typ des Parameters.
    % Parameter #2: Name des Parameters.
    %
    \newcommand{\printMethodLastParam}[2]{
        \parbox[t]{\MethodBeginWidth}{\hfill}
        \hspace{\CorrectWidthThree}
        \parbox[t]{\ParamMaxTypeWidth}{#1}
        \parbox[t]{\MaxParamWidth}{{\em #2}}
        \printMethodEnd
    }

    %-----------------------------------------------------------------------
    % Gibt den Typen und Namen eines Methodenparameters aus.
    % Parameter #1:  Typ des Parameters.
    % Parameter #2: Name des Parameters.
    % Achtung: Nicht fuer den ersten oder letzten Parameter der Methode
    % geeignet!
    %
    \newcommand{\printMethodParam}[2]{
        \parbox[t]{\MethodBeginWidth}{\hfill}
        \hspace{\CorrectWidthThree}
        \parbox[t]{\ParamMaxTypeWidth}{#1}
        \parbox[t]{\MaxParamWidth}{{\em #2},}\\[\VertSpace]
    }

    %-----------------------------------------------------------------------
    % Gibt die Beschreibung einer Methode aus.
    % Parameter #1: Beschreibung der Methode.
    %
    \newcommand{\printMethodDescr}[1]{
        \newline
        \noindent
        #1\\[\VertSpace]
    }

    %-----------------------------------------------------------------------
    % Gibt den Text ``Parameter:    `` und danach den Namen des
    % ersten Parameters der Methode, gefolgt von seiner Beschreibung
    % aus.
    % Parameter #1:  Name des Parameters. 
    % Parameter #2:  Beschreibung des Parameters.
    %
    \newcommand{\printFirstParam}[2]{
        \setlength{\Temp}{\ParamDescrWidth}
        \addtolength{\Temp}{1\CorrectWidthOne}
        \parbox[t]{\SpecialWidth}{{\bf Parameter:}}
        \parbox[t]{\MaxParamWidth}{\em #1}
        \parbox[t]{\ParamDelimWidth}{\ParamDelim}
        \parbox[t]{\Temp}{#2}\\[\VertSpace]
    }

    %-----------------------------------------------------------------------
    % Gibt einen Parameter der Methode und die dazugehoerige Beschreibung
    % aus.
    % Parameter #1:  Name des Parameters.
    % Parameter #2:  Beschreibung des Parameters.
    %
    \newcommand{\printParam}[2]{
        \setlength{\Temp}{\ParamDescrWidth}
        \addtolength{\Temp}{1\CorrectWidthOne}
        \parbox[t]{\SpecialWidth}{\hfill}
        \parbox[t]{\MaxParamWidth}{\em #1}
        \parbox[t]{\ParamDelimWidth}{\ParamDelim}
        \parbox[t]{\Temp}{#2}\\[\VertSpace]
    }

    %-----------------------------------------------------------------------
    %
    % Gibt den Text ``Rueckgabewert:  `` gefolgt von einer Beschreibung
    % des Rueckgabewertes aus.
    % Parameter #1:  Beschreibung des Rueckgabewertes.
    %
    \newcommand{\printReturn}[1]{
        \setlength{\Temp}{\SpecialRetTxtWidth}
        \addtolength{\Temp}{1\CorrectWidthTwo}
        \parbox[t]{\SpecialWidth}{{\bf R\"uckgabewert:}}
        \parbox[t]{\Temp}{#1}\\[\VertSpace]
    }

    %-----------------------------------------------------------------------
    % Gibt den Text ``Besonderheiten:  `` gefolgt von einer Beschreibung
    % der Besonderheiten aus.
    % Parameter #1:  Beschreibung der Besonderheiten.
    %
    \newcommand{\printSpecial}[1]{
        \setlength{\Temp}{\SpecialRetTxtWidth}
        \addtolength{\Temp}{1\CorrectWidthTwo}
        \parbox[t]{\SpecialWidth}{\MaxLabelTxt}
        \parbox[t]{\Temp}{#1}\\[\VertSpace]
    }

    %-----------------------------------------------------------------------
    % Gibt den Text ``Parameter:        Keine.'' aus.
    %
    \newcommand{\printNoParams}{
        \parbox[t]{\SpecialWidth}{{\bf Parameter:}}
        \parbox[t]{\NoneWidthOne}{Keine.}\\[\VertSpace]
    }

    %-----------------------------------------------------------------------
    % Gibt den Text ``Rueckgabewert:  Keiner.'' aus.
    %
    \newcommand{\printNoReturn}{
        \parbox[t]{\SpecialWidth}{{\bf R\"uckgabewert:}}
        \parbox[t]{\NoneWidthTwo}{Keiner.}\\[\VertSpace]
    }

    %-----------------------------------------------------------------------
    % Gibt den Text ``Besonderheiten:  Keine.'' aus.
    %
    \newcommand{\printNoSpecial}{
        \parbox[t]{\SpecialWidth}{\MaxLabelTxt}
        \parbox[t]{\NoneWidthOne}{Keine.}\\[\VertSpace]
    }

    %-----------------------------------------------------------------------
    % Gibt den Text:
    % ``Parameter:       Keine.
    %   Rueckgabewert:   Keiner.
    %   Besonderheiten:  Keine.''
    % aus.
    %
    \newcommand{\printNoAll}{
        \printNoParams
        \printNoReturn
        \printNoSpecial
    }

    %-----------------------------------------------------------------------
    % Gibt eine Methode ohne Parameter, Rueckgabewert und Besonderheiten
    % aus.
    % Parameter #1:  Name der Methode.
    % Parameter #2:  Beschreibung der Methode.
    %
    \newcommand{\printEmptyMethod}[2]{
        \setSpecialRetTxtWidth
        \printMethodBegin{void}{#1}
        \printMethodEnd
        \printMethodDescr{#2}
        \printNoAll
    }

    %-----------------------------------------------------------------------
    % Gibt eine Methode ohne Parameter und Besonderheiten, aber mit
    % Rueckgabewert aus.
    % Parameter #1:  Rueckgabewert der Methode.
    % Parameter #2:  Name der Methode.
    % Parameter #3:  Beschreibung der Methode.
    % Parameter #4:  Beschreibung des Rueckgabewertes der Methode.
    %
    \newcommand{\printEmptyMethodReturn}[4]{
        \setSpecialRetTxtWidth
        \printMethodBegin{#1}{#2}
        \printMethodEnd
        \printMethodDescr{#3}
        \printNoParams
        \printReturn{#4}
        \printNoSpecial
    }

    %-----------------------------------------------------------------------
    % Gibt eine Methode ohne Parameter und Rueckgabewert, aber mit
    % Besonderheiten aus.
    % Parameter #1:  Name der Methode.
    % Parameter #2:  Beschreibung der Methode.
    % Parameter #3:  Beschreibung der Besonderheiten der Methode.
    %
    \newcommand{\printEmptyMethodSpecial}[3]{
        \setSpecialRetTxtWidth
        \printMethodBegin{void}{#1}
        \printMethodEnd
        \printMethodDescr{#2}
        \printNoParams
        \printNoReturn
        \printSpecial{#3}
    }

    %-----------------------------------------------------------------------
    % Gibt eine Methode ohne Parameter, aber mit Rueckgabewert
    % und Besonderheiten aus.
    % Parameter #1:  Rueckgabewert der Methode.
    % Parameter #2:  Name der Methode.
    % Parameter #3:  Beschreibung der Methode.
    % Parameter #4:  Beschreibung des Rueckgabewertes der Methode.
    % Parameter #5:  Beschreibung der Besonderheiten der Methode.
    %
    \newcommand{\printEmptyMethodReturnSpecial}[5]{
        \setSpecialRetTxtWidth
        \printMethodBegin{#1}{#2}
        \printMethodEnd
        \printMethodDescr{#3}
        \printNoParams
        \printReturn{#4}
        \printSpecial{#5}
    }

    %-----------------------------------------------------------------------
    % Gibt eine Methode mit einem Parameter komplett mit allen
    % Beschreibungen aus.
    % Parameter #1:  Rueckgabewert der Methode.
    % Parameter #2:  Name der Methode.
    % Parameter #3:  Typ des einzigen Parameters der Methode.
    % Parameter #4:  Name des einzigen Parameters der Methode.
    % Parameter #5:  Beschreibung des einzigen Parameters der Methode.
    % Parameter #6:  Beschreibung der Methode.
    % Parameter #7:  Beschreibung des Rueckgabewertes der Methode.
    % Parameter #8:  Beschreibung der Besonderheiten der Methode.
    %
    \newcommand{\printMethodWithOneParam}[8]{
        \initParamMaxTypeWidth
        \initMaxParamWidth
        \setNewParamMaxTypeWidth{#3}
        \setNewMaxParamWidth{#4}
        \setSpecialRetTxtWidth
        \setParamDescrWidth
        \printMethodBegin{#1}{#2}
        \printMethodOneParam{#3}{#4}
        \printMethodDescr{#6}
        \printFirstParam{#4}{#5}
        \ifthenelse{\equal{#7}{Keiner.}}
            {\printNoReturn}
            {\printReturn{#7}}
        \ifthenelse{\equal{#8}{Keine.}}
            {\printNoSpecial}
            {\printSpecial{#8}}
    }

    %-----------------------------------------------------------------------
    % Gibt eine Methode mit einem Parameter komplett mit allen
    % Beschreibungen aus.
    % Parameter #1:  Rueckgabewert der Methode.
    % Parameter #2:  Beschreibung des Rueckgabewertes der Methode oder 
    %                Aufruf mit leerer Klammer. 
    % Parameter #3:  Name der Methode.
    % Parameter #4:  Beschreibung der Methode.
    % Parameter #5:  Beschreibung der Besonderheiten der Methode oder 
    %                Aufruf mit leerer Klammer.
    %
    \newcommand{\printMethodWithParamsSaved}[5]{
        \ifthenelse{\value{SavedParams} > 0}
            {\initParamMaxTypeWidth
             \initMaxParamWidth
             \setNewParamMaxTypeWidth{\ParamTypeOne}
             \setNewMaxParamWidth{\ParamOne}
             \ifthenelse{\value{SavedParams} = 2 \or \value{SavedParams} > 2}
                 {\setNewParamMaxTypeWidth{\ParamTypeTwo}
                  \setNewMaxParamWidth{\ParamTwo}}{}   
             \ifthenelse{\value{SavedParams} = 3 \or \value{SavedParams} > 3}
                 {\setNewParamMaxTypeWidth{\ParamTypeThree}
                  \setNewMaxParamWidth{\ParamThree}}{}    
             \ifthenelse{\value{SavedParams} = 4 \or \value{SavedParams} > 4}
                 {\setNewParamMaxTypeWidth{\ParamTypeFour}
                  \setNewMaxParamWidth{\ParamFour}}{}
             \ifthenelse{\value{SavedParams} = 5 \or \value{SavedParams} > 5}
                 {\setNewParamMaxTypeWidth{\ParamTypeFive}
                  \setNewMaxParamWidth{\ParamFive}}{}      
             \ifthenelse{\value{SavedParams} = 6 \or \value{SavedParams} > 6}
                 {\setNewParamMaxTypeWidth{\ParamTypeSix}
                  \setNewMaxParamWidth{\ParamSix}}{}
             \ifthenelse{\value{SavedParams} = 7 \or \value{SavedParams} > 7}
                 {\setNewParamMaxTypeWidth{\ParamTypeSeven}
                  \setNewMaxParamWidth{\ParamSeven}}{}       
             \ifthenelse{\value{SavedParams} = 8 \or \value{SavedParams} > 8}
                 {\setNewParamMaxTypeWidth{\ParamTypeEight}
                  \setNewMaxParamWidth{\ParamEight}}{}
             \ifthenelse{\value{SavedParams} = 9 \or \value{SavedParams} > 9}
                 {\setNewParamMaxTypeWidth{\ParamTypeNine}
                  \setNewMaxParamWidth{\ParamNine}}{}
             \ifthenelse{\value{SavedParams} = 10 \or 
                         \value{SavedParams} > 10}
                 {\setNewParamMaxTypeWidth{\ParamTypeTen}
                  \setNewMaxParamWidth{\ParamTen}}{}         
             \setSpecialRetTxtWidth
             \setParamDescrWidth
             \printMethodBegin{#1}{#3}
             \ifthenelse{\value{SavedParams} = 1} 
                 {\printMethodOneParam{\ParamTypeOne}{\ParamOne}}{}
                 {\printMethodFirstParam{\ParamTypeOne}{\ParamOne}}{}
             \ifthenelse{\value{SavedParams} = 2} 
                 {\printMethodLastParam{\ParamTypeTwo}{\ParamTwo}}{}
             \ifthenelse{\value{SavedParams} > 2}  
                 {\printMethodParam{\ParamTypeTwo}{\ParamTwo}}{} 
             \ifthenelse{\value{SavedParams} = 3} 
                 {\printMethodLastParam{\ParamTypeThree}{\ParamThree}}{}
             \ifthenelse{\value{SavedParams} > 3}  
                 {\printMethodParam{\ParamTypeThree}{\ParamThree}}{} 
             \ifthenelse{\value{SavedParams} = 4} 
                 {\printMethodLastParam{\ParamTypeFour}{\ParamFour}}{}
             \ifthenelse{\value{SavedParams} > 4}  
                 {\printMethodParam{\ParamTypeFour}{\ParamFour}}{} 
             \ifthenelse{\value{SavedParams} = 5} 
                 {\printMethodLastParam{\ParamTypeFive}{\ParamFive}}{}
             \ifthenelse{\value{SavedParams} > 5}  
                 {\printMethodParam{\ParamTypeFive}{\ParamFive}}{} 
             \ifthenelse{\value{SavedParams} = 6} 
                 {\printMethodLastParam{\ParamTypeSix}{\ParamSix}}{}
             \ifthenelse{\value{SavedParams} > 6}  
                 {\printMethodParam{\ParamTypeSix}{\ParamSix}}{} 
             \ifthenelse{\value{SavedParams} = 7} 
                 {\printMethodLastParam{\ParamTypeSeven}{\ParamSeven}}{}
             \ifthenelse{\value{SavedParams} > 7}  
                 {\printMethodParam{\ParamTypeSeven}{\ParamSeven}}{} 
             \ifthenelse{\value{SavedParams} = 8} 
                 {\printMethodLastParam{\ParamTypeEight}{\ParamEight}}{}
             \ifthenelse{\value{SavedParams} > 8}  
                 {\printMethodParam{\ParamTypeEight}{\ParamEight}}{}  
             \ifthenelse{\value{SavedParams} = 9} 
                 {\printMethodLastParam{\ParamTypeNine}{\ParamNine}}{}
             \ifthenelse{\value{SavedParams} > 9}  
                 {\printMethodParam{\ParamTypeNine}{\ParamNine}}{} 
             \ifthenelse{\value{SavedParams} = 10} 
                 {\printMethodLastParam{\ParamTypeTen}{\ParamTen}}{}
             \printMethodDescr{#4}
             \printFirstParam{\ParamOne}{\ParamDescrOne}
             \ifthenelse{\value{SavedParams} = 2 \or \value{SavedParams} > 2}
                  {\printParam{\ParamTwo}{\ParamDescrTwo}}{}
             \ifthenelse{\value{SavedParams} = 3 \or \value{SavedParams} > 3}
                  {\printParam{\ParamThree}{\ParamDescrThree}}{}
             \ifthenelse{\value{SavedParams} = 4 \or \value{SavedParams} > 4}
                  {\printParam{\ParamFour}{\ParamDescrFour}}{}
             \ifthenelse{\value{SavedParams} = 5 \or \value{SavedParams} > 5}
                  {\printParam{\ParamFive}{\ParamDescrFive}}{}
             \ifthenelse{\value{SavedParams} = 6 \or \value{SavedParams} > 6}
                  {\printParam{\ParamSix}{\ParamDescrSix}}{}
             \ifthenelse{\value{SavedParams} = 7 \or \value{SavedParams} > 7}
                  {\printParam{\ParamSeven}{\ParamDescrSeven}}{}
             \ifthenelse{\value{SavedParams} = 8 \or \value{SavedParams} > 8}
                  {\printParam{\ParamEight}{\ParamDescrEight}}{} 
             \ifthenelse{\value{SavedParams} = 9 \or \value{SavedParams} > 9}
                  {\printParam{\ParamNine}{\ParamDescrNine}}{}
             \ifthenelse{\value{SavedParams} = 10}
                  {\printParam{\ParamTen}{\ParamDescrTen}}{}
             \ifthenelse{\equal{#2}{}}
                 {\printNoReturn}
                 {\printReturn{#2}}
             \ifthenelse{\equal{#5}{}}
                 {\printNoSpecial}
                 {\printSpecial{#5}}}
        {\noindent
         Fehler Methode {\em printMethodWithParamsSaved}: Es sind noch keine 
         Parameter gesetzt worden!}
    }

%-----------------------------------------------------------------------
% Ende der Befehlsdefinitionen.
%-----------------------------------------------------------------------





%%% Local Variables: 
%%% mode: latex
%%% TeX-master: t
%%% End: 
'' in LaTex-Dokumente 
% (Latex2Epsilon) eingebunden und stellt Befehle zur einfachen Doku- 
% mentation von C-Funktionen/C++-Methoden bereit.
% =======================================================================

\usepackage{ifthen}

    % Definition von Konstanten:
    \newcommand{\ParamDelim}            % Trennungstext zwischen
        { \ - }                         % Parametername und -beschreibung.
    \newcommand{\NormMethodEnd}         % Endtext von Methode bei 
        {\ \ );}                        % veraenderbarer Aufrufinstanz.
    \newcommand{\ConstMethodEnd}        % Endtext von Methode bei
        {\ \ ) const;}                  % nicht veraenderbarer Aufrufinstanz.
    \newcommand{\MaxLabelTxt}           % Komponentenbeschreibungstext
        {{\bf Besonderheiten: }}        % mit maximaler Breite.
    \newcommand{\MethodBegin}           % Beginntext einer Methode.
        {(\ }
    \newcommand{\InBetween}             % Text zwischen Rueckgabewert und
        {\ \ }                          % Methodenname.

    % Definition eigener Variablen:


    \newboolean{ConstInstance}          % Gibt an, ob die Instanz, die
                                        % die Methode aufruft veraendert
                                        % werden darf oder nicht. Im
                                        % 2. Fall endet die Methode mit
                                        % ``) const;'', im 1. Fall 
                                        % mit ``);''.
    \newcounter{SavedParams}            % Anzahl der gespeicherten
                                        % Parameter.
    \setcounter{SavedParams}{0}
    \newcommand{\ParamOne}{}            % Definition der 10 
    \newcommand{\ParamTypeOne}{}        % abspeicherbaren Parameter,
    \newcommand{\ParamDescrOne}{}       % ihrer Typen und Beschreibungen.
    \newcommand{\ParamTwo}{}            
    \newcommand{\ParamTypeTwo}{}
    \newcommand{\ParamDescrTwo}{}
    \newcommand{\ParamThree}{}
    \newcommand{\ParamTypeThree}{}
    \newcommand{\ParamDescrThree}{}
    \newcommand{\ParamFour}{}
    \newcommand{\ParamTypeFour}{}
    \newcommand{\ParamDescrFour}{}
    \newcommand{\ParamFive}{}
    \newcommand{\ParamTypeFive}{}
    \newcommand{\ParamDescrFive}{}
    \newcommand{\ParamSix}{}
    \newcommand{\ParamTypeSix}{}
    \newcommand{\ParamDescrSix}{}
    \newcommand{\ParamSeven}{}
    \newcommand{\ParamTypeSeven}{}
    \newcommand{\ParamDescrSeven}{}
    \newcommand{\ParamEight}{}
    \newcommand{\ParamTypeEight}{}
    \newcommand{\ParamDescrEight}{}
    \newcommand{\ParamNine}{}
    \newcommand{\ParamTypeNine}{}
    \newcommand{\ParamDescrNine}{}
    \newcommand{\ParamTen}{}
    \newcommand{\ParamTypeTen}{}
    \newcommand{\ParamDescrTen}{}
    \newlength{\MaxParamWidth}          % Breite des breitesten 
                                        % Methodenparameters.
    \newlength{\SpecialWidth}           % Breite des Textes 
    \settowidth{\SpecialWidth}          % {\bf Besonderheiten}.
        {\MaxLabelTxt}
    \newlength{\ParamDelimWidth}        % Breite des Textes `` \ - ``.
    \settowidth{\ParamDelimWidth}
        {\ParamDelim}
    \newlength{\ParamDescrWidth}        % Max. Breite des Parametertextes.
    \newlength{\SpecialRetTxtWidth}     % Max. Breite der Texte fuer
                                        % Besonderheiten und Rueckgabewert.
    \newlength{\MethodBeginWidth}       % Breite des Methodenanfangs, d.h.
                                        % ``Rueckgabewert Methodenname ( ``.
    \newlength{\BetweenParsWidth}       % Max. zur Verfuegung stehende
                                        % Breite zwischen den beiden Klammern
                                        % der Methode.
    
    \newlength{\MethodEndWidth}         % Breite des Textes ``  );'' bzw.
    \settowidth{\MethodEndWidth}        % `` ) const;''
        {\NormMethodEnd}
    \newlength{\ParamMaxTypeWidth}      % Breite des breitesten
                                        % Parameterwert-Typs. 
    \newlength{\NoneWidthOne}           % Breite des Textes ``Keine.''.
    \settowidth{\NoneWidthOne}{Keine.}
    \newlength{\NoneWidthTwo}           % Breite des Textes ``Keiner.''.
    \settowidth{\NoneWidthTwo}{Keiner.}
    \newlength{\Temp}                   % Temporaere Hilfsvariable. 
    \newlength{\VertSpace}              % Vertikaler Abstand zwischen 
                                        % zwei Parboxen.
   
    % Trotz der automatischen Berechnung der Textbreiten treten
    % kleine Unstimmigkeiten von wenigen Punkten bei den Texten
    % zur Beschreibung der Parameter, des Rueckgabewertes und
    % der Besonderheiten auf, die mit den folgenden Variablen
    % behoben werden koennen:

    \newlength{\CorrectWidthOne}        % Korrekturbreite fuer die
                                        % Parameterbeschreibung.
    \newlength{\CorrectWidthTwo}        % Korrekturbreite fuer die
                                        % Beschreibungen von Rueckgabewert
                                        % und Besonderheiten.
    \newlength{\CorrectWidthThree}      % Korrekturbreite, da erster
                                        % Parameter einer Methode weiter
                                        % vorrueckt als die anderen.
    

% Uebersicht ueber die Parameter und ihre Bedeutung.


%
% <-- \MethodBeginWidth    -->
%------------------------------------------------------------------------
% Rueckgabewert Methodenname(   Typ_Parameter_1 Parameter_1,
%                            <->Typ_Parameter_2 Parameter_2,
%              \CorrectWideThree...
%                               Typ_Parameter_n Parameter_n   );
%                             <-------------> <--------->
%                          \ParamMaxTypeWidth \MaxParamWidth
%                             <-------------------------->
%                             \BetweenParsWidth
%                                                         <-->
%                                                         \MethodEndWidth
%------------------------------------------------------------------------
% Beschreibung der Methode...
%                             \ParamDelimWidth
%                             <->                        \CorrectWidthOne
% \VertSpace
% Parameter:       Parameter_1  - Beschreibung von Parameter 1.   <----->
%                  Parameter_2  - Beschreibung von Parameter 2.
%                  ...
%                  Parameter_n  - Beschreibung von Parameter n.
% \VertSpace                      \ParamDescrWidth
%                                 <------------------------------>
% Rueckgabewert:   Beschreibung des Rueckgabewertes bzw. der Text <-----> 
%                  Keiner.                   
%                  <----->
%                  \NoneWidthTwo               \SpecialRetTxtWidth
% \VertSpace       <--------------------------------------------->
% Besonderheiten:  Beschreibung von Besonderheiten bzw. der Text  <----->
%                  Keine.     
% <--------------> <---->
%  \SpecialWidth   \NoneWidthOne                         \CorrectWidthTwo


    % Definition eigener Befehle:


    %-----------------------------------------------------------------------
    % Gibt den Begriff ``C++'' in ansprechender Form aus.
    % Makro wurde freundlicherweise zur Verfuegung gestellt von
    % Axel W. Dietrich.
    %
    \newcommand{\cpp}{
        \mbox{\emph{\textrm{C\hspace{-1.5pt}\raisebox{1.75pt}{\scriptsize +}
        \hspace{-6pt}\raisebox{.75pt}{\scriptsize +}}}}%
    }


    %-----------------------------------------------------------------------
    % Gibt die Werte aller Parameter auf dem Bildschirm aus.
    %
    \newcommand{\showAllParams}{
        \noindent
        {\em Werte aller Parameter:}
        \begin{tabbing}
            ParamMaxTypeWidth:\ \ \=\kill
            MaxParamWidth:       \>\the\MaxParamWidth\\
            SpecialWidth:        \>\the\SpecialWidth\\
            ParamDelimWidth:     \>\the\ParamDelimWidth\\
            ParamDescrWidth:     \>\the\ParamDescrWidth\\
            SpecialRetTxtWidth:  \>\the\SpecialRetTxtWidth\\
            MethodBeginWidth:    \>\the\MethodBeginWidth\\
            BetweenParsWidth:    \>\the\BetweenParsWidth\\
            MethodEndWidth:      \>\the\MethodEndWidth\\
            ParamMaxTypeWidth:   \>\the\ParamMaxTypeWidth\\
            NoneWidthOne:        \>\the\NoneWidthOne\\
            NoneWidthTwo:        \>\the\NoneWidthTwo\\
            VertSpace:           \>\the\VertSpace\\
            CorrectWidthOne:     \>\the\CorrectWidthOne\\
            CorrectWidthTwo:     \>\the\CorrectWidthTwo\\
            CorrectWidthThree:   \>\the\CorrectWidthThree\\
            ConstInstance:       
            \ifthenelse{\boolean{ConstInstance}}
                {\>ja\\}
                {\>nein\\}
        \end{tabbing}
    }

    %-----------------------------------------------------------------------
    % Setzt einen Schalter, der angibt, dass die Instanz, die die kommende
    % Methode aufrufen kann, nicht veraendert werden darf. Dies hat fuer
    % die Dokumentation zur Folge, dass die Methode mit ``) const;'' endet.
    %
    \newcommand{\setConstInstance}{    
        \setboolean{ConstInstance}{true}
    }

    %-----------------------------------------------------------------------
    % Setzt einen Schalter, der angibt, dass die Instanz, die die kommende
    % Methode aufrufen kann, veraendert werden darf. Die Methode endet also
    % in der Dokumentation normal mit ``);''.
    %
    \newcommand{\setNormalInstance}{    
        \setboolean{ConstInstance}{false}
    }

    %-----------------------------------------------------------------------
    % Initialisiert die Breite des breitesten Methodenparametertyptextes
    % mit Null.
    %
    \newcommand{\initParamMaxTypeWidth}{
        \setlength{\ParamMaxTypeWidth}{0pt}
    }

    %-----------------------------------------------------------------------
    % Initialisiert die Breite des breitesten Parameternamens mit Null.
    %
    \newcommand{\initMaxParamWidth}{
        \setlength{\MaxParamWidth}{0pt}
    }

    %-----------------------------------------------------------------------
    % Ueberprueft, ob die Breite des uebergebenen Textes des Parametertyps
    % groesser ist als die des bisher als breitester Typtext festgelegten
    % Textes und passt den Wert evtl. an.
    % Parameter #1:  Neuer Typtext, dessen Breite zum Vergleich dient.
    %   
    \newcommand{\setNewParamMaxTypeWidth}[1]{
        \settowidth{\Temp}{#1}
        \ifthenelse{\lengthtest{\ParamMaxTypeWidth < \Temp}}
            {\setlength{\ParamMaxTypeWidth}{\Temp}}{}   
    }

    %-----------------------------------------------------------------------
    % Ueberprueft, ob die Breite des uebergebenen Parameternamens groesser
    % ist als die des bisher als am breitesten geltenden Parameternames
    % und passt den Wert evtl. an. Der Parametername steht dabei in
    % Emphasized.
    % Parameter #1:  Neuer Parametername, dessen Breite zum Vergleich dient.
    %
    \newcommand{\setNewMaxParamWidth}[1]{
        \settowidth{\Temp}{{\em #1}}
        \ifthenelse{\lengthtest{\MaxParamWidth < \Temp}}
            {\setlength{\MaxParamWidth}{\Temp}}{}
    }

    %-----------------------------------------------------------------------
    % Setzt die Breite der Beschreibungstexte fuer Rueckgabewert
    % und Besonderheiten.
    %
    \newcommand{\setSpecialRetTxtWidth}{
        \setlength{\SpecialRetTxtWidth}{\textwidth}
        \addtolength{\SpecialRetTxtWidth}{-1\SpecialWidth} 
    }

    %-----------------------------------------------------------------------
    % Setzt die Breite fuer den Beginn einer Methode, d.h. den Text
    % ``Rueckgabewert Methodenname( ``. Rueckgabewert steht dabei
    % in Emphasized, Methodenname in Bold Font.
    % Parameter #1:  Rueckgabewert der Methode.
    % Parameter #2:  Name der Methode.
    %
    \newcommand{\setMethodBeginWidth}[2]{
        \settowidth{\MethodBeginWidth}
            {{\em #1}\InBetween{\bf #2}\MethodBegin}
    }

    %-----------------------------------------------------------------------
    % Setzt die maximale Breite die an Platz zwischen der oeffnenden
    % und der schliessenden Klammer der Methode zur Verfuegung steht.
    %
    \newcommand{\setBetweenParsWidth}{
        \setlength{\BetweenParsWidth}{\textwidth}
        \addtolength{\BetweenParsWidth}{-1\MethodBeginWidth}
        \addtolength{\BetweenParsWidth}{-1\MethodEndWidth}
    }

    %-----------------------------------------------------------------------
    % Setzt die Breite fuer die Beschreibungstexte
    % der Methodenparameter.
    %
    \newcommand{\setParamDescrWidth}{
        \setlength{\ParamDescrWidth}{\textwidth}
        \addtolength{\ParamDescrWidth}{-1\SpecialWidth} 
        \addtolength{\ParamDescrWidth}{-1\MaxParamWidth}
        \addtolength{\ParamDescrWidth}{-1\ParamDelimWidth}
    }    

    %-----------------------------------------------------------------------
    % Setzt die Korrekturbreite fuer den Beschreibungstext von
    % Methodenparametern auf den uebergebenen Wert.
    % Parameter #1:  Neuer Wert fuer die Korrekturbreite.
    %
    \newcommand{\setCorrectWidthOne}[1]{
        \setlength{\CorrectWidthOne}{#1}
    }    

    %-----------------------------------------------------------------------
    % Setzt die Korrekturbreite fuer den Beschreibungstext von
    % Rueckgabewert und Besonderheiten auf den uebergebenen Wert.
    % Parameter #1:  Neuer Wert fuer die Korrekturbreite.
    %
    \newcommand{\setCorrectWidthTwo}[1]{
        \setlength{\CorrectWidthTwo}{#1}
    }    

    %-----------------------------------------------------------------------
    % Setzt die Korrekturbreite fuer den Abstand des zweiten bis n-ten
    % Parameters vom Anfang auf den uebergebenen Wert.
    % Parameter #1:  Neuer Wert fuer die Korrekturbreite.
    %
    \newcommand{\setCorrectWidthThree}[1]{
        \setlength{\CorrectWidthThree}{#1}
    }    

    %-----------------------------------------------------------------------
    % Setzt den vertikalen Abstand zwischen zwei Parboxen auf den ueber-
    % gebenen Wert.
    % Parameter #1:  Neuer Wert fuer den vertikalen Abstand.
    %
    \newcommand{\setVertSpace}[1]{
        \setlength{\VertSpace}{#1}
    }    

    %-----------------------------------------------------------------------
    % Speichert die uebergebenen Informationen ueber Methodenparameter 1
    % in internen Variablen. Achtung: Die Parameter muessen in der richtigen
    % Reihenfolge gesetzt werden!
    % Parameter #1: Der Typ des Parameters.  
    % Parameter #2: Der Name des Parameters. 
    % Parameter #3: Die Beschreibung des Parameters. 
    % 
    \newcommand{\setParamOne}[3]{
        \renewcommand{\ParamOne}{#1}
        \renewcommand{\ParamTypeOne}{#2}
        \renewcommand{\ParamDescrOne}{#3} 
        \setcounter{SavedParams}{0}
        \stepcounter{SavedParams}
    }

    %-----------------------------------------------------------------------
    % Speichert die uebergebenen Informationen ueber Methodenparameter 2
    % in internen Variablen. Achtung: Die Parameter muessen in der richtigen
    % Reihenfolge gesetzt werden!
    % Parameter #1: Der Typ des Parameters.  
    % Parameter #2: Der Name des Parameters. 
    % Parameter #3: Die Beschreibung des Parameters. 
    % 
    \newcommand{\setParamTwo}[3]{
        \renewcommand{\ParamTwo}{#1}
        \renewcommand{\ParamTypeTwo}{#2}
        \renewcommand{\ParamDescrTwo}{#3} 
        \stepcounter{SavedParams}
    }

    %-----------------------------------------------------------------------
    % Speichert die uebergebenen Informationen ueber Methodenparameter 3
    % in internen Variablen. Achtung: Die Parameter muessen in der richtigen
    % Reihenfolge gesetzt werden!
    % Parameter #1: Der Typ des Parameters.  
    % Parameter #2: Der Name des Parameters. 
    % Parameter #3: Die Beschreibung des Parameters. 
    % 
    \newcommand{\setParamThree}[3]{
        \renewcommand{\ParamThree}{#1}
        \renewcommand{\ParamTypeThree}{#2}
        \renewcommand{\ParamDescrThree}{#3} 
        \stepcounter{SavedParams}
    }

    %-----------------------------------------------------------------------
    % Speichert die uebergebenen Informationen ueber Methodenparameter 4
    % in internen Variablen. Achtung: Die Parameter muessen in der richtigen
    % Reihenfolge gesetzt werden!
    % Parameter #1: Der Typ des Parameters.  
    % Parameter #2: Der Name des Parameters. 
    % Parameter #3: Die Beschreibung des Parameters. 
    % 
    \newcommand{\setParamFour}[3]{
        \renewcommand{\ParamFour}{#1}
        \renewcommand{\ParamTypeFour}{#2}
        \renewcommand{\ParamDescrFour}{#3} 
        \stepcounter{SavedParams}
    }

    %-----------------------------------------------------------------------
    % Speichert die uebergebenen Informationen ueber Methodenparameter 5
    % in internen Variablen. Achtung: Die Parameter muessen in der richtigen
    % Reihenfolge gesetzt werden!
    % Parameter #1: Der Typ des Parameters.  
    % Parameter #2: Der Name des Parameters. 
    % Parameter #3: Die Beschreibung des Parameters. 
    % 
    \newcommand{\setParamFive}[3]{
        \renewcommand{\ParamFive}{#1}
        \renewcommand{\ParamTypeFive}{#2}
        \renewcommand{\ParamDescrFive}{#3} 
        \stepcounter{SavedParams}
    }

    %-----------------------------------------------------------------------
    % Speichert die uebergebenen Informationen ueber Methodenparameter 6
    % in internen Variablen. Achtung: Die Parameter muessen in der richtigen
    % Reihenfolge gesetzt werden!
    % Parameter #1: Der Typ des Parameters.  
    % Parameter #2: Der Name des Parameters. 
    % Parameter #3: Die Beschreibung des Parameters. 
    %     
    \newcommand{\setParamSix}[3]{
        \renewcommand{\ParamSix}{#1}
        \renewcommand{\ParamTypeSix}{#2}
        \renewcommand{\ParamDescrSix}{#3} 
        \stepcounter{SavedParams} 
    }

    %-----------------------------------------------------------------------
    % Speichert die uebergebenen Informationen ueber Methodenparameter 7
    % in internen Variablen. Achtung: Die Parameter muessen in der richtigen
    % Reihenfolge gesetzt werden!
    % Parameter #1: Der Typ des Parameters.  
    % Parameter #2: Der Name des Parameters. 
    % Parameter #3: Die Beschreibung des Parameters. 
    %    
    \newcommand{\setParamSeven}[3]{
        \renewcommand{\ParamSeven}{#1}
        \renewcommand{\ParamTypeSeven}{#2}
        \renewcommand{\ParamDescrSeven}{#3} 
        \stepcounter{SavedParams} 
    }

    %-----------------------------------------------------------------------
    % Speichert die uebergebenen Informationen ueber Methodenparameter 8
    % in internen Variablen. Achtung: Die Parameter muessen in der richtigen
    % Reihenfolge gesetzt werden!
    % Parameter #1: Der Typ des Parameters.  
    % Parameter #2: Der Name des Parameters. 
    % Parameter #3: Die Beschreibung des Parameters. 
    % 
    \newcommand{\setParamEight}[3]{
        \renewcommand{\ParamEight}{#1}
        \renewcommand{\ParamTypeEight}{#2}
        \renewcommand{\ParamDescrEight}{#3} 
        \stepcounter{SavedParams}
    }

    %-----------------------------------------------------------------------
    % Speichert die uebergebenen Informationen ueber Methodenparameter 9
    % in internen Variablen. Achtung: Die Parameter muessen in der richtigen
    % Reihenfolge gesetzt werden!
    % Parameter #1: Der Typ des Parameters.  
    % Parameter #2: Der Name des Parameters. 
    % Parameter #3: Die Beschreibung des Parameters. 
    % 
    \newcommand{\setParamNine}[3]{
        \renewcommand{\ParamNine}{#1}
        \renewcommand{\ParamTypeNine}{#2}
        \renewcommand{\ParamDescrNine}{#3}  
        \stepcounter{SavedParams}
    }

    %-----------------------------------------------------------------------
    % Speichert die uebergebenen Informationen ueber Methodenparameter 10
    % in internen Variablen. Achtung: Die Parameter muessen in der richtigen
    % Reihenfolge gesetzt werden!
    % Parameter #1: Der Typ des Parameters.  
    % Parameter #2: Der Name des Parameters. 
    % Parameter #3: Die Beschreibung des Parameters. 
    % 
    \newcommand{\setParamTen}[3]{
        \renewcommand{\ParamTen}{#1}
        \renewcommand{\ParamTypeTen}{#2}
        \renewcommand{\ParamDescrTen}{#3} 
        \stepcounter{SavedParams}
    }

    %-----------------------------------------------------------------------
    % Gibt den Text ``Rueckgabewert Methodenname(`` aus.
    % Parameter #1:  Rueckgabewert der Methode.
    % Parameter #2:  Name der Methode.
    %
    \newcommand{\printMethodBegin}[2]{
         \ifthenelse{\boolean{ConstInstance}}
             {\settowidth{\MethodEndWidth}{\ConstMethodEnd}}
             {\settowidth{\MethodEndWidth}{\NormMethodEnd}}
         \setMethodBeginWidth{#1}{#2}
         \setBetweenParsWidth
         \noindent
         \hrulefill\\
         \noindent
         \parbox[t]{\MethodBeginWidth}
             {#1\InBetween{\bf #2}\MethodBegin}
     }

    %-----------------------------------------------------------------------
    % Gibt das Ende einer Methode aus, d.h. ``\ );'' oder ``\ ) const;''.
    %
    \newcommand{\printMethodEnd}{
        \ifthenelse{\boolean{ConstInstance}}
            {\ConstMethodEnd\\}
            {\NormMethodEnd\\}
            \mbox{}\hrulefill\\
    }

    %-----------------------------------------------------------------------
    % Gibt den Typen und Namen des einzigen Parameters einer Methode
    % gefolgt von dem Text ``  );'' bzw. `` ) const;'' aus.
    % Parameter #1:  Typ des Parameters.
    % Parameter #2:  Name des Parameters.
    % 
    \newcommand{\printMethodOneParam}[2]{
        \parbox[t]{\ParamMaxTypeWidth}{#1}
        \parbox[t]{\MaxParamWidth}{{\em #2}}
        \printMethodEnd
    }

    %-----------------------------------------------------------------------
    % Gibt den Typen und den Namen des ersten Methodenparameters aus.
    % Parameter #1:  Typ des Parameters.
    % Parameter #2:  Name des Parameters.
    %
    \newcommand{\printMethodFirstParam}[2]{
        \noindent
        \parbox[t]{\ParamMaxTypeWidth}{#1}
        \parbox[t]{\MaxParamWidth}{{\em #2},}\\[\VertSpace]
    }

    %-----------------------------------------------------------------------
    % Gibt den Typen und Namen des letzten Parameters einer Methode aus.
    % Parameter #1: Typ des Parameters.
    % Parameter #2: Name des Parameters.
    %
    \newcommand{\printMethodLastParam}[2]{
        \parbox[t]{\MethodBeginWidth}{\hfill}
        \hspace{\CorrectWidthThree}
        \parbox[t]{\ParamMaxTypeWidth}{#1}
        \parbox[t]{\MaxParamWidth}{{\em #2}}
        \printMethodEnd
    }

    %-----------------------------------------------------------------------
    % Gibt den Typen und Namen eines Methodenparameters aus.
    % Parameter #1:  Typ des Parameters.
    % Parameter #2: Name des Parameters.
    % Achtung: Nicht fuer den ersten oder letzten Parameter der Methode
    % geeignet!
    %
    \newcommand{\printMethodParam}[2]{
        \parbox[t]{\MethodBeginWidth}{\hfill}
        \hspace{\CorrectWidthThree}
        \parbox[t]{\ParamMaxTypeWidth}{#1}
        \parbox[t]{\MaxParamWidth}{{\em #2},}\\[\VertSpace]
    }

    %-----------------------------------------------------------------------
    % Gibt die Beschreibung einer Methode aus.
    % Parameter #1: Beschreibung der Methode.
    %
    \newcommand{\printMethodDescr}[1]{
        \newline
        \noindent
        #1\\[\VertSpace]
    }

    %-----------------------------------------------------------------------
    % Gibt den Text ``Parameter:    `` und danach den Namen des
    % ersten Parameters der Methode, gefolgt von seiner Beschreibung
    % aus.
    % Parameter #1:  Name des Parameters. 
    % Parameter #2:  Beschreibung des Parameters.
    %
    \newcommand{\printFirstParam}[2]{
        \setlength{\Temp}{\ParamDescrWidth}
        \addtolength{\Temp}{1\CorrectWidthOne}
        \parbox[t]{\SpecialWidth}{{\bf Parameter:}}
        \parbox[t]{\MaxParamWidth}{\em #1}
        \parbox[t]{\ParamDelimWidth}{\ParamDelim}
        \parbox[t]{\Temp}{#2}\\[\VertSpace]
    }

    %-----------------------------------------------------------------------
    % Gibt einen Parameter der Methode und die dazugehoerige Beschreibung
    % aus.
    % Parameter #1:  Name des Parameters.
    % Parameter #2:  Beschreibung des Parameters.
    %
    \newcommand{\printParam}[2]{
        \setlength{\Temp}{\ParamDescrWidth}
        \addtolength{\Temp}{1\CorrectWidthOne}
        \parbox[t]{\SpecialWidth}{\hfill}
        \parbox[t]{\MaxParamWidth}{\em #1}
        \parbox[t]{\ParamDelimWidth}{\ParamDelim}
        \parbox[t]{\Temp}{#2}\\[\VertSpace]
    }

    %-----------------------------------------------------------------------
    %
    % Gibt den Text ``Rueckgabewert:  `` gefolgt von einer Beschreibung
    % des Rueckgabewertes aus.
    % Parameter #1:  Beschreibung des Rueckgabewertes.
    %
    \newcommand{\printReturn}[1]{
        \setlength{\Temp}{\SpecialRetTxtWidth}
        \addtolength{\Temp}{1\CorrectWidthTwo}
        \parbox[t]{\SpecialWidth}{{\bf R\"uckgabewert:}}
        \parbox[t]{\Temp}{#1}\\[\VertSpace]
    }

    %-----------------------------------------------------------------------
    % Gibt den Text ``Besonderheiten:  `` gefolgt von einer Beschreibung
    % der Besonderheiten aus.
    % Parameter #1:  Beschreibung der Besonderheiten.
    %
    \newcommand{\printSpecial}[1]{
        \setlength{\Temp}{\SpecialRetTxtWidth}
        \addtolength{\Temp}{1\CorrectWidthTwo}
        \parbox[t]{\SpecialWidth}{\MaxLabelTxt}
        \parbox[t]{\Temp}{#1}\\[\VertSpace]
    }

    %-----------------------------------------------------------------------
    % Gibt den Text ``Parameter:        Keine.'' aus.
    %
    \newcommand{\printNoParams}{
        \parbox[t]{\SpecialWidth}{{\bf Parameter:}}
        \parbox[t]{\NoneWidthOne}{Keine.}\\[\VertSpace]
    }

    %-----------------------------------------------------------------------
    % Gibt den Text ``Rueckgabewert:  Keiner.'' aus.
    %
    \newcommand{\printNoReturn}{
        \parbox[t]{\SpecialWidth}{{\bf R\"uckgabewert:}}
        \parbox[t]{\NoneWidthTwo}{Keiner.}\\[\VertSpace]
    }

    %-----------------------------------------------------------------------
    % Gibt den Text ``Besonderheiten:  Keine.'' aus.
    %
    \newcommand{\printNoSpecial}{
        \parbox[t]{\SpecialWidth}{\MaxLabelTxt}
        \parbox[t]{\NoneWidthOne}{Keine.}\\[\VertSpace]
    }

    %-----------------------------------------------------------------------
    % Gibt den Text:
    % ``Parameter:       Keine.
    %   Rueckgabewert:   Keiner.
    %   Besonderheiten:  Keine.''
    % aus.
    %
    \newcommand{\printNoAll}{
        \printNoParams
        \printNoReturn
        \printNoSpecial
    }

    %-----------------------------------------------------------------------
    % Gibt eine Methode ohne Parameter, Rueckgabewert und Besonderheiten
    % aus.
    % Parameter #1:  Name der Methode.
    % Parameter #2:  Beschreibung der Methode.
    %
    \newcommand{\printEmptyMethod}[2]{
        \setSpecialRetTxtWidth
        \printMethodBegin{void}{#1}
        \printMethodEnd
        \printMethodDescr{#2}
        \printNoAll
    }

    %-----------------------------------------------------------------------
    % Gibt eine Methode ohne Parameter und Besonderheiten, aber mit
    % Rueckgabewert aus.
    % Parameter #1:  Rueckgabewert der Methode.
    % Parameter #2:  Name der Methode.
    % Parameter #3:  Beschreibung der Methode.
    % Parameter #4:  Beschreibung des Rueckgabewertes der Methode.
    %
    \newcommand{\printEmptyMethodReturn}[4]{
        \setSpecialRetTxtWidth
        \printMethodBegin{#1}{#2}
        \printMethodEnd
        \printMethodDescr{#3}
        \printNoParams
        \printReturn{#4}
        \printNoSpecial
    }

    %-----------------------------------------------------------------------
    % Gibt eine Methode ohne Parameter und Rueckgabewert, aber mit
    % Besonderheiten aus.
    % Parameter #1:  Name der Methode.
    % Parameter #2:  Beschreibung der Methode.
    % Parameter #3:  Beschreibung der Besonderheiten der Methode.
    %
    \newcommand{\printEmptyMethodSpecial}[3]{
        \setSpecialRetTxtWidth
        \printMethodBegin{void}{#1}
        \printMethodEnd
        \printMethodDescr{#2}
        \printNoParams
        \printNoReturn
        \printSpecial{#3}
    }

    %-----------------------------------------------------------------------
    % Gibt eine Methode ohne Parameter, aber mit Rueckgabewert
    % und Besonderheiten aus.
    % Parameter #1:  Rueckgabewert der Methode.
    % Parameter #2:  Name der Methode.
    % Parameter #3:  Beschreibung der Methode.
    % Parameter #4:  Beschreibung des Rueckgabewertes der Methode.
    % Parameter #5:  Beschreibung der Besonderheiten der Methode.
    %
    \newcommand{\printEmptyMethodReturnSpecial}[5]{
        \setSpecialRetTxtWidth
        \printMethodBegin{#1}{#2}
        \printMethodEnd
        \printMethodDescr{#3}
        \printNoParams
        \printReturn{#4}
        \printSpecial{#5}
    }

    %-----------------------------------------------------------------------
    % Gibt eine Methode mit einem Parameter komplett mit allen
    % Beschreibungen aus.
    % Parameter #1:  Rueckgabewert der Methode.
    % Parameter #2:  Name der Methode.
    % Parameter #3:  Typ des einzigen Parameters der Methode.
    % Parameter #4:  Name des einzigen Parameters der Methode.
    % Parameter #5:  Beschreibung des einzigen Parameters der Methode.
    % Parameter #6:  Beschreibung der Methode.
    % Parameter #7:  Beschreibung des Rueckgabewertes der Methode.
    % Parameter #8:  Beschreibung der Besonderheiten der Methode.
    %
    \newcommand{\printMethodWithOneParam}[8]{
        \initParamMaxTypeWidth
        \initMaxParamWidth
        \setNewParamMaxTypeWidth{#3}
        \setNewMaxParamWidth{#4}
        \setSpecialRetTxtWidth
        \setParamDescrWidth
        \printMethodBegin{#1}{#2}
        \printMethodOneParam{#3}{#4}
        \printMethodDescr{#6}
        \printFirstParam{#4}{#5}
        \ifthenelse{\equal{#7}{Keiner.}}
            {\printNoReturn}
            {\printReturn{#7}}
        \ifthenelse{\equal{#8}{Keine.}}
            {\printNoSpecial}
            {\printSpecial{#8}}
    }

    %-----------------------------------------------------------------------
    % Gibt eine Methode mit einem Parameter komplett mit allen
    % Beschreibungen aus.
    % Parameter #1:  Rueckgabewert der Methode.
    % Parameter #2:  Beschreibung des Rueckgabewertes der Methode oder 
    %                Aufruf mit leerer Klammer. 
    % Parameter #3:  Name der Methode.
    % Parameter #4:  Beschreibung der Methode.
    % Parameter #5:  Beschreibung der Besonderheiten der Methode oder 
    %                Aufruf mit leerer Klammer.
    %
    \newcommand{\printMethodWithParamsSaved}[5]{
        \ifthenelse{\value{SavedParams} > 0}
            {\initParamMaxTypeWidth
             \initMaxParamWidth
             \setNewParamMaxTypeWidth{\ParamTypeOne}
             \setNewMaxParamWidth{\ParamOne}
             \ifthenelse{\value{SavedParams} = 2 \or \value{SavedParams} > 2}
                 {\setNewParamMaxTypeWidth{\ParamTypeTwo}
                  \setNewMaxParamWidth{\ParamTwo}}{}   
             \ifthenelse{\value{SavedParams} = 3 \or \value{SavedParams} > 3}
                 {\setNewParamMaxTypeWidth{\ParamTypeThree}
                  \setNewMaxParamWidth{\ParamThree}}{}    
             \ifthenelse{\value{SavedParams} = 4 \or \value{SavedParams} > 4}
                 {\setNewParamMaxTypeWidth{\ParamTypeFour}
                  \setNewMaxParamWidth{\ParamFour}}{}
             \ifthenelse{\value{SavedParams} = 5 \or \value{SavedParams} > 5}
                 {\setNewParamMaxTypeWidth{\ParamTypeFive}
                  \setNewMaxParamWidth{\ParamFive}}{}      
             \ifthenelse{\value{SavedParams} = 6 \or \value{SavedParams} > 6}
                 {\setNewParamMaxTypeWidth{\ParamTypeSix}
                  \setNewMaxParamWidth{\ParamSix}}{}
             \ifthenelse{\value{SavedParams} = 7 \or \value{SavedParams} > 7}
                 {\setNewParamMaxTypeWidth{\ParamTypeSeven}
                  \setNewMaxParamWidth{\ParamSeven}}{}       
             \ifthenelse{\value{SavedParams} = 8 \or \value{SavedParams} > 8}
                 {\setNewParamMaxTypeWidth{\ParamTypeEight}
                  \setNewMaxParamWidth{\ParamEight}}{}
             \ifthenelse{\value{SavedParams} = 9 \or \value{SavedParams} > 9}
                 {\setNewParamMaxTypeWidth{\ParamTypeNine}
                  \setNewMaxParamWidth{\ParamNine}}{}
             \ifthenelse{\value{SavedParams} = 10 \or 
                         \value{SavedParams} > 10}
                 {\setNewParamMaxTypeWidth{\ParamTypeTen}
                  \setNewMaxParamWidth{\ParamTen}}{}         
             \setSpecialRetTxtWidth
             \setParamDescrWidth
             \printMethodBegin{#1}{#3}
             \ifthenelse{\value{SavedParams} = 1} 
                 {\printMethodOneParam{\ParamTypeOne}{\ParamOne}}{}
                 {\printMethodFirstParam{\ParamTypeOne}{\ParamOne}}{}
             \ifthenelse{\value{SavedParams} = 2} 
                 {\printMethodLastParam{\ParamTypeTwo}{\ParamTwo}}{}
             \ifthenelse{\value{SavedParams} > 2}  
                 {\printMethodParam{\ParamTypeTwo}{\ParamTwo}}{} 
             \ifthenelse{\value{SavedParams} = 3} 
                 {\printMethodLastParam{\ParamTypeThree}{\ParamThree}}{}
             \ifthenelse{\value{SavedParams} > 3}  
                 {\printMethodParam{\ParamTypeThree}{\ParamThree}}{} 
             \ifthenelse{\value{SavedParams} = 4} 
                 {\printMethodLastParam{\ParamTypeFour}{\ParamFour}}{}
             \ifthenelse{\value{SavedParams} > 4}  
                 {\printMethodParam{\ParamTypeFour}{\ParamFour}}{} 
             \ifthenelse{\value{SavedParams} = 5} 
                 {\printMethodLastParam{\ParamTypeFive}{\ParamFive}}{}
             \ifthenelse{\value{SavedParams} > 5}  
                 {\printMethodParam{\ParamTypeFive}{\ParamFive}}{} 
             \ifthenelse{\value{SavedParams} = 6} 
                 {\printMethodLastParam{\ParamTypeSix}{\ParamSix}}{}
             \ifthenelse{\value{SavedParams} > 6}  
                 {\printMethodParam{\ParamTypeSix}{\ParamSix}}{} 
             \ifthenelse{\value{SavedParams} = 7} 
                 {\printMethodLastParam{\ParamTypeSeven}{\ParamSeven}}{}
             \ifthenelse{\value{SavedParams} > 7}  
                 {\printMethodParam{\ParamTypeSeven}{\ParamSeven}}{} 
             \ifthenelse{\value{SavedParams} = 8} 
                 {\printMethodLastParam{\ParamTypeEight}{\ParamEight}}{}
             \ifthenelse{\value{SavedParams} > 8}  
                 {\printMethodParam{\ParamTypeEight}{\ParamEight}}{}  
             \ifthenelse{\value{SavedParams} = 9} 
                 {\printMethodLastParam{\ParamTypeNine}{\ParamNine}}{}
             \ifthenelse{\value{SavedParams} > 9}  
                 {\printMethodParam{\ParamTypeNine}{\ParamNine}}{} 
             \ifthenelse{\value{SavedParams} = 10} 
                 {\printMethodLastParam{\ParamTypeTen}{\ParamTen}}{}
             \printMethodDescr{#4}
             \printFirstParam{\ParamOne}{\ParamDescrOne}
             \ifthenelse{\value{SavedParams} = 2 \or \value{SavedParams} > 2}
                  {\printParam{\ParamTwo}{\ParamDescrTwo}}{}
             \ifthenelse{\value{SavedParams} = 3 \or \value{SavedParams} > 3}
                  {\printParam{\ParamThree}{\ParamDescrThree}}{}
             \ifthenelse{\value{SavedParams} = 4 \or \value{SavedParams} > 4}
                  {\printParam{\ParamFour}{\ParamDescrFour}}{}
             \ifthenelse{\value{SavedParams} = 5 \or \value{SavedParams} > 5}
                  {\printParam{\ParamFive}{\ParamDescrFive}}{}
             \ifthenelse{\value{SavedParams} = 6 \or \value{SavedParams} > 6}
                  {\printParam{\ParamSix}{\ParamDescrSix}}{}
             \ifthenelse{\value{SavedParams} = 7 \or \value{SavedParams} > 7}
                  {\printParam{\ParamSeven}{\ParamDescrSeven}}{}
             \ifthenelse{\value{SavedParams} = 8 \or \value{SavedParams} > 8}
                  {\printParam{\ParamEight}{\ParamDescrEight}}{} 
             \ifthenelse{\value{SavedParams} = 9 \or \value{SavedParams} > 9}
                  {\printParam{\ParamNine}{\ParamDescrNine}}{}
             \ifthenelse{\value{SavedParams} = 10}
                  {\printParam{\ParamTen}{\ParamDescrTen}}{}
             \ifthenelse{\equal{#2}{}}
                 {\printNoReturn}
                 {\printReturn{#2}}
             \ifthenelse{\equal{#5}{}}
                 {\printNoSpecial}
                 {\printSpecial{#5}}}
        {\noindent
         Fehler Methode {\em printMethodWithParamsSaved}: Es sind noch keine 
         Parameter gesetzt worden!}
    }

%-----------------------------------------------------------------------
% Ende der Befehlsdefinitionen.
%-----------------------------------------------------------------------





%%% Local Variables: 
%%% mode: latex
%%% TeX-master: t
%%% End: 
'' in LaTex-Dokumente 
% (Latex2Epsilon) eingebunden und stellt Befehle zur einfachen Doku- 
% mentation von C-Funktionen/C++-Methoden bereit.
% =======================================================================

\usepackage{ifthen}

    % Definition von Konstanten:
    \newcommand{\ParamDelim}            % Trennungstext zwischen
        { \ - }                         % Parametername und -beschreibung.
    \newcommand{\NormMethodEnd}         % Endtext von Methode bei 
        {\ \ );}                        % veraenderbarer Aufrufinstanz.
    \newcommand{\ConstMethodEnd}        % Endtext von Methode bei
        {\ \ ) const;}                  % nicht veraenderbarer Aufrufinstanz.
    \newcommand{\MaxLabelTxt}           % Komponentenbeschreibungstext
        {{\bf Return Value: }}          % mit maximaler Breite.
    \newcommand{\MethodBegin}           % Beginntext einer Methode.
        {(\ }
    \newcommand{\InBetween}             % Text zwischen Rueckgabewert und
        {\ \ }                          % Methodenname.

    % Definition eigener Variablen:


    \newboolean{ConstInstance}          % Gibt an, ob die Instanz, die
                                        % die Methode aufruft veraendert
                                        % werden darf oder nicht. Im
                                        % 2. Fall endet die Methode mit
                                        % ``) const;'', im 1. Fall 
                                        % mit ``);''.
    \newcounter{SavedParams}            % Anzahl der gespeicherten
                                        % Parameter.
    \setcounter{SavedParams}{0}
    \newcommand{\ParamOne}{}            % Definition der 10 
    \newcommand{\ParamTypeOne}{}        % abspeicherbaren Parameter,
    \newcommand{\ParamDescrOne}{}       % ihrer Typen und Beschreibungen.
    \newcommand{\ParamTwo}{}            
    \newcommand{\ParamTypeTwo}{}
    \newcommand{\ParamDescrTwo}{}
    \newcommand{\ParamThree}{}
    \newcommand{\ParamTypeThree}{}
    \newcommand{\ParamDescrThree}{}
    \newcommand{\ParamFour}{}
    \newcommand{\ParamTypeFour}{}
    \newcommand{\ParamDescrFour}{}
    \newcommand{\ParamFive}{}
    \newcommand{\ParamTypeFive}{}
    \newcommand{\ParamDescrFive}{}
    \newcommand{\ParamSix}{}
    \newcommand{\ParamTypeSix}{}
    \newcommand{\ParamDescrSix}{}
    \newcommand{\ParamSeven}{}
    \newcommand{\ParamTypeSeven}{}
    \newcommand{\ParamDescrSeven}{}
    \newcommand{\ParamEight}{}
    \newcommand{\ParamTypeEight}{}
    \newcommand{\ParamDescrEight}{}
    \newcommand{\ParamNine}{}
    \newcommand{\ParamTypeNine}{}
    \newcommand{\ParamDescrNine}{}
    \newcommand{\ParamTen}{}
    \newcommand{\ParamTypeTen}{}
    \newcommand{\ParamDescrTen}{}
    \newlength{\MaxParamWidth}          % Breite des breitesten 
                                        % Methodenparameters.
    \newlength{\SpecialWidth}           % Breite des Textes 
    \settowidth{\SpecialWidth}          % {\bf Besonderheiten}.
        {\MaxLabelTxt}
    \newlength{\ParamDelimWidth}        % Breite des Textes `` \ - ``.
    \settowidth{\ParamDelimWidth}
        {\ParamDelim}
    \newlength{\ParamDescrWidth}        % Max. Breite des Parametertextes.
    \newlength{\SpecialRetTxtWidth}     % Max. Breite der Texte fuer
                                        % Besonderheiten und Rueckgabewert.
    \newlength{\MethodBeginWidth}       % Breite des Methodenanfangs, d.h.
                                        % ``Rueckgabewert Methodenname ( ``.
    \newlength{\BetweenParsWidth}       % Max. zur Verfuegung stehende
                                        % Breite zwischen den beiden Klammern
                                        % der Methode.
    
    \newlength{\MethodEndWidth}         % Breite des Textes ``  );'' bzw.
    \settowidth{\MethodEndWidth}        % `` ) const;''
        {\NormMethodEnd}
    \newlength{\ParamMaxTypeWidth}      % Breite des breitesten
                                        % Parameterwert-Typs. 
    \newlength{\NoneWidthOne}           % Breite des Textes ``Keine.''.
    \settowidth{\NoneWidthOne}{None.}
    \newlength{\NoneWidthTwo}           % Breite des Textes ``Keiner.''.
    \settowidth{\NoneWidthTwo}{None.}
    \newlength{\Temp}                   % Temporaere Hilfsvariable. 
    \newlength{\VertSpace}              % Vertikaler Abstand zwischen 
                                        % zwei Parboxen.
   
    % Trotz der automatischen Berechnung der Textbreiten treten
    % kleine Unstimmigkeiten von wenigen Punkten bei den Texten
    % zur Beschreibung der Parameter, des Rueckgabewertes und
    % der Besonderheiten auf, die mit den folgenden Variablen
    % behoben werden koennen:

    \newlength{\CorrectWidthOne}        % Korrekturbreite fuer die
                                        % Parameterbeschreibung.
    \newlength{\CorrectWidthTwo}        % Korrekturbreite fuer die
                                        % Beschreibungen von Rueckgabewert
                                        % und Besonderheiten.
    \newlength{\CorrectWidthThree}      % Korrekturbreite, da erster
                                        % Parameter einer Methode weiter
                                        % vorrueckt als die anderen.
    

% Uebersicht ueber die Parameter und ihre Bedeutung.


%
% <-- \MethodBeginWidth    -->
%------------------------------------------------------------------------
% Rueckgabewert Methodenname(   Typ_Parameter_1 Parameter_1,
%                            <->Typ_Parameter_2 Parameter_2,
%              \CorrectWideThree...
%                               Typ_Parameter_n Parameter_n   );
%                             <-------------> <--------->
%                          \ParamMaxTypeWidth \MaxParamWidth
%                             <-------------------------->
%                             \BetweenParsWidth
%                                                         <-->
%                                                         \MethodEndWidth
%------------------------------------------------------------------------
% Beschreibung der Methode...
%                             \ParamDelimWidth
%                             <->                        \CorrectWidthOne
% \VertSpace
% Parameter:       Parameter_1  - Beschreibung von Parameter 1.   <----->
%                  Parameter_2  - Beschreibung von Parameter 2.
%                  ...
%                  Parameter_n  - Beschreibung von Parameter n.
% \VertSpace                      \ParamDescrWidth
%                                 <------------------------------>
% Rueckgabewert:   Beschreibung des Rueckgabewertes bzw. der Text <-----> 
%                  Keiner.                   
%                  <----->
%                  \NoneWidthTwo               \SpecialRetTxtWidth
% \VertSpace       <--------------------------------------------->
% Besonderheiten:  Beschreibung von Besonderheiten bzw. der Text  <----->
%                  Keine.     
% <--------------> <---->
%  \SpecialWidth   \NoneWidthOne                         \CorrectWidthTwo


    % Definition eigener Befehle:


    %-----------------------------------------------------------------------
    % Gibt den Begriff ``C++'' in ansprechender Form aus.
    % Makro wurde freundlicherweise zur Verfuegung gestellt von
    % Axel W. Dietrich.
    %
    \newcommand{\cpp}{
        \mbox{\emph{\textrm{C\hspace{-1.5pt}\raisebox{1.75pt}{\scriptsize +}
        \hspace{-6pt}\raisebox{.75pt}{\scriptsize +}}}}%
    }


    %-----------------------------------------------------------------------
    % Gibt die Werte aller Parameter auf dem Bildschirm aus.
    %
    \newcommand{\showAllParams}{
        \noindent
        {\em Werte aller Parameter:}
        \begin{tabbing}
            ParamMaxTypeWidth:\ \ \=\kill
            MaxParamWidth:       \>\the\MaxParamWidth\\
            SpecialWidth:        \>\the\SpecialWidth\\
            ParamDelimWidth:     \>\the\ParamDelimWidth\\
            ParamDescrWidth:     \>\the\ParamDescrWidth\\
            SpecialRetTxtWidth:  \>\the\SpecialRetTxtWidth\\
            MethodBeginWidth:    \>\the\MethodBeginWidth\\
            BetweenParsWidth:    \>\the\BetweenParsWidth\\
            MethodEndWidth:      \>\the\MethodEndWidth\\
            ParamMaxTypeWidth:   \>\the\ParamMaxTypeWidth\\
            NoneWidthOne:        \>\the\NoneWidthOne\\
            NoneWidthTwo:        \>\the\NoneWidthTwo\\
            VertSpace:           \>\the\VertSpace\\
            CorrectWidthOne:     \>\the\CorrectWidthOne\\
            CorrectWidthTwo:     \>\the\CorrectWidthTwo\\
            CorrectWidthThree:   \>\the\CorrectWidthThree\\
            ConstInstance:       
            \ifthenelse{\boolean{ConstInstance}}
                {\>ja\\}
                {\>nein\\}
        \end{tabbing}
    }

    %-----------------------------------------------------------------------
    % Setzt einen Schalter, der angibt, dass die Instanz, die die kommende
    % Methode aufrufen kann, nicht veraendert werden darf. Dies hat fuer
    % die Dokumentation zur Folge, dass die Methode mit ``) const;'' endet.
    %
    \newcommand{\setConstInstance}{    
        \setboolean{ConstInstance}{true}
    }

    %-----------------------------------------------------------------------
    % Setzt einen Schalter, der angibt, dass die Instanz, die die kommende
    % Methode aufrufen kann, veraendert werden darf. Die Methode endet also
    % in der Dokumentation normal mit ``);''.
    %
    \newcommand{\setNormalInstance}{    
        \setboolean{ConstInstance}{false}
    }

    %-----------------------------------------------------------------------
    % Initialisiert die Breite des breitesten Methodenparametertyptextes
    % mit Null.
    %
    \newcommand{\initParamMaxTypeWidth}{
        \setlength{\ParamMaxTypeWidth}{0pt}
    }

    %-----------------------------------------------------------------------
    % Initialisiert die Breite des breitesten Parameternamens mit Null.
    %
    \newcommand{\initMaxParamWidth}{
        \setlength{\MaxParamWidth}{0pt}
    }

    %-----------------------------------------------------------------------
    % Ueberprueft, ob die Breite des uebergebenen Textes des Parametertyps
    % groesser ist als die des bisher als breitester Typtext festgelegten
    % Textes und passt den Wert evtl. an.
    % Parameter #1:  Neuer Typtext, dessen Breite zum Vergleich dient.
    %   
    \newcommand{\setNewParamMaxTypeWidth}[1]{
        \settowidth{\Temp}{#1}
        \ifthenelse{\lengthtest{\ParamMaxTypeWidth < \Temp}}
            {\setlength{\ParamMaxTypeWidth}{\Temp}}{}   
    }

    %-----------------------------------------------------------------------
    % Ueberprueft, ob die Breite des uebergebenen Parameternamens groesser
    % ist als die des bisher als am breitesten geltenden Parameternames
    % und passt den Wert evtl. an. Der Parametername steht dabei in
    % Emphasized.
    % Parameter #1:  Neuer Parametername, dessen Breite zum Vergleich dient.
    %
    \newcommand{\setNewMaxParamWidth}[1]{
        \settowidth{\Temp}{{\em #1}}
        \ifthenelse{\lengthtest{\MaxParamWidth < \Temp}}
            {\setlength{\MaxParamWidth}{\Temp}}{}
    }

    %-----------------------------------------------------------------------
    % Setzt die Breite der Beschreibungstexte fuer Rueckgabewert
    % und Besonderheiten.
    %
    \newcommand{\setSpecialRetTxtWidth}{
        \setlength{\SpecialRetTxtWidth}{\textwidth}
        \addtolength{\SpecialRetTxtWidth}{-1\SpecialWidth} 
    }

    %-----------------------------------------------------------------------
    % Setzt die Breite fuer den Beginn einer Methode, d.h. den Text
    % ``Rueckgabewert Methodenname( ``. Rueckgabewert steht dabei
    % in Emphasized, Methodenname in Bold Font.
    % Parameter #1:  Rueckgabewert der Methode.
    % Parameter #2:  Name der Methode.
    %
    \newcommand{\setMethodBeginWidth}[2]{
        \settowidth{\MethodBeginWidth}
            {{\em #1}\InBetween{\bf #2}\MethodBegin}
    }

    %-----------------------------------------------------------------------
    % Setzt die maximale Breite die an Platz zwischen der oeffnenden
    % und der schliessenden Klammer der Methode zur Verfuegung steht.
    %
    \newcommand{\setBetweenParsWidth}{
        \setlength{\BetweenParsWidth}{\textwidth}
        \addtolength{\BetweenParsWidth}{-1\MethodBeginWidth}
        \addtolength{\BetweenParsWidth}{-1\MethodEndWidth}
    }

    %-----------------------------------------------------------------------
    % Setzt die Breite fuer die Beschreibungstexte
    % der Methodenparameter.
    %
    \newcommand{\setParamDescrWidth}{
        \setlength{\ParamDescrWidth}{\textwidth}
        \addtolength{\ParamDescrWidth}{-1\SpecialWidth} 
        \addtolength{\ParamDescrWidth}{-1\MaxParamWidth}
        \addtolength{\ParamDescrWidth}{-1\ParamDelimWidth}
    }    

    %-----------------------------------------------------------------------
    % Setzt die Korrekturbreite fuer den Beschreibungstext von
    % Methodenparametern auf den uebergebenen Wert.
    % Parameter #1:  Neuer Wert fuer die Korrekturbreite.
    %
    \newcommand{\setCorrectWidthOne}[1]{
        \setlength{\CorrectWidthOne}{#1}
    }    

    %-----------------------------------------------------------------------
    % Setzt die Korrekturbreite fuer den Beschreibungstext von
    % Rueckgabewert und Besonderheiten auf den uebergebenen Wert.
    % Parameter #1:  Neuer Wert fuer die Korrekturbreite.
    %
    \newcommand{\setCorrectWidthTwo}[1]{
        \setlength{\CorrectWidthTwo}{#1}
    }    

    %-----------------------------------------------------------------------
    % Setzt die Korrekturbreite fuer den Abstand des zweiten bis n-ten
    % Parameters vom Anfang auf den uebergebenen Wert.
    % Parameter #1:  Neuer Wert fuer die Korrekturbreite.
    %
    \newcommand{\setCorrectWidthThree}[1]{
        \setlength{\CorrectWidthThree}{#1}
    }    

    %-----------------------------------------------------------------------
    % Setzt den vertikalen Abstand zwischen zwei Parboxen auf den ueber-
    % gebenen Wert.
    % Parameter #1:  Neuer Wert fuer den vertikalen Abstand.
    %
    \newcommand{\setVertSpace}[1]{
        \setlength{\VertSpace}{#1}
    }    

    %-----------------------------------------------------------------------
    % Speichert die uebergebenen Informationen ueber Methodenparameter 1
    % in internen Variablen. Achtung: Die Parameter muessen in der richtigen
    % Reihenfolge gesetzt werden!
    % Parameter #1: Der Typ des Parameters.  
    % Parameter #2: Der Name des Parameters. 
    % Parameter #3: Die Beschreibung des Parameters. 
    % 
    \newcommand{\setParamOne}[3]{
        \renewcommand{\ParamOne}{#1}
        \renewcommand{\ParamTypeOne}{#2}
        \renewcommand{\ParamDescrOne}{#3} 
        \setcounter{SavedParams}{0}
        \stepcounter{SavedParams}
    }

    %-----------------------------------------------------------------------
    % Speichert die uebergebenen Informationen ueber Methodenparameter 2
    % in internen Variablen. Achtung: Die Parameter muessen in der richtigen
    % Reihenfolge gesetzt werden!
    % Parameter #1: Der Typ des Parameters.  
    % Parameter #2: Der Name des Parameters. 
    % Parameter #3: Die Beschreibung des Parameters. 
    % 
    \newcommand{\setParamTwo}[3]{
        \renewcommand{\ParamTwo}{#1}
        \renewcommand{\ParamTypeTwo}{#2}
        \renewcommand{\ParamDescrTwo}{#3} 
        \stepcounter{SavedParams}
    }

    %-----------------------------------------------------------------------
    % Speichert die uebergebenen Informationen ueber Methodenparameter 3
    % in internen Variablen. Achtung: Die Parameter muessen in der richtigen
    % Reihenfolge gesetzt werden!
    % Parameter #1: Der Typ des Parameters.  
    % Parameter #2: Der Name des Parameters. 
    % Parameter #3: Die Beschreibung des Parameters. 
    % 
    \newcommand{\setParamThree}[3]{
        \renewcommand{\ParamThree}{#1}
        \renewcommand{\ParamTypeThree}{#2}
        \renewcommand{\ParamDescrThree}{#3} 
        \stepcounter{SavedParams}
    }

    %-----------------------------------------------------------------------
    % Speichert die uebergebenen Informationen ueber Methodenparameter 4
    % in internen Variablen. Achtung: Die Parameter muessen in der richtigen
    % Reihenfolge gesetzt werden!
    % Parameter #1: Der Typ des Parameters.  
    % Parameter #2: Der Name des Parameters. 
    % Parameter #3: Die Beschreibung des Parameters. 
    % 
    \newcommand{\setParamFour}[3]{
        \renewcommand{\ParamFour}{#1}
        \renewcommand{\ParamTypeFour}{#2}
        \renewcommand{\ParamDescrFour}{#3} 
        \stepcounter{SavedParams}
    }

    %-----------------------------------------------------------------------
    % Speichert die uebergebenen Informationen ueber Methodenparameter 5
    % in internen Variablen. Achtung: Die Parameter muessen in der richtigen
    % Reihenfolge gesetzt werden!
    % Parameter #1: Der Typ des Parameters.  
    % Parameter #2: Der Name des Parameters. 
    % Parameter #3: Die Beschreibung des Parameters. 
    % 
    \newcommand{\setParamFive}[3]{
        \renewcommand{\ParamFive}{#1}
        \renewcommand{\ParamTypeFive}{#2}
        \renewcommand{\ParamDescrFive}{#3} 
        \stepcounter{SavedParams}
    }

    %-----------------------------------------------------------------------
    % Speichert die uebergebenen Informationen ueber Methodenparameter 6
    % in internen Variablen. Achtung: Die Parameter muessen in der richtigen
    % Reihenfolge gesetzt werden!
    % Parameter #1: Der Typ des Parameters.  
    % Parameter #2: Der Name des Parameters. 
    % Parameter #3: Die Beschreibung des Parameters. 
    %     
    \newcommand{\setParamSix}[3]{
        \renewcommand{\ParamSix}{#1}
        \renewcommand{\ParamTypeSix}{#2}
        \renewcommand{\ParamDescrSix}{#3} 
        \stepcounter{SavedParams} 
    }

    %-----------------------------------------------------------------------
    % Speichert die uebergebenen Informationen ueber Methodenparameter 7
    % in internen Variablen. Achtung: Die Parameter muessen in der richtigen
    % Reihenfolge gesetzt werden!
    % Parameter #1: Der Typ des Parameters.  
    % Parameter #2: Der Name des Parameters. 
    % Parameter #3: Die Beschreibung des Parameters. 
    %    
    \newcommand{\setParamSeven}[3]{
        \renewcommand{\ParamSeven}{#1}
        \renewcommand{\ParamTypeSeven}{#2}
        \renewcommand{\ParamDescrSeven}{#3} 
        \stepcounter{SavedParams} 
    }

    %-----------------------------------------------------------------------
    % Speichert die uebergebenen Informationen ueber Methodenparameter 8
    % in internen Variablen. Achtung: Die Parameter muessen in der richtigen
    % Reihenfolge gesetzt werden!
    % Parameter #1: Der Typ des Parameters.  
    % Parameter #2: Der Name des Parameters. 
    % Parameter #3: Die Beschreibung des Parameters. 
    % 
    \newcommand{\setParamEight}[3]{
        \renewcommand{\ParamEight}{#1}
        \renewcommand{\ParamTypeEight}{#2}
        \renewcommand{\ParamDescrEight}{#3} 
        \stepcounter{SavedParams}
    }

    %-----------------------------------------------------------------------
    % Speichert die uebergebenen Informationen ueber Methodenparameter 9
    % in internen Variablen. Achtung: Die Parameter muessen in der richtigen
    % Reihenfolge gesetzt werden!
    % Parameter #1: Der Typ des Parameters.  
    % Parameter #2: Der Name des Parameters. 
    % Parameter #3: Die Beschreibung des Parameters. 
    % 
    \newcommand{\setParamNine}[3]{
        \renewcommand{\ParamNine}{#1}
        \renewcommand{\ParamTypeNine}{#2}
        \renewcommand{\ParamDescrNine}{#3}  
        \stepcounter{SavedParams}
    }

    %-----------------------------------------------------------------------
    % Speichert die uebergebenen Informationen ueber Methodenparameter 10
    % in internen Variablen. Achtung: Die Parameter muessen in der richtigen
    % Reihenfolge gesetzt werden!
    % Parameter #1: Der Typ des Parameters.  
    % Parameter #2: Der Name des Parameters. 
    % Parameter #3: Die Beschreibung des Parameters. 
    % 
    \newcommand{\setParamTen}[3]{
        \renewcommand{\ParamTen}{#1}
        \renewcommand{\ParamTypeTen}{#2}
        \renewcommand{\ParamDescrTen}{#3} 
        \stepcounter{SavedParams}
    }

    %-----------------------------------------------------------------------
    % Gibt den Text ``Rueckgabewert Methodenname(`` aus.
    % Parameter #1:  Rueckgabewert der Methode.
    % Parameter #2:  Name der Methode.
    %
    \newcommand{\printMethodBegin}[2]{
         \ifthenelse{\boolean{ConstInstance}}
             {\settowidth{\MethodEndWidth}{\ConstMethodEnd}}
             {\settowidth{\MethodEndWidth}{\NormMethodEnd}}
         \setMethodBeginWidth{#1}{#2}
         \setBetweenParsWidth
         \noindent
         \hrulefill\\
         \noindent
         \parbox[t]{\MethodBeginWidth}
             {#1\InBetween{\bf #2}\MethodBegin}
     }

    %-----------------------------------------------------------------------
    % Gibt das Ende einer Methode aus, d.h. ``\ );'' oder ``\ ) const;''.
    %
    \newcommand{\printMethodEnd}{
        \ifthenelse{\boolean{ConstInstance}}
            {\ConstMethodEnd\\}
            {\NormMethodEnd\\}
            \mbox{}\hrulefill\\
    }

    %-----------------------------------------------------------------------
    % Gibt den Typen und Namen des einzigen Parameters einer Methode
    % gefolgt von dem Text ``  );'' bzw. `` ) const;'' aus.
    % Parameter #1:  Typ des Parameters.
    % Parameter #2:  Name des Parameters.
    % 
    \newcommand{\printMethodOneParam}[2]{
        \parbox[t]{\ParamMaxTypeWidth}{#1}
        \parbox[t]{\MaxParamWidth}{{\em #2}}
        \printMethodEnd
    }

    %-----------------------------------------------------------------------
    % Gibt den Typen und den Namen des ersten Methodenparameters aus.
    % Parameter #1:  Typ des Parameters.
    % Parameter #2:  Name des Parameters.
    %
    \newcommand{\printMethodFirstParam}[2]{
        \noindent
        \parbox[t]{\ParamMaxTypeWidth}{#1}
        \parbox[t]{\MaxParamWidth}{{\em #2},}\\[\VertSpace]
    }

    %-----------------------------------------------------------------------
    % Gibt den Typen und Namen des letzten Parameters einer Methode aus.
    % Parameter #1: Typ des Parameters.
    % Parameter #2: Name des Parameters.
    %
    \newcommand{\printMethodLastParam}[2]{
        \parbox[t]{\MethodBeginWidth}{\hfill}
        \hspace{\CorrectWidthThree}
        \parbox[t]{\ParamMaxTypeWidth}{#1}
        \parbox[t]{\MaxParamWidth}{{\em #2}}
        \printMethodEnd
    }

    %-----------------------------------------------------------------------
    % Gibt den Typen und Namen eines Methodenparameters aus.
    % Parameter #1:  Typ des Parameters.
    % Parameter #2: Name des Parameters.
    % Achtung: Nicht fuer den ersten oder letzten Parameter der Methode
    % geeignet!
    %
    \newcommand{\printMethodParam}[2]{
        \parbox[t]{\MethodBeginWidth}{\hfill}
        \hspace{\CorrectWidthThree}
        \parbox[t]{\ParamMaxTypeWidth}{#1}
        \parbox[t]{\MaxParamWidth}{{\em #2},}\\[\VertSpace]
    }

    %-----------------------------------------------------------------------
    % Gibt die Beschreibung einer Methode aus.
    % Parameter #1: Beschreibung der Methode.
    %
    \newcommand{\printMethodDescr}[1]{
        \newline
        \noindent
        #1\\[\VertSpace]
    }

    %-----------------------------------------------------------------------
    % Gibt den Text ``Parameter:    `` und danach den Namen des
    % ersten Parameters der Methode, gefolgt von seiner Beschreibung
    % aus.
    % Parameter #1:  Name des Parameters. 
    % Parameter #2:  Beschreibung des Parameters.
    %
    \newcommand{\printFirstParam}[2]{
        \setlength{\Temp}{\ParamDescrWidth}
        \addtolength{\Temp}{1\CorrectWidthOne}
        \parbox[t]{\SpecialWidth}{{\bf Parameters:}}
        \parbox[t]{\MaxParamWidth}{\em #1}
        \parbox[t]{\ParamDelimWidth}{\ParamDelim}
        \parbox[t]{\Temp}{#2}\\[\VertSpace]
    }

    %-----------------------------------------------------------------------
    % Gibt einen Parameter der Methode und die dazugehoerige Beschreibung
    % aus.
    % Parameter #1:  Name des Parameters.
    % Parameter #2:  Beschreibung des Parameters.
    %
    \newcommand{\printParam}[2]{
        \setlength{\Temp}{\ParamDescrWidth}
        \addtolength{\Temp}{1\CorrectWidthOne}
        \parbox[t]{\SpecialWidth}{\hfill}
        \parbox[t]{\MaxParamWidth}{\em #1}
        \parbox[t]{\ParamDelimWidth}{\ParamDelim}
        \parbox[t]{\Temp}{#2}\\[\VertSpace]
    }

    %-----------------------------------------------------------------------
    %
    % Gibt den Text ``Rueckgabewert:  `` gefolgt von einer Beschreibung
    % des Rueckgabewertes aus.
    % Parameter #1:  Beschreibung des Rueckgabewertes.
    %
    \newcommand{\printReturn}[1]{
        \setlength{\Temp}{\SpecialRetTxtWidth}
        \addtolength{\Temp}{1\CorrectWidthTwo}
        \parbox[t]{\SpecialWidth}{{\bf Return Value:}}
        \parbox[t]{\Temp}{#1}\\[\VertSpace]
    }

    %-----------------------------------------------------------------------
    % Gibt den Text ``Besonderheiten:  `` gefolgt von einer Beschreibung
    % der Besonderheiten aus.
    % Parameter #1:  Beschreibung der Besonderheiten.
    %
    \newcommand{\printSpecial}[1]{
        \setlength{\Temp}{\SpecialRetTxtWidth}
        \addtolength{\Temp}{1\CorrectWidthTwo}
        \parbox[t]{\SpecialWidth}{\bf Caveats:}
        \parbox[t]{\Temp}{#1}\\[\VertSpace]
    }

    %-----------------------------------------------------------------------
    % Gibt den Text ``Parameter:        Keine.'' aus.
    %
    \newcommand{\printNoParams}{
        \parbox[t]{\SpecialWidth}{{\bf Parameters:}}
        \parbox[t]{\NoneWidthOne}{None.}\\[\VertSpace]
    }

    %-----------------------------------------------------------------------
    % Gibt den Text ``Rueckgabewert:  Keiner.'' aus.
    %
    \newcommand{\printNoReturn}{
        \parbox[t]{\SpecialWidth}{{\bf Return Value:}}
        \parbox[t]{\NoneWidthTwo}{None.}\\[\VertSpace]
    }

    %-----------------------------------------------------------------------
    % Gibt den Text ``Besonderheiten:  Keine.'' aus.
    %
    \newcommand{\printNoSpecial}{
        \parbox[t]{\SpecialWidth}{\bf Caveats:}
        \parbox[t]{\NoneWidthOne}{None.}\\[\VertSpace]
    }

    %-----------------------------------------------------------------------
    % Gibt den Text:
    % ``Parameter:       Keine.
    %   Rueckgabewert:   Keiner.
    %   Besonderheiten:  Keine.''
    % aus.
    %
    \newcommand{\printNoAll}{
        \printNoParams
        \printNoReturn
        \printNoSpecial
    }

    %-----------------------------------------------------------------------
    % Gibt eine Methode ohne Parameter, Rueckgabewert und Besonderheiten
    % aus.
    % Parameter #1:  Name der Methode.
    % Parameter #2:  Beschreibung der Methode.
    %
    \newcommand{\printEmptyMethod}[2]{
        \setSpecialRetTxtWidth
        \printMethodBegin{void}{#1}
        \printMethodEnd
        \printMethodDescr{#2}
        \printNoAll
    }

    %-----------------------------------------------------------------------
    % Gibt eine Methode ohne Parameter und Besonderheiten, aber mit
    % Rueckgabewert aus.
    % Parameter #1:  Rueckgabewert der Methode.
    % Parameter #2:  Name der Methode.
    % Parameter #3:  Beschreibung der Methode.
    % Parameter #4:  Beschreibung des Rueckgabewertes der Methode.
    %
    \newcommand{\printEmptyMethodReturn}[4]{
        \setSpecialRetTxtWidth
        \printMethodBegin{#1}{#2}
        \printMethodEnd
        \printMethodDescr{#3}
        \printNoParams
        \printReturn{#4}
        \printNoSpecial
    }

    %-----------------------------------------------------------------------
    % Gibt eine Methode ohne Parameter und Rueckgabewert, aber mit
    % Besonderheiten aus.
    % Parameter #1:  Name der Methode.
    % Parameter #2:  Beschreibung der Methode.
    % Parameter #3:  Beschreibung der Besonderheiten der Methode.
    %
    \newcommand{\printEmptyMethodSpecial}[3]{
        \setSpecialRetTxtWidth
        \printMethodBegin{void}{#1}
        \printMethodEnd
        \printMethodDescr{#2}
        \printNoParams
        \printNoReturn
        \printSpecial{#3}
    }

    %-----------------------------------------------------------------------
    % Gibt eine Methode ohne Parameter, aber mit Rueckgabewert
    % und Besonderheiten aus.
    % Parameter #1:  Rueckgabewert der Methode.
    % Parameter #2:  Name der Methode.
    % Parameter #3:  Beschreibung der Methode.
    % Parameter #4:  Beschreibung des Rueckgabewertes der Methode.
    % Parameter #5:  Beschreibung der Besonderheiten der Methode.
    %
    \newcommand{\printEmptyMethodReturnSpecial}[5]{
        \setSpecialRetTxtWidth
        \printMethodBegin{#1}{#2}
        \printMethodEnd
        \printMethodDescr{#3}
        \printNoParams
        \printReturn{#4}
        \printSpecial{#5}
    }

    %-----------------------------------------------------------------------
    % Gibt eine Methode mit einem Parameter komplett mit allen
    % Beschreibungen aus.
    % Parameter #1:  Rueckgabewert der Methode.
    % Parameter #2:  Name der Methode.
    % Parameter #3:  Typ des einzigen Parameters der Methode.
    % Parameter #4:  Name des einzigen Parameters der Methode.
    % Parameter #5:  Beschreibung des einzigen Parameters der Methode.
    % Parameter #6:  Beschreibung der Methode.
    % Parameter #7:  Beschreibung des Rueckgabewertes der Methode.
    % Parameter #8:  Beschreibung der Besonderheiten der Methode.
    %
    \newcommand{\printMethodWithOneParam}[8]{
        \initParamMaxTypeWidth
        \initMaxParamWidth
        \setNewParamMaxTypeWidth{#3}
        \setNewMaxParamWidth{#4}
        \setSpecialRetTxtWidth
        \setParamDescrWidth
        \printMethodBegin{#1}{#2}
        \printMethodOneParam{#3}{#4}
        \printMethodDescr{#6}
        \printFirstParam{#4}{#5}
        \ifthenelse{\equal{#7}{Keiner.}}
            {\printNoReturn}
            {\printReturn{#7}}
        \ifthenelse{\equal{#8}{Keine.}}
            {\printNoSpecial}
            {\printSpecial{#8}}
    }

    %-----------------------------------------------------------------------
    % Gibt eine Methode mit einem Parameter komplett mit allen
    % Beschreibungen aus.
    % Parameter #1:  Rueckgabewert der Methode.
    % Parameter #2:  Beschreibung des Rueckgabewertes der Methode oder 
    %                Aufruf mit leerer Klammer. 
    % Parameter #3:  Name der Methode.
    % Parameter #4:  Beschreibung der Methode.
    % Parameter #5:  Beschreibung der Besonderheiten der Methode oder 
    %                Aufruf mit leerer Klammer.
    %
    \newcommand{\printMethodWithParamsSaved}[5]{
        \ifthenelse{\value{SavedParams} > 0}
            {\initParamMaxTypeWidth
             \initMaxParamWidth
             \setNewParamMaxTypeWidth{\ParamTypeOne}
             \setNewMaxParamWidth{\ParamOne}
             \ifthenelse{\value{SavedParams} = 2 \or \value{SavedParams} > 2}
                 {\setNewParamMaxTypeWidth{\ParamTypeTwo}
                  \setNewMaxParamWidth{\ParamTwo}}{}   
             \ifthenelse{\value{SavedParams} = 3 \or \value{SavedParams} > 3}
                 {\setNewParamMaxTypeWidth{\ParamTypeThree}
                  \setNewMaxParamWidth{\ParamThree}}{}    
             \ifthenelse{\value{SavedParams} = 4 \or \value{SavedParams} > 4}
                 {\setNewParamMaxTypeWidth{\ParamTypeFour}
                  \setNewMaxParamWidth{\ParamFour}}{}
             \ifthenelse{\value{SavedParams} = 5 \or \value{SavedParams} > 5}
                 {\setNewParamMaxTypeWidth{\ParamTypeFive}
                  \setNewMaxParamWidth{\ParamFive}}{}      
             \ifthenelse{\value{SavedParams} = 6 \or \value{SavedParams} > 6}
                 {\setNewParamMaxTypeWidth{\ParamTypeSix}
                  \setNewMaxParamWidth{\ParamSix}}{}
             \ifthenelse{\value{SavedParams} = 7 \or \value{SavedParams} > 7}
                 {\setNewParamMaxTypeWidth{\ParamTypeSeven}
                  \setNewMaxParamWidth{\ParamSeven}}{}       
             \ifthenelse{\value{SavedParams} = 8 \or \value{SavedParams} > 8}
                 {\setNewParamMaxTypeWidth{\ParamTypeEight}
                  \setNewMaxParamWidth{\ParamEight}}{}
             \ifthenelse{\value{SavedParams} = 9 \or \value{SavedParams} > 9}
                 {\setNewParamMaxTypeWidth{\ParamTypeNine}
                  \setNewMaxParamWidth{\ParamNine}}{}
             \ifthenelse{\value{SavedParams} = 10 \or 
                         \value{SavedParams} > 10}
                 {\setNewParamMaxTypeWidth{\ParamTypeTen}
                  \setNewMaxParamWidth{\ParamTen}}{}         
             \setSpecialRetTxtWidth
             \setParamDescrWidth
             \printMethodBegin{#1}{#3}
             \ifthenelse{\value{SavedParams} = 1} 
                 {\printMethodOneParam{\ParamTypeOne}{\ParamOne}}{}
                 {\printMethodFirstParam{\ParamTypeOne}{\ParamOne}}{}
             \ifthenelse{\value{SavedParams} = 2} 
                 {\printMethodLastParam{\ParamTypeTwo}{\ParamTwo}}{}
             \ifthenelse{\value{SavedParams} > 2}  
                 {\printMethodParam{\ParamTypeTwo}{\ParamTwo}}{} 
             \ifthenelse{\value{SavedParams} = 3} 
                 {\printMethodLastParam{\ParamTypeThree}{\ParamThree}}{}
             \ifthenelse{\value{SavedParams} > 3}  
                 {\printMethodParam{\ParamTypeThree}{\ParamThree}}{} 
             \ifthenelse{\value{SavedParams} = 4} 
                 {\printMethodLastParam{\ParamTypeFour}{\ParamFour}}{}
             \ifthenelse{\value{SavedParams} > 4}  
                 {\printMethodParam{\ParamTypeFour}{\ParamFour}}{} 
             \ifthenelse{\value{SavedParams} = 5} 
                 {\printMethodLastParam{\ParamTypeFive}{\ParamFive}}{}
             \ifthenelse{\value{SavedParams} > 5}  
                 {\printMethodParam{\ParamTypeFive}{\ParamFive}}{} 
             \ifthenelse{\value{SavedParams} = 6} 
                 {\printMethodLastParam{\ParamTypeSix}{\ParamSix}}{}
             \ifthenelse{\value{SavedParams} > 6}  
                 {\printMethodParam{\ParamTypeSix}{\ParamSix}}{} 
             \ifthenelse{\value{SavedParams} = 7} 
                 {\printMethodLastParam{\ParamTypeSeven}{\ParamSeven}}{}
             \ifthenelse{\value{SavedParams} > 7}  
                 {\printMethodParam{\ParamTypeSeven}{\ParamSeven}}{} 
             \ifthenelse{\value{SavedParams} = 8} 
                 {\printMethodLastParam{\ParamTypeEight}{\ParamEight}}{}
             \ifthenelse{\value{SavedParams} > 8}  
                 {\printMethodParam{\ParamTypeEight}{\ParamEight}}{}  
             \ifthenelse{\value{SavedParams} = 9} 
                 {\printMethodLastParam{\ParamTypeNine}{\ParamNine}}{}
             \ifthenelse{\value{SavedParams} > 9}  
                 {\printMethodParam{\ParamTypeNine}{\ParamNine}}{} 
             \ifthenelse{\value{SavedParams} = 10} 
                 {\printMethodLastParam{\ParamTypeTen}{\ParamTen}}{}
             \printMethodDescr{#4}
             \printFirstParam{\ParamOne}{\ParamDescrOne}
             \ifthenelse{\value{SavedParams} = 2 \or \value{SavedParams} > 2}
                  {\printParam{\ParamTwo}{\ParamDescrTwo}}{}
             \ifthenelse{\value{SavedParams} = 3 \or \value{SavedParams} > 3}
                  {\printParam{\ParamThree}{\ParamDescrThree}}{}
             \ifthenelse{\value{SavedParams} = 4 \or \value{SavedParams} > 4}
                  {\printParam{\ParamFour}{\ParamDescrFour}}{}
             \ifthenelse{\value{SavedParams} = 5 \or \value{SavedParams} > 5}
                  {\printParam{\ParamFive}{\ParamDescrFive}}{}
             \ifthenelse{\value{SavedParams} = 6 \or \value{SavedParams} > 6}
                  {\printParam{\ParamSix}{\ParamDescrSix}}{}
             \ifthenelse{\value{SavedParams} = 7 \or \value{SavedParams} > 7}
                  {\printParam{\ParamSeven}{\ParamDescrSeven}}{}
             \ifthenelse{\value{SavedParams} = 8 \or \value{SavedParams} > 8}
                  {\printParam{\ParamEight}{\ParamDescrEight}}{} 
             \ifthenelse{\value{SavedParams} = 9 \or \value{SavedParams} > 9}
                  {\printParam{\ParamNine}{\ParamDescrNine}}{}
             \ifthenelse{\value{SavedParams} = 10}
                  {\printParam{\ParamTen}{\ParamDescrTen}}{}
             \ifthenelse{\equal{#2}{}}
                 {\printNoReturn}
                 {\printReturn{#2}}
             \ifthenelse{\equal{#5}{}}
                 {\printNoSpecial}
                 {\printSpecial{#5}}}
        {\noindent
         Fehler Methode {\em printMethodWithParamsSaved}: Es sind noch keine 
         Parameter gesetzt worden!}
    }

%-----------------------------------------------------------------------
% Ende der Befehlsdefinitionen.
%-----------------------------------------------------------------------





%%% Local Variables: 
%%% mode: latex
%%% TeX-master: t
%%% End: 
