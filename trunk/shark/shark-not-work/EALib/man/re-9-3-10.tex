\subsection{Selection Methods}

\subsubsection{Deterministic Selection}

%---------------------------------------------------------------------------%
\index{selectMuLambda!( Population\& parents, unsigned numElitists = 0 )}
\setNormalInstance
\setCorrectWidthThree{8pt}
\setParamOne{parents}{Population\&}{Parent population.}
\setParamTwo{numElitists}{unsigned}{Number of elitists to be
taken over from both populations. The default value is 0.}
\printMethodWithParamsSaved
{void}
{}
{selectMuLambda}
{Selects individuals from the parent population {\em parents} and/or
 from the population {\em this} for reproduction. {\em numElitists}
 elite individuals will be taken over from {\em parents} and {\em this}.
 Is {\em numElitists} set to zero, then all parent individuals will be
 dismissed; this is equal to a ($\mu$, $\lambda$)-selection. Is
 {\em numElitists} greater than zero, then this number of elitists
 will be taken over from both populations; this is equal to a 
 ($\mu + \lambda$)-selektion.
 For more information cf. \cite{Baeck}.}
{}
\setCorrectWidthThree{4pt}
%---------------------------------------------------------------------------%

\clearpage

%---------------------------------------------------------------------------%
\index{selectMuLambdaKappa!( Population\& parents, unsigned lifespan = 1, unsigned adolescence = 0 )}
\setNormalInstance
\setCorrectWidthThree{8pt}
\setParamOne{parents}{Population\&}{The parent population.}
\setParamTwo{lifespan}{unsigned}{Maximum lifespan of an individual, the
default value is 1.}
\setParamThree{adolescence}{unsigned}{An individual must have fallen
short of this age to be in the set of ``youngest'' individuals.
The default value is 0.}
\printMethodWithParamsSaved
{void}
{}
{selectMuLambdaKappa}
{Selects some individuals from the population {\em this} and its
 parent population {\em parents} for reproduction. The ``youngest''
 individuals will be chosen first, i.e. those individuals,
 whose age is less than the value of {\em adolescence}. Second,
 those individuals will be selected, which do not belong to the
 set of ``youngest'', but whose lifespan has not reached the value 
 {\em lifespan} yet. Older individuals will be dismissed.}
{}
\setCorrectWidthThree{4pt}
%---------------------------------------------------------------------------%

\subsubsection{Proportional Selection}

%---------------------------------------------------------------------------%
\index{selectOneIndividual!( )}
\setNormalInstance
\printEmptyMethodReturnSpecial
{Individual\&}
{selectOneIndividual}
{Selects an individual for reproduction by using proportional
 selection, i.e. the selection probability of all
 individuals will be adapted to scaled fitness values. After that,
 an individual will be chosen by calling the method
 {\em selectRouletteWheel}.\\
 For more information cf. \cite{EALibRef}.}
{The selected individual.}
{The fitness values of all individuals must be positive.}
%---------------------------------------------------------------------------%

\clearpage

%---------------------------------------------------------------------------%
\index{selectProportional!( Population\& parents, unsigned numElitists = 0 )}
\setNormalInstance
\setCorrectWidthThree{8pt}
\setParamOne{parents}{Population\&}{The parent population.}
\setParamTwo{numElitists}{unsigned}{Number of elitists to be
taken over. If {\em numElitists} is zero, then a proportional
selection takes place. If the value is greater than zero, {\em numElitists}
elitists will be taken over from both populations and the rest will
be selected from {\em parents}.}
\printMethodWithParamsSaved
{void}
{}
{selectProportional}
{Selects individuals from the parent population {\em parents} and
 the population {\em this} for reproduction using the proportional
 selection, i.e. all fitness values will be adapted in a way
 that they are proportional to the scaled fitness values. After that, the
 individuals are chosen for reproduction by calling the method
 {\em selectRouletteWheel}.\\
 For more information cf. \cite{EALibRef}.}
{The fitness values of all individuals of the population must be
 positive.}
\setCorrectWidthThree{4pt}
%---------------------------------------------------------------------------%

\subsubsection{Selection by Ranking}

%---------------------------------------------------------------------------%
\index{selectLinearRanking!( Population\& parents, double etaMax = 1.1, unsigned numElitists = 0 )}
\setNormalInstance
\setCorrectWidthThree{8pt}
\setParamOne{parents}{Population\&}{The parent population.}
\setParamTwo{etaMax}{double}{Maximum reproduction rate, the default value
is $1.1$.}
\setParamThree{numElitists}{unsigned}{Number of elitists to be taken
over from both populations. The default is set to 0.}
\printMethodWithParamsSaved
{void}
{}
{selectLinearRanking}
{Selects individuals from the population {\em this} for reproduction by 
 using the method of {\em Linear Ranking}, i.e. each individual
 receives a selection probability based upon its ``ranking''
 inside the population. The ranking is determined by the order of
 the individuals refering to descending fitness values.
 After evaluating the probabilities, the individuals are selected
 using the method {\em selectRouletteWheel}.
 Additionally {\em numElitists} elitists
 can be taken over from the previous population and the parent
 population.\\
 For more information cf. \cite{Baker} and \cite{AlgoWork}.}
{}
\setCorrectWidthThree{4pt}
%---------------------------------------------------------------------------%

\clearpage

%---------------------------------------------------------------------------%
\index{selectLinearRankingWhitley!( Population\& parents, double a = 1.1, unsigned numElitists = 0 )} 
\setNormalInstance
\setCorrectWidthThree{8pt}
\setParamOne{parents}{Population\&}{The parent population.}
\setParamTwo{a}{double}{Maximum reproduction rate, the default value
is set to $1.1$.}
\setParamThree{numElitists}{unsigned}{Number of elititst to be 
 taken over from both populations. The default is 0.}
\printMethodWithParamsSaved
{void}
{}
{selectLinearRankingWhitley}
{Same as above, but here the evaluation of the selection
 probability differs.\\
 For more information cf. \cite{Whitley}.}
{None.}
\setCorrectWidthThree{4pt}
%---------------------------------------------------------------------------%

\vspace*{4ex}

%---------------------------------------------------------------------------%
\index{selectUniformRanking!( Population\& parents, unsigned numElitists = 0 )}
\setNormalInstance
\setCorrectWidthThree{8pt}
\setParamOne{parents}{Population\&}{Parent population, from which
elititsts can be taken over.}
\setParamTwo{numElitists}{unsigned}{Number of elititsts. The default value 
is 0.}
\printMethodWithParamsSaved
{void}
{}
{selectUniformRanking}
{Selects individuals from the population {\em this} for reproduction
 using the method of {\em selectUniformRanking}, i.e.
 each individual receives the same selection probability and the method
 {\em selectRouletteWheel} is used for selection.
 Additionally {\em numElitists} elitits can be taken over from the
 previous population and the parent population {\em parents}.
 For more information cf. \cite{EALibRef}.}
{}
\setCorrectWidthThree{4pt}
%---------------------------------------------------------------------------%

\vspace*{4ex}

%---------------------------------------------------------------------------%
\index{reproduce!( Population\& parents, unsigned numElitists = 0 )}
\setNormalInstance
\setCorrectWidthThree{8pt}
\setParamOne{parents}{Population\&}{See above.}
\setParamTwo{numElitists}{unsigned}{See above.}
\printMethodWithParamsSaved
{void}
{}
{reproduce}
{Same as method {\em selectUniformRanking}.}
{}
\setCorrectWidthThree{4pt}
%---------------------------------------------------------------------------%

\clearpage

\subsubsection{Tournament Selection}

%---------------------------------------------------------------------------%
\index{selectTournament!( Population\& parents, unsigned q = 2, unsigned numElitists = 0 )}
\setNormalInstance
\setCorrectWidthThree{8pt}
\setParamOne{parents}{Population\&}{The parent population.}
\setParamTwo{q}{unsigned}{Number of tournament opponents, the default
number is 2.}
\setParamThree{numElitists}{unsigned}{Number of elitists to be
taken over from both populations, the default value is 0.}
\printMethodWithParamsSaved
{void}
{}
{selectTournament}
{Selects the best individual out of a randomly chosen group of {\em q}
 individuals from the parent population {\em parents} and takes it over
 to the new population {\em this}. This selection is repeated as long as
 there are free slots in the new population. Additionally
 {\em numElitists} elitists can be taken over from {\em parents} and
 {\em this}. No correction of the selection probabilities for elitists
 will take place.\\
 For more information cf. \cite{Goldberg}.}
{}
\setCorrectWidthThree{4pt}
%---------------------------------------------------------------------------%

\vspace*{4ex}

%---------------------------------------------------------------------------%
\index{selectEPTournament!( Population\& parents, unsigned q )}
\setNormalInstance
\setCorrectWidthThree{8pt}
\setParamOne{parents}{Population\&}{The parent population.}
\setParamTwo{q}{unsigned}{Number of opponents for each individual.}
\printMethodWithParamsSaved
{void}
{}
{selectEPTournament}
{Selects individuals from the parent population {\em parents} and
 the population {\em this} by using the {\em EP-style Tournament Selection}
 by D. B. Fogel. Each individual of both populations has
 to compete against {\em q} randomly chosen individuals of the
 populations {\em this} and {\em parents}. An individual wins a round,
when it has collected more wins during the last round, i.e.
when its fitness value was better than the value of its opponent. If
two individuals have the same number of wins, then the individual with
the better fitness value wins.\\
For more information cf. \cite{Fogel}.}
{}
\setCorrectWidthThree{4pt}
%---------------------------------------------------------------------------%


