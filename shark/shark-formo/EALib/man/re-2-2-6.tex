\subsection{Structure Changing Methods}

%---------------------------------------------------------------------------%
\index{resize!( unsigned n )}
    \setNormalInstance
    \printMethodWithOneParam
    {void} 
    {resize}
    {unsigned} 
    {n} 
    {New number of alleles of the chromosome.}
    {Changes the size of {\em this} (i.e. the number
     of alleles contained in {\em this}) to the new value {\em n}.
     If $n < |this|$, then $|this| - n$ alleles at the end
     of {\em this} will be deleted. If $n > |this|$, then
     $n - |this|$ new and empty alleles will be appended at the
     end of {\em this}.}
    {None.}
    {None.}
%---------------------------------------------------------------------------%

\clearpage

%---------------------------------------------------------------------------%
\index{duplicate!( unsigned start, unsigned stop, unsigned dest )}
    \setCorrectWidthThree{8pt}
    \setParamOne{start}{unsigned}{Index of the first allele of the 
sequence that shall be copied. {\em start} must be less than 
the number of alleles in {\em this}, otherwise the method will be
aborted with an error message.}
    \setParamTwo{stop}{unsigned}{Index of the last allele of the 
sequence that shall be copied. {\em stop} must be less than 
the number of alleles in {\em this}, otherwise the method will be
aborted with an error message.}
    \setParamThree{dest}{unsigned}{Position to where the sequence of
alleles shall be copied. {\em dest} must be less than 
the number of alleles in {\em this}, otherwise the method will be
aborted with an error message.}
    \printMethodWithParamsSaved
        {void}
        {}
        {duplicate}
        {Copies the sequence of alleles from index {\em start} until index
         {\em stop} to position {\em dest}. Because the chromosome
         is assumed to be cyclic for this operation, {\em start} can be 
         greater than {\em stop}.}
        {}
    \setCorrectWidthThree{4pt}
%---------------------------------------------------------------------------%

\vspace*{4ex}

%---------------------------------------------------------------------------%
\index{invert!( unsigned start, unsigned stop, unsigned granularity )}
    \setCorrectWidthThree{8pt}
    \setParamOne{start}{unsigned}{Index of the first allele of the
sequence that shall be inverted. {\em start} must be less than the
number of alleles in {\em this}, otherwise the method will be aborted
with an error message.}
    \setParamTwo{stop}{unsigned}{Index of the last allele of the
sequence that shall be inverted. {\em stop} must be less than the
number of alleles in {\em this}, otherwise the method will be aborted
with an error message.}
    \setParamThree{granularity}{unsigned}{Size of one allele block inside
the allele sequence, the default value is ``1''.}
    \printMethodWithParamsSaved
        {void}
        {}
        {invert}
        {Inverts the order of allele blocks of size {\em granularity} 
         inside the allele sequence from index {\em start} to index
         {\em stop}. Because the chromosome is assumed to be cyclic
         for this operation,
         {\em start} can be greater than {\em stop}.
         Example: {\em this} $= < 1, 2, 3, 4, 5, 6, 7, 8, 9 >$,
         {\em start} $= 0$, {\em stop} $= 8$, {\em granularity} $= 3$,
         chromosome after calling the method: {\em this}\ 
         $= < 7, 8, 9, 4, 5, 6, 1, 2, 3 >$. {\em granularity} is set to
         ``3'', so the order of the numbers inside blocks of size three
         (``1'' to ``3'', ``4'' to ``6'',
         ``7'' to ``9'') remains unchanged, only the order of the blocks
         themselves will be inverted.}
        {}
    \setCorrectWidthThree{4pt}
%---------------------------------------------------------------------------%

\vspace*{4ex}

%---------------------------------------------------------------------------%
\index{invert!( unsigned granularity )}
    \printMethodWithOneParam
    {void} 
    {invert}
    {unsigned} 
    {granularity}
    {See above.}
    {Same as above, but here the allele sequence corresponds to
     the whole chromosome {\em this}.}
    {None.}
    {None.}
%---------------------------------------------------------------------------%

\vspace*{4ex}

%---------------------------------------------------------------------------%
\index{transcribe!( unsigned start, unsigned stop, const Chromosome\& chrom )}
    \setCorrectWidthThree{8pt}
    \setParamOne{start}{unsigned}{Index of the first allele of the
sequence inside {\em chrom} that shall be transcribed to {\em this}.
{\em start} must be less than the number of alleles in {\em this},
otherwise the method will be aborted with an error message.}
    \setParamTwo{stop}{unsigned}{Index of the last allele of the
sequence inside {\em chrom} that shall be transcribed to {\em this}.
{\em stop} must be less than the number of alleles in {\em this},
otherwise the method will be aborted with an error message.}
    \setParamThree{chrom}{const Chromosome\&}{Chromosome that contains the
allele sequence that shall be transcribed to {\em this}.}
    \printMethodWithParamsSaved
        {void}
        {}
        {transcribe}
        {Transcribes the allele sequence from index {\em start} to index
{\em stop} from chromosome {\em chrom} to chromosome {\em this}. Therefore
the size of {\em this} will be changed to the size of the allele sequence.
Because {\em chrom} is assumed to be cyclic for this operation, 
{\em start} can be greater than {\em stop}.}
        {}
    \setCorrectWidthThree{4pt}
%---------------------------------------------------------------------------%

\clearpage

%---------------------------------------------------------------------------%
\index{swap!( unsigned i, unsigned j )}
    \setCorrectWidthThree{8pt}
    \setParamOne{i}{unsigned}{First allele that shall change its position.
{\em i} must be less than the number of alleles in {\em this}, otherwise
the method will be aborted with an error message.}
    \setParamTwo{j}{unsigned}{Second allele that shall change its position.
{\em j} must be less than the number of alleles in {\em this}, otherwise
the method will be aborted with an error message.}
    \printMethodWithParamsSaved
        {void}
        {}
        {swap}
        {Exchanges allele at index {\em i} with the one at index {\em j}.}
        {}
    \setCorrectWidthThree{4pt}
%---------------------------------------------------------------------------%

\vspace*{4ex}

%---------------------------------------------------------------------------%
\index{shuffle!( )}
    \printEmptyMethod
    {shuffle}
    {Randomly changes the order of the alleles inside {\em this}.}
%---------------------------------------------------------------------------%

\vspace*{4ex}

%---------------------------------------------------------------------------%
\index{replace!( unsigned i, const T\& v )}
    \setCorrectWidthThree{8pt}
    \setParamOne{i}{unsigned}{Index of the allele from which the value shall be
changed. {\em i} must be less than the number of alleles in {\em this},
otherwise the method will be aborted with an error message.}
    \setParamTwo{v}{const {\sl T}\&}{New value for the allele.}
    \printMethodWithParamsSaved
        {void}
        {}
        {replace}
        {Replaces the value of allele number {\em i} with {\em v}.}
        {}
    \setCorrectWidthThree{4pt}
%---------------------------------------------------------------------------%

\clearpage

%---------------------------------------------------------------------------%
\index{replace!( unsigned i, const Chromosome\& chrom )}
    \setCorrectWidthThree{8pt}
    \setParamOne{i}{unsigned}{Index of the first allele of {\em this} that
shall be replaced with the alleles of {\em chrom}. {\em i} must be less than
the number of alleles in {\em this}, otherwise the method will be aborted
with an error message.}
    \setParamTwo{chrom}{const Chromosome\&}{Chromosome that shall 
replace some alleles of {\em this}. The size of {\em chrom} must be less
than the number of alleles in {\em this} minus {\em i}, otherwise the
method will be aborted with an error message.}
    \printMethodWithParamsSaved
        {void}
        {}
        {replace}
        {Replaces all alleles of {\em this}, beginning with allele
         number {\em i}, with the alleles inside the chromosome {\em chrom}.}
        {}
    \setCorrectWidthThree{4pt}
%---------------------------------------------------------------------------%

\vspace*{4ex}

%---------------------------------------------------------------------------%
\index{insert!( unsigned i, const T\& allele )}
    \setCorrectWidthThree{8pt}
    \setParamOne{i}{unsigned}{Position in {\em this}, where the new value
shall be inserted. {\em i} must be less than the number of alleles
in {\em this}, otherwise the method will be aborted with an error message.}
    \setParamTwo{allele}{const {\sl T}\&}{Value that shall be inserted in
chromosome {\em this}.}
    \printMethodWithParamsSaved
        {void}
        {}
        {insert}
        {Inserts the value {\em allele} at index {\em i} in chromosome
         {\em this}.}
        {}
    \setCorrectWidthThree{4pt}
%---------------------------------------------------------------------------%

\vspace*{4ex}    

%---------------------------------------------------------------------------%
\index{insert!( unsigned i, const Chromosome\& chrom )}
    \setCorrectWidthThree{8pt}
    \setParamOne{i}{unsigned}{Position in {\em this}, where the alleles of
{\em chrom} shall be inserted. {\em i} must be less than the number of
alleles in {\em this}, otherwise the method will be aborted with an
error message.}
    \setParamTwo{chrom}{const Chromosome\&}{Chromosome, which alleles shall
be inserted in chromosome {\em this}.}

    \printMethodWithParamsSaved
        {void}
        {}
        {insert}
        {Inserts all alleles of chromosome {\em chrom} at position {\em i}
         in chromosome {\em this}.}
        {This method does not work on systems using a gcc-compiler 
         in version less than 2.91.x.}
    \setCorrectWidthThree{4pt}
%---------------------------------------------------------------------------%

\clearpage

%---------------------------------------------------------------------------%
\index{append!( const T\& allele )}
    \printMethodWithOneParam
    {void} 
    {append}
    {const {\sl T}\&} 
    {allele} 
    {Value to be appended at the end of {\em this}.}
    {Appends the value {\em allele} at the end of chromosome {\em this}.}
    {None.}
    {None.}
%---------------------------------------------------------------------------%

\vspace*{4ex}

%---------------------------------------------------------------------------%
\index{append!( const Chromosome\& chrom )}
    \printMethodWithOneParam
    {void} 
    {append}
    {const Chromosome\&}
    {chrom} 
    {Chromosome to be appended at the end of {\em this}.}
    {Appends all alleles of chromosome {\em chrom} at the end of {\em this}.}
    {None.}
    {This method does not work on systems using a gcc-compiler 
         in version less than 2.91.x.}
%---------------------------------------------------------------------------%

\vspace*{4ex}

%---------------------------------------------------------------------------%
\index{remove!( unsigned i )}
    \printMethodWithOneParam
    {void} 
    {remove}
    {unsigned}
    {i} 
    {Index of the allele that is removed from {\em this}. {\em i}
     must be less than the number of alleles in {\em this}, otherwise
     the method will be aborted with an error message.}
    {Removes the allele with index {\em i} from chromosome {\em this}.}
    {None.}
    {None.}
%---------------------------------------------------------------------------%

\vspace*{4ex}

%---------------------------------------------------------------------------%
\index{remove!( unsigned i, unsigned k )}
    \setCorrectWidthThree{8pt}
    \setParamOne{i}{unsigned}{Index of the first allele that is
removed from {\em this}. {\em i} must be less than or equal to {\em k},
otherwise no allele will be removed.}
    \setParamTwo{k}{unsigned}{Index of the last allele that is removed
from {\em this}. {\em k} must be less than the number of alleles in {\em this},
otherwise the method will be aborted with an error message.}
    \printMethodWithParamsSaved
        {void}
        {}
        {remove}
        {Removes all alleles in {\em this} from index {\em i} to index {\em k}
         from chromosome {\em this}.}
        {}
    \setCorrectWidthThree{4pt}
%---------------------------------------------------------------------------%

\clearpage

%---------------------------------------------------------------------------%
\index{rotateRight!( unsigned n )}
    \printMethodWithOneParam
    {void} 
    {rotateRight}
    {unsigned} 
    {n} 
    {Number of positions that the alleles of {\em this} shall be rotated.
     If no value is given for {\em n}, then all alleles are rotated 
     one position.}
    {Rotates all alleles inside {\em this} {\em n} positions to the right.
     This a cyclic shift, i.e. an allele at the end of the chromosome
     is not lost during the shift-operation, but is placed at the
     of the chromosome.}
    {None.}
    {This method is not defined for systems that represent a vector of
     type {\tt bool} as a packed bitstring.}
%---------------------------------------------------------------------------%

\vspace*{4ex}

%---------------------------------------------------------------------------%
\index{rotateLeft!( unsigned n )}
    \printMethodWithOneParam
    {void} 
    {rotateLeft}
    {unsigned} 
    {n} 
    {See above.}
    {The same as above, but here all alleles are rotated to the left.}
    {None.}
    {This method is not defined for systems that represent a vector of
     type {\tt bool} as a packed bitstring.}
%---------------------------------------------------------------------------%



