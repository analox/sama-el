%
%% 
%% file: replacement.tex
%%
%

% =======================================================================
	\subsection{Replacement}
% =======================================================================

A number of candidates (the offsprings) are generated from the current
population, e.\,g.\ by recombination and mutation,
and may replace some or all of the parents top produce the next
generation.

% -----------------------------------------------------------------------
	\subsubsection{Generational Replacement}
% -----------------------------------------------------------------------

In the generational replacement scheme the entire population is
replaced at once.  The generational replacement technique has some
potential drawbacks.  Even with an elitist strategy many of the best
individuals found may not reproduce at all or their chromosomes may be
lost due to \myindex{crossover} and \index{mutation}.  But generational
replacement is fairly robust against noisy objective functions\index{objective function!noisy} and
\myindex{changing environments} since each individual is re-evaluated whenenver
it turns up in the population.


% -----------------------------------------------------------------------
	\subsubsection{Steady-State Replacement}
	\label{replacement:subsubs:steadyStateReplacement}
% -----------------------------------------------------------------------

In the steady-state replacement scheme only a few members are
replaced at a time \cite{Whitley:89,Syswerda:89}.  Steady-state
replacement has one parameter---the number of new individuals to
create (the number of old individuals to replace, respectively). Note
that generational replacement is a special case of steady-state
replacement where the number of replaced individuals equals the
population size.  For a comparison of the two replacement schemes see
\cite{Syswerda:91}.

The condition can be imposed that only one copy of any individual
exists in the population at any time so that the population never
fully converges.  This variant is called \emph{steady-state without
duplicates}.  It involves some computational overhead, in that it
requires a large number of equality tests whenever an offspring is
created, but applying genetic algorithms to \emph{real} problems most
of the time is spent in evaluation.  So this additional computation
time is negligible with regard to the total computing time.

It should be noted that steady-state replacement may not work well
when the \myindex{objective function} is noisy or the environment is changing
since good individuals won't be deleted from the population even if
they are bad in the changed environment.  For further discussion see
\cite{Davis:91}, chapter 2.
